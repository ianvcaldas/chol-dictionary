\documentclass[10pt]{scrreprt}
\KOMAoptions{
  chapterprefix=false,
  headings=openany,
  headings=big,
  DIV=15,
}
\usepackage{scrhack}
\addtokomafont{disposition}{\rmfamily}
\addtokomafont{section}{\clearpage\Huge}
\renewcommand*{\raggedsection}{\centering}

% Fonts and encoding
\usepackage[T1]{fontenc}
\usepackage[utf8]{inputenc}
\DeclareUnicodeCharacter{28C}{ä}
\usepackage{beton}

\usepackage{microtype}
\frenchspacing
% \usepackage{indentfirst}
% \usepackage[parfill]{parskip}

\usepackage{multicol}

% Entries in header
\usepackage{fancyhdr}
\pagestyle{fancy}
\renewcommand{\headrulewidth}{0pt}
\renewcommand{\chaptermark}[1]{}
\renewcommand{\sectionmark}[1]{}

% Entries are "reverse indented"
\leftskip=\parindent
\parindent=-\parindent

% Whitespace between entries
\parskip=0.5\baselineskip

% Prevent underfull hbox warnings
\hbadness=10000

% Dictionary formatting

\newcommand{\entry}[1]{\textbf{#1}\markboth{#1}{#1}}
\newcommand{\maintitle}[1]{\end{multicols*}\part{#1}\begin{multicols*}{2}}
\newcommand{\alphaletter}[1]{\end{multicols*}\addsec{#1}\begin{multicols*}{2}}
\newcommand{\onedefinition}[1]{#1.}
\newcommand{\defsuperscript}[1]{\textsuperscript{(#1)}}
\newcommand{\nontranslationdef}[1]{\textit{#1}}
\newcommand{\partofspeech}[1]{\textit{#1}}
\newcommand{\spanishtranslation}[1]{#1}
\newcommand{\clarification}[1]{(\textit{#1})}
\newcommand{\cholexample}[1]{\textbf{#1}}
\newcommand{\exampletranslation}[1]{#1}
\newcommand{\dialectvariant}[1]{\\\textit{#1}:}
\newcommand{\dialectword}[1]{\textbf{#1}}
\newcommand{\alsosee}[1]{\\\textit{Véase} \textbf{#1}}
\newcommand{\relevantdialect}[1]{(\textit{#1})}
\newcommand{\culturalinformation}[1]{(\textit{#1})}
\newcommand{\secondaryentry}[1]{\\\textbf{#1}}
\newcommand{\secondpartofspeech}[1]{\textit{#1}}
\newcommand{\secondtranslation}[1]{#1}
\newcommand{\variation}[1]{\textit{Var.} \textbf{#1}}
\newcommand{\conjugationtense}[1]{[\textit{#1}}
\newcommand{\conjugationverb}[1]{\textit{de} \textbf{#1}]}
\newcommand{\otherconjugation}[1]{\textit{:} \textbf{#1}]}

\begin{document}

\tableofcontents

\begin{multicols*}{2}


\maintitle{CH'OL — ESPAÑOL}
\alphaletter{A}

\entry{a-}
\onedefinition{1}
\nontranslationdef{Prefijo que indica adjetivo posesivo de segunda persona.}
\onedefinition{2}
\nontranslationdef{Prefijo que indica pronombre personal de segunda persona.}

\entry{ab}
\partofspeech{s}
\spanishtranslation{hamaca}
\cholexample{J'mäl tyi ab.}
\exampletranslation{Está acostado en hamaca.}

\entry{abañ}
\partofspeech{s}
\spanishtranslation{laguna}
\dialectvariant{Sab.}
\dialectword{petyem}
\secondaryentry{kolem abañ}
\secondpartofspeech{s}
\secondtranslation{mar}

\entry{abäk}
\partofspeech{s}
\spanishtranslation{carbón}

\entry{abälel}
\relevantdialect{Sab.}
\partofspeech{s}
\spanishtranslation{noche}
\alsosee{ak'älel}

\entry{abi}
\defsuperscript{1}
\partofspeech{part}
\onedefinition{1}
\spanishtranslation{oye}
\cholexample{Abi ku.}
\exampletranslation{¡Oye!}
\onedefinition{2}
\spanishtranslation{así}
\cholexample{Che' abi tsa' ujtyi.}
\exampletranslation{Así sucedió, se dice.}

\entry{abi}
\defsuperscript{2}
\relevantdialect{Sab.}
\partofspeech{adv}
\spanishtranslation{ayer}
\alsosee{ak'bi}

\entry{akaras}
\relevantdialect{Tila}
\partofspeech{s}
\spanishtranslation{cántaro chico para llevar agua}

\entry{aktyañ}
\partofspeech{part}
\nontranslationdef{Palabra que introduce la oración o el discurso.}

\entry{aktye'}
\partofspeech{s}
\spanishtranslation{palo de chapaya}
\clarification{tiene muchas espinas}

\entry{Aktye'pa'}
\partofspeech{s}
\spanishtranslation{Arroyo del Palmar}
\clarification{colonia}

\entry{akuxañ}
\partofspeech{s}
\spanishtranslation{aguja}

\entry{ak'}
\defsuperscript{1}
\partofspeech{s}
\onedefinition{1}
\spanishtranslation{lengua}
\clarification{de la boca}
\onedefinition{2}
\spanishtranslation{bejuco}
\secondaryentry{iyäk'il}
\secondpartofspeech{s}
\secondtranslation{su bejuco}

\entry{ak'}
\defsuperscript{2}
\partofspeech{vt}
\spanishtranslation{dar}
\cholexample{Tsa' majli tyi ak' juñ.}
\exampletranslation{Fue a entregar una carta.}
\secondaryentry{mi iyäk' ch'ämja'}
\secondtranslation{él bautiza}
\secondaryentry{mi iyäk' tyi majañ}
\secondtranslation{él presta}
\secondaryentry{mi iyäk' tyi ñuk}
\secondtranslation{él cree}
\secondaryentry{mi iyäk' tyi tsäñañ}
\secondtranslation{lo deja que se enfríe}
\secondaryentry{x'ak'bety}
\secondpartofspeech{s}
\secondtranslation{el que lleva dinero para pagar la deuda de otra persona}

\entry{ak'ach}
\partofspeech{s}
\spanishtranslation{guajolote}
\clarification{macho o hembra}

\entry{ak'äl}
\partofspeech{adj}
\spanishtranslation{limpio}
\clarification{de vegetación}
\cholexample{Ak'äl jiñi bij.}
\exampletranslation{El camino está limpio.}
\cholexample{Mach ak'älik.}
\exampletranslation{No está limpio.}

\entry{ak'älel}
\partofspeech{s}
\spanishtranslation{noche}
\dialectvariant{Sab.}
\dialectword{abälel}

\entry{ak'bä ts'uñuñ}
\spanishtranslation{gorrioncillo}

\entry{ak'bi}
\partofspeech{adv}
\spanishtranslation{ayer}
\cholexample{Ak'bi tsa' juliyoñ ilayi.}
\exampletranslation{Ayer llegué acá.}
\dialectvariant{Sab.}
\dialectword{abi}

\entry{ak'juñ}
\partofspeech{s}
\spanishtranslation{mensajero}

\entry{ak'ñañ}
\partofspeech{vt}
\spanishtranslation{limpiar}
\clarification{sembrados}
\cholexample{Mi lakäk'ñañ jiñi cholel.}
\exampletranslation{Limpiamos la milpa.}
\secondaryentry{ak'ñäñtyel}
\secondpartofspeech{vi}
\secondtranslation{limpiarse}

\entry{ak'ñibil}
\partofspeech{adj}
\spanishtranslation{limpio}
\clarification{sembrado}
\cholexample{Ak'ñibil jiñi bu'lel.}
\exampletranslation{El frijolar está limpio.}

\entry{ach'}
\partofspeech{adj}
\spanishtranslation{mojado}
\cholexample{Ach' jiñi lum.}
\exampletranslation{La tierra está mojada.}
\secondaryentry{ach'esañ}
\secondpartofspeech{vt}
\secondtranslation{mojarlo}
\secondaryentry{mi iyäch'añ}
\secondtranslation{se moja}
\secondaryentry{mi iyäch'esañ}
\secondtranslation{lo moja}

\entry{ach'añ}
\partofspeech{vi}
\spanishtranslation{mojarse}

\entry{ach'esäbil}
\partofspeech{adj}
\onedefinition{1}
\spanishtranslation{mojado}
\cholexample{Ach'esäbil jiñi lum cha'añja'al.}
\exampletranslation{La tierra está mojada por la lluvia.}
\onedefinition{2}
\spanishtranslation{regado}
\cholexample{Ach'esäbil jiñi päk'äbäl.}
\exampletranslation{La hortaliza está regada.}

\entry{aj-}
\nontranslationdef{Prefijo que se presenta con sustantivos de los dialectos de Sabanilla y Tila para indicar que se trata de una persona; p.ej.:}
\cholexample{ajkoltyaya}
\exampletranslation{el que ayuda.}
\alsosee{x-}

\entry{aja}
\partofspeech{part}
\spanishtranslation{una respuesta}

\entry{ajakña}
\partofspeech{adj}
\spanishtranslation{quejándose}
\cholexample{Ajakña jiñi wiñik cha'añ k'ux ijol.}
\exampletranslation{Ese hombre se está quejando por el dolor de cabeza.}

\entry{ajal}
\relevantdialect{Tila}
\partofspeech{s}
\spanishtranslation{espíritu malo}
\culturalinformation{Información cultural: Se dice que es el más peligroso de los espíritus. Es el espíritu del diablo. Es capaz de aparecer en distintas formas. Puede aparecer como una mujer amante, pero no como esposa. Al pecar un hombre con ella, se vuelve un animal y muere.}

\entry{ajapam}
\relevantdialect{Sab.}
\partofspeech{adv}
\spanishtranslation{adelante}
\cholexample{Ajapam mi majlel ktyaty.}
\exampletranslation{Mi papá va adelante.}
\alsosee{ñaxañ}

\entry{ajaw}
\partofspeech{s}
\onedefinition{1}
\spanishtranslation{espíritu malo de la tierra}
\culturalinformation{Información cultural: Lo nombran <lac tat> |r“nuestro padre”|r|i. Se cree que una persona puede hacer pacto con él. Esta persona puede hacerle peticiones a este espíritu a favor o en contra de otra persona. El hombre que tiene relaciones con el <ajaw> se llama sacristán. Si un hombre o una mujer ofende al sacristán, él acude a este espíritu para maldecir a la persona, y al poco tiempo la persona muere.}
\onedefinition{2}
\spanishtranslation{espíritu del agua}
\cholexample{Mi tsa' lakchoko lakbä tyi abañ mi ik'uxoñlajiñi ajaw.}
\exampletranslation{Si nos metemos en el agua, nos comerá el espíritu del agua.}
\onedefinition{3}
\spanishtranslation{un compañero del diablo}
\cholexample{Jiñi ajaw lajaläch ijoñtyolilbajche' xiba'.}
\exampletranslation{La maldad del <ajaw> es igual a la del diablo.}

\entry{ajbaj}
\relevantdialect{Sab.}
\partofspeech{s}
\spanishtranslation{tuza}
\alsosee{baj}

\entry{ajk}
\partofspeech{s}
\spanishtranslation{tortuga negra}

\entry{ajkal}
\partofspeech{s esp}
\spanishtranslation{alcalde}
\culturalinformation{Información cultural: La autoridad máxima que encabeza a los tres inferiores: <wasil, mayor, síntico>. Él posee dos bastones, varas bien adornadas con listones. Si un culpable no respeta a los tres inferiores, entonces el <ajcal> se lo comunica al presidente municipal para que él lo tome en cuenta. Permanecen como autoridades un año. Cada uno tiene que buscar a su sucesor. Sólo se aceptan personas de buen juicio y de más conocimientos. Deben ser ejemplo para mantener el orden.}

\entry{ajkächol}
\relevantdialect{Sab.}
\partofspeech{s}
\spanishtranslation{preso}
\alsosee{xkäjchel}

\entry{ajkäñ}
\partofspeech{s}
\spanishtranslation{gemido}
\cholexample{Woli tyi ajkäñ juñtyikil xk'amäjel.}
\exampletranslation{Está un enfermo gimiendo (lit.: está dando un gemido) de dolor.}

\entry{ajkäñty'añ}
\relevantdialect{Sab.}
\partofspeech{s}
\spanishtranslation{discípulo}
\alsosee{xk'äñty'añ}

\entry{ajkok}
\relevantdialect{Sab.}
\partofspeech{s}
\spanishtranslation{tortuga chica}

\entry{ajkoltyaya}
\relevantdialect{Sab.}
\partofspeech{s}
\onedefinition{1}
\spanishtranslation{auxiliador}
\onedefinition{2}
\spanishtranslation{uno que salva}
\alsosee{xkoltyaya}

\entry{ajkum}
\partofspeech{s}
\spanishtranslation{camote}

\entry{ajkum bik'}
\spanishtranslation{papera}

\entry{ajkuñts'u'}
\partofspeech{s}
\spanishtranslation{correcamino}
\clarification{ave}

\entry{ajchumtyäl}
\relevantdialect{Sab.}
\partofspeech{s}
\spanishtranslation{habitante}
\alsosee{xchumtyäl}

\entry{-ajel}
\nontranslationdef{Sufijo que se presenta con raíces atributivas, sustantivas y transitivas para formar otra raíz sustantiva que indica algún estado; p. ej.:}
\cholexample{k'amäjel}
\exampletranslation{en estado de enfermedad.}

\entry{aji}
\partofspeech{part}
\spanishtranslation{ciertamente}
\clarification{exclamación como ‘¡ya ves!’}
\cholexample{Aji, mach kajik.}
\exampletranslation{Ciertamente no se podrá.}

\entry{ajiñ}
\partofspeech{s}
\onedefinition{1}
\spanishtranslation{cocodrilo de pantano}
\onedefinition{2}
\spanishtranslation{cocodrilo de río}
\onedefinition{3}
\spanishtranslation{lagarto, caimán}

\entry{ajlel}
\partofspeech{vi}
\onedefinition{1}
\spanishtranslation{criticarse}
\cholexample{Woli tyi ajlel.}
\exampletranslation{Se están criticando.}
\onedefinition{2}
\spanishtranslation{decirse}
\cholexample{Che' woli tyi ajlel.}
\exampletranslation{Así están diciendo.}

\entry{ajluk'}
\partofspeech{s}
\spanishtranslation{salamandra venenosa}

\entry{ajlu'}
\relevantdialect{Tila}
\partofspeech{s}
\spanishtranslation{bagre barrigón}
\clarification{pez}

\entry{ajmulil}
\relevantdialect{Sab.}
\partofspeech{s}
\spanishtranslation{pecador}
\alsosee{mulil, xmulil}

\entry{ajñibäl}
\partofspeech{s}
\spanishtranslation{lugar}
\cholexample{Tyi pejtyelel ora ya'añ iyajñib.}
\exampletranslation{Allí siempre es su lugar.}

\entry{ajñel}
\defsuperscript{1}
\partofspeech{vi}
\onedefinition{1}
\spanishtranslation{correr}
\cholexample{Wosiña jiñi ch'ityoñ kome woli tyi ajñel.}
\exampletranslation{Está jadeante el joven por estar corriendo.}
\onedefinition{2}
\spanishtranslation{llegar a una parte}
\cholexample{¿baki mi awajñel?}
\exampletranslation{¿A dónde llegas?}

\entry{ajñel}
\defsuperscript{2}
\partofspeech{vi}
\spanishtranslation{permanecer}
\cholexample{Mi iyajñel jiñi wäyibäl ya' tyi' mal otyoty.}
\exampletranslation{La cama permanece dentro de la casa.}

\entry{ajñesañ}
\partofspeech{vt}
\onedefinition{1}
\spanishtranslation{correr a, perseguir}
\cholexample{Yom mi awajñesañ majlel jiñi chityam}
\exampletranslation{Debes correr al puerco.}
\onedefinition{2}
\spanishtranslation{corretear}
\cholexample{Yom mi awajñesañ lok'el tyi awotyoty jiñi wiñik.}
\exampletranslation{Debes correr de tu casa a ese hombre.}

\entry{ajp'eltye'}
\relevantdialect{Sab.}
\partofspeech{s}
\spanishtranslation{aserrador}
\alsosee{xp'eltye'}

\entry{ajp'ujpuya}
\relevantdialect{Sab.}
\partofspeech{s}
\spanishtranslation{sembrador}
\alsosee{xpak'}

\entry{ajk'iñijel}
\relevantdialect{Sab.}
\partofspeech{s}
\spanishtranslation{participante en fiesta}

\entry{ajsubty'añ}
\relevantdialect{Sab.}
\partofspeech{s}
\spanishtranslation{orador, predicador}

\entry{*ajty'al}
\partofspeech{s}
\spanishtranslation{el hijo o la hija de la mujer}

\entry{ajtso'}
\partofspeech{s}
\onedefinition{1}
\spanishtranslation{pavo}
\clarification{ave doméstica}
\onedefinition{2}
\spanishtranslation{espíritu malo que es compañero del brujo}

\entry{ajts'ijb}
\relevantdialect{Sab.}
\partofspeech{s}
\spanishtranslation{escribano}
\alsosee{sts'ijb}

\entry{ajxujch'}
\relevantdialect{Sab.}
\partofspeech{s}
\spanishtranslation{ladrón}
\alsosee{xujch'}

\entry{al}
\defsuperscript{1}
\partofspeech{adj}
\spanishtranslation{pesado}
\cholexample{Weñ al jiñi kuchäl.}
\exampletranslation{La carga es muy pesada.}

\entry{al}
\defsuperscript{2}
\partofspeech{vi}
\spanishtranslation{decir}
\cholexample{Mi käl jiñi wolibä kña'tyañ.}
\exampletranslation{Digo lo que estoy pensando.}

\entry{-al}
\defsuperscript{1}
\nontranslationdef{Sufijo que se presenta con raíces sustantivas para formar otra raíz sustantiva que indica extensión o lugar de algo; p. ej.:}
\cholexample{ilumal}
\exampletranslation{su país.}
\variation{2*-il}

\entry{-al}
\defsuperscript{2}
\nontranslationdef{Sufijo que se presenta con raíces transitivas y neutras para formar otra raíz atributiva que indica posición; p. ej.:}
\cholexample{wa'al}
\exampletranslation{parado.}
\variation{-äl, 2*-el, 1*-ol, -ul}

\entry{*al}
\partofspeech{s}
\spanishtranslation{cría}
\cholexample{Añ iyal jiñi xña'a ts'i'.}
\exampletranslation{Esa perra tiene cría.}

\entry{alas}
\partofspeech{s}
\spanishtranslation{juego}
\cholexample{Woliyob tyi alas.}
\exampletranslation{Están jugando.}
\dialectvariant{Sab.}
\dialectword{'ñak}
\secondaryentry{alasäl}
\secondpartofspeech{s}
\secondtranslation{juguete}
\secondaryentry{alas ty'añ}
\secondpartofspeech{s}
\secondtranslation{broma, chiste}

\entry{alaxax}
\partofspeech{s}
\spanishtranslation{naranja}

\entry{alaxaxil}
\partofspeech{s}
\spanishtranslation{naranjal}

\entry{alä bij}
\spanishtranslation{senda}

\entry{aläk'äl}
\partofspeech{s}
\spanishtranslation{animal doméstico}
\dialectvariant{Sab.}
\dialectword{aläk'il}

\entry{aläl}
\partofspeech{s}
\spanishtranslation{criatura}

\entry{alämuty}
\partofspeech{s}
\onedefinition{1}
\spanishtranslation{pollito}
\onedefinition{2}
\spanishtranslation{pájaro chico}
\alsosee{muty}

\entry{aläk'il}
\relevantdialect{Sab.}
\partofspeech{s}
\spanishtranslation{animal doméstico}
\alsosee{aläk'äl}

\entry{alä tyo}
\partofspeech{adj}
\spanishtranslation{pequeño}
\cholexample{Jiñi chityam alätyo.}
\exampletranslation{Ese puerco está pequeño todavía.}

\entry{alä wakax}
\partofspeech{s}
\spanishtranslation{becerro}

\entry{almasio}
\partofspeech{s esp}
\spanishtranslation{almácigo}

\entry{almis}
\partofspeech{s}
\spanishtranslation{ocre}

\entry{alob}
\partofspeech{s}
\spanishtranslation{niño}
\dialectvariant{Sab.}
\dialectword{bi'tyal}

\entry{alo'}
\relevantdialect{Sab.}
\partofspeech{s}
\spanishtranslation{joven}
\alsosee{ch'ityoñ}

\entry{alp'eñelob}
\partofspeech{s}
\spanishtranslation{niños}
\cholexample{Añ alp'eñelob tyi kotyoty.}
\exampletranslation{Hay niños en mi casa.}

\entry{al'añ}
\partofspeech{vi}
\spanishtranslation{hacerse pesado}
\clarification{físicamente}
\cholexample{Woli iyal'añ kuchäl.}
\exampletranslation{Se está haciendo pesada la carga.}

\entry{al'iya}
\relevantdialect{Tila}
\partofspeech{s}
\spanishtranslation{regaño}

\entry{am}
\partofspeech{s}
\onedefinition{1}
\spanishtranslation{araña}
\onedefinition{2}
\spanishtranslation{tarántula}
\clarification{venenosa}

\entry{Amarko}
\partofspeech{s esp}
\spanishtranslation{Amado Nervo}
\clarification{colonia}

\entry{amäy}
\partofspeech{s}
\onedefinition{1}
\spanishtranslation{carrizo}
\onedefinition{2}
\spanishtranslation{pito}

\entry{am bä i ye'tyel}
\spanishtranslation{una autoridad}
\cholexample{Jiñi komisariado jiñäch juñtyikil ambä iye'tyel tyi ejido.}
\exampletranslation{El comisario es una de las autoridades del ejido.}

\entry{ame}
\partofspeech{part}
\spanishtranslation{no sea que}
\cholexample{Yom ik'el doktyor ame muk'ik ichämel.}
\exampletranslation{Debe consultar al doctor no sea que muera.}

\entry{amiku}
\partofspeech{s}
\spanishtranslation{hombre tzeltal}
\secondaryentry{x'amiku}
\secondpartofspeech{s}
\secondtranslation{mujer tzeltal}

\entry{amila}
\partofspeech{s}
\spanishtranslation{guaje blanco}
\clarification{árbol}

\entry{añ}
\partofspeech{vi}
\spanishtranslation{hay}
\cholexample{Weñ añ tsäñal.}
\exampletranslation{Hace (lit.: hay) mucho frío}
\secondaryentry{añ ku}
\secondtranslation{si lo hay}
\secondaryentry{añik}
\secondpartofspeech{vi}
\secondtranslation{si hubiera}
\secondaryentry{añix ora}
\secondtranslation{ya tiene tiempo}
\secondaryentry{añtyo yom}
\secondtranslation{falta todavía}
\secondaryentry{añäch}
\secondpartofspeech{vi}
\secondtranslation{si hay}

\entry{-añ}
\defsuperscript{1}
\nontranslationdef{Sufijo que se presenta con raíces atributivas para formar una raíz intransitiva; p. ej.:}
\cholexample{k'am'añ}
\exampletranslation{enfermarse.}

\entry{-añ}
\defsuperscript{2}
\nontranslationdef{Sufijo que se presenta con raíces transitivas y neutras para formar una raíz transitiva; p. ej.:}
\cholexample{jisañ}
\exampletranslation{destruir.}
\variation{-iñ}

\entry{añ a jol}
\onedefinition{1}
\relevantdialect{Tum.}
\spanishtranslation{eres inteligente}
\onedefinition{2}
\relevantdialect{Sab.}
\spanishtranslation{estás loca}

\entry{añkese}
\partofspeech{part esp}
\spanishtranslation{aunque}
\cholexample{Añkese woli ija'al, che'äch mik majlel tyi e'tyel.}
\exampletranslation{Aunque esté lloviendo, me voy a ir a trabajar.}

\entry{añkimi}
\relevantdialect{Sab.}
\partofspeech{part}
\spanishtranslation{aunque}
\cholexample{Añkimi ma'añ kmul, mi iyäl ibajñel añ.}
\exampletranslation{Aunque no tengo delito, él dice que sí.}

\entry{añk'iñil}
\partofspeech{part}
\spanishtranslation{a veces}
\cholexample{Añk'iñil mi ikajel tsäñal.}
\exampletranslation{A veces hace frío.}

\entry{ak'eñ}
\partofspeech{imper}
\spanishtranslation{¡dáselo!}
\cholexample{Ak'eñ ak'äb.}
\exampletranslation{Dale la mano.}

\entry{ak'iñ}
\partofspeech{s}
\spanishtranslation{limpieza}
\cholexample{Woli tyi ak'iñ tyi' chol.}
\exampletranslation{Está haciendo la limpieza de su milpa.}

\entry{araweño}
\partofspeech{s esp}
\spanishtranslation{hierbabuena}
\alsosee{x'araweño}

\entry{arayojil}
\partofspeech{s esp}
\onedefinition{1}
\spanishtranslation{arroyo}
\onedefinition{2}
\spanishtranslation{laguna}

\entry{arieru}
\partofspeech{s esp}
\spanishtranslation{arriero}

\entry{arus}
\partofspeech{s esp}
\spanishtranslation{arroz}

\entry{askuñ}
\partofspeech{s}
\spanishtranslation{hermano mayor}
\alsosee{*äskuñ}

\entry{askuñäl}
\partofspeech{s}
\spanishtranslation{hombre de más edad sea hermano o hijo}

\entry{*asiñ}
\partofspeech{vt}
\spanishtranslation{juguetear}
\clarification{con un objeto}
\cholexample{Woli lakäsiñ pelotya.}
\exampletranslation{Estamos jugando pelota.}

\entry{asiyal}
\partofspeech{s}
\spanishtranslation{chicote}

\entry{asukal}
\partofspeech{s esp}
\spanishtranslation{azúcar}

\entry{-atyax}
\nontranslationdef{Sufijo para enfatizar; p. ej.:}
\cholexample{Uts'atyax iyotyoty.}
\exampletranslation{Es muy bonita su casa.}

\entry{ats'am}
\partofspeech{s}
\spanishtranslation{sal}
\secondaryentry{iyäts'mil}
\secondtranslation{está salado}

\entry{ats'am tye'}
\spanishtranslation{tipo de árbol}
\clarification{sirve para postes y leña}

\entry{awilañ}
\partofspeech{part}
\spanishtranslation{ya ves}
\cholexample{Awilañ ma'añik tsa' mejli imel iye'tyel.}
\exampletranslation{Ya ves, no pudo hacer su trabajo.}

\entry{awokolik}
\partofspeech{part}
\spanishtranslation{por favor}
\cholexample{Awokolik koltyañoñ}
\exampletranslation{Por favor, ayúdame.}

\entry{awujtyaya}
\relevantdialect{Sab.}
\partofspeech{s}
\spanishtranslation{curandero}
\alsosee{xwujty}

\entry{ax}
\partofspeech{s}
\spanishtranslation{moju}
\clarification{ramón; árbol de madera dura; la fruta es comestible}
\secondaryentry{chäk ax}
\secondpartofspeech{s}
\secondtranslation{árbol con fruta colorada}

\entry{axäñtye'}
\partofspeech{s}
\nontranslationdef{Es un tipo de verdura con hojas grandes y suaves que se comen.}

\entry{axñal}
\partofspeech{s}
\spanishtranslation{sombra}
\secondaryentry{iyäxñälel tye'}
\secondtranslation{la sombra del árbol}

\entry{axux}
\partofspeech{s}
\spanishtranslation{ajo}

\entry{a'leñ}
\partofspeech{vt}
\spanishtranslation{regañar}
\cholexample{Jiñi wiñik woli iyä'leñ iyijñam.}
\exampletranslation{Ese hombre está regañando a su mujer.}

\entry{a'leya}
\partofspeech{s}
\spanishtranslation{regaño}
\cholexample{Kabäl mi icha'leñ a'leya cha'añ mich'.}
\exampletranslation{Regaña mucho (lit.: hace regaño) porque está enojado.}

\entry{a'libäl}
\relevantdialect{Sab.}
\partofspeech{s}
\spanishtranslation{nuera}
\alsosee{ä'lib}

\alphaletter{Ä}

\entry{äch'uñiyel}
\partofspeech{vi}
\spanishtranslation{rechinar}
\cholexample{Mu' tyi äch'uñiyel jiñi ab che' mi lakcha'leñ jäjmel.}
\exampletranslation{La hamaca rechina cuando nos mecemos.}

\entry{äch'uña}
\partofspeech{ve}
\spanishtranslation{rechinando}
\cholexample{Äch'uña jiñi abba' kächäl.}
\exampletranslation{La hamaca está rechinando donde está amarrada.}

\entry{äj}
\partofspeech{adv}
\spanishtranslation{aquí}
\cholexample{Äjba tsa käk'ä.}
\exampletranslation{Aquí lo puse.}
\variation{um}

\entry{äjäkña}
\partofspeech{ve}
\spanishtranslation{quejándose}
\cholexample{Äjäkña jiñi xñox cha'añ k'ux ijol.}
\exampletranslation{Ese anciano está quejándose porque le duele la cabeza.}

\entry{-äl}
\nontranslationdef{Sufijo que se presenta con raíces transitivas y neutras para formar otra raíz atributiva que indica posición; p. ej.:}
\cholexample{päkäl}
\exampletranslation{boca abajo}
\variation{2*-al, 2*-el, 1*-ol, -ul}

\entry{*älas}
\partofspeech{s}
\spanishtranslation{juguete}

\entry{ämäl}
\partofspeech{adj}
\spanishtranslation{callado}
\cholexample{Ämäl jiñi wiñik.}
\exampletranslation{Ese hombre está callado.}

\entry{*äskuñ}
\partofspeech{s}
\spanishtranslation{hermano mayor}

\entry{*äts'mil}
\partofspeech{s}
\spanishtranslation{condimento}
\cholexample{Ma'añik iyäts'mil jiñi bu'ul.}
\exampletranslation{El frijol no tiene condimento.}

\entry{*äxñilel}
\partofspeech{s}
\spanishtranslation{sombra}
\cholexample{Ya' mi ijijlelob tyi' yäxñilel jiñi kolem tye'.}
\exampletranslation{Descansan allí, a la sombra del árbol.}
\alsosee{axñal}

\entry{*ä'lib}
\partofspeech{s}
\onedefinition{1}
\spanishtranslation{nuera}
\onedefinition{2}
\spanishtranslation{suegros de la mujer}
\dialectvariant{Sab.}
\dialectword{a'libäl}

\alphaletter{B}

\entry{ba}
\partofspeech{part}
\nontranslationdef{Indica pregunta; p. ej.:}
\cholexample{¿amba ja'as?}
\exampletranslation{¿Hay plátanos?}

\entry{bak}
\partofspeech{s}
\spanishtranslation{hueso}
\cholexample{Tyo'obak jiñi ts'i'.}
\exampletranslation{El perro es puro hueso.}
\secondaryentry{ibäkel}
\secondtranslation{su hueso}

\entry{Baktye'el}
\partofspeech{s}
\spanishtranslation{Bosque de Los Huesos}
\clarification{ranchería}

\entry{baj}
\defsuperscript{1}
\partofspeech{s}
\spanishtranslation{tuza}
\dialectvariant{Sab.}
\dialectword{ajbaj}

\entry{baj}
\defsuperscript{2}
\relevantdialect{Sab., Tila}
\partofspeech{vt}
\onedefinition{1}
\spanishtranslation{clavar}
\cholexample{Yom ma'baj ochel lawux tyi tye'.}
\exampletranslation{Hay que clavar el clavo en la tabla.}
\onedefinition{2}
\spanishtranslation{meter}
\cholexample{Mi lakbaj ochel ixim tyi koxtyal.}
\exampletranslation{Metemos el maíz en el costal.}
\alsosee{ch'ij}

\entry{bajañ}
\partofspeech{adj}
\spanishtranslation{solo}
\cholexample{Tyikbajañ jach tyik mele.}
\exampletranslation{Yo solo lo hice.}

\entry{bajbeñ}
\partofspeech{vt}
\spanishtranslation{pegar}
\clarification{de golpe}

\entry{-bajk'}
\onedefinition{1}
\nontranslationdef{Sufijo numeral que usan para contar unidades de cuatrocientos; p. ej.:}
\cholexample{Tsa' ityempayob ibä jumbajk' wiñikob.}
\exampletranslation{Se reunieron cuatrocientos hombres.}
\onedefinition{2}
\partofspeech{s}
\spanishtranslation{zontle}
\clarification{reg.}
\spanishtranslation{zonte}
\cholexample{Tsa' ik'aja jumbajk'.}
\exampletranslation{Cosechó un zontle de mazorcas.}

\entry{bajk'äl}
\partofspeech{adj}
\spanishtranslation{mucho}
\cholexample{Bajk'äl wiñikob tsa' ityempayob ibä.}
\exampletranslation{Muchos hombres se reunieron.}

\entry{bajche'}
\partofspeech{adv}
\onedefinition{1}
\spanishtranslation{como}
\cholexample{Tsa' imele che'bajche' tsa' subeñtyi.}
\exampletranslation{Lo hizo como fue indicado.}
\onedefinition{2}
\spanishtranslation{¿cómo?}
\cholexample{¿bajche' awilal?}
\exampletranslation{¿Cómo estás?}
\onedefinition{3}
\spanishtranslation{¿cúanto?}
\cholexample{¿bajche' ityojol?}
\exampletranslation{¿Cuánto cuesta?}

\entry{bajche' i kach}
\spanishtranslation{comoquiera}
\clarification{sin cuidado}
\cholexample{Bajche' ikach mi imel iye'tyel jiñi wiñik.}
\exampletranslation{Ese hombre hace su trabajo comoquiera.}

\entry{bajche' yilal}
\spanishtranslation{¿cómo está?}

\entry{-bajl}
\nontranslationdef{Sufijo numeral que usan para contar rollos de algo; p. ej.:}
\cholexample{jumbajl yopom}
\exampletranslation{un rollo de hojas.}

\entry{bajlum}
\partofspeech{s}
\spanishtranslation{tigre, jaguar}
\clarification{mamífero}

\entry{bajñel}
\partofspeech{adj}
\spanishtranslation{solo}
\cholexample{Ibajñel jaxtyo añ.}
\exampletranslation{Todavía es soltero.}
\secondaryentry{ibajñel}
\secondtranslation{él solo}
\secondaryentry{ibajñelil}
\secondtranslation{solito}

\entry{bajk'el}
\partofspeech{s}
\onedefinition{1}
\spanishtranslation{parto}
\cholexample{Jiñi x'ixik woli tyibajk'el.}
\exampletranslation{Esa mujer está sufriendo dolores de parto.}
\onedefinition{2}
\spanishtranslation{llanto}
\clarification{en extremo}

\entry{bajtyuñ}
\partofspeech{s}
\spanishtranslation{fruta tierna del árbol chapaya}
\clarification{tipo de palmera}

\entry{-bal}
\nontranslationdef{Sufijo que se presenta con raíces sustantivas y transitivas para formar una raíz sustantiva: p. ej.:}
\cholexample{Woli tyi sibal.}
\exampletranslation{Está buscando leña.}

\entry{balakña}
\defsuperscript{1}
\partofspeech{adv}
\spanishtranslation{estruendosamente}
\clarification{como creciente}
\cholexample{Balakña jiñi ja' che' weñ buty'ul.}
\exampletranslation{Cuando el río está lleno, corre estruendosamente.}

\entry{balakña}
\defsuperscript{2}
\partofspeech{adj}
\spanishtranslation{moviéndose}
\clarification{como buena milpa}
\cholexample{Balakña jiñi cholel.}
\exampletranslation{La milpa está moviéndose muy bien.}

\entry{balisajlel}
\partofspeech{s esp}
\spanishtranslation{estacas}
\clarification{se meten en donde se van a sembrar las matas de café}

\entry{baki}
\partofspeech{adv}
\onedefinition{1}
\spanishtranslation{donde}
\cholexample{Tsa' majlibaki weñ uts'aty jiñi lum.}
\exampletranslation{Se fue donde la tierra es muy fértil.}
\onedefinition{2}
\spanishtranslation{¿dónde?}
\cholexample{¿baki tsajñiyety?}
\exampletranslation{¿Dónde fuiste?}
\secondaryentry{bakibä}
\secondtranslation{¿cuál es?}
\secondaryentry{baki jachbä}
\secondtranslation{cualquiera}
\secondaryentry{baki ora}
\secondtranslation{cuando, ¿cuándo?}
\secondaryentry{baki añ}
\secondtranslation{¿dónde está?}
\secondaryentry{bak añ}
\secondtranslation{¿dónde está?}

\entry{baki jach}
\spanishtranslation{dondequiera}
\dialectvariant{Sab.}
\dialectword{ba'ika}

\entry{barsiñ}
\partofspeech{adj}
\spanishtranslation{rayado}
\cholexample{Barsiñtyik iyok jiñi wakax.}
\exampletranslation{La pata de la vaca está rayada.}

\entry{batyika}
\relevantdialect{Sab.}
\partofspeech{part}
\spanishtranslation{será}
\cholexample{Ma'ix mi ikäñob. ¿jimbatyika?}
\exampletranslation{Ya no lo conocen. ¿Será que es él?}

\entry{batyiya tye'}
\partofspeech{s esp}
\spanishtranslation{batea}

\entry{bats'}
\partofspeech{s}
\spanishtranslation{mono}

\entry{Baxija'}
\partofspeech{s}
\spanishtranslation{Piedritas de Agua}
\clarification{comunidad}

\entry{bayakña}
\partofspeech{adj}
\spanishtranslation{muy gordo}
\cholexample{Bayakña jiñi chityam.}
\exampletranslation{Ese cerdo está muy gordo.}

\entry{bayil}
\partofspeech{s}
\spanishtranslation{bejuco silvestre}
\culturalinformation{Información cultural: Se usa para tejidos de muebles y canastas.}

\entry{ba'}
\partofspeech{adv}
\spanishtranslation{donde}

\entry{ba'ikach bä}
\spanishtranslation{cualquiera}
\cholexample{Ba'ikachbä mi mejlel lakmel.}
\exampletranslation{Podemos hacer cualquier cosa.}

\entry{ba'ikal}
\relevantdialect{Sab.}
\partofspeech{adv}
\spanishtranslation{dondequiera}
\cholexample{Mi choñ ikapeba'ikal mi imämbeñtyel.}
\exampletranslation{Vende su café dondequiera que se lo compran.}

\entry{ba'ixtyi}
\partofspeech{part}
\spanishtranslation{no sé}

\entry{bä}
\partofspeech{part}
\onedefinition{1}
\nontranslationdef{Convierte el adjetivo en sustantivo; p. ej.:}
\cholexample{Yom mi ayajkañ juñkojty jujp'embä.}
\exampletranslation{Debes escoger uno gordo (|ianimal|i).}
\onedefinition{2}
\nontranslationdef{Da énfasis a un pronombre; p. ej.:}
\cholexample{Ibä jach tsa' majli.}
\exampletranslation{Él se fue por su propia voluntad.}
\onedefinition{3}
\nontranslationdef{Se presenta con palabras que indican el aspecto del verbo para marcar el sujeto u objeto de una acción; p. ej.:}
\cholexample{Jiñäach tsa'bä imele'.}
\exampletranslation{Él es la persona que lo hizo.}
\onedefinition{4}
\nontranslationdef{Se usa en los pronombres reflexivos; p. ej.:}
\cholexample{Tsa' ijats' ibä.}
\exampletranslation{Él se pegó a sí mismo.}

\entry{bäbäk'eñ}
\partofspeech{s}
\spanishtranslation{peligro}
\cholexample{Bäbäk'eñ mi lakyajlel.}
\exampletranslation{Hay peligro de que caigamos.}

\entry{bäkäl}
\partofspeech{s}
\spanishtranslation{olote}

\entry{bäkbäkñiyel}
\partofspeech{vi}
\spanishtranslation{temblar}
\cholexample{Woli tyibäkbäkñiyel cha'añ tsäñal.}
\exampletranslation{Está temblando de frío.}

\entry{bäkch'ujm}
\partofspeech{s}
\spanishtranslation{pepita de calabaza}

\entry{bäkch'umtye'}
\partofspeech{s}
\spanishtranslation{cuipu}
\clarification{reg.}
\spanishtranslation{piñoncillo}
\clarification{árbol}

\entry{bäk'}
\defsuperscript{3}
\partofspeech{vt}
\spanishtranslation{envolver}
\clarification{criatura}
\cholexample{Mi lakbäk' aläl tyi majts.}
\exampletranslation{Envolvemos a la criatura con su pañal.}

\entry{bäk'}
\defsuperscript{1}
\partofspeech{s}
\onedefinition{1}
\spanishtranslation{semilla}
\cholexample{Añ ibäk' jiñi ch'ujm.}
\exampletranslation{La calabaza tiene semilla.}
\onedefinition{2}
\spanishtranslation{pupila}
\secondaryentry{ibäk' lakwuty}
\secondtranslation{niña del ojo}
\secondaryentry{ibäk' juloñib}
\secondtranslation{tiro o cartucho de una arma}
\secondaryentry{bäk' iyaty}
\secondtranslation{testículos}

\entry{bäk'}
\defsuperscript{2}
\partofspeech{adv}
\spanishtranslation{luego}
\cholexample{Tsi'bäk' tsäñsa.}
\exampletranslation{Luego lo mató.}

\entry{bäk'ñañ}
\partofspeech{vt}
\spanishtranslation{temer}
\cholexample{Mikbäk'ñañ ts'i'.}
\exampletranslation{Le tengo miedo al perro.}

\entry{bäk'tyaläl}
\partofspeech{s}
\spanishtranslation{cuerpo}
\secondaryentry{ibäk'tyal}
\secondtranslation{su cuerpo}

\entry{bäk'tyesañ}
\partofspeech{vt}
\spanishtranslation{espantar}

\entry{bäch'}
\partofspeech{vt}
\spanishtranslation{enrollar}
\clarification{hilo o ixtle}
\cholexample{Mi'bäch' chij che' yomix imel ab.}
\exampletranslation{Enrolla el ixtle cuando va a hacer hamaca.}

\entry{bäch'äl}
\partofspeech{adj}
\spanishtranslation{enrollado}
\clarification{estado}
\cholexample{Bäch'äl jiñi tye' tyi ak'.}
\exampletranslation{El árbol está enrollado por el bejuco.}

\entry{bäch'bil}
\partofspeech{adj}
\spanishtranslation{enrollado}
\clarification{por alguien}
\cholexample{Weñbäch'bil jiñi puy.}
\exampletranslation{El hilo está bien enrollado.}

\entry{bäch'tyäl}
\partofspeech{vi}
\spanishtranslation{enrollarse}
\cholexample{Mi'bäch'tyäl jiñi puy.}
\exampletranslation{Se enrolla el hilo.}

\entry{bäjäch}
\partofspeech{adj}
\spanishtranslation{culpable}
\cholexample{Bäjäch abä tsa' tyeche letyo.}
\exampletranslation{Eres culpable de comenzar el pleito.}

\entry{bäjix}
\partofspeech{adj}
\spanishtranslation{responsable}
\cholexample{Bäjix ibä mi tsa' icha' lowo ibä.}
\exampletranslation{Él es responsable si se vuelve a lastimar.}

\entry{bäjlel}
\partofspeech{vi}
\spanishtranslation{ponerse}
\clarification{el sol}
\cholexample{Woli tyibäjlel k'iñ.}
\exampletranslation{El sol está poniéndose.}

\entry{*bäjlib k'iñ}
\spanishtranslation{poniente}

\entry{bäjñel}
\partofspeech{adv}
\spanishtranslation{mucho}
\cholexample{Bäjñeltyo käk'iñ kajpelel.}
\exampletranslation{Todavía me falta limpiar mucho en mi cafetal.}

\entry{*bäjk'il}
\partofspeech{s}
\onedefinition{1}
\spanishtranslation{pañales}
\onedefinition{2}
\spanishtranslation{mortaja}
\cholexample{Wersa yom ibäjk'il jiñi ch'ujleläl.}
\exampletranslation{Es necesario que tenga mortaja el cadáver.}

\entry{bäl}
\defsuperscript{1}
\partofspeech{vt}
\spanishtranslation{enrollar}
\clarification{tela o papel}
\cholexample{Woli'bäl itsuts cha'añ mi ich'äm majlel yik'oty.}
\exampletranslation{Está enrollando su cobija para llevarla consigo.}

\entry{bäl}
\defsuperscript{2}
\partofspeech{s}
\onedefinition{1}
\spanishtranslation{contenido, muchas cosas}
\cholexample{Kabäl ibäl jiñi otyoty.}
\exampletranslation{Esa casa tiene muchas cosas adentro.}
\onedefinition{2}
\spanishtranslation{alimento}
\cholexample{Ma'añix ibäl latyu.}
\exampletranslation{Su plato ya no tiene alimento.}

\entry{bäläkña}
\partofspeech{adv}
\nontranslationdef{Se relaciona con la forma de rodar (como un lápiz).}

\entry{bäläk'}
\relevantdialect{Tila}
\onedefinition{1}
\partofspeech{adv}
\spanishtranslation{inmediatamente}
\cholexample{Tsa'bäläk' kajiyob tyi letyo.}
\exampletranslation{Inmediatamente empezaron a pelear.}
\onedefinition{2}
\partofspeech{vi}
\spanishtranslation{revolcarse}
\cholexample{Chäñkol tyibäläk' jiñi mula.}
\exampletranslation{La mula está revolcándose.}

\entry{bäläl}
\partofspeech{adj}
\spanishtranslation{empinado}
\cholexample{Bäläl jiñi wits.}
\exampletranslation{El cerro está empinado.}

\entry{bälk'uñ}
\partofspeech{vt}
\spanishtranslation{rodar}
\clarification{palo}
\cholexample{Mi lakbälk'uñ jubel tye'.}
\exampletranslation{Bajamos rodando el tronco del árbol.}

\entry{bälch'uñ}
\partofspeech{vt}
\spanishtranslation{torcer}
\clarification{con las manos}

\entry{bälmatye'el}
\partofspeech{s}
\spanishtranslation{animal silvestre}

\entry{bälñäk'äl}
\partofspeech{s}
\spanishtranslation{alimento}
\secondaryentry{ibäl lakñäk'}
\secondtranslation{nuestro alimento}

\entry{bälulañ}
\partofspeech{vt}
\spanishtranslation{enrollar}
\cholexample{Yom ma'bälulañ jiñi tsuts.}
\exampletranslation{Debes enrollar la cobija.}

\entry{bäñ}
\partofspeech{adj}
\nontranslationdef{Palabra para describir un objeto largo; p. ej.:}
\cholexample{Mi lakbäñ ch'uy letsel jiñi kukujl.}
\exampletranslation{Levantamos la viga.}

\entry{bäñäkña}
\partofspeech{adv}
\nontranslationdef{Se relaciona con el movimiento difícil o pesado (de pez o de culebra); p. ej.}
\cholexample{Bäñäkña mi imajlel jiñi kolem chäy ya' tyi ja'.}
\exampletranslation{Ese pez grande nada pesadamente en el agua.}

\entry{bäñlaw}
\partofspeech{adv}
\spanishtranslation{desordenadamente}
\cholexample{Baki jachixbäñlaw tsa' ikäyäyob jiñi tye'.}
\exampletranslation{Dejaron los palos desordenadamente, por dondequiera.}

\entry{bäñäl}
\partofspeech{adj}
\nontranslationdef{Palabra para describir forma larga y delgada; p. ej.:}
\cholexample{Jiñi lápizbäñäl tsa' käle tyi lum.}
\exampletranslation{Ese lápiz se quedó en el suelo.}

\entry{*bäkel ejäl}
\spanishtranslation{diente}

\entry{*bäkel joläl}
\spanishtranslation{calavera}

\entry{bäk'eñ}
\partofspeech{s}
\spanishtranslation{miedo}
\cholexample{Woli tyibäk'eñ jiñi x'ixik che' mi ik'el lukum.}
\exampletranslation{La mujer siente miedo al ver la culebra.}

\entry{bäk'eñ bäk'eñ jax}
\spanishtranslation{temeroso}
\cholexample{Bäk'eñbäk'eñ jax yilal jiñi wiñik.}
\exampletranslation{Parece que ese hombre es algo temeroso.}

\entry{bätye'el}
\partofspeech{s}
\spanishtranslation{mamífero silvestre}

\entry{bäwits}
\relevantdialect{Sab.}
\partofspeech{s}
\spanishtranslation{dueño del cerro, diablo}

\entry{bäx}
\partofspeech{adj}
\spanishtranslation{activo}
\cholexample{Bäx jiñi wiñik tyi' ye'tyel.}
\exampletranslation{Ese hombre es activo en su trabajo.}

\entry{*bäxlel}
\partofspeech{s}
\spanishtranslation{energía}
\cholexample{Añ ibäxlel jiñi x'ixik cha'añ e'tyel.}
\exampletranslation{Esa mujer tiene energía para trabajar.}

\entry{bek'}
\partofspeech{vt}
\spanishtranslation{derramar}

\entry{bech}
\onedefinition{1}
\partofspeech{vt}
\spanishtranslation{ladear}
\cholexample{Jiñi avióñ mi ibech ibä.}
\exampletranslation{El avión se ladea.}
\onedefinition{2}
\partofspeech{adv}
\spanishtranslation{de lado}
\cholexample{Tsa' bech yajli jiñi ch'ejew.}
\exampletranslation{El cajete se cayó de lado.}

\entry{bech'}
\partofspeech{vt}
\spanishtranslation{enrollar}

\entry{bej}
\partofspeech{adv}
\onedefinition{1}
\spanishtranslation{más}
\cholexample{Bej choko tyilel abä.}
\exampletranslation{Acércate más.}
\onedefinition{2}
\spanishtranslation{siempre}
\cholexample{Mik bej k'elety.}
\exampletranslation{Siempre te veo.}

\entry{*bejch'il}
\partofspeech{s}
\spanishtranslation{venda}

\entry{bejch'iñ}
\partofspeech{vt}
\spanishtranslation{enrollar}
\clarification{hilo o ixtle}

\entry{bejlañ}
\partofspeech{vt}
\spanishtranslation{llevar o traer carga}
\clarification{varios viajes}
\cholexample{Yom mi abejlañ tyilel xajlel.}
\exampletranslation{Hay que traer varios viajes de piedra.}

\entry{bejts'el}
\partofspeech{vi}
\spanishtranslation{pasar}
\clarification{mediodía}

\entry{bejts'em}
\partofspeech{adj}
\spanishtranslation{pasado}
\clarification{después del mediodía}

\entry{belekña}
\partofspeech{adv}
\onedefinition{1}
\nontranslationdef{Se relaciona con el movimiento en formación: p. ej.:}
\cholexample{Belekña mi icha'leñ xämbal jiñi xu'.}
\exampletranslation{Las arrieras andan en formación.}
\onedefinition{2}
\spanishtranslation{constantemente}
\cholexample{Belekña mi iñumel jiñi k'iñ.}
\exampletranslation{Los días pasan constantemente.}

\entry{belel}
\partofspeech{adv}
\onedefinition{1}
\nontranslationdef{Se relaciona con la formación en línea; p. ej.:}
\cholexample{Belel tyak tsa' awa'chokoñtyi jiñi otyoty tyak.}
\exampletranslation{Esas casas se construyen en línea.}
\onedefinition{2}
\spanishtranslation{seguido}
\cholexample{Belel ora mi ijulel jiñi avióñ.}
\exampletranslation{Ese avión viene seguido.}

\entry{beljul}
\partofspeech{vt}
\spanishtranslation{ensartar}
\clarification{gargantillas}
\cholexample{Mi kaj kbeljul jiñi uya'äl.}
\exampletranslation{Voy a ensartar los collares.}

\entry{beljulbil}
\partofspeech{adj}
\spanishtranslation{ensartado}
\clarification{gargantillas}
\cholexample{Beljulbil ujäl tyi päy.}
\exampletranslation{El collar está ensartado en hilo.}

\entry{-beñ}
\nontranslationdef{Sufijo que ndica la persona que recibe la acción del verbo; p. ej.:}
\cholexample{Säkläbeñoñ juñkojty mula.}
\exampletranslation{Busca una mula para mí.}

\entry{bety}
\partofspeech{s}
\spanishtranslation{deuda}
\cholexample{Samiktyoj kbety.}
\exampletranslation{Voy a pagar mi deuda.}

\entry{betyañ}
\partofspeech{vt}
\spanishtranslation{pedir fiado}
\cholexample{Mi kaj kbetyañ jiñi lum.}
\exampletranslation{Pediré fiado ese terreno.}

\entry{bets'}
\partofspeech{vt}
\spanishtranslation{ladear}
\cholexample{Weñ kabäl mi ibets' ibä jiñi jukub.}
\exampletranslation{Se ladea mucho ese cayuco.}

\entry{bets'el}
\partofspeech{adj}
\spanishtranslation{de lado}
\cholexample{Bets'el jiñi avióñ che' mi kaj tyi joyokñiyel.}
\exampletranslation{El avión está de lado al dar vuelta.}

\entry{bets'uña}
\partofspeech{adv}
\nontranslationdef{Se relaciona con el movimiento de algo ladeado; p. ej.:}
\cholexample{Bets'uña tsa' majli jiñi jukub.}
\exampletranslation{Ese cayuco se fue de lado.}

\entry{bexel}
\partofspeech{adj}
\spanishtranslation{desnivelado, canteado}
\cholexample{Bexel tsa' käle jiñi kotyoty.}
\exampletranslation{Se quedó desnivelada mi casa.}

\entry{-bi}
\nontranslationdef{Sufijo que se presenta con raíces atributivas para formar otra raíz atributiva que indica tiempo; p. ej.:}
\cholexample{ak'bi}
\exampletranslation{ayer}

\entry{bibesäbil}
\partofspeech{adj}
\spanishtranslation{sucio}
\cholexample{Bibesäbil jiñi ja'.}
\exampletranslation{El agua está sucia.}

\entry{bibi'}
\partofspeech{adj}
\spanishtranslation{sucio}
\cholexample{Weñ bibi' apislel.}
\exampletranslation{Tu ropa está muy sucia.}

\entry{*bibi'lel}
\partofspeech{s}
\spanishtranslation{suciedad}
\cholexample{Kabäl ibibi'lel jiñi pisil.}
\exampletranslation{Esa ropa tiene mucha suciedad.}

\entry{bibu}
\partofspeech{adj}
\spanishtranslation{listo}
\cholexample{Weñ bibu jiñi wiñik.}
\exampletranslation{Ese hombre es muy listo.}

\entry{bib'añ}
\partofspeech{vi}
\spanishtranslation{ensuciarse}
\cholexample{Tsa' bib'a ak'äb.}
\exampletranslation{Se te ensució la mano.}

\entry{bib'esañ}
\partofspeech{vt}
\spanishtranslation{ensuciar}
\cholexample{Jiñi chityam mi ibib'esañ jiñi otyoty.}
\exampletranslation{El puerco ensucia la casa.}

\entry{bik'}
\partofspeech{s}
\spanishtranslation{cuello}
\secondaryentry{ibik' lakok}
\secondtranslation{tobillo}
\secondaryentry{ibik' laj k'äb}
\secondtranslation{coyuntura}
\clarification{de la mano}

\entry{bik'tyal}
\onedefinition{1}
\partofspeech{s}
\spanishtranslation{molcate}
\clarification{reg.}
\spanishtranslation{mazorca pequeña}
\onedefinition{2}
\partofspeech{adj}
\spanishtranslation{atrofiado, mal desarrollado}
\secondaryentry{ibik'tyal ixim}
\secondtranslation{molcate del maíz}
\secondaryentry{ibik'tyal tye'}
\secondtranslation{árboles pequeños}

\entry{bik'tyi}
\partofspeech{adv}
\nontranslationdef{Se relaciona con algo convertido en pedazos pequeños; p. ej.:}
\cholexample{Tsa' ibik'tyi xulu tye'.}
\exampletranslation{Cortó el palo en pedacitos.}

\entry{bik'tyi chä'tye'}
\spanishtranslation{chicozapote}
\clarification{árbol}

\entry{bik'tyi ijts'iñ}
\spanishtranslation{sobrino, nieto}

\entry{bik'tyi päm}
\spanishtranslation{tucanete verde}
\clarification{ave}

\entry{bij}
\partofspeech{s}
\spanishtranslation{camino}
\secondaryentry{alä bij}
\secondtranslation{vereda}
\secondtranslation{senda}
\secondaryentry{kolem bij}
\secondtranslation{camino real}

\entry{*bijleyok ja'}
\spanishtranslation{zanja}

\entry{bijty'el}
\partofspeech{vi}
\spanishtranslation{echarse a un lado por susto}

\entry{bijyel}
\partofspeech{vi}
\spanishtranslation{frotarse}
\cholexample{Woli tyi bijyel tyi ts'ajk jiñi ik'äb.}
\exampletranslation{Se está frotando con medicina la mano.}

\entry{-bil}
\nontranslationdef{Sufijo que se presenta con raíces transitivas para formar otra raíz atributiva que indica que fue terminado por un agente; p. ej.:}
\cholexample{yäpbil}
\partofspeech{adj}
\exampletranslation{apagado.}

\entry{bilil}
\partofspeech{adj}
\spanishtranslation{delgado y alto}
\cholexample{Bilil tsa' koli jiñi tye'.}
\exampletranslation{Ese árbol creció muy delgado y alto.}

\entry{bilty'uñ}
\partofspeech{vt}
\spanishtranslation{frotar}
\clarification{piel}

\entry{bik'ity}
\partofspeech{adj}
\spanishtranslation{chico}
\cholexample{Weñ bik'ity iwuty jiñi kajpe'.}
\exampletranslation{Los granos del café están muy chicos.}
\dialectvariant{Tila, Sab.}
\dialectword{chuty}

\entry{*bik'itye'lel}
\partofspeech{s}
\spanishtranslation{varillas}
\cholexample{Ibik'itye'lel k'äty kächbil tyakbä jiñächba' mi ikäjchel ijamil otyoty.}
\exampletranslation{Las varillas que están atravesadas en el techo son las que sostienen el zacate.}

\entry{bits'}
\partofspeech{s}
\spanishtranslation{cocsán, cuajinicuil}
\clarification{árbol de hojas coloradas y fruta comestible}

\entry{bits'chajk}
\relevantdialect{Tila}
\partofspeech{s}
\spanishtranslation{cuajinicuil}
\clarification{árbol}
\alsosee{xlasobits'}

\entry{Bits'ol}
\partofspeech{s}
\spanishtranslation{Arboleda de Cocsán}
\clarification{ranchería}

\entry{bixeltyik}
\partofspeech{adv}
\spanishtranslation{de vez en cuando}
\cholexample{Bixeltyik mi ik'otyel tyi tyemplo.}
\exampletranslation{De vez en cuando llega al templo.}

\entry{biyulañ}
\partofspeech{vt}
\spanishtranslation{frotar}
\clarification{cosa resbalosa}
\cholexample{Yom mi abiyulañ awok yik'oty ts'ak.}
\exampletranslation{Te debes frotar con medicina el pie.}

\entry{bi'ijtyik}
\partofspeech{adj}
\spanishtranslation{asqueroso}
\cholexample{Bi'ijtyik jiñi ts'i'.}
\exampletranslation{Ese perro es asqueroso.}

\entry{bi'leñ}
\partofspeech{vt}
\spanishtranslation{tener asco por mal olor}
\cholexample{Tsa' ibi'le chämeñbätye'el.}
\exampletranslation{Tuvo asco por un animal muerto.}

\entry{bi'tyal}
\relevantdialect{Sab.}
\partofspeech{s}
\spanishtranslation{niño}
\alsosee{alob}

\entry{bi'tyibi'tyäl}
\relevantdialect{Sab.}
\partofspeech{adv}
\spanishtranslation{poquito en poquito}
\clarification{comer}
\cholexample{Bi'tyibityäl mi ik'ux.}
\exampletranslation{Come de poquito en poquito.}

\entry{bi'tyi muty}
\relevantdialect{Sab., Tila}
\spanishtranslation{pajarito}

\entry{bi'tyi xajlel}
\relevantdialect{Sab.}
\spanishtranslation{grava, piedra quebrada}

\entry{bob}
\partofspeech{s}
\spanishtranslation{flor}
\cholexample{Wolix ipasel ibob ja'as.}
\exampletranslation{Está brotando la flor del plátano.}

\entry{bok}
\partofspeech{vt}
\spanishtranslation{arrancar}
\clarification{una mata}
\cholexample{Yom mi abok jiñi ajkum.}
\exampletranslation{Debes arrancar el camote.}

\entry{bokbil}
\partofspeech{adj}
\spanishtranslation{arrancado}
\clarification{camote, papas, cebollas}
\cholexample{Bokbilix jiñi ajkum.}
\exampletranslation{Ya está arrancado el camote.}

\entry{bojlox}
\partofspeech{adj}
\spanishtranslation{con pocos granos}
\clarification{la mazorca}

\entry{*bojñib wutyäl}
\spanishtranslation{cosmético}

\entry{*bojñil}
\partofspeech{s}
\spanishtranslation{color, pintura}

\entry{bojkel}
\partofspeech{vi}
\onedefinition{1}
\spanishtranslation{arrancarse}
\cholexample{Mi ibojkel jiñi ajkum.}
\exampletranslation{Se arranca el camote.}
\onedefinition{2}
\spanishtranslation{caer}
\cholexample{Wolix tyi bojkel itsutsel ijol.}
\exampletranslation{Está cayéndose su cabello.}

\entry{bojtye'}
\partofspeech{s}
\spanishtranslation{pared de madera}

\entry{*bojtye'lel}
\partofspeech{s}
\spanishtranslation{cerca}
\clarification{de una casa}
\cholexample{Añix ibojtye'lel iyotyoty.}
\exampletranslation{Su casa ya tiene cerca.}

\entry{bojy}
\partofspeech{adj}
\spanishtranslation{resbaloso}
\cholexample{Bojy jiñi bij.}
\exampletranslation{El camino está resbaloso.}

\entry{bolomp'ejl}
\partofspeech{adj}
\spanishtranslation{nueve}

\entry{Boloñajaw}
\partofspeech{s}
\spanishtranslation{Nueve Espíritus}
\clarification{lugar}

\entry{boloñij}
\partofspeech{adv}
\spanishtranslation{el noveno día}

\entry{boloñlujump'ejl}
\partofspeech{adj}
\spanishtranslation{diecinueve}

\entry{bolo'ty'axal}
\partofspeech{adj}
\spanishtranslation{disparejo}
\clarification{poco}

\entry{boltyäl}
\relevantdialect{Sab.}
\partofspeech{s}
\spanishtranslation{cerro}
\cholexample{Mäkäl tyityokal jiñi boltyäl.}
\exampletranslation{Ese cerro está tapado por la nube.}
\alsosee{wits}

\entry{bombom jam}
\spanishtranslation{manojo de zacate}

\entry{boñ}
\partofspeech{vt}
\onedefinition{1}
\spanishtranslation{pintar}
\onedefinition{2}
\spanishtranslation{repellar}

\entry{boñtyäl}
\partofspeech{adv}
\spanishtranslation{así de grueso}
\cholexample{Che' ya boñtyäl jiñi muty.}
\exampletranslation{El pollo está así de grueso.}

\entry{*boñtyilel}
\partofspeech{s}
\spanishtranslation{el gordo}
\cholexample{Che'äch iboñtyälel jiñi muty.}
\exampletranslation{Así de gordo está el pollo.}

\entry{boñxañ}
\partofspeech{s}
\spanishtranslation{tipo de planta alta}
\clarification{la hoja sirve para techos}

\entry{boñxañtye'}
\partofspeech{s}
\spanishtranslation{palma}

\entry{borol}
\partofspeech{adj}
\spanishtranslation{trunco}
\cholexample{Borol ikäb.}
\exampletranslation{La mano está trunca.}
\secondaryentry{borolbä tye'}
\secondtranslation{tronco}
\clarification{árbol}

\entry{borol i bik'}
\spanishtranslation{sin cabeza}
\cholexample{Borol ich'och' jiñi lukum che' mi laktsep.}
\exampletranslation{La culebra se queda sin cabeza cuando se la cortamos.}

\entry{boroñch'och}
\partofspeech{s}
\spanishtranslation{papera}

\entry{borox}
\partofspeech{adj}
\spanishtranslation{sin pluma}
\cholexample{Bixeltyik borox tyak jiñi alä muty.}
\exampletranslation{A veces algunos pollitos están sin plumas.}

\entry{borxa}
\partofspeech{s esp}
\spanishtranslation{bolsa}

\entry{*borxajlel}
\partofspeech{s esp}
\spanishtranslation{bolsa}

\entry{bots}
\partofspeech{vi}
\spanishtranslation{criando}
\cholexample{Jiñi bu'ul tsa'bä p'ajtyi wolix tyi bots.}
\exampletranslation{El frijol que se cayó ya se está criando.}

\entry{bots'}
\partofspeech{vt}
\spanishtranslation{arrancar}
\clarification{poste}
\cholexample{La'lakbots' jiñi postye.}
\exampletranslation{Arranquemos ese poste.}

\entry{boxol}
\partofspeech{adv}
\spanishtranslation{hoyado}
\clarification{adentro en forma redonda}

\entry{bo'lay}
\partofspeech{s}
\spanishtranslation{jaguar}
\clarification{mamífero}

\entry{bo'oyel}
\relevantdialect{Sab.}
\partofspeech{vi}
\spanishtranslation{fastidiarse}
\cholexample{Bo'oyelix muk'oñ.}
\exampletranslation{Ya me fastidié.}
\alsosee{k'ojyel}

\entry{bo'tyek'}
\partofspeech{vt}
\spanishtranslation{patear}
\cholexample{Jiñi mulatsa' ibo'tyek'e jiñi wiñik.}
\exampletranslation{La mula pateó al hombre.}

\entry{buk'}
\relevantdialect{Sab.}
\partofspeech{vt}
\spanishtranslation{tragar}
\cholexample{Tsa' ibuk'u pastyila.}
\exampletranslation{Tragó la pastilla.}
\alsosee{mäsañ}

\entry{*buk'bal}
\partofspeech{s}
\spanishtranslation{comida}
\clarification{de animales}
\cholexample{Tsa'ix käk'e ibuk'bal chityam.}
\exampletranslation{Ya le di su comida al puerco.}

\entry{buk'sañ}
\partofspeech{vt}
\spanishtranslation{alimentar}
\clarification{animales}
\cholexample{Yom mi abuk'sañ mula.}
\exampletranslation{Dale de comer a la mula.}

\entry{buk'tsu'}
\partofspeech{s}
\spanishtranslation{mococha}

\entry{buch}
\partofspeech{vt}
\spanishtranslation{derrumbar}
\clarification{árbol, casa}
\cholexample{Tyik'äl yomix mi abuch awotyoty.}
\exampletranslation{Tal vez ya debes derrumbar tu casa.}

\entry{buchbuchña}
\partofspeech{adv}
\spanishtranslation{moviéndose}
\cholexample{Buchbuchña jiñi ñoxixbä iyotyoty.}
\exampletranslation{La casa vieja está moviéndose.}

\entry{buchchokoñ}
\partofspeech{vt}
\spanishtranslation{sentar}
\clarification{a una persona}
\cholexample{Yom mi abuchchokoñ jiñi aläl tyi tyem.}
\exampletranslation{Debes sentar al niño en el banco.}
\dialectvariant{Sab.}
\dialectword{ñakchokoñ}

\entry{buchiña}
\partofspeech{adv}
\spanishtranslation{moviéndose}
\cholexample{Buchiña jiñi ñoxixbä iyotyoty.}
\exampletranslation{Su casa vieja está moviéndose.}

\entry{*buchlib}
\partofspeech{s}
\spanishtranslation{asiento}

\entry{buchtyañ}
\partofspeech{vt}
\spanishtranslation{sentar sobre}
\cholexample{Mi lakbuchtyañ xajlel.}
\exampletranslation{Nos sentamos sobre una piedra.}

\entry{buchtyäl}
\partofspeech{vi}
\spanishtranslation{sentarse}
\cholexample{Woli tyi buchtyäl jiñi aläl tyi tyem.}
\exampletranslation{El niño se sentó en el banco.}
\dialectvariant{Sab.}
\dialectword{ñaktyäl}

\entry{buchul}
\partofspeech{adj}
\spanishtranslation{sentado}
\cholexample{Buchul jiñi aläl tyi tyem.}
\exampletranslation{El niño está sentado en el banco.}
\dialectvariant{Sab.}
\dialectword{ñakal}

\entry{buj}
\partofspeech{adv}
\spanishtranslation{con puño}
\cholexample{Tsa' ibuj jats'be ipaty.}
\exampletranslation{Le pegó con el puño en la espalda.}

\entry{bujb}
\partofspeech{s}
\spanishtranslation{renacuajo}

\entry{bujbujtyäl}
\partofspeech{adj}
\spanishtranslation{disparejo}
\clarification{terreno}
\cholexample{Ya' tyi klum puru bujbujtyäl.}
\exampletranslation{Mi terreno está muy disparejo.}

\entry{bujkäl}
\partofspeech{s}
\spanishtranslation{camisa}

\entry{bujchem}
\partofspeech{adj}
\spanishtranslation{caído}
\clarification{árbol, casa}
\cholexample{Che' mi iñumel ik' kabäl mi laktyaj bujchem tyakbä tye'.}
\exampletranslation{Cuando pasa un viento fuerte, encontramos muchos árboles caídos.}

\entry{bujchuñ}
\partofspeech{vt}
\spanishtranslation{tornear}
\clarification{mover para sacar}
\cholexample{Woli ibujchuñ postye.}
\exampletranslation{Está torneando el poste.}

\entry{bujlel}
\partofspeech{vi}
\spanishtranslation{perder la cáscara}
\cholexample{Mi ibujlel ixim che' mi ich'äxtyäl.}
\exampletranslation{El maíz pierde la cáscara cuando se hierve.}

\entry{bujlux}
\partofspeech{adj}
\spanishtranslation{subdesarrollado}
\clarification{granos, plumas}

\entry{bujtyäl}
\partofspeech{s}
\spanishtranslation{loma}

\entry{bujul}
\partofspeech{adj}
\spanishtranslation{abultado}
\cholexample{Bujul jiñi wits.}
\exampletranslation{Ese cerrito está abultado.}

\entry{bujulum}
\partofspeech{s}
\spanishtranslation{cima de una loma}

\entry{bul}
\partofspeech{adv}
\nontranslationdef{Se relaciona con la forma en que sale un tumor; p. ej.:}
\cholexample{Tsa' bul lok'i kchäkajl.}
\exampletranslation{Me salió un tumor.}

\entry{bulbulña}
\partofspeech{adv}
\nontranslationdef{Se relaciona con la forma en que brota el agua; p. ej.:}
\cholexample{Bulbulña mi ilok'el ja' tyi lum.}
\exampletranslation{El agua sale brotando del suelo.}

\entry{bultyäl}
\partofspeech{adv}
\spanishtranslation{abultadamente}
\clarification{señalando el tamaño de una hinchazón}

\entry{bulubexel}
\partofspeech{adj}
\spanishtranslation{disparejo}
\clarification{piso de la casa}

\entry{buluch}
\partofspeech{adj}
\spanishtranslation{once}

\entry{Bulujib}
\partofspeech{s}
\spanishtranslation{nombre de una colonia}

\entry{bulul}
\defsuperscript{1}
\partofspeech{adj}
\spanishtranslation{con lobanillo}
\secondaryentry{bulul ibik'}
\secondtranslation{lobanillo en el cuello}
\secondaryentry{xbulubik'}
\secondpartofspeech{s}
\secondtranslation{hombre con lobanillo}

\entry{bulul}
\defsuperscript{2}
\partofspeech{adj}
\spanishtranslation{abultado}
\cholexample{Bulul ipaty jiñi wakax.}
\exampletranslation{El cebú tiene abultada la espalda.}

\entry{bulu ok}
\relevantdialect{Tila}
\spanishtranslation{espíritu malo}
\culturalinformation{Información cultural: Se cree que vive en los cerros. Hace que las personas se pierdan en los caminos de los cerros. Uno puede defenderse al ponerse al revés la ropa.}

\entry{bulux}
\partofspeech{adj}
\onedefinition{1}
\spanishtranslation{desnudo}
\cholexample{Bulux jiñi muty.}
\exampletranslation{Ese pollo está desnudo.}
\onedefinition{2}
\spanishtranslation{brotado}
\cholexample{Bulux jiñi ja' tyi bij.}
\exampletranslation{En el camino hay agua brotada.}
\secondaryentry{bulux ja'}
\secondtranslation{brote de agua}

\entry{burburña}
\partofspeech{adv}
\spanishtranslation{ruidosamente}
\cholexample{Burburña woli tyi e'tyel jiñi makiña.}
\exampletranslation{La máquina está trabajando ruidosamente.}

\entry{burukña}
\partofspeech{adv}
\spanishtranslation{zumbando}
\clarification{una creciente o una caída}
\cholexample{Burukña jiñi ja' cha'añ weñ buty'ul.}
\exampletranslation{El río corre zumbando porque está crecido.}

\entry{busul}
\partofspeech{adj}
\spanishtranslation{montón}
\clarification{de tierra}
\cholexample{Busul lum tyi' tyi' bij.}
\exampletranslation{Hay un montón de tierra en la orilla del camino.}

\entry{buty'}
\partofspeech{vt}
\spanishtranslation{llenar}
\cholexample{Mi lakbuty' tyi ja' jiñibalde.}
\exampletranslation{Llenamos de agua el balde.}

\entry{buty'ja'}
\partofspeech{s}
\spanishtranslation{creciente}
\cholexample{Tyi ili jabil añ kabäl buty'ja'.}
\exampletranslation{En este año hubo mucha creciente.}

\entry{buty'ja'iyel}
\partofspeech{vi}
\spanishtranslation{inundarse}
\cholexample{Wolix ikajel buty'ja'iyel kome weñ woli ija'al.}
\exampletranslation{Ya se está inundando porque está lloviendo mucho.}

\entry{buty'ukña}
\partofspeech{adj}
\onedefinition{1}
\spanishtranslation{rebalsando}
\cholexample{Weñ buty'ukña jiñi ñoj ja'.}
\exampletranslation{Está rebalsando el río.}
\onedefinition{2}
\spanishtranslation{lleno}
\clarification{de gente}
\cholexample{Buty'ukña jiñi tyejklum tyi wiñikob x'ixikob.}
\exampletranslation{El pueblo está lleno de gente.}

\entry{buty'ul}
\partofspeech{adj}
\spanishtranslation{lleno}
\cholexample{Buty'ul tyi ixim jiñi koxtyal.}
\exampletranslation{El costal está lleno de maíz.}

\entry{buts}
\partofspeech{s}
\onedefinition{1}
\spanishtranslation{retoño}
\cholexample{Mi laj k'uche' itye'el kajpe' cha'añ mi ilok'el ibuts.}
\exampletranslation{Arqueamos la mata de café para que le salgan retoños.}
\onedefinition{2}
\spanishtranslation{nieto}
\cholexample{Añix lujuñtyikil ibuts jiñi chuchu'äl.}
\exampletranslation{Ya son diez nietos que tiene la abuelita.}

\entry{butstyäl}
\partofspeech{adv}
\spanishtranslation{así}
\clarification{señalando un rollo de hierba}
\cholexample{Che' ya butstyäl tsa' ich'ämä tyilel jiñi ch'ajuk.}
\exampletranslation{Nada más trajo un rollito así de verdura.}

\entry{buts'}
\partofspeech{s}
\spanishtranslation{humo}

\entry{bux}
\partofspeech{s}
\onedefinition{1}
\spanishtranslation{jícara}
\onedefinition{2}
\spanishtranslation{bejuco de tecomate}
\secondaryentry{bux pok'}
\secondpartofspeech{s}
\secondtranslation{jícara para tortillas}

\entry{bu'lel}
\partofspeech{s}
\spanishtranslation{frijolar}

\entry{bu'le waj}
\spanishtranslation{memela, tortilla con frijol}

\entry{bu'lich}
\partofspeech{s}
\spanishtranslation{sudor}

\entry{bu'ul}
\partofspeech{s}
\spanishtranslation{frijol}

\entry{bu'ultye'}
\partofspeech{s}
\spanishtranslation{hormiguillo}
\clarification{palo de marimba; árbol grande que da fruta como frijolitos; la madera es maciza y colorada y se usa para marimbas}

\alphaletter{K}

\entry{k-}
\partofspeech{pron}
\onedefinition{1}
\nontranslationdef{Prefijo que indica adjetivo posesivo de 1ª persona.}
\onedefinition{2}
\nontranslationdef{Prefijo que indica pronombre personal de 1ª persona.}

\entry{kabäl}
\partofspeech{adj}
\onedefinition{1}
\spanishtranslation{mucho}
\cholexample{Añ kabäl tsäñal.}
\exampletranslation{En enero hace mucho frío.}
\dialectvariant{Tila}
\dialectword{joboñ}
\onedefinition{2}
\spanishtranslation{muchas veces}
\cholexample{Kabäl tyak mi ityilel ijula'.}
\exampletranslation{Muchas veces le llegan visitantes.}

\entry{Kakawatyal}
\partofspeech{s esp}
\spanishtranslation{Arboleda de Cacao}
\clarification{colonia}

\entry{kaktye'}
\partofspeech{s}
\spanishtranslation{huarache}

\entry{kaj}
\partofspeech{part}
\spanishtranslation{por causa de}
\cholexample{Tyi kaj amul tsa' käjchiyoñ.}
\exampletranslation{Me encarcelaron por causa tuya.}

\entry{kajajtyañ}
\partofspeech{vt}
\spanishtranslation{pelear por}
\cholexample{Ak'bi tsa' ikajatyayob jiñi kajpe'lel kome tyi cha'tyiklelob yomob ich'äm.}
\exampletranslation{Ayer se pelearon por el cafetal, pues los dos quieren tomarlo.}

\entry{kajkaña}
\partofspeech{adv}
\nontranslationdef{Se relaciona con el movimiento trastornado; p. ej.:}
\cholexample{Ya' kajkaña wiñik tsak tyaja.}
\exampletranslation{Encontré a un hombre que caminaba trastornadamente.}
\alsosee{xkajka}

\entry{*kajchil}
\partofspeech{s}
\spanishtranslation{faja}

\entry{kajchiñäk'}
\partofspeech{s}
\spanishtranslation{cinturón}

\entry{kajel}
\defsuperscript{1}
\partofspeech{vi}
\spanishtranslation{comenzar}
\cholexample{Mux ikajel iyorajlel tyikwal.}
\exampletranslation{Va a comenzar la temporada de calor.}
\dialectvariant{Sab.}
\dialectword{ñijlel}

\entry{kajel}
\defsuperscript{2}
\partofspeech{part}
\nontranslationdef{Palabra que indica el aspecto de tiempo futuro; p. ej.:}
\cholexample{Mi kajel imajlel tyi tyejklum.}
\exampletranslation{Va a ir al pueblo.}

\entry{kajoñtye'}
\partofspeech{s esp}
\spanishtranslation{cajón}

\entry{kajpe'}
\partofspeech{s esp}
\spanishtranslation{café}
\dialectvariant{Tila}
\dialectword{kajwe'}
\dialectvariant{Sab.}
\dialectword{kape}

\entry{kajpe'lel}
\partofspeech{s esp}
\spanishtranslation{cafetal}

\entry{kaj testigo}
\partofspeech{esp}
\spanishtranslation{mi testigo}

\entry{kajwe'}
\relevantdialect{Tila}
\partofspeech{s esp}
\spanishtranslation{café}
\alsosee{kajpe'}

\entry{kal}
\partofspeech{vt}
\spanishtranslation{abrir}
\clarification{una pared}
\cholexample{Tsa' ikalai bojtye'lel iyotyoty.}
\exampletranslation{Abrió la pared de su casa.}

\entry{kalal}
\partofspeech{adj}
\spanishtranslation{abierto}
\clarification{p. ej.: una pared, una tabla}
\cholexample{Kalal ijol otyoty.}
\exampletranslation{Está abierto el techo de la casa.}

\entry{kañ}
\partofspeech{vt}
\spanishtranslation{abrir}
\clarification{los ojos}
\cholexample{Tyi ojlil ak'älel tsa' kajñi kwuty.}
\exampletranslation{A medianoche me desperté (lit.: se abrieron mis ojos).}

\entry{kañar}
\partofspeech{s esp}
\spanishtranslation{sueldo}
\cholexample{Tsa' majli tyi kañar.}
\exampletranslation{Fue a trabajar para recibir sueldo.}

\entry{kañar x'e'tyel}
\spanishtranslation{jornalero}

\entry{kañcheñek}
\relevantdialect{Sab.}
\partofspeech{s}
\spanishtranslation{frijol amarillo}

\entry{kañtyil}
\partofspeech{s esp}
\spanishtranslation{candil}
\clarification{víbora}

\entry{kañal lak wuty}
\spanishtranslation{despierto}
\dialectvariant{Sab.}
\dialectword{p'ixil}

\entry{kape}
\relevantdialect{Sab.}
\partofspeech{s}
\spanishtranslation{café}
\alsosee{kajpe'}

\entry{kapejol}
\relevantdialect{Sab.}
\partofspeech{s}
\spanishtranslation{cafetal}
\alsosee{kajpe'lel}

\entry{kapityañ muty}
\partofspeech{s esp}
\spanishtranslation{tipo de pájaro con pico largo}

\entry{kapom}
\partofspeech{adj esp}
\spanishtranslation{capado}

\entry{kareña}
\partofspeech{s esp}
\spanishtranslation{cadena}

\entry{karitya}
\partofspeech{s esp}
\spanishtranslation{garita}

\entry{kas}
\partofspeech{s esp}
\spanishtranslation{petróleo}

\entry{kasä}
\partofspeech{part esp}
\spanishtranslation{negativo}
\cholexample{Kasä joñoñik tsak mele.}
\exampletranslation{No fui yo el que lo hizo.}

\entry{kastyiya}
\partofspeech{s esp}
\spanishtranslation{castellano}

\entry{kats'}
\defsuperscript{1}
\partofspeech{adv}
\onedefinition{1}
\nontranslationdef{Se relaciona con la forma de morder; p. ej.:}
\cholexample{Jiñi ts'i' tsa' ikats' k'uxu kok.}
\exampletranslation{El perro me mordió el pie.}
\onedefinition{2}
\nontranslationdef{Se relaciona con la manera de comer una bola; p. ej.:}
\cholexample{Tsa' ikats' kämä tyi' yej alaxax.}
\exampletranslation{Se metió una naranja en la boca.}

\entry{kats'}
\defsuperscript{2}
\partofspeech{adj}
\spanishtranslation{trabado}
\cholexample{Tsa' kats' käle tyi jajp xajlel.}
\exampletranslation{Se quedó trabado en la rendija de una piedra.}

\entry{kats'tyäl}
\partofspeech{adv}
\spanishtranslation{así de ancho}
\cholexample{Che' ya' kats'tyälba' mi iñumel jiñi tye'lal.}
\exampletranslation{Así de ancho está el lugar por donde pasa el tepezcuintle.}

\entry{kaw}
\partofspeech{vt}
\spanishtranslation{abrir}
\clarification{boca}

\entry{kawakña}
\partofspeech{adj}
\spanishtranslation{abierta}
\clarification{boca}
\cholexample{Kawakña iyej jiñi alob woli ik'el pañimil.}
\exampletranslation{El niño está con la boca abierta, viendo las cosas de la calle.}

\entry{kawal}
\partofspeech{adj}
\spanishtranslation{abierta}
\clarification{boca}

\entry{kawayu'}
\partofspeech{s esp}
\spanishtranslation{caballo}

\entry{kawtyilel}
\partofspeech{s}
\spanishtranslation{abertura}
\cholexample{Ch'och'ok ikawtyilel ityi' otyoty.}
\exampletranslation{La abertura de la puerta de la casa es chica.}

\entry{kaxatye'}
\partofspeech{s esp}
\spanishtranslation{cajón}

\entry{kaxlañ}
\partofspeech{s}
\spanishtranslation{uno que no es indígena, ladino}

\entry{kaxlañ waj}
\spanishtranslation{pan}

\entry{kayajoñ}
\partofspeech{s esp}
\onedefinition{1}
\spanishtranslation{callejón}
\onedefinition{2}
\spanishtranslation{frontera}

\entry{kayu}
\partofspeech{s esp}
\spanishtranslation{gallo}

\entry{käkätye'ol}
\partofspeech{s}
\spanishtranslation{lugar donde hay muchas matas de cacaté}

\entry{käkäw}
\partofspeech{s esp}
\spanishtranslation{cacao}
\clarification{árbol}

\entry{käkäwol}
\partofspeech{s esp}
\spanishtranslation{cacaotal}

\entry{Käktye'pa'}
\partofspeech{s}
\spanishtranslation{nombre de una colonia}

\entry{käktye'pa'}
\partofspeech{s}
\spanishtranslation{zapote de agua}
\clarification{árbol; la fruta es comestible y parecida a la fruta del zapote; crece en la orilla de los arroyos}

\entry{käch}
\partofspeech{vt}
\spanishtranslation{amarrar}

\entry{kächäl}
\partofspeech{adj}
\onedefinition{1}
\spanishtranslation{amarrado}
\cholexample{Kächäl kcha'añ jiñi chityam kome ñajty mi icha'leñ xämbal.}
\exampletranslation{Tengo el cerdo amarrado porque va muy lejos.}
\onedefinition{2}
\spanishtranslation{encarcelado}
\cholexample{Kächäl jiñi wiñik kome tsa' icha'le tsäñsa.}
\exampletranslation{Ese hombre está encarcelado porque cometió un homicidio.}

\entry{käch'äkña}
\partofspeech{adv}
\nontranslationdef{Se relaciona con el sonido que produce una cuerda nueva al moverla.}

\entry{käjchel}
\partofspeech{vi}
\onedefinition{1}
\spanishtranslation{amarrarse}
\cholexample{Yom mi ikäjchel jiñi ts'i' cha'añ ma'añik mi ik'ux lakjula'.}
\exampletranslation{Debe amarrarse al perro para que no muerda a nuestra visita.}
\onedefinition{2}
\spanishtranslation{encarcelar}
\cholexample{Tsa' käjchi jiñi wiñik.}
\exampletranslation{Fue encarcelado ese hombre.}

\entry{*käjchil}
\partofspeech{s}
\spanishtranslation{amarrador}
\clarification{de tela}
\cholexample{Ma'añik ikäjchil cha'añ imajts aläl.}
\exampletranslation{No tiene amarrador para el pañal de la criatura.}

\entry{käläx}
\partofspeech{adv}
\spanishtranslation{demasiado}
\cholexample{Käläx tsi'bajbe ja'al.}
\exampletranslation{Llovió demasiado.}

\entry{kälel}
\partofspeech{vi}
\spanishtranslation{quedarse}
\cholexample{Kepel mi ikälel lake'tyel.}
\exampletranslation{Nuestro trabajo se queda pendiente.}

\entry{kälem}
\partofspeech{adj}
\spanishtranslation{dejado}
\cholexample{Kälem jiñi otyoty.}
\exampletranslation{Esa casa está dejada.}

\entry{käm}
\partofspeech{vt}
\spanishtranslation{agarrar o llevar}
\clarification{con la boca}
\cholexample{Jiñi ts'i' tsa' iwox kämä majlel tyumuty.}
\exampletranslation{El perro se llevó el huevo (en la boca).}

\entry{käñ}
\partofspeech{vt}
\onedefinition{1}
\spanishtranslation{conocer}
\cholexample{Tsa'ix jkäñä jiñi tyejklum.}
\exampletranslation{Ya conocí ese pueblo.}
\dialectvariant{Sab., Tila}
\dialectword{pojleñ}
\onedefinition{2}
\spanishtranslation{aprender}
\cholexample{Tsa'ix ikäñä juñ.}
\exampletranslation{Ya aprendió a leer.}

\entry{käñkäläñ}
\partofspeech{adv}
\nontranslationdef{Se relaciona con el ruido que hace un palo seco al caer; p. ej.:}
\cholexample{Käñkäläñ che' tsa' k'otyi tyi lum jiñi tye'.}
\exampletranslation{Sonó el palo seco al caer.}

\entry{käñkäñ}
\partofspeech{adv}
\spanishtranslation{a veces}
\cholexample{Mi ikäñkäñ chämel kome jubeñix ich'ich'el.}
\exampletranslation{A veces se desmaya porque tiene poca sangre.}

\entry{käñtyañ}
\partofspeech{vt}
\spanishtranslation{cuidar}

\entry{käñtyäbil}
\partofspeech{adj}
\onedefinition{1}
\spanishtranslation{protegido}
\cholexample{Käñtyäbil cha'añ iyerañob kome yomob itsäñsañ.}
\exampletranslation{Ese hombre está protegido por sus hermanos, porque otros lo quieren matar.}
\onedefinition{2}
\spanishtranslation{embarazada}
\cholexample{Käñtyäbil icha'añ iyalobil jiñi x'ixik.}
\exampletranslation{Esa mujer está embarazada.}

\entry{käñtyesa}
\partofspeech{s}
\spanishtranslation{enseñanza}
\secondaryentry{xkäñtyesa}
\secondpartofspeech{s}
\secondtranslation{maestro}

\entry{käñtyesañ}
\partofspeech{vt}
\spanishtranslation{enseñar}

\entry{käñtyesaya}
\partofspeech{s}
\spanishtranslation{enseñanza}

\entry{*käñtyesäbal}
\partofspeech{s}
\spanishtranslation{enseñanza}
\cholexample{Uts'aty ikäñtyesäbal.}
\exampletranslation{Su enseñanza es buena.}

\entry{*käñtyesäñtyel}
\partofspeech{s}
\spanishtranslation{instrucción}
\cholexample{Woli laktyaj laj käñtyesäñtyel.}
\exampletranslation{Estamos recibiendo instrucción.}

\entry{Käñtyok'}
\partofspeech{s}
\spanishtranslation{nombre de una colonia}

\entry{-käñañ}
\nontranslationdef{Sufijo que se presenta con raíces adjetivas que indican color y se refiere a un líquido.}

\entry{käñäl}
\partofspeech{adj}
\spanishtranslation{conocido}
\cholexample{Käñäl jiñi joñtyolbä wiñik.}
\exampletranslation{El hombre malo es conocido.}

\entry{käräkña}
\partofspeech{adj}
\spanishtranslation{gruñendo}
\cholexample{Käräkña ibik jiñibajlum.}
\exampletranslation{El jaguar está gruñendo.}

\entry{käxtyi}
\partofspeech{part}
\spanishtranslation{negativo}
\cholexample{Käxtyi añik tsa' majliyoñ.}
\exampletranslation{Pues no fui.}

\entry{käyäl}
\partofspeech{adj}
\spanishtranslation{dejado, abandonado}
\cholexample{Käyäl jiñi x'ixik.}
\exampletranslation{Esa mujer es abandonada.}

\entry{käyleñ}
\relevantdialect{Sab.}
\partofspeech{adj}
\spanishtranslation{dejado}

\entry{kibre}
\partofspeech{s esp}
\spanishtranslation{jengibre}
\clarification{tipo de zacate}

\entry{kiñtya laj ko'}
\partofspeech{esp}
\spanishtranslation{coralillo}

\entry{kokorojo'}
\partofspeech{s}
\spanishtranslation{canto del gallo}

\entry{kokoyol}
\partofspeech{s}
\spanishtranslation{coyol}
\clarification{árbol}

\entry{koch'}
\partofspeech{adv}
\nontranslationdef{Se relaciona con instrumentos delgados; p. ej.:}
\cholexample{Tsi'koch' jek'eyoñ ch'ix.}
\exampletranslation{Me picó la espina.}

\entry{kojach}
\partofspeech{adv}
\spanishtranslation{sólo}
\cholexample{Kojach juñyajl tsa' tyiliyoñ.}
\exampletranslation{Sólo una vez vine.}

\entry{kojax}
\partofspeech{adv}
\spanishtranslation{sólo esa vez}
\cholexample{Kojax tsa' tyili iliyi.}
\exampletranslation{Sólo vino esa vez.}

\entry{kojk}
\partofspeech{adj}
\spanishtranslation{sordo}
\cholexample{Kojk jiñi wiñik.}
\exampletranslation{Ese hombre está sordo.}

\entry{kojko}
\partofspeech{adv}
\onedefinition{1}
\spanishtranslation{menos}
\cholexample{Ma'añik tsa' mäjliyob icha'añ, kojkotyo jatyety.}
\exampletranslation{Si ellos no pudieron, menos tú.}
\onedefinition{2}
\spanishtranslation{qué tal}
\cholexample{Kojko tsa'ki majliyoñ ja'el.}
\exampletranslation{¡Qué tal si yo también hubiera ido!}

\entry{kojkom}
\partofspeech{s}
\spanishtranslation{bejuco}
\clarification{macizo y rollizo que se usa para amarrar cercos}

\entry{kojix}
\partofspeech{adj}
\spanishtranslation{último}
\cholexample{Jiñäch kojixbä kalobil.}
\exampletranslation{Él es mi último hijo.}

\entry{kojm}
\partofspeech{adj}
\spanishtranslation{jadeante}
\clarification{por falta de aire}
\cholexample{Kojmatyax kpusik'al cha'añ woliyoñ tyi ajñel.}
\exampletranslation{Estoy jadeante por correr.}

\entry{kojoñ}
\partofspeech{adj}
\spanishtranslation{único}
\cholexample{Kojoñ jachbä ch'ityoñ, ma'añik kerañ.}
\exampletranslation{Soy el único muchacho, no tengo hermano.}

\entry{-kojty}
\nontranslationdef{Sufijo numeral que se usa para contar animales; p. ej.:}
\cholexample{Ya' tyi potyrero añ jo'kojty kawayu'.}
\exampletranslation{En el potrero están cinco caballos.}

\entry{kojtyel}
\partofspeech{vi}
\spanishtranslation{gatear}
\cholexample{Jiñi aläl mi ikojtyel tyi lum.}
\exampletranslation{El niño gatea en el suelo.}

\entry{koj tyo bä i yalobil}
\spanishtranslation{primogénito}

\entry{kojtyom}
\partofspeech{s}
\spanishtranslation{tejón}
\clarification{mamífero}

\entry{kol}
\partofspeech{vt}
\onedefinition{1}
\spanishtranslation{soltar}
\cholexample{Yomix lakol jiñi mulacha'añ mi imajlel ijap ja'.}
\exampletranslation{Debemos soltar la mula para que vaya a tomar agua.}
\onedefinition{2}
\spanishtranslation{embrocar}
\clarification{agua}
\cholexample{Mi laj kol ochel ja' tyi p'ejty cha'añ mi lakch'äx bu'ul.}
\exampletranslation{Embrocamos agua en la olla para cocer el frijol.}

\entry{kolaj}
\defsuperscript{1}
\partofspeech{s esp}
\spanishtranslation{pegamento}

\entry{kolaj}
\defsuperscript{2}
\relevantdialect{Tila}
\partofspeech{s}
\onedefinition{1}
\spanishtranslation{diosa del agua de la creación}
\onedefinition{2}
\spanishtranslation{abuela}
\alsosee{ko'äl}

\entry{kolbil}
\partofspeech{adj}
\spanishtranslation{suelto}
\cholexample{Kolbil jiñi mula.}
\exampletranslation{La mula está suelta.}

\entry{kolkolña}
\partofspeech{adv}
\nontranslationdef{Se relaciona con el movimiento que tiene las características de un chorrito de agua que cae.}

\entry{kolel}
\onedefinition{1}
\partofspeech{adv}
\spanishtranslation{casi}
\cholexample{Kolel ichämel.}
\exampletranslation{Casi se muere.}
\onedefinition{2}
\partofspeech{vi}
\spanishtranslation{crecer}
\cholexample{Tyi ora woli tyi kolel apäk'äb.}
\exampletranslation{Está creciendo su siembra.}

\entry{kolelix}
\partofspeech{adv}
\spanishtranslation{por poco}
\cholexample{Kolelix kchämel sajmäl.}
\exampletranslation{Por poco me muero hoy.}

\entry{kolem}
\partofspeech{adj}
\spanishtranslation{grande}
\cholexample{Kolem jiñi otyoty.}
\exampletranslation{Esa casa es grande.}
\secondaryentry{kolem abal}
\secondtranslation{mar}
\secondaryentry{kolembä xiye'}
\secondtranslation{águila real}
\secondaryentry{kolem ja'}
\secondtranslation{río grande}

\entry{*kolemal}
\partofspeech{s}
\spanishtranslation{niñez}
\cholexample{Che'äch yilal k'äläl tyi' kolemal.}
\exampletranslation{Así es desde su niñez.}

\entry{kolemäyel}
\partofspeech{vi}
\spanishtranslation{criarse}
\cholexample{Wä'añoñ tyi tsäñälel k'äläl che' tyi kolemäyel.}
\exampletranslation{Aquí he estado en tierra fría desde que me vine criando.}

\entry{kolem matye' chityam}
\spanishtranslation{tamborcillo}
\spanishtranslation{marina}
\clarification{mamífero}

\entry{kolem päm}
\spanishtranslation{tucán, pico de canoa}
\clarification{ave}

\entry{kolem p'ok}
\spanishtranslation{iguana}
\clarification{reptil}

\entry{kolib}
\partofspeech{s}
\spanishtranslation{viejo}

\entry{kolmäjel}
\partofspeech{vi}
\spanishtranslation{cazar}
\cholexample{Woli tyi kolmäjel jiñi ts'i'.}
\exampletranslation{Ese perro está cazando.}

\entry{kolokña}
\partofspeech{adj}
\spanishtranslation{un bosque sin monte bajo}
\cholexample{Kolokña jiñi tye'el.}
\exampletranslation{Ese bosque no tiene monte bajo.}

\entry{*kolojbal}
\partofspeech{s}
\spanishtranslation{sobra}
\cholexample{Ma'añix ikolojbal jiñi waj.}
\exampletranslation{Ya no quedaron sobras de tortillas.}

\entry{kolol}
\partofspeech{adj}
\spanishtranslation{monte levantado}
\cholexample{Kolol jiñi wumälel.}
\exampletranslation{El acahual está levantado.}

\entry{*kolom}
\partofspeech{s}
\spanishtranslation{presa}

\entry{koloso}
\partofspeech{adj esp}
\spanishtranslation{goloso}
\cholexample{Koloso jax tyi we'elel jiñi wiñik.}
\exampletranslation{Ese hombre es muy goloso con la carne.}

\entry{*kolosojlel}
\partofspeech{s esp}
\spanishtranslation{maldad}
\cholexample{Jiñi wiñik mi itsajkañ ikolosojlel ipusik'al.}
\exampletranslation{Ese hombre sigue la maldad de su corazón.}

\entry{kolotye'}
\partofspeech{s}
\spanishtranslation{jaula grande para transportar aves}

\entry{kolo'ity}
\relevantdialect{Tila}
\partofspeech{s}
\spanishtranslation{nalgas}
\alsosee{cho'jity}

\entry{koltyañ}
\partofspeech{vt}
\onedefinition{1}
\spanishtranslation{ayudar}
\cholexample{Mik majlel jkoltyañ kijts'iñ tyi chobal.}
\exampletranslation{Voy a ayudar a mi hermanito a rozar.}
\onedefinition{2}
\spanishtranslation{salvar}
\cholexample{Ma'añix tsa' mejli ikoltyañ ibä cha'añ ikoñtyrajob.}
\exampletranslation{No se pudo salvar de sus enemigos.}
\alsosee{ajkoltyaya}
\alsosee{xkoltyaya}

\entry{koltyaya}
\partofspeech{s}
\spanishtranslation{ayuda}
\cholexample{Tsa' icha'le koltyaya tyi'tyojlel jiñi p'ump'uñbä.}
\exampletranslation{Dio ayuda al pobre.}

\entry{koltyäbil}
\partofspeech{adj}
\onedefinition{1}
\spanishtranslation{ayudado}
\cholexample{Koltyäbil jax jiñi x'ixik cha'añ mi ipäyob majlel tyi' yotyoty.}
\exampletranslation{Esa mujer fue ayudada para llegar a su casa.}
\onedefinition{2}
\spanishtranslation{salvado}
\cholexample{Koltyäbilety tyi k'äb jiñi wiñik che'bä tsa' yajliyety tyi ja'.}
\exampletranslation{Cuando caíste al agua, fuiste salvado por ese hombre.}

\entry{koltyäñtyel}
\partofspeech{s}
\onedefinition{1}
\spanishtranslation{auxilio}
\cholexample{Ipi'äl tsa' iyäk'e ikoltyäñtyel.}
\exampletranslation{Su compañero le dio auxilio.}
\onedefinition{2}
\spanishtranslation{salvación}
\cholexample{Tyilem tyi dios ikoltyäñtyel wiñik.}
\exampletranslation{La salvación del hombre viene de Dios.}

\entry{kom}
\defsuperscript{1}
\conjugationtense{1ª pers. sing.}
\conjugationverb{om}
\spanishtranslation{quiero}
\cholexample{Kom e'tyel.}
\exampletranslation{Quiero trabajar.}

\entry{kom}
\defsuperscript{2}
\partofspeech{adj}
\spanishtranslation{corto}
\cholexample{Kom jiñik tyajbal.}
\exampletranslation{Mi mecapal es corto.}

\entry{komatyax}
\partofspeech{adj}
\spanishtranslation{muy corto}
\cholexample{Komatyax iyok jiñi mesa.}
\exampletranslation{Las patas de la mesa están demasiado cortas.}

\entry{komkatsa'}
\onedefinition{1}
\partofspeech{part}
\spanishtranslation{quisiera}
\cholexample{Komkatsa'bäk' sujtyel tsa kälä.}
\exampletranslation{Quisiera regresar pronto.}
\onedefinition{2}
\partofspeech{adv}
\spanishtranslation{apenas, difícilmente}
\cholexample{Komkatsa' ats'aki.}
\exampletranslation{Difícilmente hay medicina para ti.}

\entry{kome}
\partofspeech{part}
\spanishtranslation{porque}

\entry{komol}
\partofspeech{adj}
\spanishtranslation{juntos}
\cholexample{Komol mi iyajkañob jiñi mu'bä kaj icha'leñ yumäl.}
\exampletranslation{Juntos eligen al que los va a gobernar.}

\entry{komol e'tyel}
\spanishtranslation{trabajo comunal}

\entry{komol jach lak cha'añ}
\spanishtranslation{es de nosotros juntos}
\clarification{modismo}

\entry{kompirar}
\partofspeech{s}
\spanishtranslation{tipo de comida}
\clarification{La comida que se ofrece a los santos y que se reparte para comer.}

\entry{kom'esañ}
\partofspeech{vt}
\onedefinition{1}
\spanishtranslation{acortar}
\cholexample{Yom laj kom'esañ jiñi ch'ajañ.}
\exampletranslation{Debemos acortar el mecate.}
\onedefinition{2}
\spanishtranslation{hacer breve}
\cholexample{Yom mi akom'esañ aty'añ.}
\exampletranslation{Debes hacer breve tu plática.}

\entry{koñix}
\partofspeech{v irr}
\spanishtranslation{me voy}

\entry{koñla}
\partofspeech{v irr}
\spanishtranslation{vamos}

\entry{koñtyra}
\partofspeech{s esp}
\spanishtranslation{enemigo}

\entry{koñtyrajiñ}
\partofspeech{vt esp}
\spanishtranslation{contrariar}

\entry{koñerol}
\partofspeech{s esp}
\spanishtranslation{juez rural}

\entry{Kokija'}
\partofspeech{s}
\spanishtranslation{nombre de colonia}

\entry{koraläjem}
\partofspeech{adj}
\spanishtranslation{cercado}
\cholexample{Koraläjem ipam otyoty.}
\exampletranslation{El patio de la casa está cercado.}

\entry{korälijel}
\partofspeech{vi}
\spanishtranslation{hacer cerco, cercar}
\cholexample{Woliyoñ tyi korälijel.}
\exampletranslation{Estoy haciendo un cerco.}

\entry{koroso}
\relevantdialect{Tila}
\partofspeech{s}
\spanishtranslation{manaca}
\clarification{palma}

\entry{kosañ}
\partofspeech{vt}
\spanishtranslation{criar}
\cholexample{Kabäl mi ikosañ imuty.}
\exampletranslation{Cría muchos pollos.}

\entry{kotyañety}
\partofspeech{part}
\spanishtranslation{saludo}

\entry{kotykotyña}
\partofspeech{adv}
\onedefinition{1}
\nontranslationdef{Se relaciona con el movimiento que se hace cuando se está en cuatro patas; p.ej.:}
\cholexample{Kotykotyña jiñi mulatyi pam otyoty.}
\exampletranslation{Esa mula está caminando en el patio de la casa.}
\onedefinition{2}
\spanishtranslation{en cuatro patas}
\clarification{persona caminando}
\cholexample{Kotykotyña mi icha'leñ xämbal jiñi wiñik.}
\exampletranslation{Ese hombre camina en cuatro patas.}

\entry{kotychokoñ}
\partofspeech{vt}
\spanishtranslation{poner parado}
\clarification{objetos de cuatro patas}
\cholexample{Jiñi xch'ok tsa' ikotychoko mis tyi pam mesa.}
\exampletranslation{La niña puso el gato encima de la mesa.}

\entry{*kotyilel}
\partofspeech{s}
\spanishtranslation{altura}
\clarification{de animales}
\cholexample{Chañ ikotyilel jiñi kawayu'.}
\exampletranslation{Es mucha la altura del caballo.}

\entry{kotyo}
\partofspeech{part}
\spanishtranslation{adiós}

\entry{kotyokña}
\partofspeech{adv}
\nontranslationdef{Se relaciona con la forma en que se paran los animales; p. ej.:}
\cholexample{Kotyokña jiñi wakax tyi mal potyrero.}
\exampletranslation{La vaca está parada en el potrero.}

\entry{kotyol}
\partofspeech{adj}
\onedefinition{1}
\spanishtranslation{parado}
\clarification{animales}
\cholexample{Ñajtyi tsikil kotyol jiñi kawayu'.}
\exampletranslation{El caballo se ve parado desde lejos.}
\onedefinition{2}
\spanishtranslation{agachado}
\clarification{personas}
\cholexample{Kotyol jiñi x'ixik woli isäk' iwaj.}
\exampletranslation{La mujer está agachada, lavando su maíz.}
\onedefinition{3}
\spanishtranslation{todo entero}
\cholexample{Kotyol tsa' sujp'i ochel majlel tyi ja' jiñi wiñik.}
\exampletranslation{Ese hombre se hundió en el agua con todo el cuerpo.}

\entry{kotyresiya}
\partofspeech{s esp}
\spanishtranslation{saludo}
\cholexample{Yom mi lakäk'eñ lakbä kotyresiya.}
\exampletranslation{Debemos dar saludos.}

\entry{kotytyäl}
\partofspeech{vi}
\spanishtranslation{pararse}
\clarification{animal}
\cholexample{Ma'añik mi ikotytyäl jiñik kawayu'.}
\exampletranslation{Mi caballo no se para.}

\entry{kox}
\partofspeech{s}
\spanishtranslation{pava, cojolite}
\clarification{ave}

\entry{koxtyal}
\partofspeech{s esp}
\spanishtranslation{costal}

\entry{koya'}
\partofspeech{s}
\spanishtranslation{tomate}

\entry{koyoj}
\partofspeech{s}
\spanishtranslation{chinino}
\clarification{árbol}

\entry{ko'äl}
\partofspeech{s}
\onedefinition{1}
\spanishtranslation{abuela}
\cholexample{Kuxultyo iko'.}
\exampletranslation{Todavía vive su abuela.}
\dialectvariant{Tila}
\dialectword{2*kolaj}
\onedefinition{2}
\spanishtranslation{la hermana de la madre del padre}
\onedefinition{3}
\spanishtranslation{partera}
\cholexample{Jiñi ko'äl tsa' ik'ele kalobil che'baki ora tsa' ik'ele pañimil.}
\exampletranslation{La partera atendió a mi hijo cuando nació.}

\entry{-ku}
\onedefinition{1}
\nontranslationdef{Sufijo afirmativo; p. ej.:}
\cholexample{Sí, yomku.}
\exampletranslation{Es conveniente.}
\onedefinition{2}
\nontranslationdef{Sufijo que forma otra raíz indicando afirmación; p. ej.:}
\cholexample{Woli ku.}
\exampletranslation{Sí, lo está (haciendo).}

\entry{kukluñtya'}
\partofspeech{s}
\spanishtranslation{cocuyo}
\clarification{insecto escarabajo negro que vuela por la noche, y que es relumbroso}

\entry{kuktyal}
\defsuperscript{1}
\partofspeech{s}
\spanishtranslation{tórax}
\cholexample{Ñuk ikuktyal jiñi wiñik.}
\exampletranslation{El tórax de ese hombre es grande.}

\entry{kuktyal}
\defsuperscript{2}
\partofspeech{s}
\spanishtranslation{manifestación de un espíritu}
\cholexample{Tyoj sajtyel ipusik'al cha'añ tsa' ik'elbe ikuktyal jiñi xiba.}
\exampletranslation{Se espantó al ver la manifestación del diablo.}

\entry{kuku}
\partofspeech{imp}
\spanishtranslation{¡Vete!}
\cholexample{Kuku tyi e'tyel kome iyorajlelix.}
\exampletranslation{Vete a tu trabajo, porque ya es hora.}

\entry{kukuch yopom}
\spanishtranslation{tipo de insecto verde}

\entry{kukujl}
\partofspeech{s}
\spanishtranslation{viga lateral de una casa}

\entry{kukuñuñ}
\partofspeech{imp}
\spanishtranslation{¡apresúrate!}

\entry{*kuch}
\onedefinition{1}
\partofspeech{s}
\spanishtranslation{carga}
\cholexample{Añ ikuch.}
\exampletranslation{Tiene su carga.}
\onedefinition{2}
\partofspeech{vt}
\spanishtranslation{cargar}
\cholexample{Mi kaj ikuch majlel jiñi xk'amäjel.}
\exampletranslation{Él va a cargar al enfermo en su espalda.}

\entry{kuchäl}
\partofspeech{s}
\spanishtranslation{carga}

\entry{kuchbil}
\partofspeech{adj}
\spanishtranslation{cargado}
\clarification{sobre la espalda}
\cholexample{Kuchbil tsa' majli jiñi xk'amäjel.}
\exampletranslation{Llevaron cargado al enfermo.}

\entry{kuchib pich}
\spanishtranslation{vejiga}
\alsosee{chujyib jpich}

\entry{kuchijel}
\partofspeech{vi}
\spanishtranslation{cargar}
\clarification{sobre la espalda}
\cholexample{Woli tyi kuchijel kome ma'añik imula.}
\exampletranslation{Está cargando (en la espalda) porque no tiene mula.}

\entry{kuchilu}
\partofspeech{s esp}
\spanishtranslation{cuchillo}

\entry{*kuchi'tyuñ}
\partofspeech{s}
\spanishtranslation{riñón}

\entry{kuj}
\partofspeech{vt}
\onedefinition{1}
\spanishtranslation{golpear}
\cholexample{Yom mi akuj jiñi lum tyi' joytyilel iyebal jiñi postye.}
\exampletranslation{Debes golpear la tierra alrededor del poste.}
\onedefinition{2}
\spanishtranslation{remachar}
\cholexample{Mi laj kuj tsukul tyak'iñ.}
\exampletranslation{Remachamos el fierro para componerlo.}

\entry{*kujklel}
\relevantdialect{Tila}
\partofspeech{s}
\spanishtranslation{suciedad}
\cholexample{Añ akujklel cha'añ ma'añik mi acha'leñ ts'ämel.}
\exampletranslation{Estás muy sucio (lit.: tienes mucha suciedad) porque no te bañas.}
\alsosee{bibi'lel}

\entry{-kujch}
\nontranslationdef{Sufijo numeral para contar bultos; p. ej.:}
\cholexample{Tyi jujump'ejl jab mik tyuk' jo'kujch kajpe'.}
\exampletranslation{Cada año cosecho cinco bultos de café.}

\entry{kujchel}
\partofspeech{vi}
\spanishtranslation{ser llevado}
\clarification{sobre la espalda}

\entry{*kujchil}
\partofspeech{s}
\nontranslationdef{Tela que se usa para cargar criaturas.}

\entry{kujoñib}
\partofspeech{s}
\spanishtranslation{pisón}

\entry{kujyel}
\partofspeech{vi}
\spanishtranslation{fingir}
\cholexample{Mi ikujyel tyi riko.}
\exampletranslation{Finge ser un hombre rico.}

\entry{kulak'}
\partofspeech{s}
\spanishtranslation{cabeza de negro}
\clarification{bejuco grande con espinas y muchas raíces duras}

\entry{kulañtya}
\partofspeech{s esp}
\spanishtranslation{cilantro}

\entry{kulukab}
\partofspeech{s}
\spanishtranslation{perdiz}

\entry{kumale}
\partofspeech{s esp}
\spanishtranslation{comadre}

\entry{kumkumña}
\partofspeech{adv}
\nontranslationdef{Se relaciona con el sonido que se oye al pasar una persona brincando.}

\entry{kumpale}
\partofspeech{s esp}
\spanishtranslation{compadre}

\entry{*kuñil}
\partofspeech{s}
\spanishtranslation{plataforma para maíz}
\cholexample{Woli its'äl ikuñil ixim.}
\exampletranslation{Está hacinando el maíz en la plataforma.}

\entry{kusiñaj}
\partofspeech{s esp}
\spanishtranslation{cocina}

\entry{*kuxel}
\partofspeech{s}
\spanishtranslation{moho}
\cholexample{Añ ikuxel jiñi waj.}
\exampletranslation{Esa tortilla tiene moho.}

\entry{*kuxemal}
\partofspeech{s}
\spanishtranslation{añublo}

\entry{kuxeñtyik}
\partofspeech{adj}
\spanishtranslation{mugroso}
\cholexample{Kuxeñtyik jiñi ch'ityoñ.}
\exampletranslation{El muchacho está mugroso.}

\entry{*kuxil}
\partofspeech{s}
\spanishtranslation{añublo}

\entry{*kuxtyälel}
\partofspeech{s}
\spanishtranslation{vida}
\cholexample{Ma'añik ikuxtyälel jiñi diostye'.}
\exampletranslation{El ídolo no tiene vida.}

\entry{kuxtyiyel}
\partofspeech{vi}
\spanishtranslation{vivir}

\entry{kuxu i bajñel}
\spanishtranslation{completamente solo}

\entry{kuxu i ty'ojol}
\spanishtranslation{simpático}

\entry{kuxul}
\partofspeech{adj}
\spanishtranslation{vivo}
\cholexample{Kuxultyo ktatuch.}
\exampletranslation{Todavía vive (lit.: está vivo) mi abuelo.}

\entry{kuy}
\partofspeech{vt}
\spanishtranslation{fingir}
\cholexample{Mi ikuy ibä tyi riko.}
\exampletranslation{Finge ser rico.}

\alphaletter{K'}

\entry{*k'aba'}
\partofspeech{s}
\spanishtranslation{nombre}
\cholexample{Ik'aba'äch josé.}
\exampletranslation{Su nombre es José.}

\entry{k'ak xäñ}
\relevantdialect{Sab.}
\spanishtranslation{menear}

\entry{k'ach}
\partofspeech{adj}
\spanishtranslation{torcer}
\cholexample{Mi ik'ach ibä tyablatyi k'iñ.}
\exampletranslation{La tabla se tuerce en el sol.}
\dialectvariant{Sab.}
\dialectword{ts'oty}

\entry{k'achal}
\partofspeech{adj}
\spanishtranslation{torcido}
\cholexample{K'achal jiñi tyablaya' tyi k'iñ.}
\exampletranslation{La tabla está torcida por el sol.}

\entry{k'achulañ}
\partofspeech{vt}
\spanishtranslation{subir y bajar}
\clarification{los extremos}
\cholexample{Mi laj k'achulañ jiñi tye'.}
\exampletranslation{Subimos y bajamos las puntas de un palo.}

\entry{k'achuña}
\partofspeech{adv}
\nontranslationdef{Una manera de subir y bajar los brazos o las piernas.}

\entry{k'aj}
\partofspeech{vt}
\spanishtranslation{tapiscar}
\clarification{maíz}

\entry{k'ajakña}
\relevantdialect{Sab.}
\partofspeech{adj}
\spanishtranslation{contento}
\alsosee{uts}

\entry{k'ajakñayix yo}
\relevantdialect{Sab.}
\spanishtranslation{le está gustando}

\entry{k'ajal}
\partofspeech{adj}
\spanishtranslation{acordado}
\cholexample{K'ajal icha'añ ibety.}
\exampletranslation{Él recuerda (lit.: la tiene acordada) su deuda.}

\entry{k'ajatyañ}
\relevantdialect{Sab.}
\partofspeech{vt}
\spanishtranslation{acordar}
\cholexample{Mi ik'ajatyañ ipi'älob.}
\exampletranslation{Se acuerda de sus compañeros.}
\variation{k'ajtyisañ}

\entry{k'ajbal}
\partofspeech{s}
\spanishtranslation{tapisca}

\entry{k'ajk}
\partofspeech{s}
\onedefinition{1}
\spanishtranslation{fuego}
\cholexample{Yom mi axik' jiñi k'ajk.}
\exampletranslation{Debes atizar el fuego.}
\onedefinition{2}
\spanishtranslation{calentura}
\cholexample{Tsa' iyubi k'ajk tyi ak'älel sajmäl.}
\exampletranslation{Le dio calentura ayer por la noche.}
\onedefinition{3}
\spanishtranslation{luz}
\cholexample{Ik'yoch'añjak imal kome ma'añik k'ajk.}
\exampletranslation{Está muy oscuro adentro de la casa porque no hay luz.}

\entry{*k'ajk}
\partofspeech{s}
\onedefinition{1}
\spanishtranslation{base de la cola de aves,}
\onedefinition{2}
\spanishtranslation{varilla}
\alsosee{chuñchuñ muty}

\entry{k'ajkätye'}
\partofspeech{s}
\spanishtranslation{tipo de árbol}
\clarification{de madera colorada que sirve para horcones}

\entry{k'ajchel}
\partofspeech{vi}
\spanishtranslation{despegarse}
\clarification{una parte}
\cholexample{Mi ik'ajchel lakejk'ach.}
\exampletranslation{Se despega la uña.}

\entry{k'ajchem}
\partofspeech{adj}
\spanishtranslation{despegada}
\clarification{uña}

\entry{k'aj la ko}
\spanishtranslation{descansar}
\cholexample{Mi laj k'aj lako tyi ojlil bij.}
\exampletranslation{Descansamos a medio camino.}

\entry{k'ajlel}
\partofspeech{vi}
\spanishtranslation{despegarse}
\clarification{pintura o repello}
\cholexample{Woli ik'ajlel ipajk'il jiñi otyoty.}
\exampletranslation{Se está despegando el repello de la casa.}

\entry{k'ajlem}
\partofspeech{adj}
\spanishtranslation{despegado}
\clarification{pintura o repello}

\entry{k'aj o}
\partofspeech{s}
\spanishtranslation{descanso}
\cholexample{Lakomix k'aj o che' tyi ak'älel.}
\exampletranslation{Queremos un descanso por la noche.}

\entry{k'ajo'o'}
\partofspeech{s}
\spanishtranslation{lugar de descanso}
\clarification{para la noche}
\cholexample{Jiñäch k'ajo'o' tyi paso ñarañjo.}
\exampletranslation{Paso Naranjo es un lugar de descanso para la noche.}

\entry{k'ajpam}
\partofspeech{adv}
\spanishtranslation{a la vista}
\cholexample{K'ajpam tsikil jiñi tyejklum.}
\exampletranslation{El pueblo está a la vista.}

\entry{k'ajpa'añ}
\partofspeech{adj}
\spanishtranslation{cerca, visible}
\cholexample{K'ajpa'añ jiñi otyoty tyi' tyi' bij.}
\exampletranslation{La casa está cerca (visible) de la orilla del camino.}

\entry{k'ajtyesañ}
\partofspeech{vt}
\spanishtranslation{acordarse de}
\cholexample{Woli jk'ajtyesañ chuki tsa' subeñtyiyoñ ak'bi.}
\exampletranslation{Me acordé de lo que me dijeron ayer.}

\entry{k'ajtye'}
\relevantdialect{Sab.}
\partofspeech{s}
\spanishtranslation{puente}
\variation{k'atye}
\alsosee{pañtye'}

\entry{K'ajtye'ja'}
\partofspeech{s}
\spanishtranslation{Paso de Agua}
\clarification{lugar}

\entry{k'ajtyibeñ}
\partofspeech{vt}
\onedefinition{1}
\spanishtranslation{preguntarle}
\cholexample{Yom mi ak'ajtyibeñbajche' mi'tyoj kajpe'.}
\exampletranslation{Debes preguntarle a cómo paga el café.}
\onedefinition{2}
\spanishtranslation{pedirle}
\cholexample{Yom mi ak'ajtyibeñ lum jiñi komisariado.}
\exampletranslation{Debes pedirle terreno al comisariado.}

\entry{k'ajtyibil}
\partofspeech{adj}
\onedefinition{1}
\spanishtranslation{comprometida}
\clarification{una muchacha}
\cholexample{K'ajtyibilix jiñi xch'ok.}
\exampletranslation{La muchacha ya está comprometida.}
\onedefinition{2}
\spanishtranslation{preguntado}
\cholexample{Maxtyo añik k'ajtyibil kcha'añbaki ora mi kaj ityilel.}
\exampletranslation{No le he preguntado para cuándo va a venir.}
\onedefinition{3}
\spanishtranslation{pedido}
\cholexample{Mach mejlik kchombeñety jiñi chityam kome k'ajtyibilix.}
\exampletranslation{No puedo venderte el puerco porque está pedido.}

\entry{k'ajtyiñ}
\partofspeech{vt}
\onedefinition{1}
\spanishtranslation{preguntar}
\cholexample{Woli jk'ajtyiñbajche' ma'tyoj juñkujch kajpe.}
\exampletranslation{Estoy preguntando cuánto pagas por un bulto de café.}
\onedefinition{2}
\spanishtranslation{pedir}
\cholexample{Mik majlel jk'ajtyiñ mi muk' iyäk'eñoñ tyi bety ityak'iñ.}
\exampletranslation{Voy a pedir dinero prestado.}

\entry{k'ajtyisañ}
\relevantdialect{Sab.}
\conjugationtense{variante}
\conjugationverb{k'ajatyañ}
\spanishtranslation{acordar}

\entry{k'ajyem}
\partofspeech{adj}
\spanishtranslation{hallado}

\entry{k'al}
\partofspeech{adv}
\nontranslationdef{Se relaciona con una cosa ancha que se despega; p. ej.:}
\cholexample{Tsa' ik'al lok'sa jiñi tyabla.}
\exampletranslation{Despegó la tabla (aflojando).}

\entry{-k'al}
\nontranslationdef{Sufijo numeral para contar unidades de veinte; p. ej.:}
\cholexample{juñk'al}
\partofspeech{adj}
\exampletranslation{veinte;}
\cholexample{cha'k'al}
\partofspeech{adj}
\exampletranslation{cuarenta.}

\entry{k'am}
\defsuperscript{1}
\partofspeech{adj}
\spanishtranslation{enfermo}
\cholexample{Pejtyelel ora k'am lakña'.}
\exampletranslation{Nuestra mamá siempre está enferma.}

\entry{k'am}
\defsuperscript{2}
\partofspeech{adv}
\spanishtranslation{fuertemente}
\cholexample{K'am tsi'bajbe.}
\exampletranslation{Le pegó fuertemente.}
\secondaryentry{k'am ja'al}
\secondtranslation{lluvia recia}
\secondaryentry{k'ambä ty'añ}
\secondtranslation{voz fuerte}

\entry{k'amäjel}
\partofspeech{s}
\spanishtranslation{enfermedad}
\alsosee{xk'amäjel}

\entry{*k'amel}
\partofspeech{s}
\spanishtranslation{abundancia}
\cholexample{Añ ik'amel ixim.}
\exampletranslation{Hay abundancia de maíz.}

\entry{*k'amlel}
\partofspeech{s}
\spanishtranslation{abundancia}
\cholexample{Añ ik'amlel tyuk' kajpe'.}
\exampletranslation{Es la temporada de abundancia de café.}

\entry{k'am'añ}
\partofspeech{vi}
\spanishtranslation{enfermarse}
\cholexample{Kabäl mi ik'am'añob jiñi alälob kome mach yäxik jiñi ja' mu'bä ijapob.}
\exampletranslation{Los niños siempre se enferman porque el agua que toman no está limpia.}

\entry{k'añal}
\partofspeech{adj}
\spanishtranslation{maíz amarillo}
\cholexample{K'añal jiñi ixim.}
\exampletranslation{Ese maíz es amarillo.}

\entry{k'atye'}
\relevantdialect{Sab.}
\conjugationtense{variante}
\conjugationverb{k'ajtye'}
\spanishtranslation{puente}
\alsosee{pañtye'}

\entry{k'atyiñbak}
\relevantdialect{Sab.}
\partofspeech{s}
\nontranslationdef{Dominio de los demonios y destino de los malos.}

\entry{k'axel}
\partofspeech{vi}
\onedefinition{1}
\spanishtranslation{pasar}
\cholexample{Yom laj k'axel tyi' lum juañ.}
\exampletranslation{Debemos pasar por el terreno de Juan.}
\onedefinition{2}
\spanishtranslation{cruzar}
\cholexample{Yom laj k'axel tyi ja'.}
\exampletranslation{Debemos cruzar el río.}
\onedefinition{3}
\spanishtranslation{cambiar}
\cholexample{Ijk'äl mi kaj laj k'axel tyi yambä otyoty.}
\exampletranslation{Mañana vamos a cambiarnos de casa.}

\entry{k'axibäl}
\partofspeech{s}
\onedefinition{1}
\spanishtranslation{paso}
\clarification{por un río}
\cholexample{Pek' jiñi ja' tyi k'axibäl.}
\exampletranslation{El agua está baja en el paso.}
\onedefinition{2}
\spanishtranslation{puente}

\entry{k'axtyañ}
\partofspeech{vt}
\spanishtranslation{cruzar}
\cholexample{Tyi pañtye' mi laj k'axtyañ jiñi ja'.}
\exampletranslation{Cruzamos el río por el puente.}

\entry{k'ay}
\partofspeech{s}
\spanishtranslation{canto}
\cholexample{Weñ ity'ojol jiñi k'ay.}
\exampletranslation{Ese canto es muy bonito.}

\entry{k'äbäl}
\partofspeech{s}
\spanishtranslation{mano, brazo}
\secondaryentry{ik'äb}
\secondpartofspeech{s}
\secondtranslation{su mano, su brazo}

\entry{*k'äbtye'}
\partofspeech{s}
\spanishtranslation{gajo de un árbol}
\cholexample{Tsa' yajli ik'äbtye'.}
\exampletranslation{Se cayó el gajo del árbol.}

\entry{k'äk}
\partofspeech{vt}
\spanishtranslation{encaramar}
\cholexample{Mi lak'äk letsel chikib tyi' jol otyoty.}
\exampletranslation{Encaramamos la canasta arriba de la casa.}

\entry{k'äkäkña}
\partofspeech{adv}
\onedefinition{1}
\spanishtranslation{pasando encima}
\clarification{lentamente}
\cholexample{K'äkäkña ñumel tyi' pam otyoty.}
\exampletranslation{Está pasando, caminando lentamente, encima de la casa.}
\onedefinition{2}
\spanishtranslation{pasando flotando}
\cholexample{K'äkäkña ñumel tyi pam ja' jiñi jukub.}
\exampletranslation{El cayuco está pasando por el agua.}

\entry{k'äkchokoñ}
\partofspeech{vt}
\onedefinition{1}
\spanishtranslation{encaramar}
\cholexample{Mi laj k'äkchokoñ ikukujlel otyoty.}
\exampletranslation{Encaramamos la viga de la casa.}
\onedefinition{2}
\spanishtranslation{poner}
\cholexample{Jiñi ch'ityoñ tsa' ik'äkchoko juñ tyi mesa.}
\exampletranslation{El joven puso el libro sobre la mesa.}

\entry{k'äklib}
\partofspeech{s}
\spanishtranslation{base}
\cholexample{Ik'äklib jiñi otyoty ts'ajkibil tyi puru xajlel.}
\exampletranslation{La base de la casa está hecha de concreto.}
\dialectvariant{Sab.}
\dialectword{ñaklib}

\entry{*k'äklib i kuch}
\spanishtranslation{yugo}

\entry{k'äktyäl}
\partofspeech{vt}
\spanishtranslation{colocar}
\clarification{atravesado}
\cholexample{Tyi ojlil mi ik'äktyäl ikukujlel.}
\exampletranslation{La viga se coloca en medio, atravesada.<>}

\entry{*k'äk'al}
\partofspeech{s}
\spanishtranslation{llama}
\cholexample{Chañ mi iletsel ik'äk'al jiñi k'ajk.}
\exampletranslation{La llama del fuego se levanta alto.}

\entry{k'ächk'ächña}
\partofspeech{adv}
\nontranslationdef{Se relaciona con la forma de montar; p. ej.}
\cholexample{K'ächk'ächña mi ityilel.}
\exampletranslation{Viene montada.}

\entry{k'ächchokoñ}
\partofspeech{vt}
\spanishtranslation{montar a}
\cholexample{Yom mi ak'ächchokoñ jiñi alob tyi kawayu'.}
\exampletranslation{Debes montar al niño en el caballo.}

\entry{k'ächlibäl}
\partofspeech{s}
\spanishtranslation{caballo, mula}

\entry{k'ächtyañ}
\partofspeech{vt}
\spanishtranslation{montar}
\cholexample{Mi kaj jk'ächtyañ kawayu'.}
\exampletranslation{Voy a montar el caballo.}

\entry{k'äjk'äs}
\partofspeech{s}
\spanishtranslation{luciérnaga}
\clarification{insecto}

\entry{*k'äjñibal}
\partofspeech{s}
\onedefinition{1}
\spanishtranslation{importancia}
\cholexample{Añ kabäl ik'äjñibäyel moliño cha'añ juch'bal.}
\exampletranslation{El molino tiene mucha importancia.}
\onedefinition{2}
\spanishtranslation{utilidad}
\cholexample{Añ ik'äjñibäyel ili mákiña.}
\exampletranslation{La máquina tiene utilidad.}
\onedefinition{3}
\spanishtranslation{deber}
\cholexample{Mi iweñ mel ik'äjñibäyel cha'añ komisariado.}
\exampletranslation{Cumple bien su deber de comisario.}

\entry{*k'äjñibäyel}
\partofspeech{s}
\spanishtranslation{importancia}

\entry{k'äjñel}
\partofspeech{vi}
\spanishtranslation{usarse}
\cholexample{Woli tyi k'äjñel jiñi machity.}
\exampletranslation{Se está usando el machete.}

\entry{k'äjkel}
\partofspeech{vi}
\onedefinition{1}
\spanishtranslation{llegar}
\clarification{a la cumbre}
\cholexample{Tsa'ix k'äjkiyoñla.}
\exampletranslation{Ya llegamos a la cumbre.}
\onedefinition{2}
\spanishtranslation{subir}
\cholexample{Mi laklujb'añ che' woliyoñlatyi k'äjkel.}
\exampletranslation{Nos cansamos al subir.}

\entry{k'äl}
\partofspeech{vt}
\spanishtranslation{construir}
\clarification{casa}
\cholexample{Mi kaj k'äl kotyoty.}
\exampletranslation{Voy a construir mi casa.}

\entry{k'äläl}
\partofspeech{adv}
\spanishtranslation{hasta}
\cholexample{Tsa' majli k'äläl tyi' tyi' kolem ñajb.}
\exampletranslation{Se fue hasta la orilla del mar.}

\entry{k'ämbil}
\onedefinition{1}
\partofspeech{adj}
\spanishtranslation{usado}
\cholexample{K'ämbilix jiñibatyería.}
\exampletranslation{Esa batería ya está usada.}
\onedefinition{2}
\partofspeech{vi}
\spanishtranslation{usarse}
\cholexample{K'ämbil jiñi jacha cha'añ sek' tye'.}
\exampletranslation{Se usa el hacha para tumbar árboles.}

\entry{k'äñ}
\partofspeech{adj}
\onedefinition{1}
\spanishtranslation{maduro}
\cholexample{K'äñix jiñi kajpe'.}
\exampletranslation{Ese café ya está maduro.}
\onedefinition{2}
\spanishtranslation{pálido}
\cholexample{K'äñix jiñi x'ixik cha'añ k'am.}
\exampletranslation{Esa mujer ya está pálida por la enfermedad.}

\entry{k'äñ bo'lay}
\partofspeech{s}
\spanishtranslation{jaguar}
\clarification{mamífero}

\entry{K'äñk'ämpa'}
\partofspeech{s}
\spanishtranslation{Arroyo Amarillo}
\clarification{finca}

\entry{k'äñk'äñ}
\partofspeech{adj}
\spanishtranslation{amarillo}
\cholexample{K'äñk'äñ jiñi max.}
\exampletranslation{Ese zorro es amarillo.}

\entry{k'äñk'äñ max}
\spanishtranslation{mico de noche}
\clarification{mamífero}

\entry{k'äñk'äñ uch}
\spanishtranslation{tizón, pulgón}
\clarification{enfermedad del maíz}

\entry{k'äñchejb}
\partofspeech{s}
\onedefinition{1}
\spanishtranslation{otate verde}
\clarification{planta}
\onedefinition{2.}
\spanishtranslation{bambú amarillo}
\clarification{planta}

\entry{k'äñcho}
\partofspeech{s}
\spanishtranslation{falsa nauyaca}
\clarification{reptil}

\entry{k'äñ jaläjp}
\spanishtranslation{canícula, temporada de sequía}
\clarification{agosto a septiembre}

\entry{k'äñjixil}
\partofspeech{s}
\spanishtranslation{culebra voladora}
\clarification{reptil}

\entry{k'äñjoläl}
\partofspeech{s}
\spanishtranslation{almohada}

\entry{k'äñlemañ}
\partofspeech{adj}
\onedefinition{1}
\spanishtranslation{brilloso}
\clarification{lámina en el sol}
\onedefinition{2}
\spanishtranslation{amarilla, pero sin cosechar}
\clarification{milpa}

\entry{k'äñsijñ}
\partofspeech{s}
\spanishtranslation{guachipilín}
\clarification{árbol}

\entry{k'äñtyijañ}
\partofspeech{adj}
\spanishtranslation{amarillo}
\cholexample{Ñajtyä tsikil k'äñtyijañ iñich jiñi tye'.}
\exampletranslation{Desde lejos se ve aquella flor amarilla.}

\entry{k'äñtyok'}
\partofspeech{s}
\spanishtranslation{piedra de chispa}
\culturalinformation{Información cultural: Piedra amarilla o negra que se golpea con eslabón de fierro para sacar chispas. Se necesita mecha de hilo torcido. Se mete la mecha en un casquillo de cartucho ya quemado. La piedra amarilla da más chispas que la negra.}

\entry{k'äñwaj}
\relevantdialect{Sab.}
\partofspeech{s}
\spanishtranslation{tortilla amarilla}
\alsosee{waj}

\entry{k'äñwechañ}
\partofspeech{adj}
\spanishtranslation{objeto amarillo y plano}
\cholexample{K'äñwechañ jiñi tyabla.}
\exampletranslation{Esa tabla es amarilla y plana.}

\entry{k'äñxañ}
\partofspeech{s}
\spanishtranslation{árbol grande}
\clarification{de madera amarilla y maciza, y de fruta chica.}

\entry{*k'äñajel}
\onedefinition{1}
\partofspeech{s}
\spanishtranslation{hepatitis}
\clarification{enfermedad que deja la piel amarilla}
\onedefinition{2}
\partofspeech{vi}
\spanishtranslation{madurarse}
\onedefinition{3}
\partofspeech{vi}
\spanishtranslation{ponerse amarillo; p. ej.:}
\clarification{hojas de árboles}

\entry{*k'äñel}
\partofspeech{s}
\onedefinition{1}
\spanishtranslation{maduro}
\cholexample{Añix ik'äñel jiñi jumpajl ja'as.}
\exampletranslation{Ese racimo de plátanos ya tiene algunos maduros.}
\onedefinition{2}
\spanishtranslation{yema}
\cholexample{Ma'añik ik'äñel jiñi tyumuty.}
\exampletranslation{Ese huevo no tenía yema.}

\entry{k'äñ'añ}
\partofspeech{vi}
\onedefinition{1}
\spanishtranslation{madurarse}
\cholexample{Woli tyi k'äñ'añ kajpe'.}
\exampletranslation{Está madurándose el café.}
\onedefinition{2}
\spanishtranslation{palidecer}
\cholexample{Woli tyi k'äñ'añ x'ixik cha'añ tyikwal.}
\exampletranslation{La mujer está palideciendo por el calor.}

\entry{k'äs}
\partofspeech{vt}
\spanishtranslation{quebrar}
\clarification{palo, hueso}

\entry{k'äsäb}
\partofspeech{s}
\spanishtranslation{cascajo}

\entry{k'äskujel}
\partofspeech{vi}
\spanishtranslation{quebrarse}
\clarification{palo, hueso}

\entry{k'äsix}
\partofspeech{adj}
\spanishtranslation{falta de cocimiento}
\clarification{frijol}
\cholexample{K'äsix jiñi bu'ul.}
\exampletranslation{Al frijol le falta cocerse.}

\entry{k'äslaw}
\partofspeech{adv}
\nontranslationdef{Se relaciona con el ruido de palos secos al quebrarse; p. ej.:}
\cholexample{K'äslaw tsa' majli wakax tyi ma'tye'el.}
\exampletranslation{La vaca se fue al acahual haciendo ruido al quebrar los palitos secos.}
\alsosee{woch'law}

\entry{k'ästyäl}
\partofspeech{vi}
\spanishtranslation{quebrarse}
\clarification{palo, hueso}

\entry{k'ästye' bij}
\spanishtranslation{vereda}

\entry{k'äty}
\partofspeech{adv}
\spanishtranslation{atravesado}
\cholexample{Tsa' ik'äty ak'ä tye' tyi' yojlil bij.}
\exampletranslation{Dejó atravesado un palo en medio del camino.}

\entry{k'ätyäl}
\partofspeech{adj}
\spanishtranslation{atravesado}
\cholexample{K'ätyäl jiñi tye' tyi bij.}
\exampletranslation{Está atravesado el palo en el camino.}

\entry{*k'ätyloñtye'el}
\partofspeech{s}
\onedefinition{1}
\spanishtranslation{dintel}
\onedefinition{2}
\spanishtranslation{travesaño donde se amarra el seto}

\entry{k'ätsats}
\partofspeech{s}
\onedefinition{1}
\relevantdialect{Tila}
\spanishtranslation{anona colorada o morada}
\clarification{árbol}
\onedefinition{2}
\relevantdialect{Tum.}
\spanishtranslation{guanábana}
\clarification{árbol}
\alsosee{k'ewex}

\entry{k'äyiñ}
\partofspeech{vt}
\spanishtranslation{cantar}
\cholexample{Ity'ojoljax mi ik'äyiñ jiñi k'ay.}
\exampletranslation{Canta bonito esa canción.}

\entry{k'ä'tyuñ}
\partofspeech{s}
\spanishtranslation{mano de metate}

\entry{k'ok}
\partofspeech{vt}
\spanishtranslation{cortar}
\clarification{cosa redonda}
\cholexample{Awokolik mi ak'ok jiñi k'ätsats.}
\exampletranslation{Por favor corta esa anona.}

\entry{k'ok'}
\partofspeech{adj}
\spanishtranslation{sano}

\entry{k'ok'añ}
\partofspeech{vi}
\spanishtranslation{sanar}
\cholexample{Woli ik'ok'añ jiñi xk'amäjel.}
\exampletranslation{El enfermo está sanando.}

\entry{k'ok'chi}
\partofspeech{s}
\spanishtranslation{tipo de henequén}
\culturalinformation{Información cultural: Se come la flor, pero no se utiliza la fibra.}

\entry{k'ok'lel}
\partofspeech{s}
\spanishtranslation{salud}
\cholexample{Ma'añix ik'ok'lel.}
\exampletranslation{No está bien de salud.}

\entry{k'ochilañ}
\partofspeech{vt}
\spanishtranslation{doblar}
\clarification{la mano}
\cholexample{Mi laj k'ochilañ laj k'äb.}
\exampletranslation{Doblamos la mano.}

\entry{k'ochol}
\partofspeech{adj}
\spanishtranslation{torcido}
\cholexample{K'ochol jiñi tye'.}
\exampletranslation{El palo está torcido.}

\entry{k'ocholmetyel}
\partofspeech{adj}
\spanishtranslation{sinuoso}
\clarification{camino}
\cholexample{K'ocholmetyel ibijlel kchol.}
\exampletranslation{El camino de mi milpa está sinuoso.}
\alsosee{xoyometyel}

\entry{k'oj}
\defsuperscript{1}
\partofspeech{s}
\spanishtranslation{tábano}
\clarification{insecto}

\entry{k'oj}
\defsuperscript{2}
\partofspeech{s}
\nontranslationdef{Término de desprecio para la cara; p. ej.:}
\cholexample{Mach säkik ak'oj.}
\exampletranslation{Tu cara no está limpia (hablado con desprecio).}

\entry{k'ojchel bij}
\spanishtranslation{curva del camino}

\entry{-k'ojl}
\nontranslationdef{Sufijo numeral para contar objetos redondos; p. ej.:}
\cholexample{Che' mi lakmajlel tyi cholel mi lakch'äm majlel juñk'ojl laksa'.}
\exampletranslation{Cuando vamos a la milpa llevamos una bola de pozol.}

\entry{k'ojlañ}
\partofspeech{vt}
\spanishtranslation{zangolotear, sacudir}
\cholexample{Mi ik'ojlañoñlajiñi mula.}
\exampletranslation{La mula nos zangolotea.}

\entry{k'ojlel}
\partofspeech{vi}
\spanishtranslation{arrancarse}
\cholexample{Mi ik'ojlel iyal ja'as.}
\exampletranslation{Se arranca el hijuelo del plátano.}

\entry{k'ojlom}
\partofspeech{s}
\spanishtranslation{gallina ciega}
\clarification{larva de cierto insecto}

\entry{k'ojlostyik}
\partofspeech{adj}
\onedefinition{1}
\spanishtranslation{áspero}
\cholexample{K'ojlostyik jiñi tye'.}
\exampletranslation{El palo es muy áspero.}
\onedefinition{2}
\spanishtranslation{anudado}
\cholexample{K'ojlostyik jiñi laso.}
\exampletranslation{El lazo está anudado.}

\entry{k'ojkel}
\partofspeech{vi}
\onedefinition{1}
\spanishtranslation{arrancarse}
\cholexample{Mi ik'ojkel ibäkel iyej.}
\exampletranslation{Se arranca el diente.}
\onedefinition{2}
\spanishtranslation{desprender}
\cholexample{Mi ik'ojkel lok'el lakpäk' tyi laj k'äb.}
\exampletranslation{Se desprende una verruga de la mano.}

\entry{k'ojty}
\partofspeech{s}
\spanishtranslation{retoño}

\entry{k'ojyel}
\partofspeech{vi}
\spanishtranslation{fastidiarse}
\cholexample{K'ojyelix muk'oñ.}
\exampletranslation{Ya me fastidié.}
\dialectvariant{Sab.}
\dialectword{bo'oyel}

\entry{k'oj'ojtyañ}
\partofspeech{vt}
\spanishtranslation{preocuparse}

\entry{k'oj'ol}
\partofspeech{s}
\spanishtranslation{preocupación}
\cholexample{Añ ik'oj'ol cha'añ iyalobil.}
\exampletranslation{Siente preocupación por su hijo.}

\entry{k'ol}
\partofspeech{vt}
\onedefinition{1}
\spanishtranslation{arrancar}
\clarification{hijuelo de plátano}
\cholexample{Tsa' ik'olo iyal ja'as.}
\exampletranslation{Arrancó una mata de plátano.}
\onedefinition{2}
\spanishtranslation{desenterrar}
\clarification{mata de plátano, piedra}
\cholexample{Tsa' ik'olo lok'el xajlel.}
\exampletranslation{Desenterró la piedra.}

\entry{k'olilañ}
\partofspeech{vt}
\spanishtranslation{hacer una bola}

\entry{k'oliñ}
\partofspeech{vt}
\spanishtranslation{hacer bolas}
\cholexample{Jiñi ch'ityoñ mi ik'oliñ ok'ol tyi' k'äb.}
\exampletranslation{El niño hace bolas de lodo con la mano.}

\entry{k'oliña}
\partofspeech{adv}
\nontranslationdef{Se relaciona con el sonido de algo adentro de un envase; p. ej.:}
\cholexample{K'oliña jiñi lew tyi mal jiñi latya.}
\exampletranslation{La manteca suena adentro de la lata.}

\entry{K'olipa'}
\partofspeech{s}
\spanishtranslation{nombre de colonia}

\entry{k'olok'}
\partofspeech{s}
\spanishtranslation{guarumbo}
\spanishtranslation{chancarro}
\clarification{árbol que se utiliza para hacer canales para traer agua}

\entry{K'olojil}
\partofspeech{s}
\spanishtranslation{nombre de colonia}

\entry{k'olol}
\partofspeech{s}
\spanishtranslation{roble}

\entry{K'ololil}
\partofspeech{s}
\spanishtranslation{Arboleda de Encino}
\clarification{colonia}

\entry{k'omokña}
\partofspeech{adj}
\spanishtranslation{hondonado}
\cholexample{K'omokña jiñi kampo.}
\exampletranslation{El campo está hondonado.}

\entry{k'omoch}
\partofspeech{s}
\spanishtranslation{taco}

\entry{k'omol}
\partofspeech{adj}
\spanishtranslation{hondonado}
\cholexample{K'omol jiñi lum.}
\exampletranslation{La tierra está hondonada.}

\entry{k'omtyäl}
\partofspeech{s}
\spanishtranslation{rejoya}
\cholexample{Jiñi kajpe'lel añ tyi k'omtyäl.}
\exampletranslation{El cafetal está en una rejoya.}

\entry{k'okil ts'i'}
\relevantdialect{Sab.}
\spanishtranslation{perro de monte}
\spanishtranslation{agutí}
\clarification{mamífero}

\entry{k'ok'esañ}
\partofspeech{vt}
\spanishtranslation{sanar}

\entry{k'osäl}
\partofspeech{s}
\spanishtranslation{granos}
\clarification{en la cabeza}
\secondaryentry{ik'os}
\secondpartofspeech{s}
\secondtranslation{sus granos}

\entry{k'otya}
\relevantdialect{Sab., Tila}
\partofspeech{adj}
\spanishtranslation{bonito}
\cholexample{K'otyajax awotyoty.}
\exampletranslation{Tu casa es muy bonita.}
\alsosee{ty'ojol}

\entry{k'otyel}
\partofspeech{vi}
\spanishtranslation{llegar}
\clarification{allá}
\cholexample{Tsa' k'otyi tyi tyejklum ak'bi.}
\exampletranslation{Llegó al pueblo ayer.}

\entry{k'otyem}
\partofspeech{adj}
\spanishtranslation{llegado}
\clarification{allá}
\cholexample{Maxtyo añik k'otyem jiñi wiñik.}
\exampletranslation{Ese hombre todavía no ha llegado (allá).}

\entry{k'otyib}
\partofspeech{s}
\spanishtranslation{destino}

\entry{k'oxol}
\partofspeech{adj}
\spanishtranslation{bola}
\clarification{de masa, alimento, tierra}
\cholexample{Tyikiñ k'oxol jiñi kaxlañ waj.}
\exampletranslation{Está seco el (bola de) pan.}

\entry{k'oyem}
\partofspeech{s}
\spanishtranslation{cerdito}
\spanishtranslation{puerquito}
\clarification{ave}
\dialectvariant{Sab.}
\dialectword{ch'ämpäk'}

\entry{k'o'awom}
\relevantdialect{Sab.}
\partofspeech{adv}
\spanishtranslation{acaso}

\entry{k'o'chokoñ}
\partofspeech{vt}
\spanishtranslation{colocar}
\clarification{objeto esférico}
\cholexample{Tsa' ik'o'choko juñk'ojl sa' tyi mesa.}
\exampletranslation{Colocó una bola de pozol sobre la mesa.}

\entry{k'o'loch}
\partofspeech{s}
\spanishtranslation{tipo de hongo}
\clarification{comestible}

\entry{k'o'mäkäl}
\partofspeech{adj}
\spanishtranslation{cerrado de nubes}
\cholexample{K'o'mäkäl jiñi pañimil.}
\exampletranslation{Está cerrado de nubes.}

\entry{k'o'ol}
\partofspeech{adj}
\spanishtranslation{esférico}
\clarification{p. ej.: pelota, bola de masa}

\entry{k'o'omejl}
\relevantdialect{Tila}
\partofspeech{adj}
\spanishtranslation{puede ser que podamos}
\cholexample{K'o'omejloñlatyi tyroñel.}
\exampletranslation{Puede ser que podamos trabajar.}

\entry{k'o'o'}
\relevantdialect{Sab.}
\partofspeech{part}
\spanishtranslation{es verdad}

\entry{k'ubujl}
\partofspeech{s}
\spanishtranslation{zacuilla, zanate de oro}
\clarification{ave}

\entry{k'ubul}
\partofspeech{s}
\spanishtranslation{zacua gigante}
\clarification{ave}

\entry{*k'uk'mal}
\partofspeech{s}
\spanishtranslation{plumaje}

\entry{*k'uk'umlel}
\partofspeech{s}
\spanishtranslation{plumaje}

\entry{K'uk'wits}
\partofspeech{s}
\spanishtranslation{Cerro Alto}
\clarification{Tumbalá}

\entry{k'uch}
\partofspeech{vt}
\spanishtranslation{arquear}
\cholexample{Mi laj k'uch itye'el kajpe' cha'añ mi ilok'el ibuts.}
\exampletranslation{Arqueamos el gajo de café para que le salga retoño.}

\entry{k'uch buchul}
\spanishtranslation{sentado, agachado}
\cholexample{K'uch buchul tsak tyaja.}
\exampletranslation{Lo encontré sentado con la cabeza agachada.}

\entry{k'uchukña}
\partofspeech{adv}
\nontranslationdef{Relacionado con la forma de caminar encorvándose; p. ej.:}
\cholexample{K'uchukña mi icha'leñ xjämbal jiñi ñoxbä.}
\exampletranslation{El viejo camina encorvado.}

\entry{k'uchu paty}
\spanishtranslation{jorobado}

\entry{k'uch'uchña}
\partofspeech{adv}
\nontranslationdef{Relacionado con la forma de caminar agachado.}

\entry{k'ujts}
\partofspeech{s}
\onedefinition{1}
\spanishtranslation{tabaco}
\cholexample{Mi ityuk' iyopol k'ujts cha'añ mi ityikesañ.}
\exampletranslation{Corta la hoja del tabaco para secarla.}
\onedefinition{2}
\spanishtranslation{cigarro, puro}
\cholexample{Jiñi wiñik woli'bäl ik'ujts cha'añ mi iñuk'.}
\exampletranslation{Ese hombre está enrollando su cigarro para fumarlo.}

\entry{k'ujtsijel}
\partofspeech{vi}
\spanishtranslation{fumar}

\entry{k'ujyel}
\partofspeech{vi}
\spanishtranslation{doblarse el tobillo}
\cholexample{Tsa' k'ujyi ibik' iyok.}
\exampletranslation{Se le dobló el tobillo.}

\entry{k'uñ}
\partofspeech{adj}
\onedefinition{1}
\spanishtranslation{blando}
\clarification{alimentos, tierra, palo}
\cholexample{K'uñ jiñi tye'.}
\exampletranslation{El palo es blando.}
\onedefinition{2}
\spanishtranslation{débil}
\clarification{cuerpo}
\cholexample{K'uñ pañimil cha'añ k'amäjel.}
\exampletranslation{Está muy débil por su enfermedad.}
\onedefinition{3}
\spanishtranslation{pasivo}
\cholexample{K'uñ ipusik'al jiñi wiñik.}
\exampletranslation{Ese hombre es pasivo.}

\entry{k'uñche'ix}
\partofspeech{adj}
\spanishtranslation{poco mejor}
\clarification{persona}
\cholexample{K'uñchix añ ktyatyi.}
\exampletranslation{Mi papá ya está un poco mejor.}

\entry{k'uñlajam}
\partofspeech{adj}
\spanishtranslation{tibio}
\cholexample{K'uñlajam jiñi ja'.}
\exampletranslation{Esa agua es tibia.}

\entry{k'uñlitsañ}
\partofspeech{adj}
\onedefinition{1}
\spanishtranslation{débil}
\clarification{palo largo}
\cholexample{K'uñlitsañ jiñi tye'.}
\exampletranslation{Ese palo es muy débil.}
\onedefinition{2}
\spanishtranslation{débil}
\clarification{cuerpo}
\cholexample{K'uñlitsañoñ cha'añ k'ajk.}
\exampletranslation{Estoy débil por la calentura.}

\entry{k'uñluk'añ}
\partofspeech{adj}
\spanishtranslation{suave}
\clarification{madera}
\cholexample{K'uñluk'añ jiñi tye'.}
\exampletranslation{Ese palo es suave.}

\entry{k'uñluyañ}
\partofspeech{adj}
\spanishtranslation{fofa}
\clarification{piel}
\cholexample{K'uñluyañ ipächälel aläl.}
\exampletranslation{La piel de una criatura es fofa.}

\entry{k'uñk'el}
\partofspeech{vt}
\spanishtranslation{ir por primera vez para averiguar}
\cholexample{Muk'ix lakmajlel laj k'uñk'el jiñi xk'amäjel.}
\exampletranslation{Quizás ya sea conveniente ir por primera vez para averiguar cómo está el enfermo.}

\entry{k'uñk'ixiñ}
\partofspeech{adj}
\spanishtranslation{tibio}
\cholexample{K'uñk'ixiñ ili kajpe'.}
\exampletranslation{Este café está tibio.}

\entry{k'uñtye'}
\partofspeech{adv}
\spanishtranslation{despacio}
\cholexample{K'uñtye' yom mi lakmajlel.}
\exampletranslation{Debemos ir despacio.}

\entry{k'uñ tyo}
\partofspeech{adj}
\spanishtranslation{débil todavía}
\cholexample{Añkese tsa' lajmi, k'uñtyo pañimil.}
\exampletranslation{Aunque ya sanó, todavía está débil.}

\entry{k'uñukñiyel}
\partofspeech{s}
\spanishtranslation{debilidad}

\entry{k'uñyumañ}
\partofspeech{adj}
\spanishtranslation{muy blando}
\clarification{palo}
\cholexample{K'uñyumañ mi ipäjkel jiñi tye'.}
\exampletranslation{Ese árbol se dobla porque está muy blando.}

\entry{k'uñ'añ}
\partofspeech{vi}
\onedefinition{1}
\spanishtranslation{ablandarse}
\cholexample{Mi ik'uñ'añ lum cha'añ ja'al.}
\exampletranslation{La tierra se ablanda con el agua.}
\onedefinition{2}
\spanishtranslation{debilitarse}
\cholexample{Tsa' k'uñ'a jiñi wiñik cha'añ k'amäjel.}
\exampletranslation{Ese hombre se debilitó por la enfermedad.}
\onedefinition{3}
\spanishtranslation{hacerse pacífico}
\cholexample{Añtyo yom mi ik'uñ'añ ipusik'al ili wiñik.}
\exampletranslation{Falta que se haga pacífico este hombre.}

\entry{*k'uñel}
\partofspeech{s}
\spanishtranslation{lóbulo}

\entry{k'upiñ}
\partofspeech{vt}
\spanishtranslation{ansiar}
\cholexample{Kabäl mi ik'upiñ we'eläl.}
\exampletranslation{Ansiamos comer carne.}

\entry{k'uty}
\partofspeech{vt}
\spanishtranslation{tamular, machucar}
\clarification{chile}
\cholexample{Mi ik'uty ich tyi ch'ejew.}
\exampletranslation{Tamula chile en el molcajete.}

\entry{k'utyilañ}
\partofspeech{vt}
\spanishtranslation{tamular, machucar}
\clarification{chile}

\entry{k'ux}
\defsuperscript{1}
\partofspeech{vt}
\onedefinition{1}
\spanishtranslation{comer}
\clarification{tortillas, frijol, carne}
\onedefinition{2}
\spanishtranslation{morder}
\cholexample{Mi kaj ik'uxety jiñi ts'i'.}
\exampletranslation{El perro te va a morder.}

\entry{k'ux}
\defsuperscript{2}
\onedefinition{1}
\partofspeech{vi}
\spanishtranslation{doler}
\cholexample{K'ux ijol.}
\exampletranslation{Le duele la cabeza.}
\onedefinition{2}
\partofspeech{adj}
\spanishtranslation{adolorido}
\cholexample{Añ k'ux jol.}
\exampletranslation{Tiene la cabeza adolorida.}

\entry{*k'uxbälel}
\relevantdialect{Sab.}
\partofspeech{s}
\spanishtranslation{comestible}

\entry{k'uxbibil}
\partofspeech{adj}
\spanishtranslation{querido}
\cholexample{K'uxbibil jiñi xñox.}
\exampletranslation{Ese viejo es querido.}

\entry{k'uxbil}
\partofspeech{adj}
\onedefinition{1}
\spanishtranslation{comestible}
\cholexample{K'uxbil jiñi xpatye'.}
\exampletranslation{Ese hongo es comestible.}
\onedefinition{2}
\spanishtranslation{comido}
\cholexample{K'uxbil jiñi ixim che'bä tsak tyaja.}
\exampletranslation{El maíz estaba comido cuando vine.}
\dialectvariant{Sab.}
\dialectword{k'uxbälel}

\entry{k'uxbiñ}
\partofspeech{vt}
\spanishtranslation{amar, querer}

\entry{*k'uxlel i pusik'al}
\relevantdialect{Sab.}
\spanishtranslation{envidia}
\cholexample{Añ kabäl ik'uxlel ipusik'al cha'añ ichubä'añ ipi'älob.}
\exampletranslation{Tiene mucha envidia por los bienes de sus compañeros.}

\entry{k'uxk'el}
\relevantdialect{Sab.}
\partofspeech{vt}
\spanishtranslation{odiar}
\cholexample{Tsa' ik'uxk'ele ipi'äl.}
\exampletranslation{Odió a su compañero.}

\entry{k'uxtyayem}
\relevantdialect{Sab.}
\partofspeech{adj}
\spanishtranslation{triste}
\cholexample{K'uxtyayem cha'añ ma'añ ja'al.}
\exampletranslation{Está triste porque no llueve.}
\alsosee{ch'ijiyem}

\entry{k'uxuk}
\partofspeech{vt}
\spanishtranslation{comer}
\clarification{negativo}
\cholexample{Mach ak'uxuk awaj.}
\exampletranslation{No comerás tu comida.}

\entry{k'uxul}
\partofspeech{adj}
\spanishtranslation{mordido}
\cholexample{K'uxul jiñi x'ixik tyi lukum.}
\exampletranslation{La mujer fue mordida por una culebra.}

\entry{k'uxultyañ}
\partofspeech{vt}
\spanishtranslation{estimar}

\entry{k'uyiña}
\partofspeech{adv}
\spanishtranslation{cojeando}
\cholexample{K'uyiña mi icha'leñ xämbal cha'añ k'ux iyok.}
\exampletranslation{Camina cojeando porque le duele el pie.}

\entry{k'u'}
\partofspeech{s}
\onedefinition{1}
\spanishtranslation{nido}
\clarification{de pájaros y de animales}
\onedefinition{2}
\spanishtranslation{broza}
\cholexample{Kabäl tsa' tyempäyi ik'u' cholel.}
\exampletranslation{Se juntó mucha broza de la milpa.}

\alphaletter{Ch}

\entry{chab}
\partofspeech{s}
\spanishtranslation{miel}

\entry{chabi}
\partofspeech{adv}
\spanishtranslation{pasado mañana}
\cholexample{Chabij mik majlel tyi tyejklum.}
\exampletranslation{Pasado mañana me voy al pueblo.}

\entry{chakal}
\relevantdialect{Tila}
\partofspeech{adj}
\spanishtranslation{desnudo}
\cholexample{Chakal jiñi alob.}
\exampletranslation{Ese niño anda desnudo.}
\alsosee{pits'il}

\entry{chakatyorex}
\partofspeech{s}
\spanishtranslation{hormiga colorada}
\clarification{pica fuerte}

\entry{chak wa'al}
\spanishtranslation{desnudo y parado}
\cholexample{Chak wa'al jiñi ch'ityoñ.}
\exampletranslation{Ese chamaco está parado y desnudo.}
\alsosee{wakal}

\entry{chajk}
\partofspeech{s}
\spanishtranslation{rayo}
\culturalinformation{Información cultural: Se cree que defiende a las colonias de espíritus malos.}

\entry{-chajp}
\nontranslationdef{Sufijo numeral para contar clases; p. ej.:}
\cholexample{Jiñäch juñchajpbä pisil.}
\exampletranslation{Esa es otra clase de tela.}

\entry{chajpañ}
\partofspeech{vt}
\spanishtranslation{preparar}

\entry{chajpäbil}
\partofspeech{adj}
\spanishtranslation{preparado}
\cholexample{Chajpäbil jiñi ibäl ñäk'äl.}
\exampletranslation{La comida está preparada.}

\entry{chalakña}
\partofspeech{adv}
\spanishtranslation{fuerte}
\clarification{el ruido cuando cae un aguacero}
\cholexample{Chalakña tsa' tyili ja'al.}
\exampletranslation{El aguacero vino fuerte.}

\entry{chañ}
\onedefinition{1}
\partofspeech{adj}
\spanishtranslation{alto}
\cholexample{Chañ añ iwuty.}
\exampletranslation{La fruta está alta.}
\onedefinition{2}
\relevantdialect{Tila}
\partofspeech{s}
\spanishtranslation{cielo}
\alsosee{pañchañ}
\secondaryentry{tyi chañ}
\secondpartofspeech{adv}
\secondtranslation{arriba}

\entry{*chañlel}
\partofspeech{s}
\spanishtranslation{altura}

\entry{chañtye'}
\partofspeech{s}
\spanishtranslation{chanté, cocohuite}
\clarification{árbol que sirve para postes}

\entry{chañwox}
\partofspeech{s}
\spanishtranslation{golonchaco, codorniz, bolonchaco}
\clarification{ave}

\entry{chañäl}
\partofspeech{s}
\spanishtranslation{mirada}
\cholexample{Mach acha'leñ chañäl che' woli tyi we'el lakjula'.}
\exampletranslation{No estés mirando (lit.: hacer una mirada) mientras está comiendo nuestra visita.}

\entry{*chañelal}
\partofspeech{s}
\onedefinition{1}
\spanishtranslation{parte de arriba}
\cholexample{Tyi' chañelal otyoty tsa' iyäk'ä tyi käñ'añ ja'as.}
\exampletranslation{Puso el plátano en la parte de arriba de la casa para que se madure.}
\onedefinition{2}
\spanishtranslation{altura}
\cholexample{P'isbil yom ichañelal postye.}
\exampletranslation{Debe ser medida la altura del poste.}

\entry{chapäy}
\partofspeech{s esp}
\spanishtranslation{chapaya}
\clarification{la fruta de la palma}

\entry{chaw}
\partofspeech{vt}
\onedefinition{1}
\spanishtranslation{desenvolver}
\cholexample{Yom mi achaw jiñi arus.}
\exampletranslation{Debes desenvolver el arroz.}
\onedefinition{2}
\spanishtranslation{quitar}
\clarification{pañales}
\cholexample{Yom mi achawbeñ imajts aläl.}
\exampletranslation{Debes quitarle el pañal al niño.}

\entry{chaxal}
\partofspeech{adj}
\onedefinition{1}
\spanishtranslation{seco sin hojas}
\clarification{árbol}
\onedefinition{2}
\spanishtranslation{sin paredes}
\clarification{casa}

\entry{-chaxañ}
\nontranslationdef{Sufijo que se presenta con raíces adjetivas que indican cola y se refiere a huesos.}

\entry{chaxchaxtyäl}
\partofspeech{adv}
\spanishtranslation{así de larga y delgada}
\clarification{espinas}
\cholexample{Che' ya chaxchaxtyäl jiñi uch' ja'.}
\exampletranslation{Así de delgado es el zancudo.}

\entry{chax läktyik}
\onedefinition{1}
\spanishtranslation{raíces}
\clarification{de árboles, que sobresalen del suelo.}
\onedefinition{2}
\spanishtranslation{gajos secos tirados en la milpa}

\entry{*chaxtyiklel}
\partofspeech{s}
\spanishtranslation{altura}
\clarification{de un tapesco para secar café o frijol}

\entry{*chaxwi'}
\partofspeech{s}
\spanishtranslation{raíces que salen del suelo}
\cholexample{Tsikil ichaxwi' jiñi ixim.}
\exampletranslation{Aparecen las raíces del maíz.}

\entry{cha'}
\partofspeech{adv}
\spanishtranslation{otra vez}
\cholexample{Tsa' icha' alä.}
\exampletranslation{Lo dijo otra vez.}

\entry{cha'am}
\partofspeech{s}
\spanishtranslation{muela}

\entry{cha'añ}
\partofspeech{part}
\onedefinition{1}
\spanishtranslation{de}
\onedefinition{2}
\spanishtranslation{para}
\cholexample{Tsa' tyiliyoñ cha'añ mik tsiktyesäbeñety.}
\exampletranslation{Vine para avisarte.}
\onedefinition{3}
\spanishtranslation{por}
\cholexample{Ma'añix tsa' tyili cha'añ ja'al.}
\exampletranslation{Ya no vino por la lluvia.}
\secondaryentry{kcha'añ}
\secondtranslation{mío}
\secondaryentry{acha'añ}
\secondtranslation{tuyo}
\secondaryentry{icha'añ}
\secondtranslation{suyo, suya}

\entry{cha'añiñ}
\partofspeech{vt}
\spanishtranslation{adueñarse}
\cholexample{Mi kaj acha'añiñ ichubä'añ.}
\exampletranslation{Te vas a adueñar de sus bienes.}

\entry{cha'chajp i pusik'al}
\spanishtranslation{dos caras, hipócrita}
\cholexample{Cha'chajp ipusik'al jiñi wiñik.}
\exampletranslation{Ese hombre es hipócrita.}

\entry{cha'leñ}
\partofspeech{vt}
\spanishtranslation{hacer}
\cholexample{Uts'aty mi icha'leñ e'tyel.}
\exampletranslation{Hace su trabajo muy bien.}
\secondaryentry{mi icha'leñ alas}
\secondtranslation{jugar}
\secondaryentry{mi icha'leñbäk'eñ}
\secondtranslation{temer}
\secondaryentry{mi icha'leñ k'ay}
\secondtranslation{cantar}
\secondaryentry{mi icha'leñ e'tyel}
\secondtranslation{trabajar}
\secondaryentry{mi icha'leñ ja'al}
\secondtranslation{llover}
\secondaryentry{mi icha'leñ ojbal}
\secondtranslation{toser}
\secondaryentry{mi icha'leñ oñel}
\secondtranslation{gritar}
\secondaryentry{mi icha'leñ ty'añ}
\secondtranslation{hablar}
\secondaryentry{mi icha'leñ uk'el}
\secondtranslation{llorar}
\secondaryentry{mi icha'leñ wäyel}
\secondtranslation{dormir}
\secondaryentry{mi icha'leñ xämbal}
\secondtranslation{andar}

\entry{cha'päjk}
\partofspeech{s}
\spanishtranslation{dos dobladas}

\entry{cha'p'ejl}
\partofspeech{adj}
\spanishtranslation{dos}

\entry{chäb}
\partofspeech{adj}
\spanishtranslation{dulce}
\cholexample{Sumuk jiñi ul che' chäb.}
\exampletranslation{El atole está sabroso cuando está dulce.}

\entry{chäbäkña}
\partofspeech{adj}
\spanishtranslation{oloroso}
\cholexample{Chäbäkña iya'lel jiñi sik'äb che' mi yäk'ob tyi paj'añ.}
\exampletranslation{El jugo de la caña es oloroso cuando se está fermentando.}

\entry{chäbiji}
\partofspeech{adv}
\spanishtranslation{anteayer}
\cholexample{Chäbiji tsa kujtyesa jiñi pak'.}
\exampletranslation{Terminé mi siembra anteayer.}

\entry{chäk}
\partofspeech{adj}
\onedefinition{1}
\spanishtranslation{sin hierba}
\cholexample{Chäk jiñi lum ambä tyi xajlelol.}
\exampletranslation{La tierra del pedregal es sin hierba.}
\onedefinition{2}
\spanishtranslation{limpio}
\clarification{un camino}
\cholexample{Chäk jiñi bij kome tsa' ipätyäyob ak'bi.}
\exampletranslation{El camino está limpio porque lo chaporrearon ayer.}
\onedefinition{3}
\secondtranslation{tacaño}
\cholexample{Weñ chäk ayumijel kome ma'añik mi iyäk' tyak'iñ.}
\exampletranslation{Su tío es muy tacaño porque no da dinero.}

\entry{chäkajl}
\defsuperscript{1}
\partofspeech{s}
\onedefinition{1}
\spanishtranslation{carbúnculo}
\onedefinition{2}
\spanishtranslation{tumor}

\entry{chäkajl}
\defsuperscript{2}
\partofspeech{s}
\spanishtranslation{mulato}
\spanishtranslation{copal}
\clarification{árbol de cáscara colorada, madera blanca y suave}

\entry{chäkächäkäjax}
\partofspeech{adv}
\spanishtranslation{seguidamente}
\clarification{se enferma}
\cholexample{Chäkächäkäjax mi ik'am'añ jiñi wiñik.}
\exampletranslation{Ese hombre se enferma muy seguido (lit.: seguidamente).}

\entry{chäkäl aty}
\spanishtranslation{insecto}
\clarification{p. ej.: el zancudo}

\entry{chäkä k'el}
\partofspeech{vt}
\onedefinition{1}
\spanishtranslation{cuidar}
\cholexample{Chäkä k'ele me abä tyi awe'tyel.}
\exampletranslation{Cuídate mucho en tu trabajo.}
\onedefinition{2}
\spanishtranslation{dirigir}
\clarification{a trabajadores}
\cholexample{Añ tyi' weñtya cha'añ mi ichäkä k'el jiñi x'e'tyelob ya' tyi kajpe'lel.}
\exampletranslation{Está encargado de dirigir a los trabajadores que están en el cafetal.}

\entry{chäk bajlum}
\spanishtranslation{puma}
\clarification{mamífero}

\entry{chäkbiñ}
\partofspeech{vt}
\spanishtranslation{no permite}
\cholexample{Kabäl mi ichäkbiñ imula.}
\exampletranslation{No me permite usar su mula.}

\entry{chäkbulañ}
\partofspeech{adj}
\spanishtranslation{seco}
\clarification{tierra}
\cholexample{Ma'añix ik'äjñibal jiñi chäkbulañbä lum.}
\exampletranslation{Ya no tiene utilidad ese terreno seco.}

\entry{chäkkojañ}
\partofspeech{adj}
\spanishtranslation{medio colorado}
\clarification{trapo, perro}

\entry{chäkkolañ}
\partofspeech{adj}
\onedefinition{1}
\spanishtranslation{claro}
\cholexample{Che' chäkkolañ pañimil mi lakmajlel tyi lake'tyel.}
\exampletranslation{Cuando ya esté un poco claro, nos vamos a nuestro trabajo.}
\onedefinition{2}
\spanishtranslation{limpio}
\clarification{de monte}
\cholexample{Chäxkkolañ mi ikäytyäl jiñi matye'el che'baki ora mi laj kolchokoñ.}
\exampletranslation{El monte queda limpio de abajo cuando rozamos el monte chico.}

\entry{chäkchab}
\partofspeech{s}
\onedefinition{1}
\spanishtranslation{tipo de abeja colorada}
\onedefinition{2}
\spanishtranslation{miel colorada}
\onedefinition{3}
\spanishtranslation{maíz negro}

\entry{chäk ik'lel}
\relevantdialect{Sab.}
\spanishtranslation{norte}
\clarification{mal tiempo}

\entry{chäkjocho chuch}
\spanishtranslation{ardilla}
\clarification{mamífero}

\entry{chäkjocho chup}
\spanishtranslation{osito lanudo}
\clarification{larva de cierto insecto}

\entry{chäkjol}
\partofspeech{s}
\spanishtranslation{pelo colorado}

\entry{chäklak'añ}
\partofspeech{adj}
\spanishtranslation{colorada}
\clarification{la piel}
\cholexample{Chäklak'añ lakpächälel che' sijty'emba' lojweñ.}
\exampletranslation{La piel se pone colorada cuando la parte lastimada está hinchada.}

\entry{*chäklel}
\partofspeech{s}
\spanishtranslation{tacañería}
\cholexample{Ma'añik mi iyäk' ichubä'añ kome kabäl ichäklel.}
\exampletranslation{Por su tacañería no da de sus bienes.}

\entry{chäklemañ}
\partofspeech{adj}
\spanishtranslation{claro}
\cholexample{Ñajtyä tsikil chäklemañ ik'äk'al jiñi k'ajk.}
\exampletranslation{Desde lejos se ve que está clara la llama de fuego.}

\entry{chäkluñtye'}
\partofspeech{s}
\spanishtranslation{zapotillo}
\clarification{árbol grande de madera roja y maciza}

\entry{Chäkluñtye'el}
\partofspeech{s}
\spanishtranslation{Arboleda de Madera Colorada}
\clarification{colonia}

\entry{chäkme'}
\partofspeech{s}
\spanishtranslation{venado cabrito}

\entry{chäkmuty}
\partofspeech{s}
\onedefinition{1}
\spanishtranslation{guan cornudo, pollo de pavón}
\onedefinition{2}
\spanishtranslation{hocofaisán, faisán americano}

\entry{chäkol}
\partofspeech{s}
\spanishtranslation{tipo de árbol}
\clarification{la cáscara se usa para mecate}

\entry{chäk ox}
\spanishtranslation{tipo de abeja grande y colorada}

\entry{chäkpi}
\partofspeech{s}
\spanishtranslation{tipo de árbol}
\clarification{de madera dura y amarilla con corazón colorado}

\entry{chäkpiräñ jol}
\spanishtranslation{calvo}
\alsosee{ch'umjol}

\entry{chäktye'}
\partofspeech{s}
\conjugationtense{variante}
\conjugationverb{ch'ujtye'}
\spanishtranslation{cedro}

\entry{Chäktye'pa'}
\partofspeech{s}
\spanishtranslation{Arroyo del Palo Colorado}
\clarification{colonia}

\entry{chäktyijañ}
\partofspeech{adj}
\spanishtranslation{rubio}
\cholexample{Chäktyijañ ijol jiñi ch'ityoñ.}
\exampletranslation{Ese niño tiene el cabello rubio.}

\entry{chäktsäñañ}
\partofspeech{adj}
\spanishtranslation{candente}
\clarification{fierro}
\cholexample{Chäktsäñañ yom jiñi alambre cha'añ mi lakch'ub tyabla.}
\exampletranslation{Se necesita un alambre candente para poder agujerar la tabla.}

\entry{chäkwatsañ}
\partofspeech{adj}
\spanishtranslation{colorado}
\clarification{pelo}
\cholexample{Chäkwatsañ ijol jiñi wiñik.}
\exampletranslation{El hombre tiene el pelo colorado.}

\entry{chäkwaxañ}
\partofspeech{adj}
\spanishtranslation{sonrojado}
\cholexample{Chäkwaxañ iwuty cha'añ kisiñ.}
\exampletranslation{Tiene la cara sonrojada, porque tiene vergüenza.}

\entry{chäkwa'añ}
\partofspeech{adj}
\spanishtranslation{rojo}
\clarification{llamas altas de fuego}

\entry{chäkyumañ}
\partofspeech{adj}
\spanishtranslation{colorado}
\clarification{líquido}
\cholexample{Chäkyumañ jiñi lämälbä ts'ak.}
\exampletranslation{Esa medicina en forma líquida es colorada.}

\entry{chäk'}
\partofspeech{adv}
\spanishtranslation{goteando}
\cholexample{Mi ichäk' p'ajtyel ja' tyi mal otyoty cha'añtyokolix ijamil.}
\exampletranslation{El techo de la casa está agujerado y cae el agua goteando.}

\entry{chäk'ak}
\partofspeech{s}
\spanishtranslation{tipo de zacate}
\clarification{para techo de la casa}

\entry{chäk'añ}
\partofspeech{vi}
\onedefinition{1}
\spanishtranslation{madurarse}
\cholexample{Tyi oktyubre mux kajel tyi chäk'añ kajpe'.}
\exampletranslation{El café comienza a madurarse en octubre.}
\onedefinition{2}
\spanishtranslation{sonrojarse}
\cholexample{Mi ichäk'añ iwuty jiñi x'ixik cha'añ kisiñ.}
\exampletranslation{Se sonroja la cara de la mujer por vergüenza.}

\entry{chäk'chäk'ña}
\partofspeech{adv}
\spanishtranslation{goteando}
\cholexample{Chäk'chäk'ña woli iyochel och ja' tyi mal otyoty cha'añ woli ija'al.}
\exampletranslation{El agua cae goteando por la gotera, porque está lloviendo.}

\entry{chächäk}
\partofspeech{adj}
\spanishtranslation{colorado}
\clarification{rojo}

\entry{chächäklumil}
\partofspeech{s}
\spanishtranslation{tierra colorada}

\entry{Chächäkruz}
\partofspeech{s}
\spanishtranslation{Cruz Colorada}
\clarification{colonia}

\entry{-chäjk'}
\nontranslationdef{Sufijo numeral para contar gotas; p. ej.}
\cholexample{Tsa' yajli juñchäjk' ja'.}
\exampletranslation{Cayó una gota de agua.}

\entry{chäläkña}
\partofspeech{adv}
\nontranslationdef{Se relaciona con la forma en que cae continuamente una cosa líquida; p. ej.:}
\cholexample{Chäläkña woli tyi lok'el ich'ich'elba' tsejpem.}
\exampletranslation{Continuamente le está saliendo sangre de la herida.}

\entry{chäläl}
\partofspeech{adj}
\nontranslationdef{La forma del zacate nuevo que crece después de cortar el anterior.}
\cholexample{Weñ chäläl iyal jiñi jam.}
\exampletranslation{Los retoños del zacate están disparejos.}

\entry{chämel}
\onedefinition{1}
\partofspeech{vi}
\spanishtranslation{morir}
\cholexample{Añ mi ichämel aläbob che' mi iñumel k'amäjel.}
\exampletranslation{Cuando pasa una enfermedad, a veces mueren muchos niños.}
\onedefinition{2}
\partofspeech{s}
\spanishtranslation{enfermedad}
\cholexample{Añ ichämel jiñi x'ixik k'äläl tyi wajali.}
\exampletranslation{Esa mujer ha tenido la enfermedad desde hace mucho tiempo.}
\secondaryentry{puk chämel}
\secondtranslation{desmayarse}

\entry{chämeläyel}
\partofspeech{vi}
\spanishtranslation{estar a punto de morir}
\cholexample{Woli tyi chämeläyel jiñi wiñik.}
\exampletranslation{Ese hombre ya está a punto de morir.}

\entry{chämeñ}
\partofspeech{adj}
\spanishtranslation{muerto}

\entry{*chämib}
\partofspeech{s}
\spanishtranslation{veneno}

\entry{chämibäl}
\partofspeech{s}
\spanishtranslation{veneno}

\entry{chämp'ejl}
\partofspeech{adj}
\spanishtranslation{cuatro}

\entry{chäm'al}
\partofspeech{s}
\spanishtranslation{insecto que come maíz}

\entry{chäñ}
\partofspeech{adv}
\spanishtranslation{continuamente}
\cholexample{Woli ichäñ cha'leñ letyo.}
\exampletranslation{Continuamente está peleando.}

\entry{chäñchäñ ik'}
\relevantdialect{Sab.}
\spanishtranslation{epilepsia}

\entry{chäñi}
\partofspeech{adv}
\spanishtranslation{cuatro días}
\cholexample{Chäñi mik majlel tyi tyejklum.}
\exampletranslation{Dentro de cuatro días me voy al pueblo.}

\entry{*chäñil}
\partofspeech{s}
\spanishtranslation{seres vivientes pequeños}
\secondaryentry{*chäñil chab}
\secondtranslation{abeja}
\secondaryentry{*chäñil ja'}
\secondtranslation{pez, rana, tortuga, insectos del agua}
\secondaryentry{*chäñil lakñäk'}
\secondtranslation{parásitos intestinales}
\secondaryentry{*chäñil pañimil}
\secondtranslation{insectos, animales}

\entry{chäñlujump'ejl}
\partofspeech{adj}
\spanishtranslation{catorce}

\entry{chäñtyañ}
\relevantdialect{Sab.}
\partofspeech{vt}
\spanishtranslation{mirar}
\spanishtranslation{pasear}
\cholexample{Mik majlel tyi lum cha'añ mik chäñtyañ k'iñ.}
\exampletranslation{Voy al pueblo para mirar la fiesta.}
\alsosee{wa'akñiyel}

\entry{chäp}
\partofspeech{vt}
\spanishtranslation{hervir}

\entry{chäpäkña}
\partofspeech{adj}
\spanishtranslation{caluroso}
\cholexample{Chäpäkñajax pañimil.}
\exampletranslation{Está muy caluroso.}

\entry{chäkix}
\partofspeech{adj}
\spanishtranslation{colorado y maduro}

\entry{chäy}
\defsuperscript{1}
\partofspeech{vt}
\spanishtranslation{empapar}
\cholexample{Jiñi tyäñäm kabäl mi ichäy ja'.}
\exampletranslation{El algodón se empapa mucho de agua.}

\entry{chäy}
\defsuperscript{2}
\partofspeech{s}
\spanishtranslation{pez, pescado}
\secondaryentry{ichäyil ja'}
\secondtranslation{pez o pescado de agua}

\entry{chäybil}
\partofspeech{adj}
\spanishtranslation{empapado}
\cholexample{Jiñi pisil chäybil tyi ja'.}
\exampletranslation{Este trapo está empapado de agua.}

\entry{chä'chäk'}
\partofspeech{adv}
\spanishtranslation{varias veces}
\clarification{acción repetida}
\cholexample{Mi lakchä'chäk' jek' lum cha'añ mi lakpäk' ixim.}
\exampletranslation{Hacemos los hoyos en acción repetida para sembrar el maíz.}

\entry{chä'tye'}
\partofspeech{s}
\spanishtranslation{chicozapote}
\clarification{árbol}

\entry{chebi}
\relevantdialect{Sab.}
\partofspeech{adv}
\spanishtranslation{así}

\entry{chek'chek'ña}
\partofspeech{adv}
\spanishtranslation{golpeándose}
\clarification{máquina}
\cholexample{Chek'chek'ña mi imel jiñi mákiña che' mi laj k'äñ tyi ts'ijb.}
\exampletranslation{Esa máquina funciona golpeándola cuando la usamos para escribir.}

\entry{chejachi}
\partofspeech{adv}
\spanishtranslation{así nada más}
\cholexample{Ma'añik tsa' ik'ajtyi ityojol. chejachi tsa' iyäk'eyoñ jiñi ixim.}
\exampletranslation{No me cobró el precio. Así nada más me dio el maíz.}

\entry{chejboy}
\partofspeech{s}
\spanishtranslation{lugar donde hay mucho bambú}

\entry{Chejopa'}
\relevantdialect{Tila}
\partofspeech{s}
\spanishtranslation{Lugar de bambú a la orilla del agua}
\clarification{ranchería}

\entry{chejp}
\partofspeech{s}
\spanishtranslation{bambú}
\clarification{planta}
\dialectvariant{Tila}
\dialectword{jimba}

\entry{chejpañ}
\partofspeech{vt}
\spanishtranslation{agitar}
\clarification{costal}
\cholexample{Mi lakchejpañ jiñi koxtyal cha'añ mi iweñ läts ibä ixim.}
\exampletranslation{Agitamos el costal para que se acomode el maíz.}

\entry{cheñek}
\partofspeech{s}
\spanishtranslation{animal muerto apestoso}
\culturalinformation{Información cultural: Se dice que el diablo come <cheñec>.}

\entry{che'}
\onedefinition{1}
\partofspeech{adv}
\spanishtranslation{así}
\cholexample{Che' yom mi amele'.}
\exampletranslation{Debes hacerlo así.}
\onedefinition{2}
\partofspeech{part}
\spanishtranslation{cuando}
\cholexample{Tsa' yajli iyotyoty che' tsa' ñumi ik'.}
\exampletranslation{Se cayó su casa cuando pasó el viento.}

\entry{che' bajche'}
\spanishtranslation{así como}
\cholexample{Tsa' imele iyotyoty che'bajche' ityaty.}
\exampletranslation{Hizo su casa igual que la de su padre.}

\entry{che' bä}
\spanishtranslation{cuando}
\cholexample{Jumuk' ijulel che'bä tsa' kaji ja'al.}
\exampletranslation{Acababa de llegar cuando comenzó la lluvia.}
\dialectvariant{Tila, Sab.}
\dialectword{che' ñak}

\entry{che'kuyi}
\partofspeech{part}
\spanishtranslation{así es}
\clarification{respuesta}

\entry{che'eñ}
\partofspeech{part}
\spanishtranslation{así dice}
\culturalinformation{Información cultural: Al repetir lo que ha dicho otra persona, termina con <che'en> ‘así dice’.}
\alsosee{che'ob}

\entry{che'ety}
\partofspeech{part}
\spanishtranslation{así debes decir}
\clarification{Al enviar a una persona con un mensaje, termina la orden con la expresión <che'et> 'así debes decir'.}

\entry{che'i}
\partofspeech{adv}
\spanishtranslation{así}
\cholexample{Che'i mi kaj ik'extyäñtyel jiñi ambä iye'tyel.}
\exampletranslation{Así se va a cambiar la autoridad.}

\entry{che'ix chaxtyäl k'iñ}
\spanishtranslation{ya está alto el sol}

\entry{che' jach}
\spanishtranslation{así nada más}
\cholexample{Che' jachbajche' iliyi tsa' subeñtyi.}
\exampletranslation{Así nada más se lo dijeron.}

\entry{che' ja'el}
\partofspeech{part}
\spanishtranslation{así también}
\cholexample{Mach mejlikoñ tyi majlel kome k'am'oñ. che' ja'el k'am kña'.}
\exampletranslation{No puedo ir porque estoy enfermo. Mi mamá también está enferma así.}

\entry{che' jiñi}
\spanishtranslation{entonces}
\cholexample{Tsa' chämi iñoxi'al kchich. che' jiñi tsak melbe ichol.}
\exampletranslation{Se murió el esposo de mi hermana. Entonces le hice su milpa.}

\entry{che'li}
\partofspeech{adv}
\spanishtranslation{así}
\clarification{señalando}
\cholexample{Mach uts'atyikbajche' woliyetylatyi e'tyel. che'li yom mi la'mel.}
\exampletranslation{No está bien la forma en que ustedes están trabajando. Deben de hacerlo así.}

\entry{che' ñak}
\relevantdialect{Sab., Tila}
\spanishtranslation{cuando}
\cholexample{Che' ñak tsa' tyejchi k'iñ tsa' jkäyä ktyroñel.}
\exampletranslation{Cuando comenzó la fiesta dejé mi trabajo.}
\alsosee{che'bä}

\entry{che'ob}
\partofspeech{part}
\spanishtranslation{así dicen}
\clarification{terminación}
\alsosee{che'eñ}

\entry{Che'opa'}
\partofspeech{s}
\spanishtranslation{Arroyo de Bambú}
\clarification{colonia}

\entry{chik}
\partofspeech{vt}
\spanishtranslation{colar}

\entry{*chich}
\partofspeech{s}
\spanishtranslation{hermana mayor}

\entry{chicha}
\partofspeech{s}
\spanishtranslation{aguardiente}
\clarification{la esencia del jugo de la caña ya fermentada}

\entry{chichäl}
\partofspeech{s}
\spanishtranslation{hermana mayor}

\entry{chij}
\defsuperscript{1}
\partofspeech{vt}
\spanishtranslation{cortar}
\clarification{fruta con un palo}
\cholexample{Yom mi atsep xäk'tye' cha'añ mi achij jubel alaxax.}
\exampletranslation{Hay que hacer una horqueta para cortar las naranjas.}

\entry{chij}
\defsuperscript{2}
\partofspeech{s}
\spanishtranslation{ixtle}
\culturalinformation{Información cultural: Se utiliza para hacer hamacas, costales y morrales.}

\entry{chijkañ}
\partofspeech{vt}
\spanishtranslation{colar}

\entry{*chijil}
\partofspeech{s}
\onedefinition{1}
\spanishtranslation{ixtle}
\cholexample{Mach jasälik ichijil cha'añ mi imel jump'ejl ab.}
\exampletranslation{No es suficiente el ixtle para hacer una hamaca.}
\onedefinition{2}
\spanishtranslation{vena}
\cholexample{Tsa' iyotsäbe akuxañ tyi' chijil.}
\exampletranslation{Le puso la aguja en su vena.}

\entry{chijlaw}
\defsuperscript{1}
\partofspeech{adj}
\spanishtranslation{muy mojado}
\clarification{por el rocío}
\cholexample{Chijlaw jiñi tyijil tyi bij.}
\exampletranslation{Está muy mojado el camino por el rocío.}

\entry{chijlaw}
\defsuperscript{2}
\partofspeech{adv}
\nontranslationdef{Se relaciona con el ruido que producen las hojas secas.}

\entry{chijmay}
\partofspeech{s}
\spanishtranslation{venado común}
\clarification{cola blanca}
\dialectvariant{Tila}
\dialectword{säkmé}

\entry{chij ok}
\spanishtranslation{venoso}
\cholexample{Chijlo'tyik iyok jiñi x'ixik.}
\exampletranslation{Se le nota la pierna venosa a esa mujer.}

\entry{chijpel}
\partofspeech{vi}
\spanishtranslation{zafarse}
\cholexample{Mi kaj ichijpel iwex jiñi aläl kome ma'añik ikäjchiñäk'.}
\exampletranslation{Se le va a zafar el calzón a ese niño porque no tiene cinturón.}

\entry{chijtyañ}
\partofspeech{vt}
\spanishtranslation{esperar}
\clarification{presa}
\cholexample{Mi lakmajlel lakchijtyañ me' tyi tye'el.}
\exampletranslation{Vamos a esperar al venado en el bosque.}

\entry{chijulañ}
\partofspeech{vt}
\spanishtranslation{mover}
\clarification{maíz o frijol en un balde}
\cholexample{Jiñi xch'ox woli ichijulañ ixim tyibalde.}
\exampletranslation{La niña está moviendo el maíz en el balde.}

\entry{chil}
\defsuperscript{1}
\partofspeech{vt}
\spanishtranslation{quitar}
\clarification{algo o una persona}
\cholexample{Yom ichilbeñoñ lamityal jkajpe'lel jiñi komisariado.}
\exampletranslation{El comisariado quiere quitarme una parte de mi cafetal.}

\entry{chil}
\defsuperscript{2}
\partofspeech{s}
\spanishtranslation{grillo}
\clarification{insecto}

\entry{chilikña}
\partofspeech{adv}
\nontranslationdef{El sonido que da una cosa sobre lámina; p. ej.:}
\cholexample{Chilikña woli ijubel tyilel ja'al tyi lámiña.}
\exampletranslation{La lluvia suena al caer sobre la lámina.}

\entry{chilil}
\relevantdialect{Sab.}
\partofspeech{adv}
\spanishtranslation{lo más posible}
\cholexample{Kom kpätyba'tyo chilil.}
\exampletranslation{Quiero hacer lo más posible.}

\entry{chim}
\partofspeech{s}
\spanishtranslation{red}

\entry{chimk'ubul}
\partofspeech{s}
\spanishtranslation{nido de la zacua}

\entry{chimo' chäy}
\partofspeech{s}
\spanishtranslation{red}
\clarification{para pescar}

\entry{chiñcho'ak'}
\relevantdialect{Sab.}
\partofspeech{s}
\spanishtranslation{puente de hamaca}

\entry{chiñiño}
\partofspeech{s esp}
\spanishtranslation{coyol}

\entry{chiñtyok'}
\partofspeech{s}
\spanishtranslation{árbol}
\clarification{de madera maciza, medio colorada}
\culturalinformation{Información cultural: Se utiliza para horcones y para la pechera de despulpadores de café.}

\entry{chip}
\partofspeech{vt}
\spanishtranslation{zafar}
\cholexample{Iña' mi ichipbeñ iwex iyalobil.}
\exampletranslation{Su mamá zafa el calzón de su hijo.}

\entry{chikib}
\partofspeech{s}
\spanishtranslation{canasta}

\entry{chikiñ}
\partofspeech{s}
\spanishtranslation{oreja}

\entry{*chikiñ i majts}
\spanishtranslation{el exceso de enaguas que sirve para llevar cosas}

\entry{chikityíñ}
\partofspeech{s esp}
\spanishtranslation{tipo de insecto}
\clarification{abunda en el tiempo de las rozaduras}

\entry{chikix}
\partofspeech{s}
\spanishtranslation{maraca}

\entry{chikixchañ}
\partofspeech{s}
\onedefinition{1}
\spanishtranslation{tipo de víbora venenosa}
\onedefinition{2}
\spanishtranslation{semilla de un tipo de planta}

\entry{chityam}
\partofspeech{s}
\spanishtranslation{cerdo}

\entry{chityam tye'}
\spanishtranslation{tipo de árbol}
\clarification{el olor de su madera es como el olor de la carne de jabalí; la madera es maciza y medio morada.}

\entry{chi'}
\partofspeech{s}
\spanishtranslation{nanche}
\clarification{árbol alto que da un fruto comestible de color amarillo o verde}

\entry{chobal}
\partofspeech{s}
\spanishtranslation{rozadura}
\cholexample{Iyorajlelix chobal che' tyi marzo.}
\exampletranslation{Marzo es el mes para hacer la rozadura.}

\entry{*chobälel}
\partofspeech{s}
\spanishtranslation{rozadura}
\cholexample{Mi iñuk'añ ichobälel che' jujump'ejl k'iñ mi imajlel tyi chobal.}
\exampletranslation{La rozadura se agranda cuando se va a rozar todos los días.}

\entry{chobejtyañ}
\partofspeech{vt}
\spanishtranslation{ganar}
\cholexample{Tsa' majli chobejtyañ tyak'iñ.}
\exampletranslation{Fue a ganar dinero.}

\entry{chobil}
\partofspeech{adj}
\spanishtranslation{pelado}
\clarification{frijol, plátano}
\cholexample{Jiñi ja'as wersa chobil ipaty.}
\exampletranslation{Es necesario que el plátano sea pelado.}

\entry{chok}
\partofspeech{vt}
\spanishtranslation{tirar}
\cholexample{Tsa' majli ichok tyi ja' jiñi chämeñ ts'i'.}
\exampletranslation{Se fue al río a tirar el perro muerto.}

\entry{chok majlel}
\spanishtranslation{enviar allá}

\entry{chok tyilel}
\spanishtranslation{enviar acá}
\cholexample{Mi kaj ichok tyilel juñtyikil wiñik.}
\exampletranslation{Enviará acá a un hombre.}

\entry{*choj}
\partofspeech{s}
\spanishtranslation{mejilla}

\entry{-chojk'}
\nontranslationdef{Sufijo numeral para contar golpes de hacha; p. ej.:}
\cholexample{juñchojk'}
\secondpartofspeech{adj}
\exampletranslation{un golpe de hacha;}
\cholexample{cha'chojk'}
\secondpartofspeech{adj}
\exampletranslation{dos golpes de hacha.}

\entry{chojlaw}
\partofspeech{adv}
\nontranslationdef{Se relaciona con el ruido que produce un chorrito de agua.}

\entry{chojol}
\partofspeech{adj}
\spanishtranslation{flojo}
\cholexample{Chojol tsa' käle iyok mesa.}
\exampletranslation{Quedó floja la pata de la mesa.}

\entry{chojom}
\partofspeech{s}
\spanishtranslation{pedazo de machete que se usa para desgranar maíz}

\entry{-chojp}
\nontranslationdef{Sufijo numeral para contar gajos con hojas.}

\entry{*chojk'ib}
\partofspeech{s}
\spanishtranslation{hoyuelo, pocito}
\cholexample{Añ ichojk'ib jiñi xch'ok.}
\exampletranslation{La niña tiene hoyuelos en la cara.}

\entry{*choj'ity}
\partofspeech{s}
\spanishtranslation{nalga}
\dialectvariant{Tila}
\dialectword{kolo'ity}
\dialectvariant{Sab.}
\dialectword{ñuchil}

\entry{cholel}
\partofspeech{s}
\spanishtranslation{milpa}
\secondaryentry{ichol}
\secondtranslation{su milpa}

\entry{cholokña}
\partofspeech{adv}
\nontranslationdef{Se relaciona con el ruido que produce un chorrito de agua; p. ej.:}
\cholexample{Cholokña mi ijubel jiñi ja' ya' tyi' tyi' bij.}
\exampletranslation{Se oye el ruido de un chorrito de agua que cae en la orilla del camino.}

\entry{choloñ}
\partofspeech{vt}
\spanishtranslation{rozar}

\entry{chombil}
\partofspeech{adj}
\spanishtranslation{vendido}
\cholexample{Maxtyo añik chombil jiñi kajpe'.}
\exampletranslation{El café todavía no está vendido.}

\entry{choñ}
\partofspeech{vt}
\spanishtranslation{vender}

\entry{choñkol}
\relevantdialect{Tila}
\partofspeech{part}
\nontranslationdef{Palabra que indica el aspecto continuativo; p. ej.:}
\cholexample{Choñkol tyi tyroñel jiñi wiñik.}
\exampletranslation{El hombre está trabajando.}
\alsosee{woli}

\entry{choñtyi'}
\relevantdialect{Tila}
\partofspeech{s}
\spanishtranslation{chisme}
\alsosee{u'yaj}

\entry{choñoñibäl}
\partofspeech{s}
\spanishtranslation{tienda}
\dialectvariant{Tila}
\dialectword{polmal}
\secondaryentry{xchoñoñel}
\secondpartofspeech{s}
\secondtranslation{vendedor}

\entry{-chopañ}
\nontranslationdef{Sufijo que se presenta con raíces adjetivas, indica color y se refiere a flor o papel.}

\entry{chopol}
\partofspeech{adj}
\spanishtranslation{tirada}
\clarification{rama de árbol con hojas}

\entry{choxol}
\partofspeech{adj}
\spanishtranslation{abierto abajo}
\cholexample{Choxol tyi yebal jiñi tye' cha'añ k'äkäl tyi pam jiñi lum.}
\exampletranslation{Está abierto abajo del árbol porque está encaramado encima de la tierra.}

\entry{cho'}
\partofspeech{vt}
\spanishtranslation{descascarar}
\clarification{plátano, frijol, maíz}

\entry{cho'oñ}
\partofspeech{part}
\spanishtranslation{así digo yo}
\culturalinformation{Información cultural: Cuando una persona pregunta: ‘¿Qué le decimos?’, se le contesta terminando con la expresión <cho'on> ‘así digo yo’.}

\entry{*chubä'añ}
\partofspeech{s}
\spanishtranslation{posesión}
\cholexample{Weñ kabäl ichubä'añ jiñi wiñik.}
\exampletranslation{Ese hombre tiene muchas posesiones.}

\entry{chuk}
\partofspeech{vt}
\spanishtranslation{agarrar}

\entry{chukbil}
\partofspeech{adj}
\spanishtranslation{agarrado}
\clarification{en la mano}
\cholexample{Chukbil icha'añ imachity.}
\exampletranslation{Tiene el machete agarrado en la mano.}

\entry{chukchäy}
\partofspeech{s}
\spanishtranslation{pesca}
\cholexample{Koñlatyi chukchäy ya' tyi ñoj ja'.}
\exampletranslation{Vamos a la pesca en el río.}

\entry{chuk majlel}
\spanishtranslation{llevar agarrado}
\clarification{en la mano}
\cholexample{Yom mi achuk majlel chityam.}
\exampletranslation{Vete a llevar al puerco agarrado.}

\entry{chukoch}
\partofspeech{adv}
\spanishtranslation{¿por qué?}
\cholexample{¿chukoch tsa' chilbe ilum api'äl?}
\exampletranslation{¿Por qué le quitaste el terreno a tu compañero?}

\entry{chukochka}
\partofspeech{adv}
\spanishtranslation{¿por qué será?}
\cholexample{¿chukochka weñ kabäl ja'al?}
\exampletranslation{¿Por qué está lloviendo mucho?}
\alsosee{chu'ochka}

\entry{chukoñib}
\partofspeech{s}
\spanishtranslation{asa de los trastos}
\clarification{donde se agarra}

\entry{chuk tyilel}
\spanishtranslation{traer agarrado}
\cholexample{Yom mi achuk tyilel chityam.}
\exampletranslation{Vaya y traiga agarrado al puerco.}

\entry{chuch}
\partofspeech{s}
\spanishtranslation{ardilla}
\clarification{mamífero}

\entry{chuchu'}
\partofspeech{s}
\spanishtranslation{abuela}

\entry{-chujk'}
\nontranslationdef{Sufijo numeral para contar medidas de alguna cosa espesa; p. ej.:}
\cholexample{Añix juñchujk' bu'ul.}
\exampletranslation{Ya está cocido un poco de frijol.}

\entry{chujkel}
\partofspeech{vi}
\onedefinition{1}
\spanishtranslation{agarrarse}
\onedefinition{2}
\spanishtranslation{juntarse}

\entry{chujkem}
\partofspeech{adj}
\spanishtranslation{con cría}
\clarification{vaca, perra, puerca, yegua}

\entry{chujkiñäk'äl}
\partofspeech{s}
\spanishtranslation{faja}

\entry{*chujyil}
\partofspeech{s}
\spanishtranslation{placenta}

\entry{*chujyil i pich}
\spanishtranslation{vejiga}

\entry{Chulum}
\partofspeech{s}
\spanishtranslation{nombre de colonia}

\entry{chulu' ja'}
\partofspeech{s}
\spanishtranslation{chorrito de agua}

\entry{chumchokoñ}
\partofspeech{vt}
\spanishtranslation{colocar encima de}
\cholexample{Yom lakchumchokoñ oy tyi' pam xajlal.}
\exampletranslation{Debemos colocar el horcón encima de una piedra.}

\entry{chumlibäl}
\partofspeech{s}
\spanishtranslation{habitación}
\cholexample{Ya'añ ichumlib tyi eñtyäl tyi' tyi' ja'.}
\exampletranslation{Su habitación está allá abajo, a la orilla del río.}

\entry{chumtyäl}
\partofspeech{vi}
\spanishtranslation{vivir}
\cholexample{Woli tyi chumtyäl tyi' lum tsa'bä imäñä.}
\exampletranslation{Está viviendo en el terreno que compró.}
\secondaryentry{xchumtyäl}
\secondpartofspeech{s}
\secondtranslation{varios habitantes}

\entry{chumtye'}
\partofspeech{s}
\spanishtranslation{padrón}
\clarification{de la casa}
\cholexample{Che' mi imelob iyotyoty tyi pajk' wersa yom ichumtye'lel.}
\exampletranslation{Cuando hacen su casa con pared de lodo es necesario que esté el padrón.}

\entry{chumul}
\partofspeech{adj}
\onedefinition{1}
\spanishtranslation{residente}
\cholexample{Chumuloñ tyi tyumbalá.}
\exampletranslation{Soy residente de Tumbalá.}
\onedefinition{2}
\spanishtranslation{rico}
\cholexample{Weñ chumul jiñi wiñik.}
\exampletranslation{Ese hombre es muy rico.}

\entry{*chumwi' ajkum}
\spanishtranslation{la primera raíz del camote}

\entry{*chuñchuñ muty}
\onedefinition{1}
\spanishtranslation{base de la cola de las aves}
\onedefinition{2}
\spanishtranslation{varilla}
\alsosee{*k'ajk}

\entry{chup}
\partofspeech{s}
\spanishtranslation{oruga, larva}

\entry{chup jam}
\spanishtranslation{un tipo de zacate que da flor}

\entry{chuki}
\partofspeech{part}
\onedefinition{1}
\spanishtranslation{lo que}
\cholexample{Chuki jach awom mi käk'eñety.}
\exampletranslation{Te doy lo que quieras.}
\onedefinition{2}
\spanishtranslation{¿qué?}
\cholexample{¿chuki tsa' mele sajmäl?}
\exampletranslation{¿Qué hiciste hoy?}

\entry{chuk'iña}
\partofspeech{adv}
\nontranslationdef{Se relaciona con el sonido de un líquido moviéndose dentro de un envase; p. ej.:}
\cholexample{Mi mach buty'ulik jiñi ja' tyi latya, chuk'iña che'baki ora mi laj kuch majlel.}
\exampletranslation{Si la lata de agua no está llena, se moverá mucho haciendo un sonido cuando la carguemos.}

\entry{chuty}
\relevantdialect{Sab., Tila}
\partofspeech{adj}
\spanishtranslation{chico}
\cholexample{Ya' mi lakjap sa'ba'añ chuty ja'.}
\exampletranslation{Allá vamos a tomar pozol, en el arroyo chico.}
\alsosee{bik'ity}

\entry{chuybil}
\partofspeech{adj}
\spanishtranslation{envuelto}
\clarification{tamal de elote envuelto con joloche}

\entry{*chuyib}
\partofspeech{s}
\spanishtranslation{estómago}

\entry{chuyo'}
\relevantdialect{Sab.}
\partofspeech{s}
\spanishtranslation{estómago}

\entry{chuyuk'}
\partofspeech{s}
\spanishtranslation{planta que produce baya amarilla cuando está madura}

\entry{chu'}
\partofspeech{s}
\onedefinition{1}
\spanishtranslation{pecho}
\onedefinition{2}
\spanishtranslation{ubre, teta}
\secondaryentry{cha'leñ chu'}
\secondtranslation{mamar}
\secondaryentry{ichu'}
\secondtranslation{su pecho}
\secondaryentry{ichu' chajk}
\secondtranslation{fuego de rayo}
\secondaryentry{ichu' tyuñ}
\secondtranslation{piedra que parece como pecho}
\secondaryentry{ichu' wakax}
\secondtranslation{ubre}
\secondaryentry{xchu'}
\secondpartofspeech{s}
\secondtranslation{frijol grande}
\clarification{de color amarillo}

\entry{chu'äl}
\partofspeech{s}
\spanishtranslation{pecho}

\entry{chu'ochka}
\partofspeech{adv}
\spanishtranslation{¿por qué será?}
\cholexample{¿chu'ochka weñ kabäl ja'al?}
\exampletranslation{¿Por qué será que llueve mucho?}
\alsosee{chukochka}

\entry{chu'tyuñ}
\partofspeech{s}
\spanishtranslation{estalagmita}
\clarification{lit.: la teta de piedra}

\entry{chu'umtye'}
\partofspeech{s}
\spanishtranslation{árbol}
\clarification{de madera amarilla y corriente}

\entry{chu'uñ}
\partofspeech{vt}
\spanishtranslation{mamar}
\cholexample{Jiñi aläl woli ichu'uñ ichu'.}
\exampletranslation{La criatura está mamando el pecho de su madre.}

\entry{*chu'yib i yal}
\spanishtranslation{matriz}

\alphaletter{Ch'}

\entry{ch'ak}
\partofspeech{s}
\spanishtranslation{cama}

\entry{ch'akutye'}
\partofspeech{s}
\spanishtranslation{tapesco donde se seca el frijol y el café}

\entry{ch'aj}
\partofspeech{adj}
\spanishtranslation{amargo}

\entry{ch'ajañ}
\partofspeech{s}
\spanishtranslation{capulín cimarrón}
\clarification{árbol que se usa para amarrar y para hacer mecapales}

\entry{ch'ajb}
\partofspeech{s}
\onedefinition{1}
\spanishtranslation{ayuno}
\cholexample{Yom lakcha'leñ ch'ajb che' mi lakjap ts'ak.}
\exampletranslation{El ayuno es necesario cuando tomamos medicina.}
\onedefinition{2}
\spanishtranslation{el cambio de hoja de los árboles}
\cholexample{Che' tyi yorajlel tsäñal mi icha'leñ ch'ajb jiñi tye'.}
\exampletranslation{En el invierno los árboles cambian de hojas.}

\entry{ch'ajbañ}
\partofspeech{vt}
\spanishtranslation{ayunar}

\entry{ch'ajñal}
\partofspeech{s}
\spanishtranslation{cordón de piel o de mecate}

\entry{ch'ajtyañ}
\partofspeech{vt}
\spanishtranslation{medio asar}
\cholexample{Mi isäl alambre tyi'bäk'tyal muty cha'añ mi ich'ajtyañ.}
\exampletranslation{Se ensarta un alambre en la carne de pollo para medio asarlo.}

\entry{ch'ajuk'}
\partofspeech{s}
\spanishtranslation{hierba mora}

\entry{ch'aplom}
\partofspeech{adj}
\spanishtranslation{rebelde}
\cholexample{Ch'aplom jiñi alob cha'añ ma'añik mi iweñtyoj'esañ ityaty.}
\exampletranslation{El niño es rebelde porque su padre casi no lo corrige.}

\entry{ch'a'al}
\partofspeech{adv}
\spanishtranslation{boca arriba}
\clarification{acostado}
\cholexample{Ch'a'al tsa' ñole jiñi ch'ityoñ.}
\exampletranslation{El niño se acostó boca arriba.}

\entry{ch'a'chokoñ}
\partofspeech{vt}
\spanishtranslation{colocar boca arriba}
\clarification{persona, animal}

\entry{ch'a'ch'a'ña}
\partofspeech{adv}
\spanishtranslation{manera de caminar echando la cabeza hacia atrás}

\entry{ch'a'tyäl}
\partofspeech{adv}
\spanishtranslation{acostarse boca arriba}

\entry{ch'äb}
\partofspeech{s}
\spanishtranslation{clase de bejuco grande}

\entry{ch'äbix}
\partofspeech{imp}
\spanishtranslation{¡cállate!}

\entry{ch'äbtyesañ}
\partofspeech{vt}
\spanishtranslation{consolar}
\clarification{a un niño}

\entry{ch'äk}
\defsuperscript{1}
\partofspeech{vt}
\spanishtranslation{maldecir}
\cholexample{Jiñi xwujty tsa' ipejka jiñi xiba cha'añ mi ich'äkoñ.}
\exampletranslation{El curandero habló con el diablo para maldecirme.}

\entry{ch'äk}
\defsuperscript{2}
\partofspeech{s}
\spanishtranslation{pulga}

\entry{ch'äkojel}
\partofspeech{vi}
\spanishtranslation{maldecir}
\cholexample{Woli tyi ch'äkojel jiñi xwujty cha'añ yom ityik'lañ jiñi año'bä tyi kotyoty.}
\exampletranslation{El curandero está maldiciendo porque quiere perjudicar a mi familia.}

\entry{ch'äkoñel}
\partofspeech{s}
\spanishtranslation{hechicería}
\cholexample{Yujil ch'äkoñel jiñi x'ixik.}
\exampletranslation{Esa mujer sabe hacer hechicería.}

\entry{ch'äktyäl}
\partofspeech{s}
\spanishtranslation{loma u orilla de cerro que sirve como mirador}

\entry{ch'äch'}
\partofspeech{vt}
\spanishtranslation{absorber}
\clarification{algodón, trapo, tierra}
\cholexample{Ora jach mi ich'äch' ja' jiñi lum.}
\exampletranslation{La tierra absorbe el agua en seguida.}

\entry{ch'äch'ak'}
\partofspeech{s}
\spanishtranslation{yagual}
\clarification{tipo de canasta que se usa para guardar pozol}

\entry{ch'äch'äñtyik}
\partofspeech{adj}
\spanishtranslation{agujerada}
\clarification{lámina}

\entry{ch'äjlil}
\partofspeech{s}
\spanishtranslation{adorno}

\entry{ch'äjy}
\partofspeech{adj}
\onedefinition{1}
\spanishtranslation{durable}
\cholexample{Tsa' aweñ jal'a kbujk kome ch'äjy.}
\exampletranslation{Mi camisa duró mucho porque es de tela durable.}
\onedefinition{2}
\spanishtranslation{macizo}
\cholexample{Tsäts tyi p'elol jiñi tye' komo weñ ch'äjy.}
\exampletranslation{Ese palo es muy duro de aserrar porque es macizo.}

\entry{ch'äjy'añ}
\partofspeech{vi}
\spanishtranslation{endurecerse}
\cholexample{Mach mejlix jk'ux jiñi waj kome tsa'ix ch'äjy'a.}
\exampletranslation{No puedo comer esta tortilla porque ya se endureció.}

\entry{ch'äl}
\partofspeech{vt}
\spanishtranslation{adornar}

\entry{ch'älch'älña}
\partofspeech{adv}
\spanishtranslation{desorientada}
\cholexample{Ityileläch jiñi wiñiki ch'älchälña mi imel.}
\exampletranslation{Ese hombre de por sí actúa de manera desorientada.}

\entry{ch'äloñib}
\partofspeech{s}
\spanishtranslation{adorno}

\entry{ch'äm}
\partofspeech{vt}
\spanishtranslation{tomar, llevar}
\cholexample{Ch'ämä ajuloñib cha'añ mi ajul jiñi uch.}
\exampletranslation{Lleva tu escopeta para que le tires al tlacuache.}
\secondaryentry{ch'äm tyi majañ}
\secondtranslation{recibir prestado}
\secondaryentry{ch'äm majlel}
\secondtranslation{llevar}
\clarification{en la mano}
\secondaryentry{ch'äm tyilel}
\secondtranslation{traer}
\clarification{en la mano}

\entry{ch'ämbeñ isujm}
\spanishtranslation{comprender}
\cholexample{Ch'ämbeñ isujm kome tyik'ol woli asubeñtyel.}
\exampletranslation{Comprende, porque es un consejo que él te está dando.}

\entry{ch'ämja'}
\partofspeech{s}
\spanishtranslation{bautismo}
\clarification{lit.: lleva agua}
\cholexample{Maxtyo komik mi iyäk'eñtyel ch'ämja' kalobil.}
\exampletranslation{Todavía no quiero que sea bautizado (lit.: que se le dé el bautismo) mi hijo.}

\entry{ch'ämoñel}
\partofspeech{s}
\spanishtranslation{enfermedad mortal}
\cholexample{Ch'ämoñelix jiñi k'amäjel wolibä iyubiñ.}
\exampletranslation{La enfermedad que tiene es mortal.}

\entry{ch'ämpäk'}
\relevantdialect{Sab.}
\partofspeech{s}
\spanishtranslation{cerdito}
\spanishtranslation{puerquito}
\clarification{ave}
\alsosee{k'oyem}

\entry{ch'äñch'äña}
\partofspeech{adv}
\spanishtranslation{impensadamente}
\clarification{hablar}
\cholexample{Ch'äñch'äña mi icha'leñ ty'añ jiñi wiñik.}
\exampletranslation{Ese hombre habla impensadamente lo que dice.}

\entry{ch'äx}
\partofspeech{vt}
\onedefinition{1}
\spanishtranslation{hervir}
\cholexample{Yom mi aweñ ch'äx jiñi ja' cha'añ mi ichämel ichäñil.}
\exampletranslation{Debes hervir el agua para que se mueran los microbios.}
\onedefinition{2}
\spanishtranslation{cocer}
\cholexample{Jal yom mi ach'äx bu'ul.}
\exampletranslation{Hay que cocer el frijol mucho tiempo.}

\entry{ch'äxbil}
\partofspeech{adj}
\spanishtranslation{hervido, cocido}

\entry{ch'äxoñib}
\partofspeech{s}
\spanishtranslation{olla}

\entry{ch'äyäkña}
\defsuperscript{1}
\partofspeech{adv}
\spanishtranslation{pegajosamente}
\cholexample{Ch'äyäkña mi iyajlel iyetsel tye'.}
\exampletranslation{La savia del árbol gotea pegajosamente.}

\entry{ch'äyäkña}
\defsuperscript{2}
\partofspeech{adj}
\onedefinition{1}
\spanishtranslation{babeando}
\cholexample{Bi'ityikax jiñi ch'ityoñ kome ch'äyäkñajax iya'lel iyej.}
\exampletranslation{Ese chamaco da asco porque está babeando.}
\onedefinition{2}
\spanishtranslation{pegajoso}
\cholexample{Ch'äyäkña isits' wakax.}
\exampletranslation{La saliva del ganado es pegajosa.}

\entry{ch'ä'ch'äñtyik}
\partofspeech{adj}
\spanishtranslation{picado, agujerado}
\clarification{tela, lámina}
\cholexample{Ch'ä'ch'äñtyik jiñi lámiña.}
\exampletranslation{Esa lámina está agujerada.}

\entry{ch'ä'tyesañ}
\relevantdialect{Sab.}
\partofspeech{vt}
\spanishtranslation{callar}

\entry{ch'ebel}
\partofspeech{adj}
\spanishtranslation{tirado boca arriba}
\cholexample{Ch'ebel tsa' käle tyi kale jiñi xyäk'äjel.}
\exampletranslation{El borracho se quedó tirado boca arriba en la calle.}

\entry{ch'ebñolol}
\partofspeech{adj}
\spanishtranslation{acostado boca arriba}

\entry{ch'ebtyots'ol}
\partofspeech{adj}
\spanishtranslation{acostado con las piernas estiradas}

\entry{ch'ejew}
\partofspeech{s}
\spanishtranslation{cajete hecho de barro}

\entry{ch'ejl}
\partofspeech{adj}
\onedefinition{1}
\spanishtranslation{fuerte}
\clarification{persona o animal}
\cholexample{Cha'añ jiñi tsätsbä e'tyel yom ch'ejlbä wiñik.}
\exampletranslation{Para un trabajo duro se necesita un hombre fuerte.}
\onedefinition{2}
\spanishtranslation{valiente}
\cholexample{Yom ch'ejl lakpusik'al che' mi lakñumelba' chumul joñtyolbä wiñikob.}
\exampletranslation{Debemos ser valientes al pasar por la aldea de los hombres malos.}
\onedefinition{3}
\spanishtranslation{bravo}
\cholexample{Yom tsajalety che' mi ak'ächtyañ jiñi macho kome ch'ejl.}
\exampletranslation{Debes tener cuidado al montar el macho, porque es bravo.}

\entry{ch'ejlaw}
\partofspeech{adv}
\nontranslationdef{Se relaciona con el sonido que hacen las hojas y palitos que levanta el viento y caen sobre el techo.}

\entry{*ch'ejñal tye'}
\relevantdialect{Sab.}
\spanishtranslation{hueco de un árbol}
\alsosee{*ch'eñal}

\entry{ch'eñ}
\partofspeech{s}
\onedefinition{1}
\spanishtranslation{cueva}
\cholexample{Añ ch'eñ tyi joloñelba' tsa' ochiyob lakñojtye'elob ich'ujutyesañob idios tyak.}
\exampletranslation{Hay una cueva en Joloñel, donde nuestros antepasados entraban a adorar a sus dioses.}
\onedefinition{2}
\spanishtranslation{hueco hondo}
\cholexample{Jiñi ch'eñ tyak tyi tye'el jiñäch iyotyoty tyak jiñi tye'lal.}
\exampletranslation{El tepezcuintle vive en los huecos hondos que hay en el bosque.}
\secondaryentry{ch'eñbä ja'}
\secondtranslation{agua que sale de una cueva, nombre de una vertiente}

\entry{*ch'eñal}
\defsuperscript{1}
\partofspeech{s}
\spanishtranslation{hueco de un árbol}
\dialectvariant{Sab.}
\dialectword{*ch'ejñal tye'}

\entry{*ch'eñal}
\defsuperscript{2}
\partofspeech{s}
\spanishtranslation{sepulcro}
\dialectvariant{Sab.}
\dialectword{*ch'eñäl}

\entry{ch'eñäl}
\relevantdialect{Sab.}
\partofspeech{s}
\spanishtranslation{sepulcro}

\entry{ch'ep}
\defsuperscript{1}
\partofspeech{s}
\spanishtranslation{pajarito}
\clarification{pequeño y negro; hace su nido en los árboles}

\entry{ch'ep}
\defsuperscript{2}
\partofspeech{vt}
\spanishtranslation{abrir}
\clarification{bolsa, costal, morral, ojos}
\cholexample{Woli ich'eñ ityi' koxtyal.}
\exampletranslation{Está abriendo la boca del costal.}

\entry{ch'ekek}
\partofspeech{s}
\spanishtranslation{pájaro negro}

\entry{ch'ewe}
\partofspeech{vt}
\spanishtranslation{abrir}
\clarification{bolsa, costal, morral, ojos}

\entry{ch'e'}
\partofspeech{adj}
\spanishtranslation{fuerte}
\clarification{ruido}
\cholexample{Weñ ch'e' iyuk'el jiñi pájaro.}
\exampletranslation{El ruido del pájaro es muy fuerte.}

\entry{ch'e'ekña}
\partofspeech{adv}
\spanishtranslation{cacareando}
\cholexample{Ch'e'ekña jiñi muty che'baki ora mi'bäk'tyesañ jiñi wax.}
\exampletranslation{Las gallinas andan cacareando porque el gato montés las espantó.}

\entry{ch'e'lel}
\partofspeech{s}
\spanishtranslation{llanto}
\cholexample{Kabäljax mi icha'leñ ch'e'lel jiñi aläl.}
\exampletranslation{El niño llora mucho (lit.: el llanto es mucho).}

\entry{ch'ib}
\partofspeech{s}
\spanishtranslation{una palma que da fruta comestible}

\entry{*ch'ibilel matye'el}
\spanishtranslation{palma del monte}

\entry{ch'ibol}
\partofspeech{s}
\spanishtranslation{palmar}

\entry{ch'ik}
\partofspeech{vt}
\spanishtranslation{meter}
\clarification{instrumento pequeño en un agujero}
\cholexample{Jiñi alob mi ich'ik iyal ik'äb tyi chikiñ.}
\exampletranslation{Ese niño se mete el dedo en el oído.}

\entry{ch'ich'}
\partofspeech{s}
\spanishtranslation{sangre}
\cholexample{Yomix ts'ak kchich kome jilemix ich'ich'el.}
\exampletranslation{Mi hermana mayor necesita medicina porque su sangre está agotada.}

\entry{ch'ich'tya'}
\partofspeech{s}
\spanishtranslation{disentería}
\clarification{lit.: excremento con sangre}

\entry{ch'ij}
\partofspeech{vt}
\spanishtranslation{clavar}
\dialectvariant{Sab., Tila}
\dialectword{\textsuperscript{2}baj}

\entry{ch'ijbol}
\partofspeech{s}
\spanishtranslation{lugar donde hay mucho bambú}

\entry{ch'ijch'um}
\partofspeech{s}
\spanishtranslation{chayote}

\entry{*ch'ijikñiyel}
\partofspeech{s}
\onedefinition{1}
\spanishtranslation{soledad}
\cholexample{Añ ich'ijikñiyel jiñi x'ixik cha'añ tsa' chämi iñoxi'al.}
\exampletranslation{Esa mujer vive en soledad porque se murió su esposo.}
\onedefinition{2}
\spanishtranslation{tristeza}
\cholexample{Ma'añik mi ilajmel ich'ijikñiyel ipusik'al cha'añ tsa' majli iyalobil.}
\exampletranslation{Tiene mucha tristeza, porque se fue su hijo.}

\entry{ch'ijiyem}
\partofspeech{adj}
\onedefinition{1}
\spanishtranslation{triste}
\cholexample{Ch'jijiyem ipusik'al jiñi x'ixik cha'añ u'yaj ty'añ tyi'tyojlel.}
\exampletranslation{Esa mujer está triste porque están hablando mal de ella.}
\onedefinition{2}
\spanishtranslation{quieto}
\cholexample{Ch'ijiyem jiñi otyoty cha'añ ma'añik iyum.}
\exampletranslation{Esa casa está muy quieta, porque no tiene dueño.}
\dialectvariant{Sab.}
\dialectword{k'uxtyayem}

\entry{ch'ijoñib}
\partofspeech{s}
\onedefinition{1}
\spanishtranslation{pedernal}
\clarification{piedra que se usa para afilar machetes y para hacer chispas}
\onedefinition{2}
\spanishtranslation{martillo}

\entry{ch'ijo'lawux}
\partofspeech{s}
\spanishtranslation{martillo}
\clarification{lit.: con lo que se clava}

\entry{ch'ijty}
\partofspeech{s}
\spanishtranslation{tipo de árbol de madera maciza}

\entry{ch'il}
\partofspeech{vt}
\onedefinition{1}
\spanishtranslation{freír}
\clarification{con grasa}
\onedefinition{2}
\spanishtranslation{tostar}
\clarification{café o maíz para hacer pinole}

\entry{ch'ilim}
\partofspeech{s}
\spanishtranslation{pinole}

\entry{Ch'iliñtye'el}
\partofspeech{s}
\nontranslationdef{Arboleda de árboles de madera blanca y dura y de hojas chicas.}
\clarification{colonia}

\entry{*ch'il'aty}
\partofspeech{s}
\spanishtranslation{costilla de persona o animal}

\entry{ch'iñch'iña}
\partofspeech{adj}
\spanishtranslation{tranquilo}
\cholexample{Ch'iñch'iña pañimil che' añoñlatyi lakbajñelil tyi lakotyoty.}
\exampletranslation{Está tranquilo cuando estamos solos en la casa.}

\entry{ch'iñlaw}
\partofspeech{adj}
\onedefinition{1}
\nontranslationdef{Se relaciona con el sonido que se produce al dar un manotazo; p. ej.:}
\cholexample{Ch'iñlaw pañimil che' mi iletsel tyi tye' muty.}
\exampletranslation{Ya casi está oscuro cuando se suben las gallinas a su tapezco.}
\onedefinition{2}
\partofspeech{adv}
\spanishtranslation{sonido de un manotazo}

\entry{ch'ip}
\partofspeech{vt}
\spanishtranslation{abrir}
\clarification{un poco como cartón u ojos}
\cholexample{Woli ich'ip kartyoñ cha'añ mi ik'el chuki añ tyi mal.}
\exampletranslation{Está abriendo un poco el cartón para ver lo que contiene.}

\entry{ch'ikijl}
\partofspeech{s}
\spanishtranslation{chaquiste}

\entry{ch'ityikña}
\partofspeech{adj}
\spanishtranslation{triste}
\clarification{por estar ausente alguien}
\cholexample{Ch'ityikña tsa' majli lakijts'iñ kome ma'añix tsikil tyi lakotyoty.}
\exampletranslation{Estamos tristes porque nuestro hermanito está ausente de la casa.}

\entry{ch'ityoñ}
\partofspeech{s}
\spanishtranslation{muchacho}
\clarification{de los siete hasta los quince años}
\dialectvariant{Sab.}
\dialectword{alo'}

\entry{ch'ityoñ muty}
\partofspeech{s}
\spanishtranslation{pollo chico, pollito}
\dialectvariant{Sab.}
\dialectword{bi'tyi muty}

\entry{ch'ix}
\defsuperscript{1}
\partofspeech{adv}
\nontranslationdef{Se relaciona con un objeto recto; p. ej.:}
\cholexample{Ya' ch'ix kotyol jiñi me'.}
\exampletranslation{Allí está el venado parado recto.}

\entry{ch'ix}
\defsuperscript{2}
\partofspeech{s}
\spanishtranslation{espina}
\secondaryentry{ich'ixal}
\secondtranslation{su espina}

\entry{ch'ix ajk}
\spanishtranslation{chiquiguao}
\spanishtranslation{tortuga cocodrilo}
\clarification{reptil}

\entry{ch'ixluktyik}
\partofspeech{adj}
\spanishtranslation{corte alto}
\clarification{vegetación}
\cholexample{Ch'ixluktyik tsa' käle tye'baki ora tsa' puli cholel.}
\exampletranslation{Cuando se quemó la milpa, sólo quedaron los palos altos que no fueron cortados.}

\entry{ch'ixol}
\partofspeech{s}
\spanishtranslation{lugar de muchas espinas}

\entry{ch'ixpaty}
\partofspeech{s}
\spanishtranslation{columna vertebral}

\entry{ch'ix tyak'iñ}
\spanishtranslation{alambrado}

\entry{ch'ixtyäl}
\partofspeech{adv}
\spanishtranslation{así de alto}
\cholexample{Che' ch'ixtyäl ichañlel ch'ityoñ.}
\exampletranslation{Así de alto es el niño.}

\entry{ch'ix uch, ch'ix'uch}
\spanishtranslation{puerco espín}
\clarification{mamífero}

\entry{ch'ix wiñik}
\relevantdialect{Tila}
\spanishtranslation{fantasma, un espíritu malo}
\culturalinformation{Información cultural: Se dice que aparece como hombre malo con espinas, pero es la creación de Dios. Se cree que al fin del mundo se va a comer a todos los malos.}

\entry{ch'i'omal}
\partofspeech{s}
\spanishtranslation{musgo verde}
\clarification{se forma sobre las piedras que están en los ríos; también se encuentra en los árboles}
\secondaryentry{ich'i'ojñal}
\secondtranslation{su musgo}

\entry{ch'ok aläl}
\partofspeech{s}
\spanishtranslation{criatura}

\entry{ch'ok bu'ul}
\spanishtranslation{frijol tierno}

\entry{ch'ok tyo}
\spanishtranslation{verde todavía}
\cholexample{Ch'oktyo jiñi bu'ul che' tyi ñoviembre.}
\exampletranslation{En noviembre el frijol está verde todavía.}

\entry{ch'ok'añ}
\relevantdialect{Sab., Tila}
\partofspeech{vi}
\spanishtranslation{nacer}
\alsosee{k'el pañimil}

\entry{ch'och'}
\partofspeech{s}
\spanishtranslation{garganta}
\clarification{se refiere a toda la garganta}

\entry{ch'och'oktyesañ}
\partofspeech{vt}
\spanishtranslation{achicar}
\cholexample{Yom mi ach'och'oktyesañ jiñi mesa.}
\exampletranslation{Hay que achicar la mesa.}

\entry{ch'och'okiyel}
\partofspeech{vi}
\spanishtranslation{hacerse más pequeño}
\cholexample{Mux ich'och'okiyel laklum kome jiñi komisariado tsa'ix ilamityal chili.}
\exampletranslation{Nuestro terreno se hizo más pequeño, porque el comisariado nos quitó una parte.}

\entry{ch'oj}
\onedefinition{1}
\partofspeech{vt}
\spanishtranslation{picar}
\cholexample{Mi ich'ojoñlajiñi ña' muty che' xpäklojm.}
\exampletranslation{La gallina nos pica cuando está clueca.}
\onedefinition{2}
\partofspeech{vt}
\spanishtranslation{morder}
\cholexample{Tsa'ix ich'ojoyoñ jiñi k'äñcho.}
\exampletranslation{Ya me mordió la víbora.}
\onedefinition{3}
\partofspeech{adj}
\spanishtranslation{golpeado}
\cholexample{Mi lakch'oj jats' tye' yik'oty xajlel}
\exampletranslation{Golpeamos la madera (lit.: la madera está golpeada) con una piedra}

\entry{ch'ojch'oj}
\partofspeech{adv}
\spanishtranslation{golpear ligeramente}
\cholexample{Mi ich'ojch'oj jats' tye' jiñi karpiñtyero.}
\exampletranslation{El carpintero golpea ligeramente el árbol.}

\entry{ch'ojch'ojña}
\partofspeech{adv}
\nontranslationdef{Se relaciona con el ruido que produce la madera cuando se corta con hacha, o el de los zapatos al caminar en la calle; p. ej.:}
\cholexample{Ch'ojch'ojña woli itsep tye'.}
\exampletranslation{Está cortando un palo haciendo ruido al golpear.}

\entry{ch'ojch'oñ}
\partofspeech{vt}
\spanishtranslation{picotear}
\cholexample{Jiñi alä muty mi ich'ojch'oñ motso'.}
\exampletranslation{El pollito picotea gusanos.}

\entry{ch'ojiyel}
\partofspeech{vi}
\spanishtranslation{levantarse}
\cholexample{Ik'tyo pañimil mi lakch'ojiyel.}
\exampletranslation{Nos levantamos cuando todavía está obscuro.}

\entry{ch'olokña}
\partofspeech{adv}
\nontranslationdef{Se relaciona con el ruido que produce la lluvia al caer sobre una casa; p. ej.:}
\cholexample{Ch'olokña mi iyajlel jiñi ja'al tyi' pam lakotyoty.}
\exampletranslation{La lluvia hace ruido al caer sobre el techo de nuestra casa.}

\entry{ch'omol}
\partofspeech{adj}
\spanishtranslation{hueco, hundido}
\cholexample{Ch'omol jiñi lum tyi' yojlil jiñi kajpe'lel.}
\exampletranslation{El terreno está hundido en medio de mi cafetal.}

\entry{*ch'okel}
\partofspeech{s}
\spanishtranslation{fruta verde del café}
\cholexample{Pejtyelelix yik'oty ich'okel mi kaj ktyuk' jkajpe'.}
\exampletranslation{Voy a cortar todo lo que queda de mi cosecha junto con la fruta verde.}

\entry{ch'okiyem}
\partofspeech{adj}
\spanishtranslation{retoñado}
\clarification{nuevamente}
\cholexample{Cha' ch'okiyemix iyopol kajpe'.}
\exampletranslation{Ya retoñó (lit.: está retoñado) el café.}

\entry{ch'ox}
\partofspeech{s}
\spanishtranslation{lombrices}

\entry{ch'oyol}
\partofspeech{adj}
\spanishtranslation{es de, viene de}
\cholexample{Ch'oyol jiñi wiñik tyi tyila.}
\exampletranslation{Ese hombre es de Tila.}

\entry{ch'o'ch'ok}
\partofspeech{adj}
\spanishtranslation{chiquito}

\entry{ch'ub}
\partofspeech{vt}
\spanishtranslation{agujerar}

\entry{ch'ubul}
\partofspeech{adj}
\spanishtranslation{hoyado}
\clarification{chico}
\spanishtranslation{agujerado}
\clarification{madera, balde, cuero, lámina}

\entry{ch'uch'tye'}
\partofspeech{s}
\spanishtranslation{tipo de árbol grande}
\clarification{de madera suave y blanca, como la balsa}

\entry{ch'ujbiñ}
\relevantdialect{Tila}
\partofspeech{vt}
\onedefinition{1}
\spanishtranslation{obedecer}
\cholexample{Jiñi ch'ityoñ ma'añik mi ich'ujbibeñ ity'añ ityaty.}
\exampletranslation{El joven no obedece a su padre.}
\onedefinition{2}
\spanishtranslation{tomar en cuenta}
\cholexample{Yom lakch'ujbibeñ ity'añ jiñi komisariado}
\exampletranslation{Debemos tomar en cuenta lo que dice el comisariado.}
\onedefinition{3}
\spanishtranslation{creer}
\cholexample{Añix jiñi mu'ixbä ich'ujbiñob ity'añ dios.}
\exampletranslation{Ya hay quienes han creído la Palabra de Dios.}

\entry{ch'ujch'uj}
\partofspeech{adv}
\spanishtranslation{insistentemente}
\clarification{mirar}
\cholexample{Woli ich'ujch'uj k'el jiñi ch'ityoñ cha'añ woli imulañ dulke.}
\exampletranslation{El niño mira insistentemente porque quiere el dulce.}

\entry{*ch'ujlel}
\partofspeech{s}
\onedefinition{1}
\spanishtranslation{espíritu}
\cholexample{Iwäy jiñi ktatuch lajal ich'ujlel yik'oty juñkojtybajlum, mi iyälob.}
\exampletranslation{Dicen que el compañero de mi abuelo es un espíritu que corresponde al espíritu de un jaguar.}
\onedefinition{2}
\spanishtranslation{pulso}
\cholexample{Jiñi xwujty mi ityälbeñ ich'ujlel wiñik cha'añ mi iña'tyäbeñ chuki tyi kaj tsa' ityechbe ik'amäjel.}
\exampletranslation{El curandero toma el pulso del hombre para saber cuál es la causa de su enfermedad.}
\secondaryentry{*ch'ujlel iwuty}
\secondtranslation{pupila}

\entry{ch'ujleläl}
\partofspeech{s}
\spanishtranslation{cadáver}

\entry{ch'ujm}
\partofspeech{s}
\spanishtranslation{calabaza}

\entry{ch'ujtye'}
\partofspeech{s}
\spanishtranslation{cedro}
\clarification{lit.: árbol santo}
\culturalinformation{Información cultural: Se utiliza para hacer los palitos de los tambores que se usan en las fiestas.}
\variation{chäktye'}

\entry{ch'ujukña}
\partofspeech{adv}
\spanishtranslation{lentamente}
\cholexample{Ch'ujukña mi imajlel jiñi xch'ok.}
\exampletranslation{Esa muchacha anda lentamente.}

\entry{ch'ujul}
\partofspeech{adj}
\onedefinition{1}
\spanishtranslation{permanente}
\cholexample{Ch'ujul añ jiñi diostye' kome ma'añik mi iñijkañ ibä.}
\exampletranslation{El ídolo está permanente en un solo lugar porque no se mueve.}
\onedefinition{2}
\spanishtranslation{santo}
\cholexample{Woli iyäl cha'añ ch'ujul jiñi k'iñ kome mach yujilik p'ajoñel.}
\exampletranslation{Dicen que el sol es santo porque no sabe maldecir.}

\entry{ch'ujuña'}
\partofspeech{s}
\spanishtranslation{luna}
\clarification{lit.: madre santa}

\entry{ch'ujuña'iñ}
\partofspeech{vt}
\spanishtranslation{adorar}
\clarification{como a la madre santa}
\cholexample{Kabäl wiñikob mi ich'ujuña'iñob jiñi uw.}
\exampletranslation{Muchos hombres adoran a la luna.}

\entry{*ch'ujutyaty}
\partofspeech{s}
\onedefinition{1}
\spanishtranslation{Dios}
\clarification{lit.: Padre Santo}
\cholexample{Lakch'ujutyaty tsa' imele pañimil.}
\exampletranslation{Dios hizo el mundo.}
\onedefinition{2}
\spanishtranslation{sol}
\cholexample{K'uxatyax lakch'ujutyaty.}
\exampletranslation{El sol está muy caliente.}
\onedefinition{3}
\spanishtranslation{imagen de Cristo}
\cholexample{Jok'ol tyi bik' iyejtyal lakch'ujutyaty.}
\exampletranslation{Tiene colgada en el cuello una imagen de Cristo.}
\onedefinition{4}
\spanishtranslation{sacerdote}
\cholexample{Tyal lakch'ujutyaty cha'añ mi ikäñtyesañoñlatyi resar.}
\exampletranslation{El sacerdote viene para enseñarnos a rezar.}

\entry{ch'ujutyesañ}
\partofspeech{vt}
\spanishtranslation{adorar}
\cholexample{Tyi wajali tsa' ich'ujutyesa diostye'.}
\exampletranslation{Hace mucho tiempo adoraba a los ídolos.}

\entry{ch'ujwañaj}
\relevantdialect{Tila}
\partofspeech{s}
\spanishtranslation{mayordomo en la iglesia}

\entry{ch'ujyel}
\partofspeech{vi}
\spanishtranslation{levantarse}
\clarification{una cosa pesada}
\cholexample{Woli ich'ujyel jiñi albä kuchäl.}
\exampletranslation{Esa carga pesada no se levanta fácilmente.}

\entry{ch'ujyijel}
\partofspeech{vi}
\spanishtranslation{orar}
\cholexample{Jiñi pasaro woli tyi ch'ujyijel cha'añ iyalobil.}
\exampletranslation{El exencargado ora por su hijo.}

\entry{ch'umjol}
\partofspeech{adj}
\spanishtranslation{calvo}
\clarification{lit.: cabeza de calabaza}
\alsosee{chäkpiräñ jol}

\entry{ch'um'ak'}
\partofspeech{s}
\spanishtranslation{granadilla}
\clarification{planta}

\entry{ch'upujk}
\partofspeech{s}
\spanishtranslation{higuerilla}
\clarification{planta}

\entry{ch'uy}
\partofspeech{vt}
\spanishtranslation{levantar}
\clarification{una cosa pesada}
\cholexample{Cha'tyikil yom cha'añ mi lakch'uy letsel jiñi kukujl.}
\exampletranslation{Se necesitan dos personas para levantar la viga.}

\entry{ch'uybañ}
\partofspeech{vt}
\spanishtranslation{chiflar}
\cholexample{Ity'ojoljax mi ich'uybañ jiñi k'ay.}
\exampletranslation{Él chifla muy bonito la canción.}

\entry{ch'uybil}
\partofspeech{adj}
\spanishtranslation{levantada}
\clarification{una cosa pesada}

\entry{ch'uyijel}
\relevantdialect{Tila}
\onedefinition{1}
\partofspeech{s}
\spanishtranslation{misa}
\onedefinition{2}
\partofspeech{vi}
\spanishtranslation{rezar}

\entry{*ch'uyity}
\partofspeech{s}
\spanishtranslation{ano}

\entry{ch'uyub}
\partofspeech{s}
\spanishtranslation{silbido}
\cholexample{Weñ k'am mi icha'leñ ch'uyub jiñi ch'ityoñ.}
\exampletranslation{Ese muchacho da un silbido recio.}

\alphaletter{D}

\entry{diosiñ}
\partofspeech{vt esp}
\spanishtranslation{adorar}
\cholexample{Mi idiosiñ diostye'.}
\exampletranslation{Él adora a un ídolo.}

\alphaletter{E}

\entry{ekchokoñ}
\partofspeech{vt}
\spanishtranslation{colocar}
\clarification{boca arriba}
\cholexample{Tsa' iekchoko tyi' pam mesa.}
\exampletranslation{Colocó el plato (boca arriba) encima de la mesa.}

\entry{ek'}
\defsuperscript{1}
\partofspeech{s}
\spanishtranslation{estrella}
\cholexample{Tsikiljax jiñi ek' che' ma'añik uw.}
\exampletranslation{Cuando no hay luna se ven muchas estrellas.}

\entry{ek'}
\defsuperscript{2}
\partofspeech{s}
\spanishtranslation{chaya}
\clarification{yerba comestible}
\cholexample{Mi ik'äñob jiñi ek' cha'añ kaldo.}
\exampletranslation{Se usa la chaya para el caldo.}

\entry{ek'xäñäb}
\partofspeech{s}
\spanishtranslation{constelaciones}

\entry{echem}
\partofspeech{adj}
\spanishtranslation{sobre cocido}
\cholexample{Weñ echem jiñi sa'.}
\exampletranslation{El pozol está sobre cocido.}

\entry{echmäl}
\partofspeech{vi}
\spanishtranslation{hervirse, hacerse tierno}

\entry{ejäl}
\partofspeech{s}
\spanishtranslation{boca}
\cholexample{Woli tyi jilel iyej.}
\exampletranslation{Está enfermo de la boca.}

\entry{Ejk-'ejk}
\nontranslationdef{Sufijo numeral para contar platos de comida; p. ej.:}
\cholexample{juñ'ejk}
\secondpartofspeech{adj}
\exampletranslation{un plato de comida}
\cholexample{cha'ejk}
\secondpartofspeech{adj}
\secondtranslation{dos platos de comida.}

\entry{ejk'ach}
\partofspeech{s}
\onedefinition{1}
\spanishtranslation{uña}
\onedefinition{2}
\spanishtranslation{casco}
\clarification{de caballo}
\secondaryentry{iyejk'achil}
\secondpartofspeech{s}
\secondtranslation{su uña, su casco}

\entry{ejmech}
\partofspeech{s}
\spanishtranslation{mapache}
\clarification{mamífero}

\entry{ejmel}
\partofspeech{s}
\spanishtranslation{derrumbe}
\cholexample{Kabäl tsa' ujtyi ejmel tyi ili jabil.}
\exampletranslation{Hubo muchos derrumbes este año.}

\entry{ejtyaläl}
\partofspeech{s}
\onedefinition{1}
\spanishtranslation{semejanza}
\cholexample{Lajal iyejtyal iwutybajche' ityaty.}
\exampletranslation{Hay mucha semejanza entre él y su padre.}
\onedefinition{2}
\spanishtranslation{foto}
\cholexample{Ya' tyi tyejklum mi mejlel ilok'esañ iyejtyal lakwuty.}
\exampletranslation{En el pueblo pueden sacar nuestra foto.}
\secondaryentry{iyejtyal wokol}
\secondtranslation{agüero}

\entry{-el}
\defsuperscript{1}
\nontranslationdef{Sufijo que se presenta con raíces atributivas para formar otra raíz que indica calidad o condición: p. ej.:}
\cholexample{ich'ajel}
\exampletranslation{su amargura.}

\entry{-el}
\defsuperscript{2}
\nontranslationdef{Sufijo que se presenta con raíces transitivas y neutras para formar otra raíz atributiva que indica posición: p. ej.:}
\cholexample{ts'ejel}
\exampletranslation{de un lado.}
\variation{2*-al, -äl, 1*-ol, -ul}

\entry{-el}
\defsuperscript{3}
\nontranslationdef{Sufijo que se presenta con raíces transitivas y neutras, para formar una raíz intransitiva: p. ej.:}
\cholexample{pulel}
\exampletranslation{quemarse.}

\entry{elekña}
\partofspeech{adj}
\spanishtranslation{brilloso y liso}
\cholexample{Elekña jiñi mesa.}
\exampletranslation{La mesa es lisa y brillosa.}

\entry{-em}
\nontranslationdef{Sufijo que se presenta con raíces intransitivas para formar otras raíces atributivas: p. ej.:}
\cholexample{sajtyem}
\exampletranslation{perdido.}
\variation{-eñ}

\entry{eñtyäl}
\partofspeech{adj}
\spanishtranslation{abajo}
\cholexample{Jiñi tyejklum añ tyi eñtyäl.}
\exampletranslation{El pueblo está abajo.}

\entry{ekel}
\partofspeech{adj}
\spanishtranslation{colocado}
\clarification{cosas}
\cholexample{Ekel jiñi chikib tyi mesa.}
\exampletranslation{La canasta está colocada sobre la mesa.}

\entry{erañ}
\partofspeech{s esp}
\spanishtranslation{hermano}

\entry{ermañu}
\partofspeech{s esp}
\spanishtranslation{hermano}

\entry{-es-}
\nontranslationdef{Sufijo que se presenta con raíces transitivas y atributivas para formar una raíz causativa; p. ej.:}
\cholexample{ñuk'esañ}
\exampletranslation{hacerlo grande.}
\variation{-s-, -tyes}

\entry{esmañ}
\partofspeech{vi}
\onedefinition{1}
\spanishtranslation{dedicarse}
\cholexample{Jujump'ejl k'iñ mi iyesmañ alas.}
\exampletranslation{Día tras día se dedica a jugar.}
\onedefinition{2}
\spanishtranslation{seguir}
\cholexample{Pejtyel ora mi iyesmañ e'tyel.}
\exampletranslation{Siempre sigue haciendo su trabajo.}

\entry{-ety}
\onedefinition{1}
\nontranslationdef{Sufijo que se presenta con raíces intransitivas en tiempo pasado para indicar la segunda persona de singular del sujeto.}
\onedefinition{2}
\nontranslationdef{Sufijo que se presenta con raíces transitivas en tiempo pasado para indicar la segunda persona de singular del objeto.}

\entry{-etyla}
\onedefinition{1}
\nontranslationdef{Sufijo que se presenta con raíces intransitivas en tiempo pasado para indicar la segunda persona de plural del sujeto.}
\onedefinition{2}
\nontranslationdef{Sufijo que se presenta con raíces transitivas en tiempo pasado para indicar la segunda persona de plural del objeto.}

\entry{*e'tyel}
\partofspeech{s}
\onedefinition{1}
\spanishtranslation{trabajo}
\cholexample{Añ kabäl lake'tyel.}
\exampletranslation{Tengo mucho trabajo.}
\onedefinition{2}
\spanishtranslation{encargo}
\cholexample{Añ iye'tyel jiñi wiñik.}
\exampletranslation{A ese hombre le dieron un encargo.}
\dialectvariant{Tila}
\dialectword{tyroñel}
\secondaryentry{ambä iye'tyel}
\secondtranslation{autoridad}

\entry{e'tyijibäl}
\partofspeech{s}
\spanishtranslation{el equipo de trabajo}
\clarification{como despulpadora de café}

\alphaletter{G}

\entry{gloria}
\relevantdialect{Tila}
\partofspeech{s esp}
\spanishtranslation{cielo}
\alsosee{pañchañ}

\alphaletter{I}

\entry{i-}
\onedefinition{1}
\nontranslationdef{Prefijo que indica adjetivo posesivo, 3ª persona}
\onedefinition{2}
\nontranslationdef{Prefijo que indica pronombre personal, 3ª persona}

\entry{-i}
\nontranslationdef{Sufijo que se presenta con raíces de ciertos numerales para formar raíces atributivas que indican tiempo de hoy en adelante; p. ej.:}
\cholexample{chabi}
\exampletranslation{dentro de dos días.}

\entry{ib}
\relevantdialect{Tila}
\partofspeech{s}
\spanishtranslation{armadillo}
\clarification{mamífero}
\alsosee{wech}

\entry{-ib}
\nontranslationdef{Sufijo que se presenta con raíces transitivas y neutras para formar una raíz sustantiva; p. ej.:}
\cholexample{wäyib}
\exampletranslation{cama.}

\entry{-ibal}
\nontranslationdef{Sufijo que se presenta con raíces intransitivas para formar una raíz sustantiva que indica un punto en el tiempo; p. ej.:}
\cholexample{jilibal}
\exampletranslation{su fin.}

\entry{-ik}
\nontranslationdef{Sufijo que se presenta con raíces adjetivas e intransitivas, para formar expresiones negativas.}

\entry{ik'}
\partofspeech{s}
\onedefinition{1}
\spanishtranslation{aire}
\cholexample{Ma'añik mi iyochel ik' che' weñ mäkäl lakotyoty.}
\exampletranslation{Cuando nuestra casa está bien cerrada no entra aire.}
\onedefinition{2}
\spanishtranslation{viento}
\cholexample{Tsa' aweñ yajli tyi ik' jiñi kchol.}
\exampletranslation{El viento tumbó al suelo mi milpa.}

\entry{ik'ajel}
\partofspeech{s}
\spanishtranslation{crepúsculo}
\clarification{ya entrando la noche, pero todavía no muy oscuro}

\entry{ik'añ}
\partofspeech{vi}
\spanishtranslation{anochecer}

\entry{ik'atyax}
\partofspeech{adv}
\spanishtranslation{madrugada}
\cholexample{Ik'atyax mi ich'ojyel jiñi x'ixik cha'añ mi imel waj.}
\exampletranslation{Muy de madrugada se levanta la mujer para hacer las tortillas.}

\entry{ik'bäty}
\partofspeech{s}
\spanishtranslation{árbol}
\clarification{sirve para hacer tablas}

\entry{ik'bo'lay}
\partofspeech{s}
\spanishtranslation{tigre frijolillo}
\clarification{reg.}
\spanishtranslation{ocelote}

\entry{ik' k'uts}
\spanishtranslation{flor de gusano}
\clarification{hierba}

\entry{ik'chäy}
\partofspeech{s}
\spanishtranslation{mojarra}
\clarification{un pez negro}

\entry{ik'ch'ipañ}
\partofspeech{adj}
\spanishtranslation{oscuro}
\clarification{dentro de la casa o de una cueva}

\entry{ik'jowañ}
\partofspeech{adj}
\onedefinition{1}
\spanishtranslation{negro}
\cholexample{Ik'jowañ tyi buts' ijol otyoty.}
\exampletranslation{El techo de su casa está negro por el humo.}
\onedefinition{2}
\spanishtranslation{oscuro}
\cholexample{Ik'jowañ jiñi ch'eñ.}
\exampletranslation{La cueva está oscura.}

\entry{ik' lukum}
\relevantdialect{Tila}
\spanishtranslation{cola de fuego}
\spanishtranslation{lagartijera olivácea}
\clarification{reptil}

\entry{ik'motyañ}
\partofspeech{adj}
\spanishtranslation{negro}
\cholexample{Che' laj i'ik' jiñi muty ik'motyañ mi icha'leñ xämbal.}
\exampletranslation{Una partida de gallinas negras está caminando.}

\entry{ik'sajp}
\partofspeech{s}
\spanishtranslation{ocelote}
\clarification{mamífero}

\entry{ik'selañ}
\partofspeech{adj}
\onedefinition{1}
\spanishtranslation{manchado}
\clarification{con mancha redonda y negra, no permanente}
\cholexample{Ik'selañ iwuty tyi abäk.}
\exampletranslation{Tiene la cara manchada de negro por el carbón.}
\onedefinition{2}
\spanishtranslation{marca de nacimiento}
\cholexample{Ik'selañ iwuty k'äläl tsa' iyilapañimil.}
\exampletranslation{Tiene un lunar en la cara.}

\entry{ik'sowañ}
\partofspeech{adj}
\spanishtranslation{sucio}
\clarification{ropa, hilo}

\entry{ik'tye'}
\partofspeech{s}
\spanishtranslation{tipo de árbol de madera negra}

\entry{ik' tyo}
\partofspeech{adv}
\spanishtranslation{madrugada}

\entry{Ik'tyumpa'}
\relevantdialect{Tila}
\partofspeech{s}
\spanishtranslation{Arroyo de Piedra Negra}
\clarification{ranchería}

\entry{ik'ty'ojal}
\relevantdialect{Tila}
\partofspeech{s}
\nontranslationdef{Espíritu que según se cree es bueno.}

\entry{ik'ty'ojañ}
\partofspeech{adj}
\spanishtranslation{oscuro y bajo}
\cholexample{Ik'ty'ojañ jiñityokal tyi tyejklum.}
\exampletranslation{La nube está sobre el pueblo, oscura y baja.}

\entry{ik'ty'ojñal}
\relevantdialect{Tila}
\partofspeech{adj}
\spanishtranslation{oscuro}
\cholexample{Che' ik'ty'ojñal ma'añ uw.}
\exampletranslation{Cuando está muy oscuro no hay luna.}

\entry{ik'uyel}
\partofspeech{s}
\spanishtranslation{aire}
\clarification{del estómago}
\cholexample{Añ icha'añ ik'uyel.}
\exampletranslation{Él tiene aire en el estómago.}

\entry{ik'wa'añ}
\partofspeech{adv}
\spanishtranslation{oscuramente}
\cholexample{Ik'wa'añ woli ityilel jiñi wiñik.}
\exampletranslation{Ese hombre viene vestido oscuramente, con ropa negra.}

\entry{ik'xi'}
\partofspeech{s}
\spanishtranslation{tipo de árbol}
\clarification{de madera amarilla y maciza; tiene fruta grande}

\entry{ik'yoch'añ}
\partofspeech{adj}
\spanishtranslation{oscuro}
\clarification{dentro de la casa o de una cueva o afuera cuando no hay luna}
\cholexample{Ik'yoch'añ imal otyoty kome ma'añik k'ajk.}
\exampletranslation{Dentro de la casa está oscuro porque no hay luz.}

\entry{ich}
\partofspeech{s}
\spanishtranslation{chile}

\entry{*ichak'}
\partofspeech{s}
\spanishtranslation{hijo de la hermana de mi padre}
\clarification{primo hermano}

\entry{*ichañ}
\partofspeech{s}
\spanishtranslation{tío}
\clarification{hermano de mi madre}

\entry{ichitye'}
\relevantdialect{Sab.}
\partofspeech{s}
\spanishtranslation{jocotillo}
\spanishtranslation{jobillo}
\clarification{árbol}
\alsosee{yäxluluy}

\entry{ichtye'}
\partofspeech{s}
\spanishtranslation{malamujer}

\entry{ichtyo'}
\partofspeech{s}
\spanishtranslation{pimienta de Jamaica}
\clarification{árbol}

\entry{ich'iñtye'}
\partofspeech{s}
\spanishtranslation{árbol malamujer}

\entry{*ij}
\partofspeech{s}
\spanishtranslation{nieto}
\alsosee{*buts}

\entry{ijk'al max}
\spanishtranslation{mono araña}
\clarification{mamífero}

\entry{ijk'äl}
\partofspeech{part}
\spanishtranslation{mañana}
\cholexample{Yom lakpejkañ lakbä ijk'äl che' jiñi.}
\exampletranslation{Entonces debemos hablarnos mañana.}

\entry{-ijel}
\nontranslationdef{Sufijo que se presenta con raíces atributivas y neutras para formar una raíz intransitiva; p. ej.:}
\cholexample{ñuxijel}
\exampletranslation{nadar.}
\variation{iyel}

\entry{*ijñam}
\partofspeech{s}
\spanishtranslation{esposa}

\entry{*ijty'añ}
\partofspeech{s}
\spanishtranslation{hermana}

\entry{*ijts'iñ}
\partofspeech{s}
\spanishtranslation{hermanito}

\entry{-il}
\defsuperscript{1}
\nontranslationdef{Sufijo que se presenta con raíces transitivas para formar otra raíz sustantiva que indica instrumento; p. ej.:}
\cholexample{imajkil}
\exampletranslation{su tapa.}
\cholexample{Ak'eñ imajkil p'ejty.}
\exampletranslation{Pon una tapa sobre la olla.}

\entry{-il}
\defsuperscript{2}
\nontranslationdef{Sufijo que se presenta con raíces sustantivas para formar otra raíz sustantiva que indica extensión o lugar de algo; p. ej.:}
\cholexample{alaxaxil}
\exampletranslation{naranjal}
\variation{1*-al}

\entry{*ilal}
\partofspeech{adj}
\nontranslationdef{Expresa la condición en que está; p. ej.}
\cholexample{¿bajche' awilal?}
\exampletranslation{¿Cómo estás?}

\entry{-ilañ}
\nontranslationdef{Sufijo que se presenta con raíces transitivas para formar una raíz transitiva que indica movimiento; p. ej.:}
\cholexample{k'utyilañ}
\exampletranslation{tamular.}

\entry{*ilañ}
\defsuperscript{1}
\partofspeech{vt}
\onedefinition{1}
\spanishtranslation{ver}
\cholexample{Tsa kilajosé tyi tyejklum.}
\exampletranslation{Vi a José en el pueblo}
\onedefinition{2}
\spanishtranslation{visitar}
\cholexample{Jiñi wiñik tsajñi iyilañ ipi'älob.}
\exampletranslation{El hombre fue a visitar a sus familiares.}
\onedefinition{3}
\spanishtranslation{curar}
\clarification{por hechicería}
\secondaryentry{*ilañ pañimil}
\secondtranslation{nacer}

\entry{*ilañ}
\defsuperscript{2}
\partofspeech{vt}
\onedefinition{1}
\spanishtranslation{reconocer}
\clarification{sus pensamientos}
\cholexample{Woli jach kiläbeñ ipusik'al cha'añ mik ña'tyañbajche' mi iyäl.}
\exampletranslation{Estoy nada más reconociendo sus pensamientos para saber que va a decir.}
\onedefinition{2}
\spanishtranslation{probar}
\clarification{el corazón}
\cholexample{Jiñi xk'el e'tyel mi iyiläbeñoñlalakpusik'al mi uts'atyäch tsa'ix lakñopo.}
\exampletranslation{El mayordomo nos prueba para saber si en verdad hemos aprendido.}

\entry{ilawä}
\relevantdialect{Sab.}
\partofspeech{adv}
\spanishtranslation{aquí}

\entry{ilayi}
\partofspeech{adv}
\spanishtranslation{aquí}
\cholexample{Tsa' tyiliyoñ tyi wäyel ilayi.}
\exampletranslation{Vine a dormir aquí.}

\entry{ili}
\partofspeech{adj}
\spanishtranslation{este, esta}
\cholexample{Woli jk'äñ ili machity.}
\exampletranslation{Estoy usando este machete.}

\entry{iliyi}
\partofspeech{adj}
\spanishtranslation{este, esta}
\cholexample{Iliyi jiñäch kotyoty.}
\exampletranslation{Esta casa es mía.}

\entry{ilol}
\partofspeech{s}
\spanishtranslation{acción de curar}
\clarification{lo que hace un curandero}

\entry{-iñ}
\conjugationtense{variante}
\conjugationverb{\textsuperscript{2}-añ}
\nontranslationdef{Sufijo que se presenta con raíces transitivas y neutras para formar una raíz transitiva; p. ej.:}
\cholexample{lotyiñ}
\exampletranslation{engañar}

\entry{ik'ix}
\partofspeech{adj}
\onedefinition{1}
\spanishtranslation{ya (es) muy tarde}
\cholexample{Ik'ix, koñixlatyi sujtyel.}
\exampletranslation{Ya es muy tarde para que regresemos.}
\onedefinition{2}
\spanishtranslation{ya (está) oscuro}
\cholexample{Ik'ix tsa' k'otyiyoñ tyi yajalóñ.}
\exampletranslation{Ya estaba oscuro cuando llegué a Yajalón.}

\entry{i sujm}
\spanishtranslation{cierto, verdadero}
\cholexample{Isujm chuki woli isub.}
\exampletranslation{Es cierto lo que está diciendo.}
\secondaryentry{isujmlel}
\secondpartofspeech{s}
\secondtranslation{verdad}

\entry{i tyajol}
\spanishtranslation{a veces}
\cholexample{Añ ityajol mi ijobel ityojol kajpe'.}
\exampletranslation{A veces baja el precio de café.}

\entry{i tyilel}
\spanishtranslation{de por sí}
\cholexample{Ityilel che'äch mi ityeñe esmañ tyi pejtyelel ora.}
\exampletranslation{De por sí, así hace siempre.}

\entry{i tyojol}
\onedefinition{1}
\spanishtranslation{precio}
\cholexample{Wolix iletsel ityojol lakbujk.}
\exampletranslation{El precio de las camisas está subiendo.}
\onedefinition{2}
\spanishtranslation{pago}
\cholexample{Maxtyo añik tsa' iyäk'eyoñ ityojol jiñi kajpe'.}
\exampletranslation{Todavía no me ha dado el pago del café.}

\entry{ixäch}
\partofspeech{adv}
\spanishtranslation{allá}
\clarification{señalando}
\cholexample{Ixäch kotyol tsa' jk'ele jiñi mula.}
\exampletranslation{Vi que allá estuvo parada la mula.}

\entry{ixba'añ}
\partofspeech{adv}
\spanishtranslation{allí donde está eso}

\entry{ixku}
\partofspeech{part}
\spanishtranslation{en cuanto a}
\cholexample{¿ixku jatyety mu'ba amejlel tyi majlel?}
\exampletranslation{En cuanto a usted, ¿puede ir?}

\entry{*ixikp'eñel}
\partofspeech{s}
\onedefinition{1}
\spanishtranslation{hija grande}
\onedefinition{2}
\spanishtranslation{hija de hombre}

\entry{*ixik'al}
\partofspeech{s}
\spanishtranslation{hija de mujer}

\entry{ixim}
\partofspeech{s}
\spanishtranslation{maíz}

\entry{ixixi}
\partofspeech{adv}
\spanishtranslation{allá}
\cholexample{Ixixi cha' wa'al jiñi wiñik.}
\exampletranslation{Allá está parado otra vez ese hombre.}

\entry{ixiyi}
\partofspeech{pron}
\spanishtranslation{aquél}

\entry{ixmañ}
\partofspeech{vt}
\spanishtranslation{desgranar}
\clarification{maíz}

\entry{ixom}
\partofspeech{s}
\spanishtranslation{acción de desgranar}
\cholexample{La'lakcha'leñ ixom.}
\exampletranslation{Vamos a desgranar (lit.: vamos a hacer la acción de desgranar).}

\entry{ixtye'}
\partofspeech{s}
\spanishtranslation{malamujer, chechén}
\clarification{árbol}

\entry{Ixtye'ja'}
\partofspeech{s}
\spanishtranslation{nombre de un rancho}

\entry{ixtyi}
\partofspeech{adv}
\spanishtranslation{allá}
\clarification{hasta allá}
\cholexample{Ixtyi yambä tyieñda mi ichojñel weñ tyakbä pisil.}
\exampletranslation{Allá en la otra tienda se vende buena ropa.}

\entry{-ixtyi-}
\nontranslationdef{Sufijo que indica opinión; p. ej.:}
\cholexample{Mach mi lakña'tyañ mi ity'ojolixtyika jiñi lum.}
\exampletranslation{No sabemos si el terreno está bonito.}

\entry{ixtyok'}
\partofspeech{s}
\spanishtranslation{pimienta de la tierra}
\clarification{árbol}

\entry{-iyel}
\conjugationtense{variante}
\conjugationverb{-ijel}
\nontranslationdef{Sufijo que se presenta con raíces atributivas y neutras para formar una raíz intransitiva; p. ej.:}
\cholexample{ñajiyel}
\exampletranslation{olvidarse.}

\entry{i'ik'}
\partofspeech{adj}
\spanishtranslation{negro}
\secondaryentry{i'ik'bä way ja'as}
\secondtranslation{zapote prieto}
\clarification{árbol}
\secondaryentry{i'ik' max}
\secondtranslation{mico negro}
\secondaryentry{i'ik' muty}
\secondtranslation{cuervo}

\entry{i'ik'ax}
\partofspeech{adj}
\spanishtranslation{muy oscuro}
\cholexample{I'ik'ax pañimil cha'añ ma'añik uw.}
\exampletranslation{Está muy oscuro, porque no hay luna.}

\entry{i'ik'tye'}
\partofspeech{s}
\spanishtranslation{tipo de árbol}
\clarification{de cáscara negra y da fruta como la ciruela}

\entry{i'ik'ix tyi buts'}
\spanishtranslation{ahumado}
\cholexample{I'ik'ix tyi buts' jiñi pixoläl.}
\exampletranslation{El sombrero está ahumado.}

\alphaletter{J}

\entry{jab}
\partofspeech{s}
\spanishtranslation{año}

\entry{*jabilel}
\partofspeech{s}
\spanishtranslation{edad}

\entry{jak}
\partofspeech{vt}
\spanishtranslation{desgajar}

\entry{jak'}
\partofspeech{vt}
\onedefinition{1}
\spanishtranslation{contestar}
\cholexample{Ma'añik tsa' iweñ jak'ä chuki tsa' jk'ajtyibe.}
\exampletranslation{No contestó bien lo que le pregunté.}
\onedefinition{2}
\spanishtranslation{obedecer}
\cholexample{Ma'añik tsa' ijak'beyoñ jiñi tsa'bä ksube.}
\exampletranslation{No me obedeció en lo que le dije.}

\entry{jak' ik'}
\relevantdialect{Sab.}
\spanishtranslation{jadear}
\cholexample{Yäkel tyi jak' ik' kome ajñel yäkel.}
\exampletranslation{Como está corriendo, está jadeando.}

\entry{jak' i yoj}
\spanishtranslation{suspirar}
\cholexample{Woli ijak' iyoj jiñi x'ixik cha'añ ch'ijiyem.}
\exampletranslation{Esta mujer está suspirando de tristeza.}

\entry{jak'ol}
\partofspeech{s}
\spanishtranslation{acción de obedecer}

\entry{jach}
\partofspeech{adv}
\spanishtranslation{sólo}
\cholexample{Ibajñel jach tsa' majli tyi e'tyel.}
\exampletranslation{Sólo él se fue a trabajar.}

\entry{jachaj}
\partofspeech{s esp}
\spanishtranslation{hacha}

\entry{jacha lak mam}
\spanishtranslation{obsidiana}
\culturalinformation{Información cultural: Algunos pedazos son verde oscuro, otros amarillos y otros negros; algunos son planos, otros redondos. Se dice que se producen por los rayos. Se encuentran en los arroyos.}

\entry{jach'}
\partofspeech{vt}
\spanishtranslation{masticar}

\entry{jaj}
\partofspeech{s}
\spanishtranslation{mosca}

\entry{jajakña}
\partofspeech{adv}
\spanishtranslation{a carcajadas}
\cholexample{Jajakña tyi tse'ñal jiñi ch'ityoñ kome tsa' yajli iyijts'iñ.}
\exampletranslation{Ese chamaco se ríe a carcajadas porque se cayó su hermanito.}

\entry{jajayajl}
\partofspeech{part}
\spanishtranslation{cada vez}
\cholexample{Jajayajl mi amajlel tyi awe'tyel mach buchtyäl jach mi amajlel.}
\exampletranslation{Cada vez que vas a tu trabajo no vayas solamente a sentarte.}

\entry{jaja'}
\partofspeech{part}
\spanishtranslation{sí, pues}
\clarification{una respuesta}

\entry{-jajk}
\nontranslationdef{Sufijo numeral para contar extremidades, brazos o piernas; p. ej.:}
\cholexample{Tsak mäñä juñjajk iya' wakax.}
\exampletranslation{Compré una pierna de res.}

\entry{jajkuñ}
\partofspeech{vt}
\spanishtranslation{descuartizar}
\cholexample{Ora jach mi lakjakulañ jiñi chityam che' chämeñix.}
\exampletranslation{Cuando el cerdo ya está muerto, rápidamente lo descuartizamos.}

\entry{jajch}
\partofspeech{s}
\spanishtranslation{jilote}

\entry{-jajl}
\nontranslationdef{Sufijo numeral para contar brazadas; p. ej.:}
\cholexample{Jiñi kotyoty añ wäkjajl iñajtylel.}
\exampletranslation{Mi casa tiene seis brazadas.}

\entry{jajlañ}
\partofspeech{vt}
\spanishtranslation{medir}
\clarification{con brazadas}

\entry{jajmel}
\onedefinition{1}
\partofspeech{s}
\spanishtranslation{tiempo de seca}
\cholexample{Wolix iläk'tyiyel iyorajlel jajmel.}
\exampletranslation{Ya se está acercando el tiempo de seca.}
\onedefinition{2}
\partofspeech{vi}
\spanishtranslation{aclararse}
\cholexample{Mach ñoj yomix jajmel pañimil.}
\exampletranslation{El tiempo ya no quiere aclararse.}
\onedefinition{3}
\partofspeech{vi}
\spanishtranslation{abrir}
\cholexample{Mux ijajmel jiñi korreo che' tyi bolomp'ejl ora.}
\exampletranslation{Se va a abrir el correo a las nueve.}

\entry{jajmeñ}
\partofspeech{adj}
\onedefinition{1}
\spanishtranslation{claro}
\cholexample{Jajmeñix pañimil che'bä tsa' majliyoñ tyi chobal.}
\exampletranslation{Ya estaba claro el día cuando me fui a rozar.}
\onedefinition{2}
\spanishtranslation{abierto}
\cholexample{Che' tyi säk'añ jajmeñix kotyoty.}
\exampletranslation{Ya está abierta mi casa temprano en la mañana.}

\entry{-jajp}
\nontranslationdef{Sufijo numeral para contar hendiduras o rajaduras.}

\entry{*jajp}
\partofspeech{s}
\spanishtranslation{hendidura, rajadura}
\cholexample{Kabäl ijajp tyak lum tyi kajpelel.}
\exampletranslation{En mi cafetal hay muchas hendiduras.}

\entry{jajpa'}
\partofspeech{s}
\spanishtranslation{arroyo}
\clarification{pasa en medio de dos cerritos}

\entry{jajpiñ}
\partofspeech{vt}
\onedefinition{1}
\spanishtranslation{acariciar}
\cholexample{Mi ijajpiñ iyalobil cha'añ mi iwäytyesañ.}
\exampletranslation{Acaricia a su hijo para adormecerlo.}
\onedefinition{2}
\spanishtranslation{frotar}
\clarification{con medicina}
\cholexample{Jiñi x'ixik mi ijajpibeñ ts'ak tyi' paty iyalobil.}
\exampletranslation{Esa mujer frota con medicina la espalda de su hijo.}

\entry{jajp wits}
\spanishtranslation{paso}
\clarification{entre dos cerros}

\entry{jaj sepoñel}
\relevantdialect{Sab.}
\spanishtranslation{rastrojo}

\entry{jajtyel}
\partofspeech{vi}
\spanishtranslation{rajarse}
\cholexample{Ma'añik mi imejlel tyi jajtyel jiñi si'.}
\exampletranslation{Esa leña no se puede rajar.}

\entry{*jajtyemal}
\partofspeech{s}
\spanishtranslation{rajada}
\cholexample{Kabäl jax ijajtyemal jiñi tyabla.}
\exampletranslation{Tiene muchas rajadas esa tabla.}

\entry{-jajts'}
\nontranslationdef{Sufijo numeral para contar períodos de música; p. ej.:}
\cholexample{juñjajts'}
\partofspeech{adj}
\exampletranslation{un período de música que ocupa un instrumento.}

\entry{*jajts'}
\partofspeech{s}
\spanishtranslation{viento leve}
\cholexample{Ijajts' jach ik' tsa' iyäsa jiñi tye'.}
\exampletranslation{Un viento leve hizo caer el árbol.}

\entry{jajwel}
\partofspeech{vi}
\spanishtranslation{partirse}
\cholexample{Mach wokolik mi ijajwel jiñi poytye' che' mi lakjaty.}
\exampletranslation{El corcho se parte fácilmente cuando lo rajamos.}

\entry{jal}
\defsuperscript{1}
\partofspeech{vt}
\spanishtranslation{tejer}
\clarification{red, hamaca, morral, costal}

\entry{jal}
\defsuperscript{2}
\partofspeech{adv}
\spanishtranslation{mucho tiempo}
\cholexample{Jal tsi'tyojbeyoñ ibety.}
\exampletranslation{Dilató mucho tiempo en pagarme su deuda.}

\entry{jalaj}
\partofspeech{adv}
\spanishtranslation{¿cuándo?}
\cholexample{¿jalaj mi kaj ityilel?}
\exampletranslation{¿Cuándo va a venir?}

\entry{jalaña'}
\partofspeech{s}
\spanishtranslation{madrina}

\entry{jalatyaty}
\partofspeech{s}
\spanishtranslation{padrino}

\entry{jalaw}
\relevantdialect{Sab.}
\partofspeech{s}
\spanishtranslation{tepescuintle}
\clarification{mamífero}

\entry{*jala'al}
\partofspeech{s}
\spanishtranslation{ahijado}

\entry{jaläjp}
\partofspeech{s}
\onedefinition{1}
\spanishtranslation{arco}
\clarification{puente}
\cholexample{Ili ora ma'añix mi ik'äñob jaläjp lakpi'älob.}
\exampletranslation{Nuestros compañeros ya no usan el puente de arco.}
\onedefinition{2}
\spanishtranslation{arco}
\clarification{arma}
\cholexample{Weñtyoj yom imelol jaläjp cha'añtyoj mi laktyaj mu'bä lakjul.}
\exampletranslation{El arco tiene que ser perfecto para poder alcanzar a lo que le tiramos.}

\entry{jaläl}
\partofspeech{s}
\spanishtranslation{flauta}
\culturalinformation{Información cultural: Es hecha de carrizo. Tiene una lengüeta de cera. Hay dos tamaños: uno de veintinueve centímetros y otro de treinta y siete centímetros. La más corta tiene tres agujeros, y la más larga tiene siete.}

\entry{jaläxki}
\partofspeech{adv}
\spanishtranslation{¿cuándo?}
\cholexample{¿jalixki tsa' k'otyi jiñi wiñik?}
\exampletranslation{¿Cuándo llegó ese hombre?}

\entry{jalbal}
\partofspeech{s}
\spanishtranslation{tejido}
\cholexample{Ora jach tsi'bäk' ñopo jalbal yik'oty chij.}
\exampletranslation{Aprendió luego a hacer el tejido con ixtle.}

\entry{jalba'am}
\partofspeech{s}
\spanishtranslation{tela de araña, telaraña}

\entry{jalbil}
\partofspeech{adj}
\spanishtranslation{trenzado}
\cholexample{Ity'ojoljax jiñi xch'ok che' jalbil ijol.}
\exampletranslation{Esa muchacha se ve bonita cuando trae el cabello trenzado.}

\entry{jalijel}
\partofspeech{vi}
\spanishtranslation{tardarse}

\entry{jal'añ}
\partofspeech{vi}
\spanishtranslation{tardarse}

\entry{jam}
\defsuperscript{1}
\partofspeech{vt}
\spanishtranslation{abrir}
\clarification{casa, libro, caja}
\secondaryentry{jam ibä}
\secondpartofspeech{vi}
\secondtranslation{abrirse}

\entry{jam}
\defsuperscript{2}
\partofspeech{s}
\spanishtranslation{zacate}
\secondaryentry{ijamil}
\secondpartofspeech{s}
\secondtranslation{paja de un techo}

\entry{jamakña}
\partofspeech{adj}
\spanishtranslation{despejado}
\cholexample{Jamakña pañimil che' tyi' yorajlel jajmel.}
\exampletranslation{En el tiempo de secas el cielo está despejado.}

\entry{-jamañ}
\nontranslationdef{Sufijo que se presenta con raíces adjetivas que indican color; se refiere a la parte interior de una casa.}

\entry{jamäl}
\partofspeech{adj}
\spanishtranslation{buen tiempo}
\cholexample{Jamälix pañimil che' mi ilajmel ja'al.}
\exampletranslation{Cuando termine la lluvia hará buen tiempo.}

\entry{jamil}
\partofspeech{s}
\spanishtranslation{zacatal}
\dialectvariant{Sab.}
\dialectword{majamol}

\entry{jamoñ}
\relevantdialect{Tila}
\partofspeech{s}
\spanishtranslation{tapir}
\clarification{mamífero}
\alsosee{tsimiñ}

\entry{jamoñib}
\partofspeech{s}
\spanishtranslation{destapador}
\clarification{de botellas}

\entry{jañ}
\partofspeech{s}
\spanishtranslation{flor de maíz}

\entry{jap}
\partofspeech{vt}
\spanishtranslation{beber}

\entry{japal}
\partofspeech{adj}
\onedefinition{1}
\spanishtranslation{abierto}
\clarification{tablas, setos}
\cholexample{Japal tyak ibojtye'lel lakotyoty.}
\exampletranslation{El seto de nuestra casa está abierto.}
\onedefinition{2}
\spanishtranslation{cuenca}
\clarification{de río}
\cholexample{Japal jiñi lumba' mi iñumel kolem ja'.}
\exampletranslation{Por donde pasa el río hay cuenca.}
\onedefinition{3}
\spanishtranslation{dividido}
\clarification{cerro, montaña}
\cholexample{Japal jiñi witsba' mi iñumel jiñi bij.}
\exampletranslation{El cerro está dividido en donde pasa el camino.}
\onedefinition{4}
\spanishtranslation{marcado}
\clarification{con cicatriz}
\cholexample{Japal ichoj wiñikba' tsa' tsejpi tyi machity.}
\exampletranslation{La mejilla del hombre está marcada con la cicatriz de cuando se cortó con el machete.}

\entry{jap ik'}
\spanishtranslation{respirar}
\cholexample{Wokol jax mi ijap ik' kome weñ mäkäl iñi' cha'añ sijmal.}
\exampletranslation{Le es difícil respirar, porque su nariz está tapada por el catarro.}

\entry{japtyäl}
\partofspeech{s}
\spanishtranslation{zanjón}
\cholexample{Bäk'eñbäk'eñjax mi ayajlel ya' tyi japtyäl.}
\exampletranslation{Hay peligro de que caigas en el zanjón.}

\entry{japuñ}
\partofspeech{vt}
\spanishtranslation{abrir}
\clarification{brecha, zanja}
\cholexample{Yom mi lakjapuñ ibijlel yok ja' cha'añ mi icha'leñ ajñel.}
\exampletranslation{Debemos abrir una zanja para que pueda correr el agua.}

\entry{jasäl}
\partofspeech{adj}
\spanishtranslation{suficiente}
\cholexample{Jasäl ts'itya' jkajpe'lel cha'añ mi mejlel kweñ käñtyañ.}
\exampletranslation{Con un poquito de mi cafetal es suficiente para que yo pueda atenderlo.}

\entry{jastyiyel}
\partofspeech{vi}
\spanishtranslation{alcanzar}
\clarification{alimento, paga, ropa}
\cholexample{Ma'añik mi ijastyiyel ityojol jkajpe'.}
\exampletranslation{No me alcanza la paga de mi café.}

\entry{jaty}
\partofspeech{vt}
\spanishtranslation{rajar}
\cholexample{Jiñi ch'ityoñ mi imulañ ijaty si'.}
\exampletranslation{A ese muchacho le gusta rajar leña.}

\entry{jatyety}
\partofspeech{pron}
\spanishtranslation{tú}

\entry{jats'}
\partofspeech{vt}
\onedefinition{1}
\spanishtranslation{pegar}
\clarification{persona o animal}
\onedefinition{2}
\spanishtranslation{golpear}
\clarification{con objeto}

\entry{jaw}
\partofspeech{vt}
\onedefinition{1}
\spanishtranslation{partir}
\clarification{naranja, calabaza, corcho}
\cholexample{Añ mu'bä ijawob jiñi alaxax che' mi ik'ux.}
\exampletranslation{Hay algunos que parten la naranja cuando la comen.}
\onedefinition{2}
\spanishtranslation{cambiar}
\clarification{moneda}
\cholexample{Samik jaw ktyak'iñ cha'añ miktyoj kwiñik.}
\exampletranslation{Voy a cambiar mi dinero para poder pagar a mis jornaleros.}

\entry{*jawäñ}
\partofspeech{s}
\spanishtranslation{cuñada de mujer}

\entry{*jawtyälel}
\partofspeech{s}
\spanishtranslation{espacio}
\clarification{de una casa}
\cholexample{Ñuk ijawtyälel imal jiñi otyoty.}
\exampletranslation{El espacio de adentro de la casa es grande.}

\entry{jax}
\defsuperscript{1}
\partofspeech{vt}
\spanishtranslation{retorcer}
\clarification{hilo, ixtle}

\entry{jax}
\defsuperscript{2}
\partofspeech{adv}
\spanishtranslation{rozando}
\cholexample{Che' mik jul jiñi muty mu' jach ijax ñumel tyi' ts'ejtyä'lel.}
\exampletranslation{Cuando le tiro a ese pájaro solamente le pasa rozando por los lados.}

\entry{jaxakña}
\onedefinition{1}
\partofspeech{adj}
\spanishtranslation{ajustado}
\cholexample{Jaxakña tsa' käle ilamiñajlel otyoty tyi' joytyilel.}
\exampletranslation{La lámina quedó ajustada a la medida de toda la casa.}
\onedefinition{2}
\partofspeech{adv}
\spanishtranslation{justamente}
\cholexample{Jaxakña tsa' jili kpak' che' tsa' ujtyi kpäk' kchol.}
\exampletranslation{Se terminó mi semilla justamente cuando acabé de sembrar mi milpa.}

\entry{jaxal}
\relevantdialect{Sab.}
\partofspeech{adj}
\onedefinition{1}
\spanishtranslation{parejo}
\cholexample{Jaxal tyak jiñi tyablaambä tyi mesa.}
\exampletranslation{Las tablas están parejas en la mesa.}
\onedefinition{2}
\spanishtranslation{colindado}
\cholexample{Jaxal ilum sabañilayik'oty ilum progreso.}
\exampletranslation{El terreno de Sabanilla está colindado con el del Progreso.}

\entry{jaxäl}
\partofspeech{adj}
\spanishtranslation{contiguo}
\cholexample{Ya' jaxäl ikajpe'lel tyi' tyi' bij.}
\exampletranslation{Su cafetal está contiguo a la orilla del camino.}

\entry{jaxulañ}
\partofspeech{vt}
\spanishtranslation{enrollar}
\clarification{ixtle o hilo}
\cholexample{Mi iñaxañ jaxulañ chij cha'añ mi imel ab.}
\exampletranslation{Primero enrolla el ixtle cuando va a hacer una hamaca.}

\entry{jaxuñ}
\partofspeech{vt}
\onedefinition{1}
\spanishtranslation{sobar}
\clarification{brazo}
\onedefinition{2}
\spanishtranslation{untar}
\clarification{grasa en una tortilla}

\entry{jay}
\partofspeech{adj}
\onedefinition{1}
\spanishtranslation{delgado}
\cholexample{Ityileläch jay jiñi wiñik.}
\exampletranslation{Ese hombre es de por sí delgado.}
\onedefinition{2}
\spanishtranslation{afilado}
\cholexample{Yom jay jiñi lakmachity cha'añ k'uñ mi laktsep jiñi pimel.}
\exampletranslation{Nuestro machete debe estar bien afilado para que sea fácil cortar el monte.}

\entry{jay-}
\partofspeech{pref}
\spanishtranslation{¿cuánto?}

\entry{jayäb}
\partofspeech{s}
\spanishtranslation{bostezo}
\cholexample{Cha'añ wolix tyi jayäb jiñi ch'ityoñ che' woliyoñ ja'el.}
\exampletranslation{Los bostezos del niño ya se me contagiaron.}

\entry{*jayel}
\partofspeech{s}
\spanishtranslation{sien}

\entry{*jaylel}
\defsuperscript{1}
\partofspeech{s}
\spanishtranslation{grueso}
\clarification{de tabla, papel}
\cholexample{Yom mi ap'is ijaylel jiñi tyabla.}
\exampletranslation{Debes medir el grueso de la tabla.}

\entry{*jaylel}
\defsuperscript{2}
\partofspeech{s}
\spanishtranslation{mollera}
\cholexample{Che' alätyo jiñi aläl añ ijaylel ijol.}
\exampletranslation{Cuando la criatura aún es pequeña tiene mollera.}

\entry{jaymejl}
\partofspeech{adv}
\spanishtranslation{ya no}
\cholexample{Jaymejloñ ma'añix mik chäñ majlel.}
\exampletranslation{Ya no vuelvo a ir.}

\entry{jaypochañ}
\partofspeech{adj}
\spanishtranslation{con mucho filo}
\cholexample{Jaypochañ kmachity kome weñ juk'bil.}
\exampletranslation{Mi machete tiene mucho filo.}

\entry{ja'}
\partofspeech{s}
\onedefinition{1}
\spanishtranslation{agua}
\onedefinition{2}
\spanishtranslation{vertiente}
\onedefinition{3}
\spanishtranslation{arroyo, río}
\secondaryentry{awa'al}
\secondtranslation{agua de usted}
\secondaryentry{kolem ja'}
\secondtranslation{río, mar}
\secondaryentry{iya'al}
\secondtranslation{agua de él}
\secondaryentry{ja' lumil}
\secondtranslation{tremedal, aguazal}

\entry{ja'al}
\partofspeech{s}
\spanishtranslation{lluvia}

\entry{ja'al pech}
\spanishtranslation{garza}
\clarification{lit.: pato de agua}

\entry{ja'al ts'i'}
\spanishtranslation{nutria}
\clarification{lit.: perro de agua}

\entry{*ja'añ}
\partofspeech{s}
\spanishtranslation{cuñado de hombre}

\entry{ja'as}
\partofspeech{s}
\spanishtranslation{plátano}

\entry{ja'el}
\partofspeech{adv}
\spanishtranslation{también}
\cholexample{Che' ja'el joñoñ mi mejlel kmajlel.}
\exampletranslation{Así también yo puedo ir.}

\entry{ja'jabil}
\partofspeech{s}
\spanishtranslation{cada año}
\cholexample{Tyi ja'jabil mi'bajbeñ ja'al ilayi tyi tyumbalá.}
\exampletranslation{Cada año llueve bastante aquí en Tumbalá.}

\entry{ja'lel}
\partofspeech{s}
\spanishtranslation{aguacero}
\secondaryentry{ja'lel k'iñ}
\secondtranslation{temporada de lluvia}

\entry{ja'lumil}
\partofspeech{s}
\spanishtranslation{ciénaga, aguazal}

\entry{ja'tye'}
\partofspeech{s}
\spanishtranslation{pimienta de la tierra}
\clarification{árbol}

\entry{ja'tsijm}
\partofspeech{s}
\spanishtranslation{estornudo}
\cholexample{Mi lakcha'leñ ja'tsijm che' woliyoñlatyi sijmal.}
\exampletranslation{Estornudamos (lit.: hacer un estornudo) cuando tenemos catarro.}

\entry{jäk'jäk'ña}
\partofspeech{adv}
\onedefinition{1}
\nontranslationdef{Se relaciona con la forma de rebuznar; p. ej.:}
\cholexample{Ñajtyi tsikil jäk'jäk'ña woli tyi uk'el jiñi mula.}
\exampletranslation{De lejos se conoce la mula por la manera en que rebuzna.}
\onedefinition{2}
\spanishtranslation{lloriqueando}
\cholexample{Chäñ jäk'jäk'ñatyo tyi uk'el jiñi alob.}
\exampletranslation{El niño sigue lloriqueando.}

\entry{jäch'bik'}
\partofspeech{s}
\spanishtranslation{hipo}
\cholexample{Woli ityik'lañoñ jäch'bik'.}
\exampletranslation{Me está molestando el hipo.}

\entry{jäjlel}
\partofspeech{vi}
\spanishtranslation{resbalar}
\clarification{en camino}
\cholexample{Kabäl mi ijäjlel lakok che' añ ok'ol.}
\exampletranslation{Resbalamos mucho cuando hay lodo.}

\entry{jäjluñ}
\partofspeech{vt}
\spanishtranslation{chicotear}
\clarification{persona o animal}

\entry{jäjmañ}
\partofspeech{vt}
\spanishtranslation{mecer}
\clarification{en hamaca}
\cholexample{Yom mi ajäjmañ jiñi aläl cha'añ mi iwäyel.}
\exampletranslation{Debes mecer al niño para que se duerma.}

\entry{jäjmel}
\partofspeech{vi}
\spanishtranslation{mecer}
\clarification{en hamaca}

\entry{jäjmesañ}
\partofspeech{vt}
\spanishtranslation{mecer}
\clarification{criatura}

\entry{jäläkña}
\partofspeech{adv}
\spanishtranslation{arrastrando}
\cholexample{Jäläkña tsa' majli jiñi lukum.}
\exampletranslation{La culebra se fue arrastrando.}

\entry{jäläl}
\partofspeech{adj}
\spanishtranslation{tirado}
\clarification{un objeto largo}
\cholexample{Jäläl juñty'ujm laso tsak tyaja tyi bij.}
\exampletranslation{Encontré una soga tirada en el camino.}

\entry{jälol}
\partofspeech{adv}
\nontranslationdef{Concuerda con todo lo largo como palo, tabla, soga, alambre; p. ej.:}
\cholexample{Tyi jälol mi kaj jk'äñ jiñi tye'.}
\exampletranslation{Voy a ocupar todo el largo de ese palo.}

\entry{jältyäl}
\onedefinition{1}
\partofspeech{adv}
\spanishtranslation{así de largo}
\onedefinition{2}
\partofspeech{adj}
\spanishtranslation{largo}
\cholexample{Che'tyo jältyäl jiñi tye' tsa'bä ksek'e.}
\exampletranslation{Es bien largo el palo que tumbé.}

\entry{jäm}
\partofspeech{adv}
\spanishtranslation{rápidamente}
\clarification{tomar objeto}
\cholexample{Tsa' ijäm ch'ämä juloñib cha'añ mi ijul muty.}
\exampletranslation{Tomó su escopeta rápidamente para tirarle a un pájaro.}

\entry{jämchokoñ}
\partofspeech{vt}
\spanishtranslation{acostar}
\clarification{en una hamaca}
\cholexample{Yom mi ajämchokoñ aläl tyi ab.}
\exampletranslation{Debes acostar al niño en la hamaca.}

\entry{jämch'äm}
\partofspeech{vt}
\spanishtranslation{tomar, agarrar}
\clarification{un objecto rápidamente}

\entry{jämjämña}
\partofspeech{adj}
\spanishtranslation{mareado}
\cholexample{Jämjämña ijol cha'añ k'am.}
\exampletranslation{Está mareado por la enfermedad.}

\entry{jämtyäl}
\partofspeech{vi}
\spanishtranslation{acostarse}
\clarification{en hamaca}
\cholexample{Woli tyi jämtyäl tyi ab.}
\exampletranslation{Está acostado en la hamaca.}

\entry{jämts'uñ}
\partofspeech{vt}
\spanishtranslation{mecer}
\cholexample{Mi lakjämts'uñ xña' muty cha'añ ma'añix mi ipäktyäl.}
\exampletranslation{Mecemos a la gallina para que ya no se eche.}

\entry{jämuña}
\partofspeech{adv}
\spanishtranslation{meciéndose}
\cholexample{Jämuña tyi ab woli ik'aj iyo.}
\exampletranslation{Está descansando, meciéndose en la hamaca.}

\entry{jäñäkña}
\partofspeech{adv}
\spanishtranslation{zumbando}
\cholexample{Jäñäkña woli imajlel ichäñil chab cha'añ mi isäklañ iñich pimel.}
\exampletranslation{La abeja se va zumbando para buscar flores.}

\entry{jäpjäkña}
\partofspeech{adv}
\nontranslationdef{Se relaciona con la manera de soplar (susurrando); p. ej.:}
\cholexample{Jäpäkña jiñi ik' che' mi iyochelba' kalal jiñi otyoty.}
\exampletranslation{El viento entra susurrando donde está abierta la casa.}

\entry{jäp'}
\partofspeech{adv}
\nontranslationdef{Se relaciona con la forma de dar un machetazo; p. ej.:}
\cholexample{Tsa' ijäp' ty'ojbe ik'äb jiñi wiñik.}
\exampletranslation{Le cortó la mano al hombre de un machetazo.}

\entry{jäp'uña}
\partofspeech{adj}
\spanishtranslation{jadeando}
\cholexample{Jäp'uña jiñi aläl cha'añ woli tyi k'ajk.}
\exampletranslation{El niño está jadeando porque tiene fiebre.}

\entry{jäx}
\partofspeech{s}
\spanishtranslation{escarabajo ciervo}
\clarification{insecto}

\entry{jeb}
\partofspeech{vt}
\spanishtranslation{sacar}
\clarification{líquidos}

\entry{jek'}
\partofspeech{vt}
\spanishtranslation{picar, apuñalar}
\clarification{con cuchillo o aguja}

\entry{jejex}
\partofspeech{adv}
\nontranslationdef{Se relaciona con la forma de sacar una cosa larga; p. ej.:}
\cholexample{Woli ijejex tyujk'añ lok'el ibik'itye'lel iyotyoty.}
\exampletranslation{Está sacando las varillas de madera para su casa.}

\entry{jejmel}
\partofspeech{vi}
\spanishtranslation{descomponerse}
\cholexample{Tsa' jejmel jiñi bij cha'añ ja'lel.}
\exampletranslation{Ese camino se descompuso por los aguaceros.}

\entry{jejñex}
\partofspeech{adj}
\spanishtranslation{áspero}
\clarification{piedra, tabla}

\entry{jejtyel}
\partofspeech{vi}
\spanishtranslation{poner en el fuego}

\entry{jel}
\partofspeech{vi}
\spanishtranslation{turnar}
\cholexample{Yom mi la'jel la'bä tyi bij che' mi la'kuch majlel jiñi juñkujch kajpe'.}
\exampletranslation{Hay que turnarse en el camino cuando se lleva cargado el bulto de café.}

\entry{jelchojk}
\onedefinition{1}
\partofspeech{adj}
\spanishtranslation{distinto}
\cholexample{Jelchojk tyak its'ijbal pisil.}
\exampletranslation{Los colores de la tela son distintos.}
\onedefinition{2}
\partofspeech{adv}
\spanishtranslation{diferente}
\cholexample{Jelchojk mi ik'uxbiñ iyalobilob jiñi ityaty.}
\exampletranslation{El padre no ama igual a sus hijos porque los trata diferente.}

\entry{jelel}
\partofspeech{adj}
\spanishtranslation{desajustado}
\cholexample{Jelel tyak tsa' käle jiñi tye' che'bä tsa' tsejpi.}
\exampletranslation{La madera quedó desajustada cuando se cortó.}

\entry{jem}
\partofspeech{vt}
\onedefinition{1}
\spanishtranslation{destruir}
\cholexample{Jiñi yok ja' mi ijem jiñi bij.}
\exampletranslation{El corrental (corriente impetuosa) destruye el camino.}
\onedefinition{2}
\spanishtranslation{desbaratar}
\cholexample{Mi kaj kjem jiñi otyoty.}
\exampletranslation{Voy a desbaratar mi casa.}
\onedefinition{3}
\spanishtranslation{anular}
\clarification{un plan}
\cholexample{Tsa' ijemeyob jiñi plañ.}
\exampletranslation{Anularon el plan.}

\entry{jemachtyika}
\partofspeech{part}
\spanishtranslation{¡increíble!}
\clarification{expresión de sorpresa}

\entry{*jembal}
\partofspeech{s}
\spanishtranslation{demolición}

\entry{jeñk'extyik}
\partofspeech{adj}
\spanishtranslation{áspero}
\clarification{árbol}
\cholexample{Weñ jeñk'extyik jax ipaty jiñi tye'.}
\exampletranslation{La cáscara de ese árbol es muy áspera.}

\entry{jesuña}
\partofspeech{adv}
\spanishtranslation{fatigadamente}
\cholexample{Jesuña mi iletsel tyi wits jiñi wiñik.}
\exampletranslation{Ese hombre sube el cerro fatigadamente.}

\entry{jety}
\partofspeech{vt}
\spanishtranslation{poner}
\clarification{olla en el fuego}
\cholexample{Tsa' ijetye ip'ejtyal we'eläl tyi k'ajk.}
\exampletranslation{Puso la olla de carne en el fuego.}
\secondaryentry{jejtyel}
\secondpartofspeech{vi}
\secondtranslation{poner en el fuego}

\entry{jetyejty}
\partofspeech{s}
\spanishtranslation{pirinola}
\cholexample{Weñ yujil iyäsiñtyel ijetyejty jiñi alob.}
\exampletranslation{Ese niño sabe jugar bien su pirinola.}

\entry{jetyjetyña}
\partofspeech{adj}
\spanishtranslation{arrastrando}
\cholexample{Jetyjetyña tyi lum jiñi alob.}
\exampletranslation{El niño se está arrastrando en el suelo.}

\entry{jety'}
\partofspeech{adv}
\nontranslationdef{Se relaciona con el ruido que produce una tela o papel al romperse; p. ej.:}
\cholexample{Tsa' ijety' tsili pisil.}
\exampletranslation{Se oyó el ruido de la tela al romperse.}

\entry{jets}
\partofspeech{vt}
\spanishtranslation{bajar}
\clarification{asiento}
\cholexample{Jiñi ch'ilim mi ijets ibä ixixil che' jal mi iyajñel tyi vaso.}
\exampletranslation{Cuando el pinole se queda en el vaso, se asienta.}

\entry{jexbañ}
\partofspeech{vt}
\spanishtranslation{arrastrar}
\clarification{viga, persona, animal}
\cholexample{Mi kaj jkexbañ tyilel itye'el kotyoty.}
\exampletranslation{Voy a arrastrar la madera hacia mi casa.}

\entry{jexk'uñ}
\partofspeech{vt}
\spanishtranslation{arrastrar}
\clarification{viga, persona, animal}
\cholexample{Woli ijexk'uñ majlel imulacha'añ mi ichok.}
\exampletranslation{Está arrastrando su mula muerta para llevarla a tirar.}

\entry{jexekña}
\partofspeech{adv}
\spanishtranslation{arrastrándose}
\cholexample{K'uñtye' jexekña woli tyi majlel jiñi uchuchañ.}
\exampletranslation{Esa boa se va arrastrando despacio.}

\entry{jexjexña}
\partofspeech{adv}
\spanishtranslation{arrastrándose}
\clarification{en el suelo}
\cholexample{Jiñi machbä weñik iyok jexjexña mi imajlel.}
\exampletranslation{La persona inválida va arrastrándose por el suelo.}

\entry{jeyaj}
\partofspeech{imp}
\spanishtranslation{¡Oye tú!}

\entry{jik'tyañ}
\partofspeech{vt}
\spanishtranslation{ahogar}
\clarification{con comida}
\cholexample{Tsa' ijik'tya iwaj jiñi alob cha'añ woli tyi ty'añ che' woli tyi we'el.}
\exampletranslation{El chamaco se ahogó con su comida por hablar mientras comía.}

\entry{jichikña}
\partofspeech{adj}
\spanishtranslation{cantando}
\clarification{la chicharra}
\cholexample{Jichikña jiñi jichityiñ ya' tyi' mal chobalel.}
\exampletranslation{Está cantando la chicharra en medio de la rozadura.}

\entry{jichityiñ}
\partofspeech{s}
\spanishtranslation{cigarra}
\clarification{insecto}

\entry{jich'}
\partofspeech{vt}
\onedefinition{1}
\spanishtranslation{lazar}
\onedefinition{2}
\spanishtranslation{colgar}
\cholexample{Mi lakjich' lámpara tyi ch'ajañ.}
\exampletranslation{Colgamos la lámpara con mecate.}

\entry{jich'bil}
\partofspeech{adj}
\spanishtranslation{colgado}
\clarification{por alguien}

\entry{jich'chokobil}
\partofspeech{adj}
\spanishtranslation{colgado}
\clarification{libremente}
\cholexample{Jich'chokobil jump'ejlbalde lew tyi yojlil imal otyoty.}
\exampletranslation{Un balde de manteca está colgado en medio de la casa.}

\entry{jich'chokoñ}
\partofspeech{vt}
\spanishtranslation{colgar}
\clarification{libremente}
\cholexample{Mi lakjich'chokoñ lámpara tyi yojlil imal otyoty.}
\exampletranslation{Colgamos la lámpara en medio de la casa.}

\entry{jich'il}
\partofspeech{adj}
\spanishtranslation{colgado}
\clarification{cosa chica}

\entry{jich'tyäl}
\partofspeech{vi}
\spanishtranslation{colgarse}

\entry{jich'ye'}
\partofspeech{vt}
\spanishtranslation{llevar guindado de la mano}
\cholexample{Che' mi imajlel tyi ja' jiñi x'ixik mi ijich'ye' majlel jiñibalde.}
\exampletranslation{Cuando la mujer va al arroyo, lleva el balde guindado de la mano.}

\entry{jich'ye'el}
\partofspeech{adj}
\spanishtranslation{guindado de la mano}
\cholexample{Jich'ye'el icha'añ ibalde tsa' majli tyi ja' jiñi x'ixik.}
\exampletranslation{La mujer se fue al arroyo con su balde colgando de la mano.}

\entry{jijch}
\partofspeech{s}
\spanishtranslation{chicharra}
\clarification{insecto}

\entry{jijik' ojbal}
\spanishtranslation{tos ferina}

\entry{jijlel}
\partofspeech{vi}
\spanishtranslation{descansar}
\clarification{por la noche}
\cholexample{Tsa' k'otyi tyi jijlel tyi pam xajlel.}
\exampletranslation{Llegó a descansar donde hay una piedra.}

\entry{-jijlel}
\nontranslationdef{Sufijo numeral para contar la distancia entre dos puntos de descanso; p. ej.:}
\cholexample{Juñjijlel tsa' ixäñä.}
\exampletranslation{Anduvo la distancia entre dos puntos de descanso.}

\entry{jijleñ}
\partofspeech{imp}
\spanishtranslation{¡descansa!}

\entry{jijlibäl}
\partofspeech{s}
\spanishtranslation{descansadero}
\clarification{en camino}
\cholexample{Tsa'ix laktyaja jijlibäl.}
\exampletranslation{Ya hemos llegado al descansadero.}

\entry{jijtye'ol}
\partofspeech{s}
\spanishtranslation{arboleda de avellano}

\entry{jijts'el}
\partofspeech{vi}
\spanishtranslation{sufrir luxación}
\clarification{al tener un movimiento violento de las coyunturas de los huesos}
\cholexample{Tsa' jijts'i ibäkel kya' che'bä tsak cha'le ajñel.}
\exampletranslation{Cuando corrí sufrí una luxación en mi cadera.}

\entry{jilel}
\partofspeech{vi}
\spanishtranslation{terminarse}

\entry{jilib}
\partofspeech{s}
\spanishtranslation{terminación}
\secondaryentry{ijilibal jabil}
\secondtranslation{fin del año}
\secondaryentry{ijilibal semaña}
\secondtranslation{sábado}

\entry{jimba}
\relevantdialect{Tila}
\partofspeech{s}
\spanishtranslation{bambú}
\alsosee{chejp}

\entry{jimbä}
\partofspeech{pron}
\spanishtranslation{éste}

\entry{jiñkuyi}
\partofspeech{adv}
\onedefinition{1}
\spanishtranslation{sí}
\clarification{afirmativo}
\onedefinition{2}
\spanishtranslation{eso es}

\entry{jiñkwäyi}
\relevantdialect{Sab.}
\partofspeech{adv}
\onedefinition{1}
\spanishtranslation{sí}
\clarification{afirmativo}
\onedefinition{2}
\spanishtranslation{eso es}

\entry{jiñi}
\onedefinition{1}
\partofspeech{pron}
\spanishtranslation{él, ella, ése, ésa, éste, ésta}
\onedefinition{2}
\partofspeech{artículo}
\spanishtranslation{el}
\spanishtranslation{la}

\entry{jiñikña}
\partofspeech{adj}
\spanishtranslation{gruñendo}
\cholexample{Jiñikña jiñi ts'i' che' yom ik'uxoñla.}
\exampletranslation{El perro está gruñendo porque nos quiere morder.}

\entry{jiñi jach}
\spanishtranslation{sólo ése}

\entry{jiñjiñi}
\partofspeech{pron}
\spanishtranslation{ese, es aquél}

\entry{jiñtyo}
\partofspeech{prep}
\spanishtranslation{hasta}
\cholexample{Jiñtyo tyi yambä jabil miktyojbeñety kbety.}
\exampletranslation{Te pago mi deuda hasta el otro año.}

\entry{jiñäch}
\partofspeech{adv}
\spanishtranslation{eso es, ése es}

\entry{jis}
\partofspeech{s}
\spanishtranslation{creta}

\entry{jisañ}
\partofspeech{vt}
\onedefinition{1}
\spanishtranslation{destruir}
\cholexample{Tsa' ujtyi ijisañ otyoty jiñi ik'.}
\exampletranslation{El viento acaba de destruir la casa.}
\onedefinition{2}
\spanishtranslation{matar}
\cholexample{Mi kaj kjisañ ityi'lel jiñi wiñik.}
\exampletranslation{Voy a matar a ese hombre.}

\entry{*jisälel}
\partofspeech{s}
\spanishtranslation{daño}
\cholexample{Añ ijisälel mi tsa' laj k'uxu lew che' woli lakjap ts'ak.}
\exampletranslation{Se dice que hace daño comer manteca cuando estamos tomando medicina.}

\entry{jisil}
\partofspeech{adj}
\spanishtranslation{prohibido}

\entry{jity}
\partofspeech{vt}
\spanishtranslation{desatar}

\entry{jitybil}
\partofspeech{adj}
\spanishtranslation{desatado}
\cholexample{Tsa' puts'i jiñi mulakome jitybil tsa' jkäyä.}
\exampletranslation{La mula huyó porque la dejé desatada.}

\entry{jity'bil}
\partofspeech{adj}
\spanishtranslation{amarrado}
\clarification{cerco, puente}

\entry{jits'kuyel}
\partofspeech{vi}
\spanishtranslation{descoyuntarse}
\cholexample{Tsa' ujtyi tyi jits'kuyel ik'äb.}
\exampletranslation{Acaba de descoyuntarse el brazo.}

\entry{jits'kwäyel}
\relevantdialect{Sab.}
\partofspeech{vi}
\spanishtranslation{desmayarse}
\cholexample{Tyi jits'kwäyi jiñi ajk'am'añ.}
\exampletranslation{Se desmayó el enfermo.}

\entry{ji'}
\partofspeech{s}
\spanishtranslation{arena}

\entry{ji'il}
\defsuperscript{1}
\partofspeech{s}
\spanishtranslation{arenal}

\entry{ji'il}
\defsuperscript{2}
\onedefinition{1}
\partofspeech{adv}
\spanishtranslation{en formación}
\cholexample{Ji'ilob tsa' icha'leyob marcha jiñi wiñikob che' tyi k'iñ.}
\exampletranslation{El día de la fiesta, los hombres desfilaron en formación.}
\onedefinition{2}
\partofspeech{adj}
\spanishtranslation{tendido}
\clarification{en el suelo}
\cholexample{Ji'ilix jiñi kajpe' tyi k'iñ.}
\exampletranslation{El café ya está tendido en el sol.}

\entry{ji'ilob}
\partofspeech{adj}
\spanishtranslation{en filas}
\cholexample{Ji'ilob jiñi wiñikob ya tyi kale.}
\exampletranslation{Los hombres están en filas en la calle.}

\entry{ji'lumil}
\partofspeech{s}
\spanishtranslation{nombre de lugar arenoso}

\entry{ji'tye'ol}
\partofspeech{s}
\onedefinition{1}
\spanishtranslation{hamaca hecha de palitos}
\onedefinition{2}
\spanishtranslation{camino compuesto por palos atravesados}

\entry{Jobel}
\partofspeech{s}
\spanishtranslation{San Cristóbal de las Casas}

\entry{jobeñ}
\partofspeech{s}
\spanishtranslation{tablero}
\clarification{de moler}
\cholexample{Pek'atyax ijobeñba' mi ijuch' isa'.}
\exampletranslation{Está muy bajo el tablero donde muele el pozol.}

\entry{jobeñtye' xux}
\partofspeech{s}
\spanishtranslation{avispa polistes}

\entry{joboñ}
\relevantdialect{Tila}
\partofspeech{adj}
\spanishtranslation{mucho}
\cholexample{Joboñ orakióñ.}
\exampletranslation{Reza mucho.}
\alsosee{kabäl}

\entry{joktyäl}
\partofspeech{s}
\onedefinition{1}
\spanishtranslation{planada}
\cholexample{Mi iweñ mejlel ixim ya' tyi joktyäl.}
\exampletranslation{Se da bien el maíz en las planadas.}
\onedefinition{2}
\spanishtranslation{valle}
\cholexample{Ya' tyi joktyäl mi kaj kmel kotyoty.}
\exampletranslation{Voy a hacer mi casa en el valle.}
\secondaryentry{ijoktyälel}
\secondtranslation{su planada, su valle}

\entry{jok'}
\defsuperscript{1}
\partofspeech{vt}
\spanishtranslation{sacar}
\clarification{con la mano}
\cholexample{Mi lakjok' lok'el chab tyi mal tye'.}
\exampletranslation{Sacamos (con la mano) la miel del árbol.}

\entry{jok'}
\defsuperscript{3}
\partofspeech{s}
\spanishtranslation{pozo}

\entry{jok'}
\defsuperscript{2}
\partofspeech{vt}
\spanishtranslation{colgar}
\clarification{contra una pared}
\cholexample{Pejtyelel ora ya' mik jok' kpixol tyi lawux.}
\exampletranslation{Siempre cuelgo mi sombrero en el clavo.}

\entry{jok'chokoñ}
\partofspeech{vt}
\spanishtranslation{colgar}
\clarification{contra una pared}
\cholexample{Mi lakjok'chokoñ morral tyi lawux ts'äpälbä tyi oy.}
\exampletranslation{Colgamos nuestro morral en un clavo sembrado en el horcón.}

\entry{*jok'lib}
\partofspeech{s}
\spanishtranslation{garabato, gancho para colgar cosas, clavo para colgar cosas}

\entry{jok'ol}
\partofspeech{adj}
\spanishtranslation{colgado}
\cholexample{Jok'ol tyak jiñi bujkäl ya' tyi choñoñibäl.}
\exampletranslation{Las camisas están colgadas en las tiendas.}

\entry{joch}
\partofspeech{vt}
\spanishtranslation{quitar}
\clarification{ropa}
\cholexample{Che' mi lakcha'leñ ts'ämel mi lakjoch lakpislel.}
\exampletranslation{Cuando nos bañamos nos quitamos nuestra ropa.}

\entry{jochitye'}
\partofspeech{s}
\spanishtranslation{guacamayo}
\clarification{árbol}

\entry{jochokña}
\partofspeech{adj}
\spanishtranslation{desocupado}
\cholexample{Jochokña tsa' käle jiñi otyoty cha'añ tsa' majli iyum.}
\exampletranslation{Esa casa se quedó desocupada porque se fue el dueño.}

\entry{jochojch}
\partofspeech{s}
\spanishtranslation{yulo}
\clarification{tipo de gusano}

\entry{jochol}
\partofspeech{adj}
\onedefinition{1}
\spanishtranslation{vacío}
\cholexample{Jochol jiñi latya.}
\exampletranslation{La lata está vacía.}
\onedefinition{2}
\spanishtranslation{desocupado}
\cholexample{Jochol jiñi otyoty.}
\exampletranslation{Esa casa está desocupada.}
\secondaryentry{jocholbä lum}
\secondtranslation{terreno desocupado}

\entry{jochtyesañ}
\partofspeech{vt}
\spanishtranslation{desocupar}
\cholexample{Wersa mi lakjochtyesañ lakotyoty che' mi iñumel jiñi xmul otyoty.}
\exampletranslation{Es necesario desocupar la casa cuando pase el rociador.}

\entry{jochtyiyel}
\partofspeech{vi}
\spanishtranslation{desocuparse}
\cholexample{Jiñi otyoty mux ikaj tyi jochtyiyel kome samix iyum.}
\exampletranslation{Esa casa ya se va a desocupar porque ya se va su dueño.}

\entry{joch'}
\defsuperscript{1}
\partofspeech{vt}
\spanishtranslation{inyectar}

\entry{joch'}
\defsuperscript{3}
\partofspeech{s}
\spanishtranslation{maíz picado}
\secondaryentry{ijoch'il}
\secondtranslation{lo picado del maíz}

\entry{joch'}
\defsuperscript{2}
\partofspeech{vt}
\spanishtranslation{bordar}
\clarification{tela}

\entry{joch'añ}
\partofspeech{vi}
\spanishtranslation{picarse}
\clarification{maíz}

\entry{joch' lo' patye'}
\spanishtranslation{hongo}
\clarification{de árboles, blanco, comestible}

\entry{jojchel}
\partofspeech{vi}
\onedefinition{1}
\spanishtranslation{caer}
\clarification{el calzón}
\cholexample{Mi kaj ijojchel iwex jiñi ch'ityoñ kome ma'añik ikajchiñäk'.}
\exampletranslation{Se le va a caer el pantalón a ese chamaco porque no tiene cinturón.}
\onedefinition{2}
\spanishtranslation{quitarse}
\clarification{ropa}
\cholexample{Yom ijojchel ibujk.}
\exampletranslation{Quiere quitarse la camisa.}

\entry{jojmañ}
\partofspeech{vt}
\spanishtranslation{comer demasiado}
\cholexample{Woli ijojmañ bu'ul. mach jasälix icha'añ ipi'äl.}
\exampletranslation{Está comiendo demasiado frijol. No hay suficiente para su esposa.}

\entry{jojmay}
\partofspeech{s}
\spanishtranslation{garza}

\entry{*jojmäñtyel}
\partofspeech{s}
\spanishtranslation{gran consumo}
\clarification{de alimento}
\cholexample{Kabäläch tsa' ujtyi ijojmäñtyel bu'ul ilayi.}
\exampletranslation{Hubo un gran consumo de frijol aquí.}

\entry{jojmel}
\partofspeech{vi}
\spanishtranslation{acabar, acabarse}
\cholexample{Ts'ityajax yom jojmel kchobal.}
\exampletranslation{Ya falta poco para que acabe mi rozadura.}

\entry{-jojp}
\nontranslationdef{Sufijo numeral para contar puñados; p. ej.:}
\cholexample{Jiñi kerañ tsa' iyäk'eyoñ juñjojp galetya.}
\exampletranslation{Mi hermano me dio un puñado de galletas.}

\entry{jojp'el}
\partofspeech{vi}
\spanishtranslation{acusar falsamente}
\cholexample{Woli ijojp'el imul jiñi wiñik.}
\exampletranslation{Se está acusando falsamente a ese hombre.}

\entry{jojyel}
\partofspeech{vi}
\spanishtranslation{avanzar}
\cholexample{Ma'añik mi ijojyel icha'añ iye'tyel.}
\exampletranslation{No puede avanzar con su trabajo.}

\entry{-jojyel}
\nontranslationdef{Sufijo numeral para medir la circunferencia; p. ej.:}
\cholexample{Juñjojyel jach mi lakcha'leñ ajñel tyi' yebal jiñi tye'.}
\exampletranslation{Solamente corremos una vuelta alrededor del árbol.}

\entry{*jol}
\partofspeech{s}
\spanishtranslation{cabeza}
\secondaryentry{*jol chu'}
\secondtranslation{teta}
\secondaryentry{*jol ik'aba'}
\secondtranslation{apellido}
\secondaryentry{*jol ixim}
\secondtranslation{pelusa}
\secondaryentry{*jol otyoty}
\secondtranslation{techo}
\secondtranslation{caballete}
\secondaryentry{*jol tyak'iñ}
\secondtranslation{interés}
\clarification{de dinero}
\secondaryentry{*jol ya'}
\secondtranslation{cadera}

\entry{Jolako}
\relevantdialect{Tila}
\partofspeech{s}
\onedefinition{1}
\spanishtranslation{nombre de colonia}
\onedefinition{2}
\spanishtranslation{tipo de árbol}

\entry{Jolaktye'pa'}
\relevantdialect{Tila}
\partofspeech{s}
\secondtranslation{nombre de colonia}
\clarification{cabeza y arboleda de palmas}

\entry{Joljamil}
\partofspeech{s}
\spanishtranslation{Cabeza del Zacatal}
\clarification{colonia}

\entry{Jolja'}
\partofspeech{s}
\spanishtranslation{Cabeza del Arroyo}
\clarification{colonia}

\entry{Jolñopa'}
\relevantdialect{Tila}
\partofspeech{s}
\spanishtranslation{Cabeza del Arroyo}
\clarification{colonia}

\entry{jolokña}
\partofspeech{adv}
\spanishtranslation{arrastrando}
\cholexample{Jolokña majlel jiñi lukum.}
\exampletranslation{La culebra se va arrastrando.}

\entry{joloñtyesañ}
\partofspeech{vt}
\spanishtranslation{terminar}
\clarification{un trabajo}
\cholexample{Yom mi ajoloñtyesañ awe'tyel ya' tyi chañ.}
\exampletranslation{Tienes que dejar terminado tu trabajo allí arriba.}

\entry{Joloñel}
\partofspeech{s}
\spanishtranslation{nombre de colonia}
\clarification{lit.: terminación}

\entry{joloñel}
\partofspeech{vi}
\spanishtranslation{terminarse}
\cholexample{Wolix ijoloñel majlel jiñi e'tyel.}
\exampletranslation{El trabajo está terminándose.}

\entry{Jolpañchil}
\partofspeech{s}
\spanishtranslation{nombre de colonia}

\entry{Jolwits}
\partofspeech{s}
\spanishtranslation{Cabeza del Cerro}
\clarification{colonia}

\entry{jom}
\partofspeech{s}
\spanishtranslation{granos podridos}
\clarification{mazorca}
\cholexample{Añ ijom jiñi juñts'ijty ixim.}
\exampletranslation{Esa mazorca tiene granos podridos.}

\entry{*jomil}
\partofspeech{s}
\spanishtranslation{mucho maíz podrido}
\cholexample{Ma'añik tsa' aweñ mejli kchol kome kabäl tsa' lok'i ijomil.}
\exampletranslation{Mi milpa no se dio bien porque salieron muchas mazorcas podridas.}

\entry{jomjomña}
\partofspeech{adv}
\nontranslationdef{Se relaciona con el movimiento de una muchedumbre de hombres, animales o insectos; p. ej.:}
\cholexample{Jomjomñayob mi iñumelob wiñikob x'ixikob ya' tyi tyejklum che' tyi' yoralel k'iñ.}
\exampletranslation{Hay una muchedumbre de gente moviéndose por el pueblo en el día de la fiesta.}

\entry{jomokña}
\partofspeech{adv}
\spanishtranslation{moviéndose}
\clarification{muchedumbre de hombres, animales o insectos}

\entry{jomojch'}
\partofspeech{s}
\spanishtranslation{joloche}
\spanishtranslation{cáscara del maíz}

\entry{jomol}
\partofspeech{adj}
\spanishtranslation{amontonado}
\clarification{gente}
\cholexample{Ya' jomol wiñikob tyi' yotyoty komisariado.}
\exampletranslation{Los hombres están amontonados allí, en la casa del comisariado.}

\entry{jomtyäl}
\partofspeech{adv}
\spanishtranslation{amontonados así}
\clarification{señalando personas o animales}
\cholexample{Che' jomtyäl wiñikob woli ik'elob jiñi xch'ujleläl.}
\exampletranslation{Los hombres están amontonados así, viendo el cadáver.}

\entry{joñtyol}
\relevantdialect{Sab.}
\partofspeech{adj}
\spanishtranslation{malo}
\cholexample{Joñtyol lakts'i' kome mach yomik ik'el lakjula'.}
\exampletranslation{Nuestro perro es malo porque no quiere que lleguen visitas.}

\entry{*joñtyolel}
\relevantdialect{Sab.}
\partofspeech{s}
\spanishtranslation{su maldad}

\entry{*joñtyolil}
\partofspeech{s}
\spanishtranslation{su maldad}

\entry{joñochañ}
\partofspeech{s}
\spanishtranslation{joñon}
\clarification{insecto que hace su nido en la tierra y que pica fuerte}

\entry{joñoñ}
\partofspeech{pron}
\spanishtranslation{yo}

\entry{jop}
\defsuperscript{1}
\relevantdialect{Sab.}
\partofspeech{vt}
\spanishtranslation{tratar}

\entry{jop}
\defsuperscript{2}
\partofspeech{vt}
\spanishtranslation{juntar}
\clarification{una cosa seca}

\entry{jopjopña}
\partofspeech{adv}
\nontranslationdef{Se relaciona con el movimiento de gusanos u hormigas; p. ej.:}
\cholexample{Jopjopña mi iñumel motso' tyi yojlil bij.}
\exampletranslation{Hay un montón de gusanos moviéndose en el camino.}

\entry{jop ochel}
\partofspeech{vt}
\spanishtranslation{juntar}
\clarification{café, frijol, maíz}
\cholexample{Muk'ix lakjop ochel jiñi kajpe' che' tyikiñix.}
\exampletranslation{Vamos a juntar el café cuando ya esté seco.}

\entry{joptyäl}
\partofspeech{adv}
\spanishtranslation{así de medida como un puñado}
\cholexample{Che' ya joptyäl bu'ul tsa' iyäk'eyoñ.}
\exampletranslation{Nada más así me dio un puñado de frijol.}

\entry{jop'}
\defsuperscript{1}
\partofspeech{vt}
\spanishtranslation{amarrar}
\clarification{enaguas}

\entry{jop'}
\defsuperscript{2}
\partofspeech{vt}
\spanishtranslation{acusar}

\entry{jop'ty'añ}
\partofspeech{s}
\spanishtranslation{chisme}
\cholexample{Kabäl mi imulañ jop'ty'añ jiñi x'ixik.}
\exampletranslation{A esa mujer le gusta mucho el chisme.}

\entry{joty}
\partofspeech{vt}
\spanishtranslation{cortar}
\clarification{con machete}

\entry{joty'}
\partofspeech{vt}
\spanishtranslation{rascar}
\clarification{la cabeza}

\entry{jots'}
\partofspeech{vt}
\onedefinition{1}
\spanishtranslation{arrancar}
\cholexample{Ora jach mi lakjots' lok'el postye che' mach tyamik ts'äpbil.}
\exampletranslation{Cuando el poste no está enterrado muy hondo, fácilmente lo arrancamos.}
\onedefinition{2}
\spanishtranslation{desenvainar}
\cholexample{Mach jalik mi'bäk' jots' imachity jiñi wiñik che' yom itsepoñla.}
\exampletranslation{Ese hombre desenvaina rápido su machete cuando nos quiere machetear.}
\onedefinition{3}
\spanishtranslation{sacar}
\clarification{diente}
\cholexample{Wokol mi lakjots' ibäkel lakej che' tsätstyo añ.}
\exampletranslation{Es algo difícil sacar el diente cuando todavía está duro.}

\entry{jots'amäy}
\partofspeech{s}
\spanishtranslation{cerbatana}
\clarification{hecha de carrizo}

\entry{jots'bil}
\partofspeech{adj}
\spanishtranslation{arrancado, sacado, extraído}
\clarification{dientes, postes}
\cholexample{Laj joty'bil ibäkel iyej jiñi x'ixik.}
\exampletranslation{Todos los dientes de esa mujer han sido extraídos.}

\entry{joy}
\onedefinition{1}
\partofspeech{adv}
\nontranslationdef{Se relaciona con la forma de cercar; p. ej.:}
\cholexample{Tsa' ijoy mäkä tyi ch'ix tyak'iñ ipotyrero jiñi wiñik.}
\exampletranslation{Ese hombre cercó con alambre su potrero.}
\onedefinition{2}
\partofspeech{vt}
\spanishtranslation{rodear}
\cholexample{Mi ijoyob chityam.}
\exampletranslation{Rodean el puerco.}
\secondaryentry{joy bij}
\secondtranslation{vuelta de camino}

\entry{Joyetya'}
\partofspeech{s}
\spanishtranslation{nombre de colonia}

\entry{joyokña}
\partofspeech{adv}
\spanishtranslation{rodeando}
\cholexample{Joyokña mi iñumel tyi xchumtyäl jiñi xchoñoñel.}
\exampletranslation{El comerciante pasa rodeando toda la ranchería.}
\secondaryentry{joyokñabä bij}
\secondtranslation{vuelta}
\clarification{desviación}

\entry{joyol}
\partofspeech{adj}
\spanishtranslation{circundado}
\cholexample{Joyol tyi wits jiñi yajalóñ.}
\exampletranslation{Yajalón está circundado por cerros.}
\secondaryentry{joyolbä ja'}
\secondtranslation{remolino}
\clarification{agua}
\secondaryentry{joyol bij}
\secondtranslation{vuelta}
\clarification{desviación}

\entry{Joyo'ja'}
\partofspeech{s}
\spanishtranslation{Circundado por Agua}
\clarification{colonia}

\entry{*joytyilel}
\partofspeech{s}
\spanishtranslation{alrededores}
\cholexample{Añ kajpe'lel tyi' joytyilel xchumtyäl.}
\exampletranslation{En los alrededores de la ranchería hay cafetales.}

\entry{joyxujk}
\partofspeech{adj}
\spanishtranslation{cuadrado}
\cholexample{Joyxujk tsak mele kotyoty.}
\exampletranslation{Hice mi casa cuadrada.}

\entry{jo'}
\partofspeech{vt}
\spanishtranslation{lavar}
\clarification{cabeza}
\cholexample{Tsa'tyo ñaxañ majli ijo' ijol.}
\exampletranslation{Primero se fue a lavar la cabeza.}

\entry{jo'jo'p'ejl}
\partofspeech{adj}
\spanishtranslation{cada cinco}

\entry{jo'lujump'ejl}
\partofspeech{adj}
\spanishtranslation{quince}

\entry{jo'lujuñk'al}
\partofspeech{adj}
\spanishtranslation{trescientos}

\entry{jo'ñij}
\partofspeech{adv}
\spanishtranslation{quinto día del presente}

\entry{jo'ñal}
\partofspeech{s}
\onedefinition{1}
\spanishtranslation{interior del cuerpo, estómago, abdomen}
\onedefinition{2}
\spanishtranslation{interior de un árbol}

\entry{jo'ox}
\partofspeech{s}
\spanishtranslation{achiote}
\clarification{árbol que produce una fruta que es condimento}

\entry{jo'p'ejl}
\partofspeech{adj}
\spanishtranslation{cinco}

\entry{Jo'xil}
\partofspeech{s}
\spanishtranslation{nombre de colonia}

\entry{jubel}
\partofspeech{vi}
\spanishtranslation{bajar}
\cholexample{Che' mi laklok'el tyi tyumbalá mi lakjubel majlel k'äläl tyi hidalgo.}
\exampletranslation{Al salir de Tumbalá bajamos hasta llegar a Hidalgo.}

\entry{jubeñ}
\partofspeech{adj}
\spanishtranslation{bajo}
\clarification{precios}
\cholexample{Weñ jubeñ ityojol kajpe'.}
\exampletranslation{El precio del café está muy bajo.}

\entry{juk}
\defsuperscript{1}
\partofspeech{adv}
\spanishtranslation{así}
\clarification{señalando la forma de meter algo}
\cholexample{Jiñi aläl tsa' ijuk otsa ik'äb ya' tyi' p'ejtyal we'eläl.}
\exampletranslation{El niño metió la mano así en la olla de carne.}

\entry{juk}
\defsuperscript{2}
\partofspeech{s}
\spanishtranslation{tipo de insecto}

\entry{jukub}
\partofspeech{s}
\spanishtranslation{cayuco}

\entry{juk'}
\defsuperscript{1}
\partofspeech{vt}
\onedefinition{1}
\spanishtranslation{cepillar}
\clarification{madera, diente}
\cholexample{Tsa'ix kñopo juk' tye'.}
\exampletranslation{Ya aprendí a cepillar madera.}
\onedefinition{2}
\spanishtranslation{afilar}
\cholexample{Yom mi ajuk' amachity.}
\exampletranslation{Debes afilar tu machete.}

\entry{juk'}
\defsuperscript{2}
\partofspeech{s}
\spanishtranslation{quequeste}
\clarification{planta}

\entry{juk'o'ejäl}
\partofspeech{s}
\spanishtranslation{cepillo dental}

\entry{juk'o' pisil}
\spanishtranslation{plancha}

\entry{juk'utyuñ}
\partofspeech{s}
\spanishtranslation{piñanona}
\clarification{bejuco}
\dialectvariant{Sab., Tila}
\dialectword{poñch'ox}

\entry{juk'xiñ}
\partofspeech{vt}
\spanishtranslation{restregar}
\cholexample{Jiñi mulawoli ijuk'xiñ ipaty ya' tyi tye'.}
\exampletranslation{La mula se está restregando la espalda en el árbol.}

\entry{juchtyek'}
\partofspeech{vt}
\spanishtranslation{no cumplir}
\cholexample{Jiñi wiñik mu' jach ijuchtyek'beñ imañdar jiñi yumälob.}
\exampletranslation{Ese hombre no cumple las leyes de las autoridades.}

\entry{juch'}
\partofspeech{vt}
\spanishtranslation{moler}
\clarification{maíz, café}

\entry{juch'bal}
\partofspeech{s}
\spanishtranslation{proceso de moler}
\clarification{maíz o café}
\cholexample{Jiñi x'ixik mi iweñ mulañ juch'bal.}
\exampletranslation{A esa mujer le gusta (el proceso de) moler.}

\entry{juch'bil}
\partofspeech{adj}
\spanishtranslation{molido}
\clarification{maíz, café}
\cholexample{Ma'añix mi kajel tyi juch'bal kome juch'bilix iwaj.}
\exampletranslation{Ya no va a moler porque ya está molido su nixtamal.}

\entry{juch'oñibäl}
\partofspeech{s}
\spanishtranslation{metate}

\entry{judas}
\relevantdialect{Sab.}
\partofspeech{s}
\spanishtranslation{traicionero, diablo}

\entry{judío}
\relevantdialect{Sab.}
\partofspeech{s}
\spanishtranslation{diablo}
\cholexample{Woli iyäl jiñi judío tsa' itsäñsa laktyaty jesús.}
\exampletranslation{Dicen que el judío (diablo) mató a nuestro Padre Jesús.}

\entry{juj}
\defsuperscript{1}
\partofspeech{s}
\spanishtranslation{árbol de madera blanca y suave}
\clarification{se chupa la fruta que es dulce}

\entry{juj}
\defsuperscript{2}
\partofspeech{s}
\spanishtranslation{iguana}
\clarification{reptil}

\entry{jujk'uñ}
\partofspeech{vt}
\spanishtranslation{embarrar}
\cholexample{Jiñi alob tsa' ijujk'u ibä tyi ok'ol.}
\exampletranslation{El niño se embarró de lodo.}

\entry{jujch}
\partofspeech{s}
\spanishtranslation{concha}

\entry{jujchiñ}
\partofspeech{vt}
\onedefinition{1}
\spanishtranslation{raspar}
\cholexample{Che' ujtyemix lapits' jiñi chityam mi lakjujchibeñ itsutsel.}
\exampletranslation{Al terminar de chamuscar al cerdo raspamos el pelo.}
\onedefinition{2}
\spanishtranslation{afeitar}
\cholexample{Wersa mik jujchiñ ktsuktyi'.}
\exampletranslation{Es necesario que me afeite.}

\entry{jujl p'ok}
\partofspeech{s}
\spanishtranslation{iguana}
\clarification{reptil}

\entry{jujp'el}
\partofspeech{vi}
\spanishtranslation{engordarse}

\entry{jujp'em}
\partofspeech{adj}
\spanishtranslation{gordo}

\entry{jujujña}
\partofspeech{adv}
\spanishtranslation{bramando}
\cholexample{Jujujña mi imajlel jiñibajlum tyi tye'el.}
\exampletranslation{El jaguar va bramando por el bosque.}

\entry{jujump'ejl}
\partofspeech{adj}
\spanishtranslation{uno por uno, cada uno}
\cholexample{Lujump'ejl keñtyavo ityojol jujump'ejl alaxax.}
\exampletranslation{El precio de cada naranja es de diez centavos.}
\secondaryentry{jujump'ejl k'iñ}
\secondtranslation{cada día}

\entry{jujuñchajp}
\partofspeech{adj}
\spanishtranslation{cada clase, toda clase}
\cholexample{Tsa' imäñä jujuñchajp ipak' tyi garitya.}
\exampletranslation{Compró toda clase de semilla en la garita.}

\entry{jujuñyajl}
\partofspeech{adv}
\spanishtranslation{cada vez}
\cholexample{Jujuñyajl che' mik majlel tyi cholel mik tyaj iyejtyal iyok ejmech.}
\exampletranslation{Cada vez que voy a mi milpa encuentro el rastro de los pies del mapache.}

\entry{jul}
\partofspeech{vt}
\spanishtranslation{tirar}
\clarification{con escopeta, piedra, tirador}

\entry{jula'}
\partofspeech{s}
\onedefinition{1}
\spanishtranslation{visitante}
\cholexample{Sajmäl tsa' juliyob kabäl xjula'.}
\exampletranslation{Hoy vinieron a mí muchos visitantes.}
\onedefinition{2}
\spanishtranslation{huésped}
\cholexample{Jiñi xjula'jiñiäch juñtyikil maestyro.}
\exampletranslation{El huésped es un maestro.}

\entry{jula'al}
\relevantdialect{Tila}
\partofspeech{s}
\spanishtranslation{visita}
\cholexample{Mux klok'el tyi jula'al.}
\exampletranslation{Voy a salir para hacer una visita.}

\entry{jula'añ}
\relevantdialect{Sab.}
\partofspeech{vt}
\spanishtranslation{visitar}

\entry{jula'tyañ}
\partofspeech{vt}
\spanishtranslation{visitar}

\entry{julbäl}
\partofspeech{s}
\spanishtranslation{cacería}
\cholexample{Koñlatyi julbäl.}
\exampletranslation{Vamos a la cacería (lit.: tirando).}

\entry{julel}
\partofspeech{vi}
\spanishtranslation{llegar}
\clarification{acá}

\entry{julio}
\relevantdialect{Tila}
\partofspeech{s}
\spanishtranslation{julio}
\culturalinformation{Información cultural: Dicen que es el mes de los judíos y del diablo, y que no se debe sembrar plátano o café.}

\entry{juloñib}
\partofspeech{s}
\spanishtranslation{escopeta, rifle, arma}

\entry{jultyesañ i pusik'al}
\spanishtranslation{corregir}
\cholexample{Jujump'ejl k'iñ woli ijultyesañ ipusik'al iyalobil.}
\exampletranslation{Cada día está corrigiendo a su hijo.}

\entry{-jumpaty}
\partofspeech{adj}
\spanishtranslation{afuera}

\entry{jump'ejl}
\partofspeech{adj}
\spanishtranslation{uno}
\secondaryentry{jump'ejl icha'k'al}
\secondtranslation{veintiuno}

\entry{-jump'ejlel}
\partofspeech{adj}
\spanishtranslation{todo, cosa entera}
\cholexample{Tyi jump'ejlel ipusik'al tsa' iña'tya ipäy iyijñam.}
\exampletranslation{Pensó con todo su corazón en casarse.}

\entry{jumuk'}
\partofspeech{adv}
\spanishtranslation{ratito}
\cholexample{Jumuk'tyo mi kaj kmajlel awik'oty.}
\exampletranslation{En un ratito voy contigo.}

\entry{juñ}
\defsuperscript{1}
\partofspeech{s}
\onedefinition{1}
\spanishtranslation{papel}
\cholexample{Mi ik'äñ juñ cha'añ mi ipix asukal.}
\exampletranslation{Se usa papel para envolver azúcar.}
\onedefinition{2}
\spanishtranslation{libro}
\cholexample{Wolityo tyi k'e'l juñ tyi eskuelo.}
\exampletranslation{Todavía está estudiando libros en la escuela.}
\onedefinition{3}
\spanishtranslation{carta}
\cholexample{Mi kaj kts'ijbañ majlel juñ.}
\exampletranslation{Voy a escribir una carta para enviarla.}

\entry{juñ}
\defsuperscript{2}
\partofspeech{s}
\spanishtranslation{amate}
\spanishtranslation{matapalo}
\spanishtranslation{higuero}
\clarification{árbol}
\cholexample{Weñ kolem mi ikolel jiñi juñ.}
\exampletranslation{El amate crece muy grande.}

\entry{juñkujyel}
\partofspeech{adv}
\spanishtranslation{con una sola palabra}
\cholexample{Juñkujyel mi jak'beñ ity'añ ityaty jiñi alob.}
\exampletranslation{Ese niño obedece a su padre cuando le habla con una sola palabra.}

\entry{juñk'al}
\partofspeech{adj}
\spanishtranslation{veinte}

\entry{juñlajal}
\partofspeech{adv}
\spanishtranslation{igual}
\cholexample{Juñlajal tsa' kajiyob tyi chobal jiñi kerañob.}
\exampletranslation{Mis hermanos empezaron a rozar iguales.}

\entry{juñlujump'ejl}
\partofspeech{adj}
\spanishtranslation{once}

\entry{juññumel jach}
\spanishtranslation{una sola vez}
\cholexample{Juññumel jach tsa' tyili ja'al.}
\exampletranslation{Una sola vez llovió.}

\entry{juñsujm}
\partofspeech{adj}
\spanishtranslation{un, una}
\clarification{género, clase}
\cholexample{Juñsujm jachbä e'tyel mi iyesmañ.}
\exampletranslation{Solamente hace una clase de trabajo.}

\entry{juñwa'le}
\relevantdialect{Sab.}
\partofspeech{adv}
\spanishtranslation{una hora antes}
\cholexample{Juñwa'le mach p'äjk k'iñ tsa' chämi.}
\exampletranslation{Se murió una hora antes de ocultarse el sol.}

\entry{juñyajlel}
\partofspeech{adv}
\onedefinition{1}
\spanishtranslation{primera vez}
\cholexample{Ijuñyajlel jaxtyo tsa' majliyoñ tyi yajalóñ.}
\exampletranslation{Fue la primera vez que fui a Yajalón.}
\onedefinition{2}
\spanishtranslation{de una vez}
\cholexample{Juñyajlel tsa' itsäñsa ikoñtyra.}
\exampletranslation{De una vez mató a su enemigo.}

\entry{juñajb}
\partofspeech{s}
\spanishtranslation{una cuarta}
\clarification{la punta del dedo pulgar hasta la punta del meñique}

\entry{juk'ilañ}
\partofspeech{vt}
\spanishtranslation{restregar}
\cholexample{Woli ijuk'ilañ ik'äb yik'oty xapom.}
\exampletranslation{Está restregándose las manos con jabón.}

\entry{jux}
\partofspeech{s}
\spanishtranslation{piedra para afilar}

\entry{juxlum}
\partofspeech{s}
\spanishtranslation{notata}
\clarification{tipo de lagartija}

\entry{juxk'iyel}
\partofspeech{vi}
\spanishtranslation{resbalar}
\clarification{sobre un palo o piedra}

\entry{juxukña}
\partofspeech{adv}
\spanishtranslation{resbalando}
\cholexample{K'uñtye' juxukña woli tyi jubel tyilel jiñi xajlel ya' tyi emel.}
\exampletranslation{Esa piedra del derrumbe se está resbalando despacio hacia abajo.}

\entry{juy}
\partofspeech{vt}
\spanishtranslation{mover con palo}
\clarification{atole, pinole}

\entry{juyib}
\partofspeech{s}
\spanishtranslation{palito para mover atole o pinole}

\entry{juyts'iñ}
\partofspeech{vt}
\spanishtranslation{mover}
\clarification{atole, pinole o maíz cocido}

\entry{ju'ju'ña}
\partofspeech{adj}
\spanishtranslation{vociferando}
\cholexample{Ju'ju'ñayob jiñi xyäk'ajelob tyi kañtyiña.}
\exampletranslation{Los borrachos están vociferando en la cantina.}

\entry{ju'sañ}
\partofspeech{vt}
\spanishtranslation{bajar}
\clarification{alguna cosa}

\alphaletter{L}

\entry{labityo}
\partofspeech{s}
\spanishtranslation{armónica}

\entry{lak}
\defsuperscript{1}
\partofspeech{part}
\onedefinition{1}
\nontranslationdef{adjetivo posesivo, 1ª per. pl. incl.}
\onedefinition{2}
\nontranslationdef{pronombre personal, 1ª per. pl. incl.}
\cholexample{Mi lakmel.}
\exampletranslation{Hacemos.}

\entry{lak}
\defsuperscript{2}
\partofspeech{adj}
\spanishtranslation{agarrado}
\clarification{objeto largo}
\cholexample{Lakye'el icha'añ ijuloñib tyi' k'äb.}
\exampletranslation{Tiene su escopeta agarrada en la mano.}

\entry{lakal}
\partofspeech{adj}
\spanishtranslation{puesto}
\clarification{objeto largo}
\cholexample{Ya' lakal ijuloñib tyi' pam jiñi mesa.}
\exampletranslation{Ahí está puesta su escopeta encima de la mesa.}

\entry{laj}
\defsuperscript{1}
\partofspeech{adj}
\spanishtranslation{todo}
\cholexample{Tsa' ilaj japäyob sa'.}
\exampletranslation{Tomaron todo el pozol.}

\entry{laj}
\defsuperscript{2}
\partofspeech{vt}
\spanishtranslation{igualar}
\clarification{un objeto con otro}

\entry{lajal}
\partofspeech{adv}
\spanishtranslation{igual}
\cholexample{Yom lajal mi lakmel otyotybajche' jiñi.}
\exampletranslation{Debemos hacer la casa igual que aquélla.}

\entry{lajk'}
\partofspeech{s}
\spanishtranslation{ronda}
\clarification{tipo de hormiga grande}

\entry{lajchämp'ejl}
\partofspeech{adj}
\spanishtranslation{doce}

\entry{lajchiñ}
\partofspeech{vt}
\spanishtranslation{rascar}

\entry{lajiñ}
\partofspeech{vt}
\spanishtranslation{igualar}
\cholexample{Mach mejlik laklajiñ lakbä yik'oty jiñi weñ yujilbä e'tyel.}
\exampletranslation{No podemos igualarnos con uno que sabe trabajar bien.}

\entry{lajlaj}
\partofspeech{adj}
\onedefinition{1}
\spanishtranslation{palmeando}
\clarification{acción repetida de palmear}
\cholexample{Woli ilajlaj jats' ik'äb.}
\exampletranslation{Está palmeándose.}
\onedefinition{2}
\spanishtranslation{golpeando}
\clarification{acción repetida de golpear ligeramente}
\cholexample{Woli ilajlaj jats'beñ ipaty ipi'äl cha'añ ch'ijiyem ipusik'al.}
\exampletranslation{Está golpeando ligeramente la espalda de su esposa porque está triste.}

\entry{lajlajye'}
\partofspeech{vt}
\spanishtranslation{palpar}
\cholexample{Jiñi xpots' mi ilajlajye' chuki tyak mi ityaj ityäl.}
\exampletranslation{El ciego palpa cualquier objeto que alcanza a tocar.}

\entry{-lajm}
\nontranslationdef{Sufijo numeral para contar capas o pisos}
\cholexample{Jiñi chañbä otyoty añ chäñlajm imal.}
\exampletranslation{Esta casa alta tiene cuatro pisos.}

\entry{lajmañ}
\partofspeech{vt}
\spanishtranslation{escoger}
\cholexample{Che' mi lakñusañ jiñi waj tyi mesa mi laklajmañ jiñi weñ tyakbä.}
\exampletranslation{Cuando pasamos las tortillas a la mesa escogemos las buenas.}

\entry{lajmel}
\partofspeech{vi}
\onedefinition{1}
\spanishtranslation{sanarse}
\cholexample{Mach yomix lajmel ik'amäjel.}
\exampletranslation{Ya no se quiere sanar de su enfermedad.}
\onedefinition{2}
\spanishtranslation{componerse}
\clarification{el tiempo}
\cholexample{Tsa' lajmi ja'al yik'oty ik'.}
\exampletranslation{Se compuso el mal tiempo.}
\onedefinition{3}
\spanishtranslation{componerse}
\clarification{en sentido figurado}
\cholexample{Che' jach tsa' lajmi jiñi letyo tsa'bä ñaxañ tyejchi ya' tyi komisariado.}
\exampletranslation{Así nada más se compuso la disputa que comenzó con el comisariado.}

\entry{lajmesañ}
\partofspeech{vt}
\spanishtranslation{sanar, curar}

\entry{lajtye'}
\partofspeech{s}
\spanishtranslation{tambor}
\culturalinformation{Información cultural: Está hecho de madera y piel de jaguar o venado. Se hacen los bolillos de palitos de cedro.}

\entry{Lajtye'wits}
\partofspeech{s}
\spanishtranslation{Cerro del Tambor}
\clarification{cerro}

\entry{-lajts}
\nontranslationdef{Sufijo numeral para contar montones; p. ej.:}
\cholexample{Ya' tyi otyoty añ cha'lajts jiñi si'.}
\exampletranslation{En esa casa hay dos montones de leña.}

\entry{lalaktyäl}
\partofspeech{adv}
\spanishtranslation{así}
\clarification{señalando los pedazos}
\cholexample{Che' lalaktyäl ip'akil jiñi sik'äb tsa'bä iyäk'eyoñ.}
\exampletranslation{Así son los pedazos de los cañutos de la caña que me dio.}

\entry{*lamiñajlel}
\partofspeech{s esp}
\spanishtranslation{lámina}
\secondaryentry{*lamiñajlel otyoty}
\secondtranslation{láminas de la casa}

\entry{lamityal}
\relevantdialect{Sab.}
\partofspeech{s}
\spanishtranslation{la mitad}
\cholexample{Lamityal jaxtyo tsak choño jkajpe'.}
\exampletranslation{Únicamente he vendido la mitad de mi café.}
\alsosee{ojlil, xiñol}

\entry{lañal}
\partofspeech{adj}
\spanishtranslation{cundido}
\clarification{granos de sarampión}
\cholexample{Lañal lächix ipulib tyi' kuktyal jiñi alob.}
\exampletranslation{El cuerpo de ese chamaco está cundido de sarampión.}

\entry{latyu}
\partofspeech{s esp}
\spanishtranslation{plato}

\entry{-law}
\nontranslationdef{Sufijo que se presenta con raíces atributivas para formar otra raíz atributiva que indica cantidad; p. ej.:}
\cholexample{chijlaw}
\exampletranslation{mucho (rocío).}

\entry{lawux}
\partofspeech{s esp}
\spanishtranslation{clavo}

\entry{la'}
\partofspeech{imp}
\spanishtranslation{¡ven!}

\entry{la'-}
\onedefinition{1}
\nontranslationdef{Prefijo que indica adjetivo posesivo de segunda persona de plural.}
\onedefinition{2}
\nontranslationdef{Prefijo que indica pronombre personal de segunda persona de plural.}

\entry{la'ñuñ}
\partofspeech{imp}
\spanishtranslation{¡apúrate!}

\entry{la'tyika}
\relevantdialect{Sab.}
\partofspeech{part}
\spanishtranslation{a ver}
\cholexample{La'tyika mi isujmäch awujiläch e'tyel.}
\exampletranslation{A ver si es cierto que sabes trabajar.}

\entry{la' tyo}
\spanishtranslation{deja}
\cholexample{La'tyo imelbajche' yom.}
\exampletranslation{Deja que lo haga como él quiera.}

\entry{läbäkña}
\partofspeech{adj}
\spanishtranslation{infinito}
\cholexample{Läbäkña xiñich mi iletsel tyi lakok che' woliyoñlatyi ak'iñ.}
\exampletranslation{Una infinidad de hormigas se nos suben a los pies cuando estamos limpiando.}

\entry{läk'}
\partofspeech{adv}
\spanishtranslation{cerca}
\cholexample{Wolix lakläk' majlel jiñi tyejklum.}
\exampletranslation{Ya estamos llegando cerca del pueblo.}

\entry{läk'äl}
\partofspeech{adj}
\onedefinition{1}
\spanishtranslation{cerca}
\cholexample{Läk'äl añ jiñi cholel tyi kolem xajlel.}
\exampletranslation{La milpa está cerca al peñasco.}
\onedefinition{2}
\spanishtranslation{próximo}
\cholexample{Läk'älix iyorajlel k'iñ cha'añ karñaval.}
\exampletranslation{Ya está próxima la fiesta del carnaval.}

\entry{läk'tyesañ}
\partofspeech{vt}
\spanishtranslation{acercar}
\cholexample{Muk'ix kaj kläk'tyesañ majlel kotyoty ya' tyi' tyi' tyejklum.}
\exampletranslation{Ya voy a acercar mi casa a la orilla del pueblo.}

\entry{läch}
\partofspeech{vt}
\spanishtranslation{rascar}

\entry{-läjts}
\nontranslationdef{Sufijo numeral para contar tongas, pilas o montones; p. ej.:}
\cholexample{Ya' tyi otyoty ya'añ cha'läjts ixim.}
\exampletranslation{En esa casa están dos montones de maíz.}

\entry{läjwel}
\partofspeech{vi}
\spanishtranslation{remendarse}

\entry{*läjwil}
\partofspeech{s}
\spanishtranslation{remiendo}
\alsosee{läwoñib}

\entry{läm}
\partofspeech{vt}
\spanishtranslation{calmar}
\clarification{enfermedad}
\cholexample{Yom mi ajap ts'ak cha'añ mi iläm ak'amäjel.}
\exampletranslation{Debes tomar medicina para calmar tu enfermedad.}

\entry{lämäkña}
\partofspeech{adv}
\onedefinition{1}
\nontranslationdef{Se relaciona con la forma en que se termina el agua o el caldo de una olla.}
\onedefinition{2}
\nontranslationdef{Se relaciona con la forma como se calma una enfermedad.}

\entry{lämäl}
\partofspeech{adj}
\onedefinition{1}
\spanishtranslation{encharcado}
\cholexample{Lämäl tsa' käle jiñi ja' ya'ba'tyokol jiñi lum.}
\exampletranslation{El agua quedó encharcada en el hueco de la tierra.}
\onedefinition{2}
\spanishtranslation{quieto}
\cholexample{Laj lämäl jiñi wiñikob woli iñich'tyañob chuki woli iyäl jiñi presideñtye.}
\exampletranslation{Los hombres están quietos escuchando lo que el presidente está diciendo.}

\entry{lämlämña}
\partofspeech{adv}
\spanishtranslation{ondeante}
\cholexample{K'uñtye' lämlämña woli tyi ajñel jiñi ja' ya' tyi bij.}
\exampletranslation{El agua está fluyendo ondeante y tranquilamente en el camino.}

\entry{lämp'ejl}
\relevantdialect{Tila}
\partofspeech{adj}
\spanishtranslation{diez}
\alsosee{lujump'ejl}

\entry{lämtyäl}
\relevantdialect{Sab.}
\partofspeech{vi}
\onedefinition{1}
\spanishtranslation{encharcarse}
\clarification{agua}
\cholexample{Ya' jach mi ilämtyäl jiñi ja'ba' joyol jiñi lum.}
\exampletranslation{Allí no más se quedó encharcada el agua donde está hueca la tierra.}
\onedefinition{2}
\spanishtranslation{calmarse}
\clarification{enfermedad}
\cholexample{Mi ilämtyäl k'ajk che' mik buk'lats'ak.}
\exampletranslation{Se calma la calentura cuando tomamos medicina.}
\alsosee{lämäl}

\entry{lämulañ}
\partofspeech{vt}
\spanishtranslation{mover}
\clarification{líquido}
\cholexample{Wolik lämulañ ja' tyi kolem p'ejty.}
\exampletranslation{Estoy moviendo agua en una olla grande.}

\entry{lämuña}
\partofspeech{adj}
\spanishtranslation{de una manera ondulante}
\clarification{agua}
\cholexample{Ity'ojol jax tyi k'elol che' lämuña jiñi ñajb.}
\exampletranslation{Es bonito ver la mar cuando se muestra de una manera ondulante.}

\entry{läp}
\partofspeech{vt}
\spanishtranslation{poner}
\clarification{ropa}
\cholexample{Mi kaj kläp kbujk.}
\exampletranslation{Voy a ponerme la camisa.}

\entry{läp'}
\onedefinition{1}
\partofspeech{vt}
\spanishtranslation{pegar}
\clarification{papel, tela}
\cholexample{Mi kaj kläp' juñ tyi pajk'.}
\exampletranslation{Voy a pegar el papel en la pared.}
\onedefinition{2}
\partofspeech{adj}
\spanishtranslation{pegajoso}
\cholexample{Läp' iyetsel jiñi ñi'uk'.}
\exampletranslation{La trementina del chayote es pegajosa.}

\entry{läp'tyäl}
\partofspeech{vi}
\spanishtranslation{pegarse}
\cholexample{Mi iläp'tyäl jiñi juñ ya' tyi pajk' yik'oty tya'chäb.}
\exampletranslation{El papel se pega en la pared con cera.}

\entry{läty'}
\partofspeech{vt}
\spanishtranslation{aguantar}
\clarification{cosa pesada, dolor}
\cholexample{Weñ oñ mik läty' kuchäl che' joktyäl bij mi jkuch majlel.}
\exampletranslation{Aguanto mucha carga cuando la llevo cargada en camino plano.}

\entry{läts}
\partofspeech{vt}
\spanishtranslation{hacinar}
\clarification{leña, maíz}
\cholexample{Che' mi iyujtyel jk'aj ixim mi kajel kläts.}
\exampletranslation{Cuando termine de tapiscar mi milpa comenzaré a hacinar.}

\entry{lätsäl}
\partofspeech{adj}
\spanishtranslation{hacinado}
\clarification{leña, maíz}
\cholexample{Machtyojik lätsäl jiñi ixim.}
\exampletranslation{El maíz está hacinado disparejo.}

\entry{lätschokoñ}
\partofspeech{vt}
\spanishtranslation{hacinar}
\clarification{leña, maíz}
\cholexample{Mi kaj klätschokoñ jump'ejl tyarea si'.}
\exampletranslation{Voy a hacinar una tarea de leña.}

\entry{läts'äl}
\partofspeech{adj}
\spanishtranslation{angosto}
\clarification{camino}
\cholexample{Läts'äl jiñi bijba' mi lakñumel.}
\exampletranslation{El camino donde pasamos está angosto.}

\entry{läw}
\partofspeech{vt}
\spanishtranslation{remendar}

\entry{läwäl}
\partofspeech{adj}
\spanishtranslation{remendado}
\cholexample{Läwäl ibujk jiñi ch'ityoñ yik'oty chächäk pisil.}
\exampletranslation{La camisa del niño está remendada con tela roja.}

\entry{*läwoñib}
\partofspeech{s}
\spanishtranslation{remiendo}
\cholexample{Che' tsijlemix jiñi bujkäl wersa yom iläwoñib.}
\exampletranslation{Cuando la camisa está rota es necesario ponerle un remiendo.}
\alsosee{läjwil}

\entry{läwoñel}
\partofspeech{s}
\spanishtranslation{actividad de remendar ropa}
\cholexample{Jiñi x'ixik mi imulañ läwoñel.}
\exampletranslation{A esa mujer le gusta estar remendando.}

\entry{leb}
\partofspeech{vt}
\spanishtranslation{partir}
\clarification{piedra}
\cholexample{Mi laj k'äñ marro cha'añ mi lakleb xajlel.}
\exampletranslation{Usamos un marro para partir piedra.}

\entry{leko}
\partofspeech{adj}
\onedefinition{1}
\spanishtranslation{desagradable}
\cholexample{Leko mi ich'äl ibä jiñi x'ixik.}
\exampletranslation{Esa mujer se viste en forma desagradable.}
\onedefinition{2}
\spanishtranslation{deficiente}
\cholexample{Leko jax tsa' iyäk'ña ichol jiñi wiñik kome chañatyax tsa' itsepe jiñi pimel.}
\exampletranslation{Fue muy deficiente la forma en que limpió su milpa ese hombre; cortó muy alto el monte.}
\onedefinition{3}
\spanishtranslation{grosero}
\cholexample{Leko mi icha'leñ ty'añ jiñi wiñik.}
\exampletranslation{Es muy grosera la manera de hablar de ese hombre.}
\secondaryentry{leko iyujts'il}
\secondtranslation{mal olor}

\entry{lek'}
\partofspeech{vt}
\spanishtranslation{lamer}
\cholexample{Jiñi ts'i' yom ilek' ja'.}
\exampletranslation{El perro quiere tomar agua (lit.: quiere lamer agua).}

\entry{lech}
\partofspeech{vt}
\spanishtranslation{sacar}
\clarification{alimento}
\cholexample{Mi laklech lakbu'ul yik'oty ixejty'il waj.}
\exampletranslation{Sacamos nuestro frijol con un pedazo de tortilla.}

\entry{lecho'k'ajk}
\partofspeech{s}
\spanishtranslation{tenazas para sacar tizones}

\entry{legra}
\relevantdialect{Sab.}
\partofspeech{s esp}
\spanishtranslation{regla}

\entry{-lejb}
\nontranslationdef{Sufijo numeral para contar pedazos; p. ej.:}
\cholexample{Mi laj k'ux juñlejb waj.}
\exampletranslation{Comemos un pedazo de tortilla.}

\entry{lejbel}
\partofspeech{vi}
\spanishtranslation{quebrarse}
\clarification{un pedazo de piedra, madera, diente}
\cholexample{Tsa' lejbi juñtyejk ibäkel iyej.}
\exampletranslation{Se quebró un pedazo de diente.}

\entry{lejbeñ}
\partofspeech{adj}
\spanishtranslation{roto}
\clarification{piedra, tabla o diente}
\cholexample{Wokol tyi lok'säñtyel ibäkel iyej kome lejbeñix.}
\exampletranslation{Ya es difícil quitar el diente porque está roto.}

\entry{*lejbil}
\partofspeech{s}
\spanishtranslation{pedazo}
\clarification{de piedra, diente o madera}

\entry{-lejch}
\nontranslationdef{Sufijo numeral para contar cucharadas; p. ej.:}
\cholexample{Jiñi aläl mi iyotsäbeñtyel juñlejch ibäl iñäk' tyi' yej.}
\exampletranslation{Al niño le dan su alimento por cucharadas.}

\entry{lejchempaty}
\partofspeech{s}
\spanishtranslation{choza}
\cholexample{Yom mi lakmel laklejchempaty.}
\exampletranslation{Hagamos una choza.}

\entry{lejchiñpaty}
\relevantdialect{Sab.}
\partofspeech{s}
\spanishtranslation{champa}

\entry{lejeñsia}
\partofspeech{adv}
\spanishtranslation{para probar}
\cholexample{Lejeñsia jach mik majlel jk'el jiñi doktyor ame muk'ikixtyo klajmel.}
\exampletranslation{Voy a ver al doctor para probar y ver si acaso puedo sanar.}

\entry{lejlej}
\partofspeech{s}
\spanishtranslation{páncreas}
\clarification{de ganado}

\entry{lejlejña}
\partofspeech{adj}
\spanishtranslation{acezando, jadeando}
\cholexample{Lejlejña jiñi ts'i' wolibä iyajñesañbätye'el.}
\exampletranslation{Está acezando el perro que está correteando al animal.}

\entry{lejles}
\partofspeech{adj}
\spanishtranslation{palmeando}
\clarification{acción repetida con regla o machete}
\cholexample{Woli jach ilejles jats' yik'oty wechelbä tye'.}
\exampletranslation{Está nada más palmeando con una regla.}

\entry{-lel}
\defsuperscript{1}
\nontranslationdef{Sufijo que se presenta con raíces atributivas para formar otra raíz que indica calidad o condición; p. ej.:}
\cholexample{iyutslel}
\exampletranslation{su bondad.}

\entry{-lel}
\defsuperscript{2}
\nontranslationdef{Sufijo que se presenta con raíces sustantivas para formar otra raíz sustantiva; p. ej.:}
\cholexample{kajpe'lel}
\exampletranslation{cafetal.}

\entry{lem}
\onedefinition{1}
\partofspeech{vt}
\spanishtranslation{lamer}
\cholexample{Jiñi ts'i' woli ilem ip'ejtyal we'eläl.}
\exampletranslation{El perro está lamiendo la olla de la carne.}
\onedefinition{2}
\partofspeech{vi}
\spanishtranslation{tomar}
\clarification{bebidas alcohólicas}
\cholexample{Mi kajel klem.}
\exampletranslation{Voy a tomar.}

\entry{-lemañ}
\nontranslationdef{Sufijo que se presenta con raíces adjetivas que indican color y se refiere a una calidad brillosa.}

\entry{lembal}
\partofspeech{s}
\spanishtranslation{aguardiente}

\entry{lemel}
\partofspeech{adj}
\spanishtranslation{borracho}
\cholexample{Lemel icha'añ jiñi wiñik.}
\exampletranslation{Ese hombre está borracho.}
\dialectvariant{Sab.}
\dialectword{k'ixiñ}

\entry{lemla}
\partofspeech{adj}
\spanishtranslation{flamante}
\cholexample{Weñ lemlajiñi k'ajk che' tyikiñ isi'il.}
\exampletranslation{El fuego es muy flamante cuando la leña está seca.}

\entry{lemlemña}
\partofspeech{adv}
\nontranslationdef{Se relaciona con la forma en que arden las llamas; p. ej.:}
\cholexample{Lemlemña woli tyi lejmel jiñi k'ajk.}
\exampletranslation{El fuego está ardiendo con llamas altas.}

\entry{lemoñel}
\partofspeech{s}
\spanishtranslation{acción de tomar bebidas alcohólicas}
\cholexample{Kabäl lächix mi imulañ lemoñel jiñi wiñik.}
\exampletranslation{A ese hombre le gusta bastante tomar tragos.}

\entry{leñ}
\partofspeech{adv}
\spanishtranslation{por lo pronto}
\cholexample{Tsa' ileñ käyä iye'tyel jiñi wiñik cha'añ tsa' majli ik'el iyalobil.}
\exampletranslation{Por lo pronto, ese hombre dejó su trabajo para ver a su hijo.}

\entry{*leñtyilel}
\partofspeech{s}
\spanishtranslation{anchura}
\cholexample{Ñuk ileñtyilel imal otyoty.}
\exampletranslation{Es grande la anchura de esa casa.}

\entry{les}
\partofspeech{vt}
\spanishtranslation{pegar}
\clarification{con alas}
\cholexample{Mi ilesob ibä yik'oty iwich' jiñi tyaty muty che' mi icha'leñob letyo.}
\exampletranslation{Cuando los gallos se pelean, se pegan con las alas.}

\entry{lesia}
\partofspeech{s esp}
\spanishtranslation{iglesia}

\entry{letyo}
\partofspeech{s esp}
\onedefinition{1}
\spanishtranslation{disputa}
\cholexample{Tsa' tyejchi letyo tyi juñtya.}
\exampletranslation{Se levantó una disputa en la junta.}
\onedefinition{2}
\spanishtranslation{pleito}
\cholexample{Tsa' kajiyob tyi letyo tyi bij.}
\exampletranslation{Comenzaron un pleito en el camino.}
\dialectvariant{Sab., Tila}
\dialectword{periyal}

\entry{letsañ}
\partofspeech{vt}
\spanishtranslation{levantar}
\spanishtranslation{subir de precio}

\entry{letsel}
\partofspeech{vi}
\spanishtranslation{subir}
\spanishtranslation{ascender}

\entry{-letsel}
\nontranslationdef{Sufijo numeral para contar jornadas; p. ej.:}
\cholexample{Che' mi ilok'el juñletsel lake'tyel mi lakcha'leñ uch'el.}
\exampletranslation{Al terminar una jornada de nuestro trabajo, vamos a tomar nuestros alimentos.}

\entry{letsem}
\partofspeech{adj}
\onedefinition{1}
\spanishtranslation{subido}
\cholexample{Maxtyo añik letsem iyopol otyoty.}
\exampletranslation{Todavía no han subido la paja de la casa.}
\onedefinition{2}
\spanishtranslation{caro}
\cholexample{Letsem ityojol jiñi ixim che' tyi juñio.}
\exampletranslation{El maíz es caro en junio.}

\entry{lew}
\partofspeech{s}
\spanishtranslation{manteca}

\entry{lewa}
\partofspeech{s esp}
\spanishtranslation{legua}

\entry{le'}
\partofspeech{adv}
\nontranslationdef{Movimiento de la mano al abrir un costal; p. ej.}
\cholexample{Tsa' ile' ch'ipi ityi' koxtyal.}
\exampletranslation{Abrió la boca del costal con las manos.}

\entry{le'ekña}
\partofspeech{adj}
\spanishtranslation{acostado}
\clarification{persona acostada boca arriba y con las piernas abiertas}

\entry{-lib}
\nontranslationdef{Sufijo que se presenta con raíces neutras para formar una raíz sustantiva que indica el instrumento; p. ej.:}
\cholexample{k'ächlib}
\exampletranslation{montura.}

\entry{likchokoñ}
\partofspeech{vt}
\spanishtranslation{colgar en un palo}
\cholexample{Pejtyelel ipislel jiñi x'ixik ya' mi ilikchokoñba' k'äty ch'ijbil jiñi tye'.}
\exampletranslation{Esa mujer cuelga toda su ropa en el palo atravesado que tiene clavado.}

\entry{lich'}
\partofspeech{vt}
\spanishtranslation{tender}
\clarification{ropa}

\entry{lich'bil}
\partofspeech{adj}
\spanishtranslation{tendido}
\cholexample{Lich'bil jiñi pisil tyi k'iñ.}
\exampletranslation{La ropa está tendida al sol.}

\entry{lich'k'äbañ}
\partofspeech{vt}
\spanishtranslation{llamar}
\clarification{con la mano}
\cholexample{Woli ilich'k'äbañ ipi'äl jiñi x'ixik.}
\exampletranslation{Esa mujer está llamando a su compañera.}

\entry{lij}
\partofspeech{vt}
\spanishtranslation{inclinarse}
\cholexample{Yom laklij lakbä che' woli lak'äleñtyel.}
\exampletranslation{Nos inclinamos al ser regañados.}

\entry{-lijk}
\nontranslationdef{Sufijo numeral para contar pedazos de tela; p. ej.:}
\cholexample{Sajmäl tsak mäñä juñlijk jiñi pisil.}
\exampletranslation{Hoy compré un pedazo de tela.}

\entry{lijkañ}
\partofspeech{vt}
\spanishtranslation{sacudir}

\entry{lijil}
\partofspeech{adj}
\spanishtranslation{inclinado}
\clarification{la cabeza}
\cholexample{Lijil ijol jiñi wiñik kome woli ityi orakióñ.}
\exampletranslation{La cabeza de ese hombre está inclinada porque está orando.}

\entry{lijk'el}
\partofspeech{vi}
\spanishtranslation{agobiarse}
\cholexample{Wolix tyi lijk'el jiñi x'ixik cha'añ k'uñix.}
\exampletranslation{Esa mujer está agobiándose porque está muy débil.}

\entry{limetye}
\partofspeech{s esp}
\spanishtranslation{botella}

\entry{limetyoñ}
\partofspeech{s esp}
\spanishtranslation{garrafón}

\entry{limil}
\partofspeech{adj}
\spanishtranslation{acostado}
\clarification{toda la familia por enfermedad}
\cholexample{Limilob tyi wäyib iyalobilob jiñi wiñik cha'añ k'amäjel.}
\exampletranslation{Todos los hijos de ese hombre están acostados en la cama por enfermedad.}

\entry{limóñ jam}
\partofspeech{s}
\nontranslationdef{Tipo de zacate que sirve para hacer un té.}

\entry{lik'ikña}
\partofspeech{adv}
\nontranslationdef{Se relaciona con la forma en que se dobla algo; p. ej.:}
\cholexample{Lik'ikña mi ijubel ik'äb tye'.}
\exampletranslation{La rama del árbol se dobla por el peso de la fruta.}

\entry{lik'il}
\partofspeech{adj}
\spanishtranslation{inclinado}
\clarification{persona}
\cholexample{Lik'il ijol jiñi wiñik che' yäk.}
\exampletranslation{La cabeza de ese hombre está inclinada porque está borracho.}

\entry{litsil}
\partofspeech{adj}
\spanishtranslation{guindado}
\clarification{la cabeza de un pavo o el cuerpo de una culebra}
\cholexample{Litsil jiñi kolem lukum ya' tyi xäk'tye'.}
\exampletranslation{La culebra está guindada en la horqueta de un árbol.}

\entry{litslitsña}
\partofspeech{adv}
\spanishtranslation{doblada}
\clarification{manera}
\cholexample{Litslitsña mi imajlel jiñi tye'.}
\exampletranslation{El palo está doblándose.}

\entry{lok}
\partofspeech{vt}
\spanishtranslation{doblar}
\clarification{alambre, palo delgado}

\entry{lokokña}
\partofspeech{adj}
\spanishtranslation{curvado}
\clarification{árbol, palo}
\cholexample{Mach mejlik lakp'el jiñi tye' kome lokokñajax.}
\exampletranslation{No podemos aserrar ese palo porque está muy curvado.}

\entry{lokol}
\partofspeech{adj}
\spanishtranslation{curvado}
\clarification{palo, alambre}

\entry{lok'}
\partofspeech{vt}
\spanishtranslation{sacar}
\cholexample{Yomix alok' jiñi we'eläl ya' tyi' p'ejtyal.}
\exampletranslation{Ya debes sacar la comida de la olla.}

\entry{lok' jol}
\spanishtranslation{cortar el pelo}

\entry{*lok'omlel}
\relevantdialect{Sab.}
\partofspeech{s}
\spanishtranslation{copia}
\cholexample{Jiñi bujkäl ilok'omlelächbajche' kcha'añ.}
\exampletranslation{Esa camisa parece ser una copia de la mía.}

\entry{lok'oñ baj}
\relevantdialect{Tila}
\partofspeech{s}
\spanishtranslation{cuadro}
\spanishtranslation{retrato}

\entry{lok'sañ}
\partofspeech{vt}
\spanishtranslation{sacar}
\cholexample{Sami klok'sañ kerañ tyi mäjkibäl.}
\exampletranslation{Voy a sacar a mi hermano de la cárcel.}

\entry{lochol}
\partofspeech{adj}
\spanishtranslation{torcido}
\clarification{palo, camino, alambre}
\cholexample{Lochol jax iyoyel iyotyoty.}
\exampletranslation{Está muy torcido el horcón de su casa.}

\entry{loch'}
\defsuperscript{1}
\partofspeech{vt}
\spanishtranslation{sacar líquido con las manos}
\cholexample{Woli jach iloch' lok'el ja' tyi' k'äb cha'añ mi ijap.}
\exampletranslation{Está sacando agua para tomar con las manos.}

\entry{loch'}
\defsuperscript{2}
\partofspeech{adv}
\spanishtranslation{en los brazos}
\cholexample{Loch' mek'el icha'añ aläl jiñi xch'ok.}
\exampletranslation{Esa muchacha tiene a la criatura en los brazos.}

\entry{loch'ilañ}
\partofspeech{vt}
\spanishtranslation{sacar agua con la mano}
\cholexample{Che' jach yatyoktyälba' mi lakloch'ilañ jiñi ja' yik'oty laj k'äb.}
\exampletranslation{Es un hueco donde se saca el agua con la mano.}

\entry{loj}
\defsuperscript{1}
\partofspeech{vt}
\spanishtranslation{aflojar}
\cholexample{Yom mi alojbeñ ichiñchajlel mula.}
\exampletranslation{Debes aflojar la cincha de la mula.}

\entry{loj}
\defsuperscript{2}
\partofspeech{s}
\spanishtranslation{gemelos}
\dialectvariant{Sab.}
\dialectword{luty}

\entry{lojk}
\partofspeech{s}
\onedefinition{1}
\spanishtranslation{espuma}
\cholexample{Kabäl mi iyäk' ilojk jiñi xapom.}
\exampletranslation{Ese jabón da mucha espuma.}
\onedefinition{2}
\spanishtranslation{acción de hervir}
\cholexample{Wolix tyi lojk jiñi bu'ul.}
\exampletranslation{El frijol está hirviendo.}

\entry{lojmel}
\partofspeech{vi}
\spanishtranslation{sumirse}

\entry{lojoñ}
\partofspeech{pron}
\spanishtranslation{nosotros}
\clarification{excluyendo a la persona con quien se está hablando}

\entry{lojkesañ}
\partofspeech{vt}
\spanishtranslation{hacer hervir}

\entry{lojwel}
\onedefinition{1}
\partofspeech{vi}
\spanishtranslation{lastimarse}
\cholexample{Tsa' poj ujtyi tyi lojwel yik'oty machity.}
\exampletranslation{Acaba de lastimarse con machete.}
\onedefinition{2}
\partofspeech{s}
\spanishtranslation{herida}
\cholexample{Mach käläx ñukik ilojwel.}
\exampletranslation{No está tan grande su herida.}

\entry{lojwem}
\partofspeech{adj}
\spanishtranslation{lastimado}
\cholexample{Mach mejlik icha'leñ xämbal jiñi wiñik kome lojwem iyok.}
\exampletranslation{Ese hombre no puede caminar, porque tiene lastimado el pie.}

\entry{*lojweñal}
\partofspeech{s}
\spanishtranslation{cicatriz}

\entry{lolom jach}
\spanishtranslation{de balde}
\cholexample{Lolom jach tsa' majliyoñ ksätye' k'iñ.}
\exampletranslation{De balde fui a perder el día.}

\entry{lom}
\partofspeech{vt}
\spanishtranslation{quebrar}
\clarification{olla o caja}

\entry{*lomojlel i paty}
\partofspeech{s esp}
\spanishtranslation{lomo}
\clarification{de animal}

\entry{*lomtyilel}
\partofspeech{s}
\spanishtranslation{hondonada del terreno}

\entry{lokilañ}
\partofspeech{vt}
\spanishtranslation{doblar}
\clarification{palo, alambre}
\cholexample{Woli ilokilañ ibä jiñi tye' cha'añ mach tsätsik.}
\exampletranslation{Se está doblando ese palo porque no está duro.}

\entry{lok'el}
\partofspeech{vi}
\onedefinition{1}
\spanishtranslation{salir}
\cholexample{Sebtyo mi ilok'el tyi' yotyoty cha'añ mi imajlel tyi' ye'tyel.}
\exampletranslation{Sale muy temprano de su casa para irse a su trabajo.}
\onedefinition{2}
\spanishtranslation{resultar}
\cholexample{Ma'añix mi iweñ lok'el lakweñtya che' mi lakloñ mel cholel kome ma'añix mi iweñ mejlel.}
\exampletranslation{Ya no resulta hacer la milpa por nuestra cuenta porque ya no se da muy bien.}

\entry{*lok'ib ja'}
\spanishtranslation{vertiente}

\entry{loty}
\defsuperscript{1}
\partofspeech{s}
\spanishtranslation{mentira}
\cholexample{¿chukoch mi acha'leñ loty?}
\exampletranslation{¿Por qué dices una mentira?}
\secondaryentry{xloty}
\secondpartofspeech{adj}
\secondtranslation{mentiroso}
\secondaryentry{xlotyiya}
\secondpartofspeech{s}
\secondtranslation{testigo falso}

\entry{loty}
\defsuperscript{2}
\partofspeech{vt}
\onedefinition{1}
\spanishtranslation{recoger}
\cholexample{Yom mi aloty jiñi ixim tsa'bä p'ajtyi.}
\exampletranslation{Debes recoger el maíz que se cayó.}
\onedefinition{2}
\spanishtranslation{guardar}
\cholexample{Wolix iweñ loty ityak'iñ.}
\exampletranslation{Está guardando bien su dinero.}

\entry{lotyiñ}
\partofspeech{vt}
\spanishtranslation{engañar}

\entry{lotyiñtyel}
\partofspeech{vi}
\spanishtranslation{engañarse}

\entry{lotyiya}
\conjugationtense{variante}
\conjugationverb{lo'loya}
\spanishtranslation{engaño}

\entry{low}
\partofspeech{vt}
\spanishtranslation{lastimar}
\cholexample{Jiñi ch'ityoñ tsa' ilowo iyok yik'oty limetyej.}
\exampletranslation{El muchacho se lastimó el pie con un vidrio.}

\entry{lo'chij}
\partofspeech{s}
\spanishtranslation{calambre}

\entry{lo'chijtyik}
\partofspeech{adj}
\spanishtranslation{venoso}
\clarification{en los pies}
\cholexample{Lo'chijtyik iyok jiñi lakña'.}
\exampletranslation{A esa señora se le ven los pies muy venosos.}

\entry{lo'lmomoy}
\partofspeech{s}
\spanishtranslation{árbol}
\clarification{la madera es comestible cuando está tierna}

\entry{lo'loñ}
\partofspeech{vt}
\spanishtranslation{engañar}

\entry{lo'loya}
\partofspeech{s}
\spanishtranslation{engaño}
\cholexample{Woli jach tyi lo'loya jiñi wiñik kome mach isujmik chuki woli iyäl.}
\exampletranslation{Lo que dice ese hombre es nada más un engaño, porque no es verdad.}
\variation{lotyiya}

\entry{luko' chäy}
\partofspeech{s}
\spanishtranslation{anzuelo}

\entry{lukum}
\partofspeech{s}
\spanishtranslation{cualquier culebra}

\entry{luk'bäl}
\partofspeech{s}
\spanishtranslation{pesca}
\cholexample{Ijk'äl weñ sebtyo mik majlel tyi lukbäl ya' tyi ñoja'}
\exampletranslation{Mañana muy temprano me voy a la pesca en el río.}

\entry{luk'law}
\partofspeech{adj}
\spanishtranslation{débil}
\cholexample{Luk'law jiñi tyablacha'añ jay.}
\exampletranslation{La tabla está débil porque es delgada.}

\entry{luk'luk'ña}
\partofspeech{adv}
\spanishtranslation{doblándose, meneándose}
\cholexample{Luk'luk'ña jiñi pañtye' che' mi laj k'axel.}
\exampletranslation{El puente se menea al pasar.}

\entry{luch}
\partofspeech{vt}
\spanishtranslation{sacar}
\clarification{alimento o agua de un objeto}
\cholexample{Mi lakluch lok'el bu'ul tyi' p'ejtyal che' tyik'añix.}
\exampletranslation{Sacamos el frijol de la olla cuando ya está cocido.}

\entry{luchoñib}
\partofspeech{s}
\spanishtranslation{cuchara}

\entry{lucho'ja'}
\partofspeech{s}
\onedefinition{1}
\spanishtranslation{taza para sacar agua}
\onedefinition{2}
\spanishtranslation{cubeta}

\entry{lujb}
\partofspeech{adj}
\spanishtranslation{cansado}
\clarification{persona o animal}
\cholexample{Ma'añik lujb jiñi kawayu' kome mach alik ikuch.}
\exampletranslation{El caballo no está cansado porque su carga no es pesada.}

\entry{*lujbel}
\partofspeech{s}
\spanishtranslation{cansancio}
\cholexample{Añix kabäl ilujbel kome ñoxix.}
\exampletranslation{Él está cansado (lit.: tiene cansancio) porque está viejo .}

\entry{lujbeñ}
\partofspeech{adj}
\spanishtranslation{cansado}
\cholexample{Lujbeñ jiñi alob kome mach ñämälik tyi kabäl xämbal.}
\exampletranslation{El niño está cansado porque no está acostumbrado a andar mucho.}

\entry{*lujbeñal}
\partofspeech{s}
\spanishtranslation{cansancio}
\cholexample{Ma'añix mik weñ mejlel tyi e'tyel kome añix kabäl ilujbeñal kbäk'tyal.}
\exampletranslation{Ya no puedo trabajar por el cansancio de mi cuerpo.}

\entry{lujb'añ}
\partofspeech{vi}
\spanishtranslation{cansarse}
\cholexample{Wolix klujb'añ kome ñajtyix tsak cha'le xämbal.}
\exampletranslation{Ya me estoy cansando porque he caminado desde muy lejos.}

\entry{-lujch}
\nontranslationdef{Sufijo numeral para contar cucharadas de sal o azúcar; p. ej.:}
\cholexample{Ya' tyi p'etyal bu'ul mi iyochel juñlujch ats'am.}
\exampletranslation{En la olla del frijol se echa una cucharada de sal.}

\entry{lujchiñ}
\partofspeech{vt}
\spanishtranslation{cornear}
\cholexample{Jiñi wakax yom ilujchiñoñ.}
\exampletranslation{El toro quiere cornearme.}

\entry{lujchiya}
\partofspeech{s}
\spanishtranslation{acción de cornear}
\cholexample{Jiñi tyaty wakax ma'añik mi imejlel tyi lujchiya kome ma'añik ixulub.}
\exampletranslation{Ese toro no puede cornear porque no tiene cuernos.}

\entry{lujchuñ ej}
\partofspeech{s}
\spanishtranslation{uña de gato}
\clarification{bejuco con espina curvada}

\entry{lujk'el}
\partofspeech{vi}
\spanishtranslation{pandearse}
\cholexample{Mi ilujk'el jiñi tyablatyi' yojlil kome jayatyax.}
\exampletranslation{La tabla se pandea en medio porque está muy delgada.}

\entry{-lujty}
\nontranslationdef{Sufijo numeral para contar mazorcas; p. ej.:}
\cholexample{Ya' tyi lum ya'añ uxlujty ixim.}
\exampletranslation{En el piso hay tres mazorcas.}

\entry{lujty'el}
\partofspeech{vi}
\spanishtranslation{brincar}
\cholexample{Woli tyi lujty'el jiñi alob kome woli'bajbeñtyel yik'oty asiyal.}
\exampletranslation{Ese niño está brincando porque le están pegando con chicote.}

\entry{lujump'ejl}
\partofspeech{adj}
\spanishtranslation{diez}
\dialectvariant{Tila}
\dialectword{lämp'ejl}
\secondaryentry{lujump'ejl icha'k'al}
\secondtranslation{treinta}

\entry{lujuñij}
\partofspeech{adv}
\spanishtranslation{el décimo día}

\entry{luluy}
\partofspeech{s}
\onedefinition{1}
\spanishtranslation{jobo}
\spanishtranslation{ciruelo}
\clarification{árbol}
\onedefinition{2}
\spanishtranslation{ujtui}
\clarification{árbol}

\entry{lum}
\partofspeech{s}
\onedefinition{1}
\spanishtranslation{terreno}
\cholexample{Mach weñik jiñi lum kome kabäl xajlel.}
\exampletranslation{Este terreno no es bueno porque tiene muchas piedras.}
\onedefinition{2}
\spanishtranslation{pueblo}
\relevantdialect{Sab.}
\cholexample{Mik majlel tyi lum cha'añ mik mäñ xapom yik'oty käts'am.}
\exampletranslation{Voy al pueblo a comprar jabón y sal.}
\onedefinition{3}
\spanishtranslation{suelo}
\cholexample{Yom mi awotsäbeñ ja' jiñi lum cha'añ mi ikolel jiñi päk'äbäl.}
\exampletranslation{Debes regar el suelo con agua para que el sembrado crezca.}
\onedefinition{4}
\spanishtranslation{piso}
\cholexample{Tyi lum jach mi imel ik'ajk.}
\exampletranslation{Hace lumbre en el piso de su casa.}

\entry{*lumal}
\partofspeech{s}
\onedefinition{1}
\spanishtranslation{tierra}
\clarification{de alguien}
\cholexample{Käläx ñajtyatyax klumal.}
\exampletranslation{Mi tierra está demasiado lejos.}
\onedefinition{2}
\spanishtranslation{país}
\clarification{de alguien}
\cholexample{Jiñi guatyemalajiñäch ilumal jiñi wiñik.}
\exampletranslation{Guatemala es el país de ese hombre.}

\entry{lumija'}
\partofspeech{s}
\spanishtranslation{ciénaga}

\entry{*lumil}
\partofspeech{s}
\spanishtranslation{pedazo de terreno}
\clarification{de una siembra}
\cholexample{Mach tyañtyo ñukik ilumil kajpe'lel.}
\exampletranslation{No está tan grande el pedazo de terreno del cafetal.}
\secondaryentry{xlumil bu'ul}
\secondtranslation{frijol de tierra}

\entry{lukijel}
\partofspeech{vi}
\spanishtranslation{pescar}
\cholexample{Wä' tyi jumuk' mik majlel tyi lukijel yik'oty kpi'älob.}
\exampletranslation{Al rato me voy a pescar con mis compañeros.}

\entry{luty}
\relevantdialect{Sab.}
\partofspeech{s}
\spanishtranslation{gemelos}
\alsosee{loj}

\entry{luty'ul}
\partofspeech{adj}
\spanishtranslation{apretada}
\clarification{las piernas}
\cholexample{Luty'ul mi iyäk' iyok che' mi ibuchtyäl.}
\exampletranslation{Cuando se sienta pone las piernas apretadas.}

\entry{luts}
\partofspeech{adj}
\spanishtranslation{agachado}
\cholexample{Luts buchul jiñi wiñik tyi yebal tye'.}
\exampletranslation{El hombre está sentando en forma agachada debajo del árbol.}

\entry{lutsañ}
\partofspeech{adj}
\spanishtranslation{empapado}
\clarification{personas, animales}
\cholexample{Ach' lutsañ tsa' k'otyi kerañ tyi' yotyoty.}
\exampletranslation{Mi hermano llegó a su casa empapado.}

\entry{lutsukña}
\partofspeech{adj}
\spanishtranslation{triste y agachado}
\clarification{gallo, gallina}
\cholexample{Lutsukña mi imel ibä jiñi xña'muty che' woli tyi ñumel ichämel.}
\exampletranslation{Triste y agachada se pone la gallina cuando está pasando la peste.}

\entry{lutsul}
\partofspeech{adj}
\spanishtranslation{acurrucado}
\clarification{persona, animal}
\cholexample{Lutsul tyi jumpaty jiñi xñox; woli tyi k'ix k'iñ.}
\exampletranslation{El viejo está acurrucado afuera, asoleándose.}

\entry{luts'}
\partofspeech{vt}
\spanishtranslation{abrazar con alas}
\cholexample{Jiñi xña'muty mi iluts' iyal che' alätyo.}
\exampletranslation{La gallina abraza sus crías con sus alas cuando todavía están chicas.}

\entry{lux}
\partofspeech{vt}
\onedefinition{1}
\spanishtranslation{doblar}
\clarification{rodilla o brazo}
\cholexample{Wokol jax mi ilux ipix jiñi wiñik.}
\exampletranslation{A ese hombre se le hace difícil doblar la rodilla.}
\onedefinition{2}
\spanishtranslation{doblar, desviar}
\clarification{callejón}
\cholexample{Jiñi kalejóñ mi ilux ibä ya'ba'añ kolembä xajlel.}
\exampletranslation{El callejón se desvia donde está la roca grande.}

\entry{luxul}
\partofspeech{adj}
\spanishtranslation{doblado}
\clarification{rodilla, camino}
\cholexample{Luxul jach iyok, ma'añik mi mejlel isäts'.}
\exampletranslation{Su rodilla está doblada; no puede estirarla.}

\entry{lu'}
\partofspeech{pron}
\spanishtranslation{todos}
\cholexample{Tsa' ilu' japäyob sa'.}
\exampletranslation{Todos tomaron el pozol.}

\entry{lu'chäy}
\partofspeech{s}
\spanishtranslation{filín}
\clarification{pez grande sin espinas}

\entry{lu'ichtyik}
\partofspeech{adj}
\spanishtranslation{arrugado}
\clarification{persona}

\alphaletter{M}

\entry{makom}
\partofspeech{s}
\spanishtranslation{mora}

\entry{mako'}
\partofspeech{s}
\spanishtranslation{cucaracha}
\clarification{insecto}

\entry{makuliñ}
\partofspeech{s}
\nontranslationdef{Cáscara de un árbol que se usa como remedio para el paludismo.}

\entry{makulis}
\partofspeech{s}
\spanishtranslation{matilisguate}
\clarification{árbol}

\entry{machik}
\partofspeech{part}
\spanishtranslation{si no hubiera}
\cholexample{Machik tsa' koltyayety tyi yäk'ñäñtyel acholi, ma'añik tsa' ujtyi acha'añ.}
\exampletranslation{Si no te hubiera ayudado a limpiar tu milpa, no hubieras terminado.}

\entry{machiki}
\partofspeech{part}
\spanishtranslation{si no fuera por eso}
\clarification{negativo y condicional}
\cholexample{Machiki tsa' jkotya ma'añik tsa' ujtyi iyäk'ñañ ikajpe'lel.}
\exampletranslation{Si no fuera porque lo ayudé, no habría terminado de limpiar su cafetal.}

\entry{machity}
\partofspeech{s esp}
\spanishtranslation{machete}

\entry{machtyika}
\partofspeech{adv}
\spanishtranslation{quizás}
\cholexample{Machtyika kajik imel ichol.}
\exampletranslation{Quizás ya no podrá hacer su milpa.}

\entry{mach...-ik}
\partofspeech{part}
\nontranslationdef{Forma negativa usada con verbos y atributivos; p. ej.:}
\cholexample{mach lajalik}
\exampletranslation{no es igual.}
\secondaryentry{mach jalik}
\secondtranslation{pronto (lit.: no tarda)}
\secondaryentry{mach mejlik}
\secondtranslation{imposible}
\secondaryentry{mach ñukik}
\secondtranslation{no es grande}
\secondaryentry{mach weñik}
\secondtranslation{no está bueno}
\secondaryentry{mach wokolik}
\secondtranslation{fácil (lit.: no es difícil)}

\entry{mach'añ}
\relevantdialect{Sab.}
\partofspeech{adv}
\spanishtranslation{no}
\cholexample{Mach'añ lujbeñoñ.}
\exampletranslation{No estoy cansado.}

\entry{maestyru}
\partofspeech{s esp}
\spanishtranslation{maestro}

\entry{majamol}
\relevantdialect{Sab.}
\partofspeech{s}
\spanishtranslation{zacatal}
\alsosee{jamil}

\entry{majañ}
\partofspeech{adv}
\spanishtranslation{prestado}
\cholexample{Tyi majañ jach tsa' iyäk'eyoñ jiñi mula.}
\exampletranslation{Sólo me dio su mula prestada.}
\secondaryentry{*majañ *äskuñ}
\secondtranslation{hermanastro mayor}
\secondaryentry{*majañ kalobil}
\secondtranslation{hijastro}
\secondaryentry{*majañ chich}
\secondtranslation{hermanastra}
\secondaryentry{*majañ ijtyi'añ}
\secondtranslation{hermanastro, hermanastra}
\secondaryentry{*majañ ña'}
\secondtranslation{madrastra}
\secondaryentry{*majañ tyaty}
\secondtranslation{padrastro}

\entry{majas}
\partofspeech{s}
\spanishtranslation{palito para mover atole}

\entry{majaw}
\partofspeech{s}
\spanishtranslation{corcho}

\entry{*majk}
\partofspeech{s}
\spanishtranslation{tapa}
\cholexample{Mach ip'isolik imajk jiñi limetye.}
\exampletranslation{La tapa de esa botella no es de la medida.}

\entry{majchil}
\partofspeech{s}
\onedefinition{1}
\spanishtranslation{pariente}
\onedefinition{2}
\spanishtranslation{clan}
\spanishtranslation{familia extendida}
\clarification{Tila; no se ha encontrado término que corresponda en el dialecto de Tumbalá.}

\entry{majchki}
\relevantdialect{Tila}
\partofspeech{pron}
\spanishtranslation{¿quién?}
\alsosee{majki}

\entry{majlel}
\partofspeech{vi}
\onedefinition{1}
\spanishtranslation{ir}
\cholexample{Che' tyi yambä tsik mux kmajlel tyi yajalóñ.}
\exampletranslation{Para el próximo mes voy a ir a Yajalón.}
\onedefinition{2}
\spanishtranslation{resultar}
\cholexample{Mach wiñik woli tyi majlel ke'tyel.}
\exampletranslation{No está resultando bien mi trabajo.}
\conjugationtense{3ª pers. sing. pret.:}
\otherconjugation{tsajni}
\alsosee{Gram. 6.17}

\entry{*majlib k'iñ}
\relevantdialect{Tila}
\spanishtranslation{Oeste}

\entry{majmaj ul}
\spanishtranslation{tipo de mariposa chica}
\clarification{sale en mayo}

\entry{majñañ}
\partofspeech{vt}
\spanishtranslation{pedir prestado}
\cholexample{Yom mi amajñañ palacha'añ mi ak'äñ tyityop' lum.}
\exampletranslation{Debes pedir prestada una pala para cavar tierra.}

\entry{majñäbeñ}
\partofspeech{vt}
\spanishtranslation{pedirle prestado}
\cholexample{Mik majlel kmajñäbeñ ijuloñib.}
\exampletranslation{Voy a pedirle prestada su escopeta.}

\entry{majñäñtyel}
\partofspeech{vi}
\spanishtranslation{prestar}
\cholexample{Woli jach iyäk' tyi majñäñtyel ityak'iñ.}
\exampletranslation{Sólo está dando a prestar su dinero.}

\entry{majki}
\partofspeech{pron}
\onedefinition{1}
\spanishtranslation{quien}
\cholexample{La'lakyajkañ majki mi iyochel.}
\exampletranslation{Nosotros escogemos quien se pueda encargar de la responsabilidad.<>}
\onedefinition{2}
\spanishtranslation{¿quién?}
\cholexample{¿majki tsa' isek'e jiñi tye'?}
\exampletranslation{¿Quién tumbó ese árbol?}
\dialectvariant{Tila}
\dialectword{majchki}

\entry{majtyañäl}
\partofspeech{s}
\spanishtranslation{regalo}

\entry{majts aläl}
\spanishtranslation{pañal}

\entry{majtsäl}
\partofspeech{s}
\spanishtranslation{enaguas}
\spanishtranslation{falda}

\entry{majts'añ}
\partofspeech{vt}
\spanishtranslation{chupar}
\cholexample{Woli imajts'añ ik'äb jiñi aläl.}
\exampletranslation{Esa criatura se está chupando la mano.}

\entry{mal}
\partofspeech{adv}
\spanishtranslation{adentro}
\secondaryentry{imal lakok}
\secondtranslation{planta del pie}
\secondaryentry{imal laj k'äb}
\secondtranslation{palma de la mano}
\secondaryentry{imal otyoty}
\secondtranslation{adentro de la casa}

\entry{*malojlel}
\relevantdialect{Sab.}
\partofspeech{s}
\spanishtranslation{pecado}

\entry{mam}
\defsuperscript{1}
\partofspeech{s}
\onedefinition{1}
\spanishtranslation{hermano de la abuela paterna}
\onedefinition{2}
\spanishtranslation{partera que atiende un nacimiento}

\entry{mam}
\defsuperscript{2}
\relevantdialect{Tila}
\partofspeech{s}
\onedefinition{1}
\spanishtranslation{nieto}
\onedefinition{2}
\spanishtranslation{descendiente}
\alsosee{buts}

\entry{mañko}
\partofspeech{s esp}
\spanishtranslation{mango}

\entry{mañdal}
\partofspeech{s esp}
\spanishtranslation{orden}
\cholexample{Tsa' iyäk'eyoñ mañdal cha'añ mi jkäñtyañ jiñi wiñik.}
\exampletranslation{Me dio la orden de cuidar a ese hombre.}

\entry{mañdar}
\partofspeech{s esp}
\onedefinition{1}
\spanishtranslation{ley del país o del pueblo}
\cholexample{Wersa yom lakp'is tyi weñtya imañdar gobierño.}
\exampletranslation{Es necesario tomar en cuenta la ley del gobierno.}
\onedefinition{2}
\spanishtranslation{órdenes y reglas de una familia}
\cholexample{Jiñi tyatyäl jiñäch mu'bä icha'leñ mañdar tyi'tyojlel iyalobilob yik'oty iyijñam.}
\exampletranslation{El padre es el que da órdenes a sus hijos y a su esposa.}

\entry{mañsojiyel}
\partofspeech{vi esp}
\spanishtranslation{amansarse}
\clarification{caballo, mula}
\cholexample{Wersa mi imañsojiyel jiñi mula.}
\exampletranslation{Es necesario que se amanse la mula.}

\entry{mañtyekaty}
\partofspeech{s esp}
\spanishtranslation{puerco capado}

\entry{maña}
\partofspeech{adj esp}
\spanishtranslation{mañoso}
\cholexample{Wem mañajaxbajche' mi ixujch'iñ tyak'iñ.}
\exampletranslation{Es muy mañoso para robar dinero.}

\entry{mañax bu'ul}
\spanishtranslation{cacahuate}

\entry{masamuñija'}
\partofspeech{s}
\onedefinition{1}
\spanishtranslation{planta medicinal}
\onedefinition{2}
\spanishtranslation{nombre de una finca en el municipio de Sabanilla}

\entry{Masoñija'}
\partofspeech{s}
\spanishtranslation{nombre de colonia}

\entry{matyemuty}
\partofspeech{s}
\spanishtranslation{pájaro del monte de cualquier tipo}

\entry{matye' chityam}
\spanishtranslation{jabalí de collar}
\clarification{mamífero}

\entry{matye'el ts'i'}
\spanishtranslation{coyote}
\clarification{mamífero}

\entry{matsa'}
\partofspeech{part}
\spanishtranslation{no poder}
\cholexample{Matsa mik ña'tyañ mi muk'äch kaj iyäk'.}
\exampletranslation{No puedo saber si va a dar su mula.}

\entry{max}
\partofspeech{s}
\spanishtranslation{mico}
\clarification{mamífero}

\entry{max tyo}
\spanishtranslation{todavía}

\entry{mayor}
\partofspeech{s esp}
\spanishtranslation{autoridad}
\culturalinformation{Información cultural: El <mayor> ordena al <wasil.> También acompaña al <wasil> a cumplir la misión que se le encomendó.}

\entry{ma'}
\partofspeech{adv}
\spanishtranslation{no}
\cholexample{Mi atsa' añik mi kaj kmel kchol.}
\exampletranslation{No voy a hacer mi milpa.}

\entry{ma'añ}
\relevantdialect{Sab., Tila}
\partofspeech{adv}
\onedefinition{1}
\spanishtranslation{no}
\onedefinition{2}
\spanishtranslation{no hay}

\entry{ma'añik}
\partofspeech{adv}
\onedefinition{1}
\spanishtranslation{no}
\cholexample{Ma'añik mi kaj kmajlel tyi tyejklum ijk'äl.}
\exampletranslation{Mañana no voy al pueblo.}
\onedefinition{2}
\spanishtranslation{no hay}
\cholexample{Ma'añik kcha'añ bu'ul tyi kotyoty.}
\exampletranslation{No hay frijol en mi casa.}

\entry{Ma'chäyil}
\partofspeech{s}
\spanishtranslation{Lugar de Bastante Pescado}
\clarification{rancho}

\entry{ma'tyika}
\partofspeech{adv}
\spanishtranslation{quizás}
\cholexample{Ma'tyika muk'ix imejlel lakmajlel.}
\exampletranslation{Quizás ya pueda uno irse.}

\entry{mäk}
\partofspeech{vt}
\spanishtranslation{tapar}
\cholexample{Jiñi wiñik woli imäk iyotyoty cha'añ mach mi iyochel ochja'.}
\exampletranslation{Ese hombre está tapando el techo de su casa para que no tenga goteras.}

\entry{mäkäkña}
\partofspeech{adv}
\spanishtranslation{cerrándose}
\clarification{con nubes}
\cholexample{Mäkäkña pañimil tyityokal.}
\exampletranslation{Está cerrándose con nubes.}

\entry{mäkäl}
\partofspeech{adj}
\spanishtranslation{brumoso, oscuro}

\entry{mäktyañ}
\partofspeech{vt}
\spanishtranslation{atajar}
\cholexample{Jiñi xmäñ ixim mi imajlel tyi bij cha'añ mi imäktyañ wiñikob wolibä ikuchob tyilel ixim.}
\exampletranslation{El comprador de maíz se va por el camino para atajar a los que traen cargado su maíz para venderlo.}

\entry{mäktya'}
\partofspeech{s}
\spanishtranslation{estreñimiento}
\cholexample{Cha'añ mäktya' tsa' chämi jiñi wiñik.}
\exampletranslation{Ese hombre se murió de estreñimiento.}

\entry{mäk'}
\partofspeech{vt}
\spanishtranslation{comer}
\clarification{alimento blando}
\cholexample{Jiñi uch woli imäk' ja'as.}
\exampletranslation{El tlacuache está comiendo plátano.}

\entry{mäk'bil}
\partofspeech{adj}
\spanishtranslation{comestible}

\entry{mäk'lañ}
\partofspeech{vt}
\spanishtranslation{mantener}
\cholexample{Añ tyi weñtya jiñi x'ixik cha'añ mi imäk'lañ iyalobilob.}
\exampletranslation{Esa mujer tiene la responsabilidad de mantener a sus hijos.}

\entry{mächäkña}
\partofspeech{adv}
\onedefinition{1}
\spanishtranslation{atardeciendo}
\cholexample{Mächäkña iyik'añ tsa' k'otyiyoñ tyi kotyoty.}
\exampletranslation{Llegué a mi casa atardeciendo.}
\onedefinition{2}
\spanishtranslation{débilmente}
\cholexample{Mächäkña jax mi icha'leñ xämbal jiñi wiñik kome ñoxix.}
\exampletranslation{Ese hombre camina débilmente porque ya está viejo.}

\entry{mäch p'ok}
\partofspeech{s}
\spanishtranslation{iguana}
\clarification{reptil}

\entry{mäjkel}
\partofspeech{vi}
\spanishtranslation{nublarse}
\cholexample{Wolix imäjkel pañimil kome tyalix ja'al.}
\exampletranslation{Se está nublando porque ya va a llover.}

\entry{mäjkibäl}
\partofspeech{s}
\spanishtranslation{cárcel}

\entry{*mäjkib chityam}
\spanishtranslation{chiquero}

\entry{mäl}
\partofspeech{vt}
\onedefinition{1}
\spanishtranslation{aguantar}
\clarification{trabajo}
\cholexample{Ma'añik mi imäl jump'ejl k'iñ e'tyel jiñi wiñik.}
\exampletranslation{Ese hombre no aguanta un día de trabajo.}
\onedefinition{2}
\spanishtranslation{vencer}
\cholexample{Weñ mi imäl alas jiñi ch'ityoñ.}
\exampletranslation{El joven vence en sus juegos.}

\entry{mäläl}
\partofspeech{s}
\spanishtranslation{pupo barrigón}
\clarification{pez; vivíparo}

\entry{mämäksijlel ja'}
\partofspeech{s}
\spanishtranslation{tipo de insecto de agua}

\entry{mäñ}
\partofspeech{vt}
\spanishtranslation{comprar}

\entry{mäñtyañ}
\partofspeech{vt}
\spanishtranslation{conseguir}
\clarification{agua}
\cholexample{Tsak mäñtya ja'ba'añ paskual.}
\exampletranslation{Conseguí agua de Pascual.}

\entry{mäp}
\relevantdialect{Sab.}
\partofspeech{s}
\spanishtranslation{cocoyol}
\clarification{árbol}

\entry{mäsañ}
\partofspeech{vt}
\spanishtranslation{tragar}
\cholexample{K'ux kbik' che' mik mäsañ iya'lel kej.}
\exampletranslation{Me duele la garganta al tragar saliva.}
\dialectvariant{Sab.}
\dialectword{buk'}

\entry{mäsmäsña}
\partofspeech{adv}
\spanishtranslation{semifuerte}
\cholexample{Mäsmäsña woli jiñi ja'al che' woli tyi tyejchel.}
\exampletranslation{Cuando la lluvia comienza, cae semifuerte.}

\entry{mätyk'äb}
\partofspeech{s}
\spanishtranslation{anillo}

\entry{mätsab}
\partofspeech{s}
\spanishtranslation{ceja}

\entry{mäy}
\partofspeech{vt}
\spanishtranslation{voltear}
\clarification{la cara}
\cholexample{Mu' jach wa' mäy iwuty jiñi wiñik che' mi ik'eloñ, kome mich' kik'oty.}
\exampletranslation{Cuando ese hombre me mira, voltea la cara para otro lado porque está enojado conmigo.}

\entry{mäyäjl}
\partofspeech{s}
\spanishtranslation{tsílica}
\clarification{tipo de calabaza}

\entry{mäyxuñ}
\partofspeech{vt}
\spanishtranslation{apagar}
\clarification{la vista}
\cholexample{Mi imäyxuk lakwuty jiñi k'iñ.}
\exampletranslation{El sol nos apaga la vista.}

\entry{-me}
\nontranslationdef{Sufijo que señala una precaución; p. ej.:}
\cholexample{Käybiletyme.}
\exampletranslation{Tú estás encargado de cuidar.}
\cholexample{Ñaxañetyme.}
\exampletranslation{Tú irás adelante.}
\cholexample{Tsajaletyme.}
\exampletranslation{Con cuidado.}

\entry{meba'}
\partofspeech{s}
\spanishtranslation{dejado huérfano, viudo}
\secondaryentry{meba' aläl}
\secondtranslation{huérfano}
\secondaryentry{meba' ch'ityoñ}
\secondtranslation{muchacho dejado huérfano}
\secondaryentry{meba' wiñik}
\secondtranslation{viudo}
\secondaryentry{meba' xch'ok}
\secondtranslation{muchacha dejada huérfana}
\secondaryentry{meba' x'ixik}
\secondtranslation{viuda}

\entry{meba' jijch}
\spanishtranslation{langosta}
\clarification{tipo que se come}

\entry{meku}
\partofspeech{part}
\spanishtranslation{ya ves}
\cholexample{Meku che'ächbajchew' tsak subeyety.}
\exampletranslation{Ya ves, así es como te dije.}

\entry{mek'}
\partofspeech{vt}
\spanishtranslation{abrazar}
\cholexample{Mi lakmek' aläl che' alätyo.}
\exampletranslation{Abrazamos a la criatura cuando está chica.}

\entry{mek' ajtso'}
\relevantdialect{Tila}
\partofspeech{s}
\spanishtranslation{ocelote}
\clarification{mamífero}

\entry{*mek'bal}
\partofspeech{s}
\spanishtranslation{cosa o criatura abrazada}
\cholexample{Añ imek'bal jiñi x'ixik.}
\exampletranslation{Esa mujer tiene abrazada a una criatura.}

\entry{mech}
\partofspeech{adj}
\spanishtranslation{inmovilizado}
\cholexample{Buchul jach kome mech.}
\exampletranslation{Sólo está sentado porque está inmovilizado.}

\entry{mech'}
\partofspeech{adj}
\spanishtranslation{perjudicial}
\cholexample{Weñ mech' jiñi alob komo mi ibok kajpe'.}
\exampletranslation{Ese chamaco es muy perjudicial porque arranca las matas de café.}

\entry{mejlel}
\partofspeech{vi}
\spanishtranslation{poder}
\cholexample{Mach mi lakña'tyañ mi muk' imejlel tyi e'tyel.}
\exampletranslation{No sabemos si podrá trabajar.}
\dialectvariant{Tila}
\dialectword{tyok'e}

\entry{mel}
\partofspeech{vt}
\onedefinition{1}
\spanishtranslation{hacer}
\cholexample{Uts'atybajche' tsa' imele ichol.}
\exampletranslation{Está bien como hizo su milpa.}
\onedefinition{2}
\spanishtranslation{juzgar}
\cholexample{Mi kaj imel ibäba'añ ambä iye'tyel.}
\exampletranslation{La autoridad va a juzgar su caso.}
\secondaryentry{mi imel ipusik'al}
\secondtranslation{estar preocupado}

\entry{melel}
\partofspeech{adv}
\spanishtranslation{es cierto}
\cholexample{Melel chuki woli ksubeñety.}
\exampletranslation{Es cierto lo que te estoy diciendo.}

\entry{melojel}
\partofspeech{s}
\spanishtranslation{proceso}
\cholexample{Jiñi wiñikob tsa' päjyiyob tyi melojel cha'añ tsa' ijats'äyob ibä.}
\exampletranslation{Esos hombres fueron llevados a proceso porque se golpearon.}

\entry{meloñel}
\partofspeech{s}
\spanishtranslation{proceso}
\cholexample{Tsa' tyejchi meloñel tyi'tyojlelob jiñi wiñikob cha'añ tsa' icha'leyob tsäñsa.}
\exampletranslation{Se comenzó un proceso contra esos hombres que cometieron un homicidio.}

\entry{melo'mulil}
\partofspeech{s}
\spanishtranslation{tribunal}

\entry{meme'}
\partofspeech{s}
\spanishtranslation{criatura}
\cholexample{Weñ ity'ojoljax imeme' jiñi x'ixik.}
\exampletranslation{La criatura de esa mujer es muy bonita.}

\entry{mep'}
\partofspeech{s}
\spanishtranslation{cangrejo}

\entry{mek'el}
\partofspeech{adj}
\spanishtranslation{abrazado}
\cholexample{Mek'el icha'añ aläl.}
\exampletranslation{Tiene abrazada una criatura.}

\entry{mero}
\partofspeech{adv esp}
\spanishtranslation{casi}
\cholexample{Mero lajal iwuty jiñi ch'ityoñbajche' ityaty.}
\exampletranslation{La cara del chamaco es casi igual a la de su papá.}

\entry{mety}
\partofspeech{s}
\spanishtranslation{nido}

\entry{metyañ}
\partofspeech{vt}
\spanishtranslation{acostarse}
\clarification{sobre}
\cholexample{Jiñi ts'i' mi imetyañ koxtyal.}
\exampletranslation{El perro se acuesta sobre el costal.}

\entry{me'}
\partofspeech{s}
\spanishtranslation{venado}
\secondaryentry{chäkme'}
\secondpartofspeech{s}
\secondtranslation{venado colorado}
\clarification{mediano}
\secondaryentry{wajch'me'}
\secondpartofspeech{s}
\secondtranslation{venado cabrito}
\clarification{chico}
\secondaryentry{yäxme'}
\secondpartofspeech{s}
\secondtranslation{venado azul}
\clarification{grande}

\entry{me'uñ}
\partofspeech{s}
\spanishtranslation{quequexte}
\clarification{planta}

\entry{mi}
\partofspeech{part}
\spanishtranslation{si}
\cholexample{Mi tsa' tyili ja'al tyi ora, ma'añik mi kaj ichämel jiñi cholel.}
\exampletranslation{Si llueve luego, no se secará la milpa.}

\entry{mich'}
\partofspeech{adj}
\spanishtranslation{enojado}
\cholexample{Pejtyelel ora mich' mi jk'el kyumijel.}
\exampletranslation{Todo el tiempo estoy enojado con mi tío.}

\entry{mich'ajel}
\partofspeech{s}
\spanishtranslation{coraje}
\cholexample{Jiñi alob wili' cha'leñ mich'ajel cha'añ ma'añik tsa' se' ak'eñtyi iwaj.}
\exampletranslation{Ese chamaco tiene coraje porque no le dieron luego su comida.}

\entry{mich'añ}
\partofspeech{vi}
\spanishtranslation{enojarse}
\cholexample{Che' añ chuki leko mik subeñtyel, ora jach mik mich'añ.}
\exampletranslation{Cuando me dice algo desagradable, de inmediato me enojo.}

\entry{mich'esañ}
\partofspeech{vt}
\spanishtranslation{enojar}
\spanishtranslation{enfadar}
\cholexample{Mach yomik mi amich'esañ atyaty.}
\exampletranslation{No debes hacer enojar a tu papá.}

\entry{mich'ikña}
\partofspeech{adj}
\spanishtranslation{enojado}
\cholexample{Mich'ikña jax mi ik'eloñlajiñi xñox.}
\exampletranslation{Ese viejo siempre nos ve enojado.}

\entry{*mich'lel}
\partofspeech{s}
\onedefinition{1}
\spanishtranslation{coraje}
\cholexample{Jiñi x'ixik ma'añik mi iñoj käy imich'lel.}
\exampletranslation{Esa mujer nunca deja su coraje.}
\onedefinition{2}
\spanishtranslation{envidia}
\cholexample{Añ imich'lel ipusik'al jiñi wiñik cha'añ weñ kabäl mi ityuk' kajpe' iyerañ.}
\exampletranslation{Ese hombre tiene envidia porque su hermano corta mucho café.}

\entry{mich'leñ}
\partofspeech{vt}
\spanishtranslation{enojarse con}
\clarification{constantemente}
\cholexample{Jiñi wiñik woli imich'leñ ityaty.}
\exampletranslation{Ese hombre se enoja con su padre.}

\entry{mijts'ity}
\partofspeech{s}
\spanishtranslation{anguila, culebra de agua}

\entry{mil}
\partofspeech{vt}
\spanishtranslation{estrangular}
\cholexample{Jiñi x'ixik mi imil jiñi muty che' yom itsäñsañ cha'añ mi ik'ux.}
\exampletranslation{La mujer estrangula la gallina cuando la quiere matar para comerla.}

\entry{milik}
\partofspeech{adv}
\spanishtranslation{de balde}
\cholexample{Milik tsa' loñ ak'e ts'ak, ma'añik tsa' lajmi.}
\exampletranslation{De balde le di medicina; no sanó.}

\entry{mis}
\partofspeech{s}
\spanishtranslation{gato}

\entry{Misolaj}
\partofspeech{s}
\spanishtranslation{nombre de río}

\entry{mistyuñtyik}
\partofspeech{adj}
\spanishtranslation{sucio y pinto}
\cholexample{Mistyuñtyik iwuty jiñi alob cha'añ ma'añik mi ipok.}
\exampletranslation{Ese chamaco tiene toda la cara sucia y pinta, porque no se la lava.}

\entry{misujel}
\partofspeech{vi}
\spanishtranslation{barrer}
\cholexample{Woliyoñtyo tyi misujel.}
\exampletranslation{Estoy barriendo todavía.}

\entry{misujeläl}
\partofspeech{s}
\spanishtranslation{basura}

\entry{*misujib}
\partofspeech{s}
\spanishtranslation{escoba}

\entry{misuñ}
\partofspeech{vt}
\spanishtranslation{barrer}
\cholexample{Ma'añik mi imulañ imisuñ iyotyoty.}
\exampletranslation{No le gusta barrer su casa.}

\entry{misuñtyel}
\partofspeech{vi}
\spanishtranslation{barrer}
\culturalinformation{Información cultural: Es parte de una ceremonia. Se dice que cuando un hombre va para que lo cure un curandero, lo primero que éste hace es pulsarle la mano para saber cuál es el motivo de su enfermedad. Cuando termina con esto, el curandero ya sabe si el enfermo se ha caído en el agua o en el camino, porque dicen que si alguien se cae, allí se queda su espíritu. Entonces el curandero se va al sitio en donde el hombre se cayó para llamar al espíritu. Se va por el camino, barriendo con ramas para traer al espíritu del hombre que se cayó en el camino.}

\entry{mits'ijty}
\partofspeech{s}
\spanishtranslation{falsa anguila}
\clarification{pez}

\entry{mits'tyi'añ}
\partofspeech{vt}
\spanishtranslation{saborear}
\cholexample{Jiñi x'ixik mi imits'tyi'añ iya'lel jiñi we'eläl.}
\exampletranslation{Esa mujer saborea el caldo de la carne.}

\entry{mix}
\partofspeech{part}
\spanishtranslation{llamada que se hace al gato}

\entry{mixuñ}
\defsuperscript{1}
\partofspeech{vt}
\spanishtranslation{cegar}
\cholexample{Jiñi k'iñ mi imixuñ lakwuty.}
\exampletranslation{El sol nos ciega los ojos.}

\entry{mixuñ}
\defsuperscript{2}
\partofspeech{vt}
\spanishtranslation{llamar}
\clarification{gato}
\cholexample{Jiñi aläl woli imixuñ jiñi mis.}
\exampletranslation{La criatura está llamando al gato.}

\entry{mi'}
\partofspeech{part}
\nontranslationdef{Palabra que indica el aspecto de lo habitual; p. ej.:}
\cholexample{Mi imel ichol che' tyi' yorajlel tyikwal.}
\exampletranslation{Hace su milpa en el tiempo de calor.}

\entry{mi' mel i pusik'al}
\spanishtranslation{ponerse triste}
\spanishtranslation{estar preocupado}
\clarification{lit.: le hace su corazón}
\cholexample{Kabäl mi imel ipusik'al kome mach mejlix imel ichol.}
\exampletranslation{Está muy triste porque ya no puede hacer su milpa.}

\entry{mok'tyäl}
\partofspeech{adv}
\nontranslationdef{Relativo a la forma de subir usando las dos piernas; p. ej.}
\cholexample{Woli tyi mok'tyäl letsel tyi tye' jiñi ch'ityoñ.}
\exampletranslation{Ese chamaco se está subiendo al árbol.}

\entry{mochilañ}
\partofspeech{vt}
\spanishtranslation{amarrar}
\clarification{animal}

\entry{mochiñ}
\partofspeech{vt}
\onedefinition{1}
\spanishtranslation{amarrar}
\clarification{animal}
\cholexample{Jiñi ch'ityoñ woli imochiñ tyi laso jiñi chityam.}
\exampletranslation{El joven está amarrando con soga el cerdo.}
\onedefinition{2}
\spanishtranslation{amarrar repetidas veces}
\clarification{costal}

\entry{mochol}
\partofspeech{adj}
\spanishtranslation{acostado}
\clarification{con los pies encogidos}
\cholexample{Ya' mochol tyi wäyib cha'añ tsäñal jiñi laktyaty.}
\exampletranslation{Nuestro padre está acostado en su cama con los pies encogidos por el frío.}

\entry{moch'}
\defsuperscript{1}
\partofspeech{vt}
\spanishtranslation{empuñar}
\clarification{mano}
\variation{mop'}

\entry{moch'}
\defsuperscript{2}
\partofspeech{s}
\spanishtranslation{pigua}
\clarification{crustáceos}

\entry{moch'ol}
\partofspeech{adj}
\spanishtranslation{empuñada}
\clarification{mano}
\cholexample{Moch'ol ik'äb cha'añ mi ijats'oñ.}
\exampletranslation{Su mano está empuñada para pegarme.}

\entry{moch'ye'}
\partofspeech{vt}
\spanishtranslation{empuñar}
\clarification{mano}
\variation{mop'ye'}

\entry{-mojañ}
\nontranslationdef{Sufijo que se presenta con raíces adjetivas que indican color y se refiere al reflejo del color.}

\entry{mojch'}
\partofspeech{s}
\spanishtranslation{capa para taparse}

\entry{mojch'añ}
\partofspeech{vi}
\spanishtranslation{taparse}
\cholexample{Tsak ch'ämä tyilel kmojch' cha'añ mik mojch'añ che' woli ja'al.}
\exampletranslation{Traje mi capa para taparme cuando está lloviendo.}

\entry{mojtyañ}
\partofspeech{vt}
\spanishtranslation{amontonarse}
\clarification{moscas}
\cholexample{Woli imojtyañ us jiñi we'eläl.}
\exampletranslation{Las moscas se están amontonando encima de la carne.}

\entry{mojtyoy}
\partofspeech{s}
\spanishtranslation{quiba}
\clarification{tipo de palma; árbol}

\entry{mol}
\partofspeech{s}
\spanishtranslation{tornamil}
\culturalinformation{Información cultural: Se siembra después de la cosecha de la milpa del año. No se quema.}
\alsosee{sijom}

\entry{momotyña}
\partofspeech{adv}
\spanishtranslation{agrupadamente}
\cholexample{Momotyña woliyob tyi e'tyel jiñi wiñikob.}
\exampletranslation{Esos hombres están trabajando agrupadamente.}

\entry{momoy}
\partofspeech{s}
\spanishtranslation{hierba santa}
\clarification{arbusto}

\entry{moñ}
\partofspeech{vt}
\spanishtranslation{alentar}
\clarification{niño}
\cholexample{Yom mi amoñ jiñi aläl.}
\exampletranslation{Alienta al niño.}

\entry{mop'}
\conjugationtense{variante}
\conjugationverb{moch'}
\spanishtranslation{empuñar}
\clarification{mano}

\entry{mop'ye'}
\conjugationtense{variante}
\conjugationverb{moch'ye}
\spanishtranslation{empuñar}
\clarification{mano}

\entry{mos}
\partofspeech{vt}
\spanishtranslation{tapar}
\cholexample{Jiñi wiñik mi imos ibä yik'oty imosil che' tyi ak'älel.}
\exampletranslation{Ese hombre se tapa con cobija en la noche.}
\secondaryentry{imosil}
\secondtranslation{su tapadera, su cobija, su chamarra}

\entry{mosbil}
\partofspeech{adj}
\spanishtranslation{tapado}
\cholexample{Mosbil yom jiñi aläl che' añ tsäñal.}
\exampletranslation{Debe estar tapada la criatura cuando hace frío.}

\entry{moso}
\partofspeech{s esp}
\spanishtranslation{criado}

\entry{mosojäñtyel}
\partofspeech{vi}
\spanishtranslation{vivir y trabajar con un patrón por un sueldo bajo}

\entry{mosojil}
\relevantdialect{Sab.}
\partofspeech{s}
\spanishtranslation{esclavitud}
\cholexample{Pätyuñ mosojil muk' tyi' yum.}
\exampletranslation{Se mantiene en esclavitud por su amo.}

\entry{mosol}
\partofspeech{adj}
\spanishtranslation{tapado}
\cholexample{Mosol iwuty jiñi x'ixik yik'oty irebus.}
\exampletranslation{Está tapada la cara de esa mujer con su rebozo.}

\entry{moskiyel}
\partofspeech{vi}
\spanishtranslation{taparse}
\cholexample{Jiñi tsuts ch'och'okax, ma'añik mik laj moskiyel.}
\exampletranslation{La cobija es muy chica; no puede taparme todo el cuerpo.}

\entry{moty}
\partofspeech{vt}
\spanishtranslation{juntar}
\clarification{leña, frijol, maíz}
\cholexample{Wolityo kmoty ksi'.}
\exampletranslation{Todavía estoy juntando mi leña.}

\entry{-motyañ}
\nontranslationdef{Sufijo que se presenta con raíces adjetivas que indican color y se refiere a un grupo de personas o animales.}

\entry{motyol}
\partofspeech{adj}
\spanishtranslation{juntos}
\clarification{casas}
\cholexample{Jiñäch ya'ba' motyolob iyotyoty.}
\exampletranslation{Allá es donde están juntas sus casas.}

\entry{motyomaj}
\partofspeech{s}
\spanishtranslation{cuidador de la iglesia}
\clarification{el que limpia la iglesia y atiende las velas}

\entry{motyk'iñ}
\partofspeech{vt}
\spanishtranslation{juntar}
\clarification{leña}

\entry{motso'}
\partofspeech{s}
\spanishtranslation{gusano}
\secondaryentry{imotso'lel}
\secondtranslation{sus gusanos}

\entry{mo'ch'}
\partofspeech{adv}
\nontranslationdef{Relativo a la forma en que se lleva un manojo; p. ej.:}
\cholexample{Tsa' imo'ch' ye'e ixim.}
\exampletranslation{Llevó un manojo de maíz.}

\entry{mo'och}
\partofspeech{s}
\spanishtranslation{tipo de guaco}
\clarification{ave}

\entry{mo'tye'}
\partofspeech{s}
\spanishtranslation{tipo de árbol}
\clarification{sirve para postes crianderos}

\entry{muk}
\partofspeech{vt}
\onedefinition{1}
\spanishtranslation{esconder}
\cholexample{Jiñi ch'ityoñ woli imuk ibä tyi yebal ch'ak.}
\exampletranslation{Ese chamaco se está escondiendo debajo de la cama.}
\onedefinition{2}
\spanishtranslation{enterrar}
\cholexample{Mach ñajtyikbaki tsajñi kmuk jiñi ktyaty.}
\exampletranslation{No está lejos el lugar donde fui a enterrar a mi padre.}

\entry{mukbil}
\partofspeech{adj}
\spanishtranslation{escondido}
\cholexample{Mukbil tyak abi imachity tyi mal jamil.}
\exampletranslation{Sus machetes están escondidos dentro del zacate.}

\entry{mukoñibäl}
\partofspeech{s}
\onedefinition{1}
\spanishtranslation{lugar de entierro}
\onedefinition{2}
\spanishtranslation{sepulcro}
\variation{mujkibäl}

\entry{mukuk pisil}
\partofspeech{s}
\spanishtranslation{bolsa donde se envasa el azúcar}

\entry{mukujk}
\partofspeech{s}
\spanishtranslation{bolsa}
\clarification{de tela}

\entry{mukuty'añ}
\partofspeech{vi}
\spanishtranslation{cuchichear}
\cholexample{Mach yomik mi lakcha'leñ mukuty'añ che' wili tyi sujbel ity'añ dios.}
\exampletranslation{No debemos cuchichear cuando se está predicando la Palabra de Dios.}

\entry{mukuy}
\partofspeech{s}
\spanishtranslation{paloma}

\entry{muk'}
\partofspeech{s}
\spanishtranslation{corva}

\entry{muk'uñ}
\partofspeech{adj}
\spanishtranslation{tiene ganas de obrar}

\entry{much'chokoñ}
\partofspeech{vt}
\spanishtranslation{amontonar}
\clarification{granos de café, piedras}
\cholexample{Yom mi amuch'chokoñ jiñi xajlel ya' tyi' paty otyoty.}
\exampletranslation{Debes amontonar las piedras detrás de la casa.}

\entry{much'kiñ}
\partofspeech{vt}
\spanishtranslation{juntar}
\clarification{frijol, café, maíz}

\entry{much'tyäl}
\partofspeech{adv}
\spanishtranslation{así}
\clarification{señalando un puñado de alimento seco}
\cholexample{Che' ya much'tyäl bu'ul mi iyäk'eñoñlatyi cha'p'ejl peso.}
\exampletranslation{Sólo un puñado así de frijol nos da por un peso.}

\entry{much'ul}
\partofspeech{adj}
\spanishtranslation{amontonado}
\clarification{como granos de café, piedras}
\cholexample{Ya' much'ul jiñi ixim tyi xo'tyäl.}
\exampletranslation{El maíz está amontonado en el rincón.}

\entry{*mujk}
\partofspeech{s}
\spanishtranslation{ombligo}

\entry{mujkuñ}
\partofspeech{vi}
\onedefinition{1}
\spanishtranslation{esconderse}
\cholexample{Woli imujkuñ ibä jiñi ch'ityoñ tyi matye'el.}
\exampletranslation{Ese chamaco está escondiéndose en el monte.}
\onedefinition{2}
\spanishtranslation{esconderse de}
\cholexample{Woli imujkuñety.}
\exampletranslation{Está escondiéndose de ti.}

\entry{mujchik'}
\partofspeech{adj}
\spanishtranslation{arrugado}
\clarification{piel, fruta seca}
\cholexample{Mujchik'tyik mi imajlel jiñi alaxax che' tyikiñix.}
\exampletranslation{La naranja se arruga (lit.: se vuelve arrugada) cuando se seca.}

\entry{mujch'äl}
\relevantdialect{Sab.}
\partofspeech{s}
\spanishtranslation{rebozo}
\alsosee{rebus}

\entry{*mujl}
\partofspeech{s}
\spanishtranslation{nido}
\secondaryentry{*mujl xu'}
\secondtranslation{nido de hormiga, arriera}

\entry{mujlañ}
\partofspeech{vt}
\spanishtranslation{cubrir}
\clarification{con arena, hojas, tierra, zacate}
\cholexample{Tsa' imujlabij yik'oty iyopol tye' ya' tyi' yojlil ichol.}
\exampletranslation{Cubrió con hojas el camino en medio de su milpa.}

\entry{mujläyem}
\partofspeech{adj}
\spanishtranslation{sumergido}
\cholexample{Mujläyem jiñi kolem xajlel tyi ja'.}
\exampletranslation{Esa piedra grande está sumergida en el agua.}

\entry{mujkibäl}
\partofspeech{s}
\onedefinition{1}
\spanishtranslation{lugar de entierro}
\onedefinition{2}
\spanishtranslation{sepulcro}
\alsosee{mukoñibäl}

\entry{mul}
\defsuperscript{1}
\partofspeech{vt}
\onedefinition{1}
\spanishtranslation{mojar}
\cholexample{Tsa' imuluyoñ tyi ja' jiñi ch'ityoñ.}
\exampletranslation{El niño me mojó con agua.}
\onedefinition{2}
\spanishtranslation{regar}
\cholexample{Yom mi lakmul jiñi päk'äbäl.}
\exampletranslation{Debemos regar los sembrados.}

\entry{mul}
\defsuperscript{2}
\partofspeech{s}
\spanishtranslation{culpa}
\cholexample{Tyi amul tsa' käjchiyoñ.}
\exampletranslation{Por tu culpa fui encarcelado.}

\entry{mulañ}
\partofspeech{vt}
\spanishtranslation{gustar}
\cholexample{Kabäl mi imulañ imäk' dulke jiñi ch'ityoñ.}
\exampletranslation{A ese chamaco le gusta comer dulce.}

\entry{-mulañ}
\nontranslationdef{Sufijo que se presenta con raíces adjetivas que indican color y se refiere al aspecto moteado.}

\entry{mulawil}
\relevantdialect{Sab., Tila}
\partofspeech{s}
\spanishtranslation{mundo}
\alsosee{pañimil}

\entry{mulil}
\partofspeech{s}
\onedefinition{1}
\spanishtranslation{maldad}
\cholexample{Tyi pejtyelel ora mi ityaj imul jiñi wiñik.}
\exampletranslation{Ese hombre siempre se mete en la maldad.}
\dialectvariant{Sab.}
\dialectword{simaroñiyel}
\onedefinition{2}
\spanishtranslation{delito}
\cholexample{Añ imul jiñi wiñik kome tsa' icha'le tsäñsa.}
\exampletranslation{Ese hombre tiene culpa porque cometió un delito.}
\onedefinition{3}
\spanishtranslation{culpa}
\cholexample{Tsa' tsiktyesäbeñtyi imul tyi'tyojlel jiñi año'bä iye'tyel.}
\exampletranslation{Su culpa fue reconocida por las autoridades.}
\dialectvariant{Sab.}
\dialectword{ajmulil}

\entry{Mulipa'}
\partofspeech{s}
\spanishtranslation{nombre de un arroyo}

\entry{mumo}
\partofspeech{s}
\nontranslationdef{Es una planta de tallo comestible cuando está tierno.}

\entry{murux}
\partofspeech{adj}
\spanishtranslation{encrespado}
\cholexample{Muruxtyik itsutsel ijol jiñi x'ixik.}
\exampletranslation{El pelo de esa mujer está encrespado.}

\entry{musmusña}
\partofspeech{adj}
\spanishtranslation{lloviznando}
\cholexample{Mach k'amik jiñi ja'al, musmusña jach.}
\exampletranslation{El agua no está fuerte; sólo está lloviznando.}

\entry{muty}
\partofspeech{s}
\onedefinition{1}
\spanishtranslation{gallina}
\onedefinition{2}
\spanishtranslation{gallo}
\onedefinition{3}
\spanishtranslation{pájaro}

\entry{muts}
\partofspeech{adv}
\nontranslationdef{Se relaciona con la forma de cerrar la boca de una bolsa o costal para amarrarlo; p. ej.:}
\cholexample{Tsa' imuts kächbe ityi' koxtyal che' tsa' ujtyi ibuty' tyi kajpe'.}
\exampletranslation{Al llenar el costal con café, lo cerró atándole la boca.}

\entry{muts'}
\partofspeech{vt}
\spanishtranslation{cerrar}
\clarification{los ojos}
\cholexample{Mi lakmuts' lakwuty che' mi lakwäyel.}
\exampletranslation{Cuando dormimos, cerramos los ojos.}

\entry{muts'wutyañ}
\partofspeech{vt}
\nontranslationdef{Hacer señas con los ojos para comunicar al compañero que el otro está mintiendo; p. ej.:}
\cholexample{Jiñi x'ixik woli jach imuts'wutyañ ipi'äl.}
\exampletranslation{Esa mujer sólo le está haciendo señas con los ojos a su compañera para comunicarle que el otro está mintiendo.}

\entry{mux}
\partofspeech{adv}
\spanishtranslation{ya}
\cholexample{Mux ikajel tyi chämel jiñi muty.}
\exampletranslation{Ya se va a morir esa gallina.}

\entry{muyilañ}
\partofspeech{vt}
\spanishtranslation{torcer la boca de un lado al otro}
\cholexample{Mi imuyilañ iyej jiñi alob che' mach mi ich'ujbibeñ ity'añ ityaty.}
\exampletranslation{Ese chamaco sólo tuerce la boca cuando no quiere obedecer lo que le dice su padre.}

\entry{muyul}
\partofspeech{adj}
\spanishtranslation{inclinado}
\cholexample{Muyul jiñi otyoty kome ok'beñix iyoyel.}
\exampletranslation{La casa está inclinada porque los horcones están podridos.}

\entry{*mu'}
\partofspeech{s}
\spanishtranslation{cuñada del hombre}

\alphaletter{Ñ}

\entry{ñaña}
\partofspeech{s}
\spanishtranslation{mamá}

\entry{Ñaylum}
\relevantdialect{Sab.}
\partofspeech{s}
\spanishtranslation{nombre de colonia}

\entry{ñew}
\partofspeech{adv}
\nontranslationdef{Se relaciona con la forma de inclinarse; p. ej.:}
\cholexample{Tsa' iñew choko ik' jiñi otyoty.}
\exampletranslation{El viento forzó la casa, inclinándola.}

\entry{*ñich k'ajk}
\partofspeech{s}
\spanishtranslation{brasa}

\entry{ñichikña}
\partofspeech{adj}
\spanishtranslation{floreciente}
\cholexample{Wäle ñichikña jiñi kajpe'.}
\exampletranslation{El café está floreciente ahora.}

\entry{ñichim}
\partofspeech{s}
\onedefinition{1}
\spanishtranslation{flor}
\onedefinition{2}
\spanishtranslation{vela}
\secondaryentry{iñich}
\secondtranslation{flor de planta o árbol}

\entry{Ñichimbäja'}
\partofspeech{s}
\spanishtranslation{Flor del Agua}
\clarification{colonia}

\entry{ñich'käm}
\partofspeech{vt}
\spanishtranslation{comer poco}

\entry{ñich'tyañ}
\partofspeech{vt}
\spanishtranslation{escuchar}
\cholexample{Yom mi aweñ ñich'tyañ chuki mi asubeñtyel.}
\exampletranslation{Debes escuchar bien lo que te dicen.}

\entry{ñijkañ}
\partofspeech{vt}
\onedefinition{1}
\spanishtranslation{mover}
\cholexample{Jiñi mulama'añix mi iñijkañ ibä kome chämeñix.}
\exampletranslation{Esa mula ya no se mueve porque está muerta.}
\onedefinition{2}
\spanishtranslation{arrear}
\cholexample{Woli iñijkañ majlel imuty jiñi x'ixik.}
\exampletranslation{Esa mujer está arreando su pollo.}

\entry{ñijkäbeñ i pusik'al}
\spanishtranslation{tocar el corazón}
\cholexample{Tsa' ñjijkäbeñtyi ipusik'al jiñi wiñik.}
\exampletranslation{Le tocó el corazón a ese hombre.}

\entry{ñijkäñtyel}
\partofspeech{vi}
\spanishtranslation{arrear}
\cholexample{Yom mi iñijkäñtyel lok'el jiñi muty ya' tyi mal.}
\exampletranslation{Debe sacar arreando al pollo.}

\entry{ñijlel}
\relevantdialect{Sab.}
\partofspeech{vi}
\spanishtranslation{comenzar}
\alsosee{kajlel}

\entry{ñijkel}
\partofspeech{vi}
\spanishtranslation{moverse}
\cholexample{Ma'añik mi iñijkel jiñi xajlel kome tyam ts'äpäl.}
\exampletranslation{Esa piedra no se mueve porque está enterrada muy hondo.}

\entry{*ñij'al}
\partofspeech{s}
\onedefinition{1}
\spanishtranslation{suegro del hombre}
\onedefinition{2}
\spanishtranslation{yerno}

\entry{*ñij'añ ña'}
\relevantdialect{Sab.}
\spanishtranslation{suegra}

\entry{ñip}
\partofspeech{adv}
\nontranslationdef{Se relaciona con la forma de sacar una cosa chica con algún instrumento; p. ej.:}
\cholexample{Woli iñip lok'el motso' yik'oty isajl tye'.}
\exampletranslation{Está sacando el gusano con una astilla.}

\entry{ñip'}
\partofspeech{adv}
\nontranslationdef{Se relaciona con la forma de agarrar con la boca o el pico; p. ej.}
\cholexample{Jiñi muty tsa' iñip' kämä majlel ixim.}
\exampletranslation{El pollo se llevó el maíz en el pico.}

\entry{ñik'i}
\partofspeech{adv}
\spanishtranslation{continuamente}
\cholexample{¿chuki mi la'k'otyel lañik'i mel ya' tyi tyejklum?}
\exampletranslation{¿Qué cosa es lo que ustedes llegan a hacer al pueblo continuamente?}

\entry{*ñi'}
\partofspeech{s}
\spanishtranslation{nariz}
\secondaryentry{*ñi' iwuty}
\secondtranslation{última fruta}
\secondaryentry{*ñi' muty}
\secondtranslation{pico de pájaro}

\entry{ñi'law}
\partofspeech{adv}
\nontranslationdef{Se relaciona con la forma en que se mueven muchos gusanos; p. ej.}
\cholexample{Ñi'law jachix woli iñijkañ ibä jiñi motso' ya' tyi' lojwemal wakax.}
\exampletranslation{Muchos gusanos se están moviendo en la herida del ganado.}

\entry{ñi'tsil}
\partofspeech{s}
\nontranslationdef{Término recíproco usado por los parientes de las dos personas que se casan.}

\entry{ñi'uk'}
\partofspeech{s}
\spanishtranslation{chayote}

\entry{*ñi'wits}
\partofspeech{s}
\spanishtranslation{cumbre}

\alphaletter{Ñ}

\entry{-ña}
\nontranslationdef{Sufijo que se presenta con raíces atributivas para formar otra raíz atributiva que indica calidad o condición; p. ej.:}
\cholexample{tyijikña}
\exampletranslation{feliz}

\entry{ñak}
\defsuperscript{1}
\partofspeech{s}
\spanishtranslation{intención}
\cholexample{Woli tyi ñak ipusik'al cha'añ yomix ñujp'uñijel.}
\exampletranslation{Tiene la intención de casarse.}

\entry{ñak}
\defsuperscript{3}
\relevantdialect{Sab., Tila}
\partofspeech{adv}
\spanishtranslation{cuando}
\cholexample{Che' ñak tsa' bujty'i jiñi kolem ja' ma'ix tsa' mejli k'axel.}
\exampletranslation{Cuando vino la creciente en el río ya no pude pasar.}

\entry{ñak}
\defsuperscript{2}
\relevantdialect{Sab.}
\partofspeech{s}
\spanishtranslation{juego}
\cholexample{Kabäl mi imulañ ñak jiñi alob.}
\exampletranslation{A ese niño le gusta mucho el juego.}
\alsosee{alas}

\entry{ñakal}
\relevantdialect{Sab.}
\partofspeech{adv}
\spanishtranslation{sentado}
\clarification{Se entiende en Tumbalá, pero no suena normal y provoca risa.}
\alsosee{buchul}

\entry{ñakchokoñ}
\relevantdialect{Sab.}
\partofspeech{vt}
\spanishtranslation{sentarlo}
\clarification{Se entiende en Tumbalá, pero no suena normal y provoca risa.}
\alsosee{buchchokoñ}

\entry{*ñaklib}
\relevantdialect{Sab.}
\partofspeech{s}
\spanishtranslation{base}
\clarification{de casa}
\alsosee{k'uklib}

\entry{ñakob}
\partofspeech{s}
\spanishtranslation{codorniz}
\clarification{ave del tamaño de un pollo, de color oscuro, con copete, sin cola, y anda en el suelo}
\variation{ñakol}

\entry{ñaktyäl}
\relevantdialect{Sab.}
\partofspeech{vi}
\spanishtranslation{sentarse}
\clarification{Se entiende en Tumbalá, pero no suena normal y provoca risa.}
\alsosee{buchtyäl}

\entry{ñakulañ}
\partofspeech{vt}
\spanishtranslation{remover}
\clarification{piedra}
\cholexample{Wokol mi lakñakulañ jiñi xajlel.}
\exampletranslation{Será difícil remover esa piedra.}

\entry{ñajal}
\partofspeech{s}
\spanishtranslation{sueño}
\secondaryentry{cha'leñ ñajal}
\secondpartofspeech{vt}
\secondtranslation{soñar}

\entry{ñajañ}
\relevantdialect{Sab.}
\partofspeech{adv}
\onedefinition{1}
\spanishtranslation{adelante}
\cholexample{Ñajañ yom mi amajlel.}
\exampletranslation{Debes ir adelante.}
\onedefinition{2}
\spanishtranslation{primero}
\cholexample{Ñajañ tsa' ujtyi ipäk ichol.}
\exampletranslation{Él terminó primero de doblar su maíz.}
\alsosee{ñaxañ}

\entry{ñajätyesañ}
\partofspeech{vt}
\spanishtranslation{olvidar}
\cholexample{Wolik ñajätyesañ ibety jiñi wiñik.}
\exampletranslation{Estoy olvidando la deuda del hombre.}

\entry{ñajäyel}
\partofspeech{vi}
\spanishtranslation{olvidarse}
\cholexample{Tsa' ñajäyi icha'añ jiñi juñtya.}
\exampletranslation{Se le olvidó la junta.}

\entry{ñajäyem}
\partofspeech{adj}
\spanishtranslation{olvidado}
\cholexample{Ñajäyem kcha'añ jiñi tsa'bä isubeyoñ.}
\exampletranslation{Ya está olvidado lo que me dijo.}

\entry{ñajb}
\partofspeech{s}
\spanishtranslation{mar}

\entry{-ñajb}
\nontranslationdef{Sufijo numeral para contar cuartas de la mano; p. ej.:}
\cholexample{Jiñi lápiz añ juññajb ichañlel.}
\exampletranslation{Este lápiz tiene una cuarta de altura.}

\entry{ñajbañ}
\partofspeech{vt}
\spanishtranslation{medir con la cuarta de la mano}
\cholexample{Mi lakñajbañ ileñtyälel mesa.}
\exampletranslation{Medimos con la cuarta de la mano la superficie de la mesa.}

\entry{ñajleñ}
\partofspeech{vt}
\spanishtranslation{soñar}

\entry{ñajp'äk}
\partofspeech{s}
\spanishtranslation{oruga agrimensora}
\clarification{larva}

\entry{ñajty}
\partofspeech{adv}
\spanishtranslation{lejos}
\cholexample{Ñajtyba' tsak mele kchol.}
\exampletranslation{Está lejos donde hice mi milpa.}

\entry{ñaj'añ}
\partofspeech{vi}
\spanishtranslation{llenarse}
\clarification{de comida}
\cholexample{Mach komix bu'ul, ñajoñix.}
\exampletranslation{Ya no quiero frijol, estoy satisfecho.}

\entry{ñawaxax}
\partofspeech{s esp}
\spanishtranslation{navaja}

\entry{ñaxañ}
\onedefinition{1}
\partofspeech{adv}
\spanishtranslation{adelante}
\cholexample{Ñaxañ mi imajlel tyi bij jiñi tyatyäl.}
\exampletranslation{El padre va adelante en el camino.}
\onedefinition{2}
\partofspeech{adj}
\spanishtranslation{primero}
\cholexample{Jiñi ñaxambä wolibä tyilel jiñäch käskuñ.}
\exampletranslation{El primero que viene es mi hermano mayor.}
\dialectvariant{Sab.}
\dialectword{ajapam, ñajañ}

\entry{ña'}
\partofspeech{s}
\spanishtranslation{madre}

\entry{*ña'al}
\defsuperscript{1}
\partofspeech{s}
\nontranslationdef{palos a lo largo de la casa que están encaramados paralelos a los <cucujl>}

\entry{*ña'al}
\defsuperscript{2}
\partofspeech{s}
\spanishtranslation{hembra de animal}
\secondaryentry{iña'al}
\secondtranslation{gallina}
\secondaryentry{ña' ak'ache}
\secondtranslation{pava}
\secondaryentry{*ña'al laj k'äb}
\secondtranslation{dedo pulgar}
\secondaryentry{*ña'al tsäñal}
\secondtranslation{hielo, helado}

\entry{*ña'al}
\defsuperscript{3}
\partofspeech{s}
\spanishtranslation{dios de la abundancia de plantas y animales}
\culturalinformation{Información cultural: Se dice que aparece en forma concreta en el maíz, frijol, pollos y puercos. Los ídolos antiguos de los choles fueron hechos por este dios.}

\entry{ña'atyuñ}
\partofspeech{s}
\spanishtranslation{metate}

\entry{ña'ik'}
\partofspeech{s}
\spanishtranslation{viento fuerte}
\clarification{madre de viento}

\entry{ña'iñ}
\partofspeech{vi}
\spanishtranslation{saludar a una vieja}

\entry{*ña'jel}
\partofspeech{s}
\spanishtranslation{tía}
\clarification{hermana de la madre}

\entry{ña' muty}
\partofspeech{s}
\spanishtranslation{gallina}

\entry{ña'tyañ}
\partofspeech{vt}
\onedefinition{1}
\spanishtranslation{pensar}
\cholexample{Yom mi aña'tyañbajche' mi asubeñ ambä iye'tyel.}
\exampletranslation{Debes pensar bien cómo vas a hablar con la autoridad.}
\onedefinition{2}
\spanishtranslation{comprender}
\cholexample{Yom mi aña'tyañ pañimil, mach yomik ch'äñch'äña mi acha'leñ ty'añ.}
\exampletranslation{Hay que comprender las cosas para no hablar comoquiera.}
\onedefinition{3}
\spanishtranslation{recordar}
\cholexample{Yom mi aña'tyañ chuki tsa' subeñtyiyety cha'añ mi amäñ majlel.}
\exampletranslation{Hay que recordar lo que te dijeron que compraras.}

\entry{*ña'tyäbal}
\partofspeech{s}
\onedefinition{1}
\spanishtranslation{inteligencia}
\cholexample{Weñ yujil imel iye'tyel jiñi wiñik kome añ kabäl iña'tyäbal.}
\exampletranslation{Ese hombre sabe hacer bien su trabajo porque tiene mucha inteligencia.}
\onedefinition{2}
\spanishtranslation{comprensión}
\cholexample{Jiñi wiñik añix iye'tyel tyi komisariado kome añ iña'tyäbal.}
\exampletranslation{Ese hombre tiene la responsabilidad de comisario por su comprensión.}

\entry{ña'tyäñtyel}
\partofspeech{vi}
\spanishtranslation{pensar}
\cholexample{Woli iña'tyäñtyelbajche' mi kaj ityojtyäl.}
\exampletranslation{Está pensando cómo le va a pagar.}

\entry{ñäkäb}
\partofspeech{s}
\spanishtranslation{acción de dormitar}
\cholexample{Woli tyi ñäkäb jiñi wiñik cha'añ weñ lujb.}
\exampletranslation{Ese hombre está dormitando (lit.: está haciendo la acción de dormitar) porque está muy cansado.}

\entry{ñäkäkña}
\partofspeech{adv}
\spanishtranslation{atravesando}
\cholexample{Ñäkäkña mi iñumel jiñi bij ya' tyi' yojlil wits.}
\exampletranslation{El camino pasa atravesando en medio del cerro.}

\entry{ñäkäl}
\partofspeech{adj}
\spanishtranslation{inclinado}
\cholexample{Ñäkäl tyi bojtye' jiñi lakña'.}
\exampletranslation{Nuestra mamá está inclinada contra la pared.}

\entry{ñäkbuchul}
\partofspeech{adj}
\spanishtranslation{sentado}
\clarification{inclinado hacia atrás}
\cholexample{Ñäkbuchul jiñi wiñik ya' tyi xo'tyäl.}
\exampletranslation{Ese hombre está sentado, reclinado en el rincón.}

\entry{ñäktyäl}
\partofspeech{vi}
\spanishtranslation{reclinarse}
\cholexample{Kom ñäktyäl tyi bojtye'.}
\exampletranslation{Quiero reclinarme en la pared.}

\entry{ñäkwa'al}
\partofspeech{adj}
\spanishtranslation{parado e inclinado}
\cholexample{Ñäkwa'al jach jiñi wiñik ya' tyi xujk otyoty.}
\exampletranslation{Ese hombre está nada más parado e inclinado en la esquina de la casa.}

\entry{ñäkye'el}
\partofspeech{adj}
\spanishtranslation{inclinado en la mano}
\cholexample{Ñäkye'el icha'añ ijuloñib tyi' käb.}
\exampletranslation{Tiene su escopeta inclinada en la mano.}

\entry{*ñäk'}
\partofspeech{s}
\spanishtranslation{estómago}
\secondaryentry{ibäl ñäk'äl}
\secondtranslation{comida}

\entry{ñäch'äl}
\partofspeech{adj}
\onedefinition{1}
\spanishtranslation{callado}
\cholexample{Ñäch'äl tsa' käle jiñi wiñik che' tsa' ujtyi tyi a'leñtyel.}
\exampletranslation{Ese hombre se quedó callado cuando terminaron de regañarlo.}
\onedefinition{2}
\spanishtranslation{quieto}
\cholexample{Ñäch'äl jiñi aläl che' wäyäl.}
\exampletranslation{La criatura está quieta cuando está durmiendo.}

\entry{ñäch'chokoya}
\partofspeech{s}
\spanishtranslation{paz}
\cholexample{Jiñi xmeloñel mi icha'leñ ñäch'chokoya che' añ letyo.}
\exampletranslation{El juez hace la paz cuando hay pleitos.}

\entry{ñäch'tyesañ}
\partofspeech{vt}
\spanishtranslation{aquietar}
\clarification{a una persona}

\entry{*ñäch'tyilel}
\partofspeech{s}
\spanishtranslation{paz}
\cholexample{Dios mi iyäk'eñoñlai ñäch'tyilel lakpusik'al.}
\exampletranslation{Dios nos da paz (lit.: la tranquilidad de nuestros corazones).}

\entry{ñäjch'el}
\partofspeech{vi}
\onedefinition{1}
\spanishtranslation{aquietarse}
\cholexample{Mach yomik ñäjch'el tyi uk'el jiñi aläl.}
\exampletranslation{La criatura no quiere aquietarse.}
\onedefinition{2}
\spanishtranslation{calmarse}
\clarification{enfermedad}
\cholexample{Mach yomik ñäjch'el ik'amäjel.}
\exampletranslation{Su enfermedad no quiere calmarse.}
\dialectvariant{Sab.}
\dialectword{lämtyäl}

\entry{ñäjch'em}
\partofspeech{adj}
\spanishtranslation{tranquilo}
\cholexample{Ñäjch'em jiñi tyejklum che' tyi ak'älel.}
\exampletranslation{El pueblo está tranquilo en la noche.}

\entry{ñämtyesañ}
\partofspeech{vt}
\spanishtranslation{acostumbrar}

\entry{ñäm'añ}
\partofspeech{vi}
\spanishtranslation{acostumbrarse}
\cholexample{Jiñi alob wolix iñäm'añ tyi e'tyel.}
\exampletranslation{Ese chamaco está acostumbrándose a trabajar.}

\entry{ñäp'}
\partofspeech{vt}
\spanishtranslation{pegar}
\clarification{con pegamento}
\cholexample{Mi lakñäp' juñ tyi pajk'.}
\exampletranslation{Pegamos el papel en la pared.}

\entry{ñäxäkña}
\partofspeech{adv}
\spanishtranslation{derechamente}
\cholexample{Ñäxäkña tsa' ilok'o jiñi tyabla.}
\exampletranslation{Hizo la tabla derechamente.}

\entry{ñä'tye'}
\relevantdialect{Sab.}
\partofspeech{s}
\spanishtranslation{bordón}

\entry{ñek}
\relevantdialect{Tila}
\partofspeech{s}
\spanishtranslation{espíritu malo}
\culturalinformation{Información cultural: Dicen que es un hombre negro que vive en las cuevas. Sale durante la cuaresma a comerse la lengua de la gente. Se toca la flauta para defenderse de él.}

\entry{ñechekña}
\partofspeech{adv}
\nontranslationdef{Se relaciona con el sonido que hace la lluvia al caer; p. ej.:}
\cholexample{Ñechekña woli ityilel ja'al.}
\exampletranslation{Se oye que la lluvia viene acercándose.}

\entry{ñechñechña}
\partofspeech{adv}
\nontranslationdef{Se relaciona con la forma en que muchos insectos mueven las alas; p. ej.:}
\cholexample{Ñechñechña woli tyi wejlel jiñi chab.}
\exampletranslation{Las abejas están enjambrando, y se ve cómo mueven las alas.}

\entry{ñej}
\partofspeech{s}
\spanishtranslation{cola}

\entry{ñejñañ}
\partofspeech{vi}
\onedefinition{1}
\spanishtranslation{apuntar}
\cholexample{Che' mi lakñejñañ lakjuloñib,tyoj mi laktyaj chuki mi lakjul.}
\exampletranslation{Cuando apuntamos nuestra escopeta derecho, se le pega a lo que se tira.}
\onedefinition{2}
\spanishtranslation{ver con algún instrumento}
\cholexample{Woli iñejñäbeñ ibalisajlelba' mi kaj ipäk' ikajpe'.}
\exampletranslation{Está viendo con sus palitos dónde va a sembrar su café.}
\onedefinition{3}
\spanishtranslation{analizar}
\cholexample{Jiñi doktyor mi iñejñäbeñ ich'ich'el xk'amäjelob.}
\exampletranslation{El doctor analiza la sangre de los enfermos.}

\entry{ñejp'añ}
\partofspeech{vi}
\onedefinition{1}
\spanishtranslation{madurar}
\cholexample{Maxtyo añik woli tyi ñejp'añ jiñi ja'as.}
\exampletranslation{El plátano no se está madurando.}
\onedefinition{2}
\spanishtranslation{envejecer}
\cholexample{Wolix iñejp'añ jiñi x'ixik.}
\exampletranslation{Esa mujer ya está envejeciendo.}

\entry{ñejp'äyel}
\partofspeech{vi}
\spanishtranslation{madurarse}
\cholexample{Wolix iñejp'äyel jiñi ja'as.}
\exampletranslation{El plátano ya se está madurando.}

\entry{*ñejtyib joläl}
\partofspeech{s}
\spanishtranslation{prensador de pelo}

\entry{*ñejty'il}
\partofspeech{s}
\onedefinition{1}
\spanishtranslation{cuña}
\cholexample{Juñts'jity tye' mi ik'äñ cha'añ iñejty'il ityi' iyotyoty.}
\exampletranslation{Usa un palo para la cuña de la puerta de su casa.}
\onedefinition{2}
\spanishtranslation{cerradura}
\cholexample{Yom lakotsäbeñ iñejty'il ipuertyajlel otyoty.}
\exampletranslation{Debemos poner una cerradura en la puerta de la casa.}
\secondaryentry{iñejty'il joläl}
\secondtranslation{prensa pelo}

\entry{ñelekña}
\partofspeech{adv}
\nontranslationdef{Se relaciona con la forma en que se pasa algo por un lado; p. ej.:}
\cholexample{Ñelekña jach tsa' ñumi jiñi kalejóñ.}
\exampletranslation{Por un lado de mi casa pasó el callejón.}

\entry{ñelel}
\partofspeech{adj}
\spanishtranslation{por un lado}
\cholexample{Ñelel jach tsa' ñumi jiñi kalejóñ ya' tyi kotyoty.}
\exampletranslation{El callejón pasó mi casa por un lado.}

\entry{ñety}
\partofspeech{vt}
\spanishtranslation{echar candado}
\cholexample{Yom mi añety jiñi otyoty che' mi alok'el.}
\exampletranslation{Debes echar candado a la casa al salir.}

\entry{ñety'}
\partofspeech{vt}
\spanishtranslation{aplastar}
\cholexample{Tsa' iñety'e iyok tyi xajlel.}
\exampletranslation{Se aplastó el pie con una piedra.}

\entry{ñokchokoñ}
\partofspeech{vt}
\spanishtranslation{arrodillar}
\cholexample{Mach yomik mi lakñokchokoñ lakbä tyi'tyojlel diostye'.}
\exampletranslation{Uno no se debe arrodillar ante los ídolos.}

\entry{ñokol}
\partofspeech{adj}
\spanishtranslation{arrodillado}
\cholexample{Ñokol woli tyi orakióñ jiñi x'ixik.}
\exampletranslation{Esa mujer está orando arrodillada.}

\entry{ñoktyäl}
\partofspeech{vi}
\spanishtranslation{arrodillarse}

\entry{ñoch}
\partofspeech{vt}
\spanishtranslation{acercar}
\cholexample{Wolix lakñoch majlel yajalóñ.}
\exampletranslation{Ya estamos acercándonos a Yajalón.}

\entry{ñochol}
\onedefinition{1}
\partofspeech{adj}
\spanishtranslation{pegado}
\cholexample{Ñochol ityi' otyoty yik'oty pajk'.}
\exampletranslation{La puerta está pegada a la pared de la casa.}
\onedefinition{2}
\partofspeech{adv}
\spanishtranslation{cerca}
\cholexample{Ya' jach ñochol añ iyotyoty jiñi wiñik.}
\exampletranslation{La casa de aquel hombre está muy cerca.}

\entry{ñochtyañ}
\partofspeech{vt}
\spanishtranslation{acercarse}
\cholexample{Jiñi alob ma'añik mi iñochtyañ iyäskuñ kome joñtyol.}
\exampletranslation{El niño no se acerca a su hermano mayor porque es malo.}

\entry{ñoj}
\defsuperscript{3}
\partofspeech{adv}
\spanishtranslation{siempre}
\cholexample{Jiñi jach josé mi iñoj tyech juñtya.}
\exampletranslation{Sólo José dirige siempre la junta.}
\dialectvariant{Sab.}
\dialectword{ñuj}

\entry{ñoj}
\defsuperscript{1}
\partofspeech{adj}
\spanishtranslation{grande}
\secondaryentry{ñoj bij}
\secondtranslation{camino principal}
\secondaryentry{ñoj chich}
\secondtranslation{hermana mayor}
\secondaryentry{ñoj ek'}
\secondtranslation{planeta}
\secondaryentry{ñoj *äskuñ}
\secondtranslation{hermano mayor}
\secondaryentry{ñoj wits}
\secondtranslation{cerro grande}

\entry{ñoj}
\defsuperscript{2}
\onedefinition{1}
\partofspeech{adj}
\spanishtranslation{derecho}
\cholexample{Jiñi ñojbä laj k'äb jiñäch mu'bä laj k'äñ tyi chobal.}
\exampletranslation{Nuestra mano derecha es la que usamos para rozar.}
\onedefinition{2}
\partofspeech{s}
\spanishtranslation{lado derecho}

\entry{*ñojal}
\partofspeech{s}
\spanishtranslation{altura}
\clarification{de niños o animales}
\cholexample{Che'äch iñojal che'li.}
\exampletranslation{Así es su altura (señalando).}

\entry{ñoja'}
\partofspeech{s}
\spanishtranslation{río}
\dialectvariant{Sab.}
\dialectword{ñojpa}

\entry{ñojk'}
\partofspeech{s}
\spanishtranslation{ronquido}
\cholexample{Tsa' kajñi kwuty cha'añ tsa kubi ñojk'.}
\exampletranslation{Un ronquido me despertó.}

\entry{*ñojel}
\partofspeech{s}
\spanishtranslation{tamaño grande}
\clarification{mazorca}
\cholexample{Tsa'ix iyajka iñojemal ixim.}
\exampletranslation{Ya escogió las grandes mazorcas.}

\entry{ñojlel}
\partofspeech{vi}
\spanishtranslation{rodarse}
\cholexample{Woli iñojlel tyilel xajlel ya' tyi wits.}
\exampletranslation{Una piedra viene rodándose del cerro.}

\entry{Ñoj lum}
\partofspeech{s}
\spanishtranslation{pueblo principal, nombre de Tumbalá}

\entry{ñojpa'}
\relevantdialect{Sab.}
\partofspeech{s}
\spanishtranslation{río}
\alsosee{ñoja'}

\entry{ñojk'ijel}
\partofspeech{vi}
\spanishtranslation{roncar}

\entry{ñojtye'el}
\relevantdialect{Sab.}
\partofspeech{s}
\spanishtranslation{bosque}

\entry{ñolchokoñ}
\partofspeech{vt}
\spanishtranslation{acostar}
\cholexample{Ñolchokoñ jiñi aläl ya' tyi ch'ak.}
\exampletranslation{Acuesta al niño en la cama.}

\entry{ñolch'iñ}
\partofspeech{vt}
\onedefinition{1}
\spanishtranslation{revolcar}
\clarification{persona con persona o animal con animal}
\onedefinition{2}
\spanishtranslation{violar}
\clarification{a una mujer}

\entry{ñoliña}
\partofspeech{adj}
\spanishtranslation{dando vueltas}
\cholexample{Ñoliña jiñi kañika tyi mal latyu.}
\exampletranslation{La canica está dando vueltas en la lata.}

\entry{ñolok'}
\partofspeech{s}
\spanishtranslation{revolcándose}
\cholexample{Woli icha'leñ ñolok' jiñi alob ya' tyi jamil.}
\exampletranslation{El chamaco está revolcándose en el zacatal.}

\entry{ñolk'iñ}
\partofspeech{vt}
\spanishtranslation{rodar}
\cholexample{Mi lakñolk'iñ jubel xajlel tyi wits.}
\exampletranslation{Bajamos rodando una piedra del cerro.}

\entry{ñoltyäl}
\partofspeech{vi}
\spanishtranslation{acostarse}
\cholexample{Mach yomik ñoltyäl jiñi alob.}
\exampletranslation{Ese niño no quiere acostarse.}

\entry{ñomach}
\partofspeech{adv}
\spanishtranslation{verdad que así}
\cholexample{¿ñomach che'äch tsa' iyälä?}
\exampletranslation{¿Verdad que dijo así?}

\entry{ñoñochtye'}
\partofspeech{s}
\spanishtranslation{trepatronco gigante}
\clarification{ave}

\entry{ñop}
\partofspeech{vt}
\onedefinition{1}
\spanishtranslation{aprender}
\cholexample{Jalatyax woli iñop juñ jiñi ch'ityoñ.}
\exampletranslation{Ese muchacho está tardando mucho en aprender.}
\onedefinition{2}
\spanishtranslation{creer}
\cholexample{Ma'añik wolik ñop chuki woli isubeñoñ.}
\exampletranslation{No estoy creyendo lo que me está diciendo.}
\onedefinition{3}
\spanishtranslation{probar}
\cholexample{Mik majlel kñop mi muk'äch imejlel cholel ya' tyi yambä lum.}
\exampletranslation{Voy a probar si se da la milpa en otro terreno.}

\entry{ñokeb ja'as}
\spanishtranslation{plátano enano}
\alsosee{xpek' ja'as}

\entry{ñok'ijel}
\partofspeech{vi}
\spanishtranslation{hilar}
\cholexample{Weñ mi imulañ ñok'ijel jiñi lakña'.}
\exampletranslation{A esa señora le gusta mucho hilar.}

\entry{ñok'ijibäl}
\partofspeech{s}
\spanishtranslation{malacate}

\entry{ñok'iñ}
\partofspeech{vt}
\spanishtranslation{hilar}
\culturalinformation{Información cultural: Las ancianas hilan con un palito en una piedra redonda.}

\entry{ñoroch'}
\partofspeech{adj}
\spanishtranslation{crespo}
\cholexample{Jiñi x'ixik weñ ñoroch' ijol.}
\exampletranslation{Esa mujer tiene el pelo muy crespo.}

\entry{ñoroch'iyel}
\partofspeech{vi}
\spanishtranslation{encresparse}
\cholexample{Jiñi x'ixik tsa' iyäk'ä tyi ñoroch'iyel ijol.}
\exampletranslation{Esa mujer se encrespó el pelo.}

\entry{ñoty}
\partofspeech{vt}
\spanishtranslation{pegar}
\cholexample{Wolik ñoty juñ tyi pajk'.}
\exampletranslation{Estoy pegando papel en la pared.}

\entry{ñotyñotyña}
\partofspeech{adv}
\spanishtranslation{sube pegándose}
\cholexample{Ñotyñotyña mi iletsel majlel tyi tye' jiñi xch'ejku.}
\exampletranslation{El carpintero sube pegándose al árbol al ir ascendiendo.}

\entry{ñotyol}
\partofspeech{adj}
\spanishtranslation{pegado}
\cholexample{Ñotyol jiñi juñ ya' tyi tyabla.}
\exampletranslation{El papel está pegado en la tabla.}

\entry{ñox}
\partofspeech{adj}
\spanishtranslation{viejo}
\variation{xñox}

\entry{ñoxix}
\partofspeech{adj}
\spanishtranslation{ya está viejo}

\entry{*ñoxi'al}
\partofspeech{s}
\spanishtranslation{esposo}

\entry{ñoxi'aliñ}
\partofspeech{vt}
\spanishtranslation{recibir como esposo}
\cholexample{Jiñi x'ixik mi kaj iñoxi'aliñ pedro.}
\exampletranslation{Esa mujer va a recibir a Pedro como esposo.}

\entry{ñox'añ}
\partofspeech{vi}
\spanishtranslation{envejecerse}
\cholexample{Wolix iñox'añ jiñi kmula.}
\exampletranslation{Mi mula ya se está envejeciendo.}

\entry{ñuk}
\defsuperscript{1}
\partofspeech{adj}
\onedefinition{1}
\spanishtranslation{grande}
\cholexample{Weñ ñuk jiñi ch'ujtye'.}
\exampletranslation{Ese cedro está muy grande.}
\onedefinition{2}
\spanishtranslation{importante}
\cholexample{Ñuk iye'tyel jiñi jefe de zoña.}
\exampletranslation{El trabajo del jefe de zona es importante.}
\alsosee{ñoj}

\entry{ñuk}
\defsuperscript{2}
\partofspeech{adv}
\onedefinition{1}
\nontranslationdef{Se relaciona con la forma doblada; p. ej.:}
\cholexample{Ñuk yajleñ tye' tyi' yojlil bij.}
\exampletranslation{Un árbol se quedó doblado al caer a la mitad del camino.}
\onedefinition{2}
\spanishtranslation{agachada}
\clarification{forma de persona caída en el camino}
\cholexample{Tsa' ñuk yajliyoñ tyi ok'lel.}
\exampletranslation{Me caí agachado en el lodazal.}

\entry{ñukchokoñ}
\partofspeech{vt}
\spanishtranslation{invertir, poner boca abajo}
\cholexample{Ñukchokoñ jiñi p'ejty ya' tyi jobeñ.}
\exampletranslation{Pon la olla invertida en el tablero.}

\entry{*ñuklel}
\partofspeech{s}
\onedefinition{1}
\spanishtranslation{tamaño}
\cholexample{Iñuklel iyotyoty che' lajalbajche kcha'añ.}
\exampletranslation{El tamaño de su casa es igual al de la mía.}
\onedefinition{2}
\spanishtranslation{importancia}
\cholexample{Yom lakp'is tyi weñtya chuki mi iyäl kome añ iñuklel jiñi wes.}
\exampletranslation{Debemos tomar en cuenta lo que dice porque ese juez tiene importancia.}

\entry{ñukñukña}
\partofspeech{adv}
\spanishtranslation{agachado}
\cholexample{Ñukñukña mi icha'leñ xämbal jiñi xñox.}
\exampletranslation{Ese anciano camina medio agachado.}

\entry{ñuk ñumejach}
\spanishtranslation{muy fácil}

\entry{ñuktyäl}
\partofspeech{vi}
\spanishtranslation{agacharse}
\cholexample{Woli tyi ñuktyäl cha'añ mi ijap ja'.}
\exampletranslation{Está agachándose para tomar agua.}

\entry{ñukul}
\partofspeech{adj}
\spanishtranslation{embrocado}
\cholexample{Ñukul jiñi chikib ya' tyi lum.}
\exampletranslation{El canasto está embrocado en el suelo.}

\entry{ñukye'el}
\partofspeech{adj}
\spanishtranslation{boca abajo, boca agachado}
\clarification{persona, cubeta}

\entry{ñuk'}
\partofspeech{vt}
\spanishtranslation{fumar}
\cholexample{Tsa'ix iweñ ñäm'a jiñi x'ixik iñuk' k'ujts.}
\exampletranslation{Esa mujer ya se acostumbró a fumar mucho cigarro.}

\entry{ñuk'añ}
\partofspeech{vi}
\spanishtranslation{crecer}
\cholexample{Ma'añik woli tyi ñuk'añ iyalobil.}
\exampletranslation{Su hijo no está creciendo.}

\entry{ñuchil}
\relevantdialect{Tila}
\partofspeech{s}
\spanishtranslation{nalgas}
\alsosee{choj'ity}

\entry{*ñuchil}
\partofspeech{s}
\nontranslationdef{tres palos en cada lado de la casa que sostiene el caballete}
\clarification{puntales}

\entry{ñuj}
\relevantdialect{Sab.}
\partofspeech{adv}
\spanishtranslation{siempre}
\alsosee{\textsuperscript{3}ñoj}

\entry{ñujp}
\partofspeech{s}
\onedefinition{1}
\spanishtranslation{par}
\clarification{de animales, pájaros}
\onedefinition{2}
\spanishtranslation{lo que acompaña, lo que ajusta}

\entry{ñujpañ}
\partofspeech{vt}
\onedefinition{1}
\spanishtranslation{sustituir}
\cholexample{Che' mach ya'añ jiñi komisariado mi iñujpañ jiñi supleñtye.}
\exampletranslation{Cuando el comisariado no está, lo sustituye el suplente.}
\onedefinition{2}
\spanishtranslation{formar un par}
\cholexample{Jiñi wiñik mi iñujpañ ibä yik'oty jiñi x'ixik.}
\exampletranslation{El hombre forma una pareja con la mujer.}

\entry{ñujpuñel}
\partofspeech{vi}
\spanishtranslation{casarse}
\cholexample{Mach jalix mi kaj iñujpuñel jiñi xch'ok.}
\exampletranslation{Ya pronto se va a casar esa muchacha.}

\entry{ñujp'ibäl}
\partofspeech{s}
\spanishtranslation{cárcel}

\entry{*ñujp'ib la kuk'ib}
\spanishtranslation{angina}

\entry{*ñujp'il}
\partofspeech{s}
\spanishtranslation{puerta}

\entry{ñul}
\partofspeech{vt}
\spanishtranslation{chupar}
\cholexample{Jiñi ch'ityoñ mi imulañ iñul dulke.}
\exampletranslation{A ese niño le gusta chupar dulce.}

\entry{ñumel}
\partofspeech{vi}
\onedefinition{1}
\spanishtranslation{pasar}
\cholexample{Ya' jach mi kaj kñumel tyi' tyi' xchumtyäl.}
\exampletranslation{Sólo voy a pasar a la orilla de la ranchería.}
\onedefinition{2}
\spanishtranslation{exceder}
\cholexample{Añ iñumel ijol ityak'iñ mi ikajtyiñ.}
\exampletranslation{El interés que pide por su dinero excede a lo que debe ser.}

\entry{-ñumel}
\nontranslationdef{Sufijo numeral para contar vueltas, veces que se repite; p. ej.:}
\cholexample{Uxñumel mi laktyuk' jiñi kajpe'.}
\exampletranslation{Cortamos el café en tres vueltas.}

\entry{ñumeñächix}
\partofspeech{imp}
\spanishtranslation{¡pásate adelante!}

\entry{ñuñ}
\partofspeech{adv}
\spanishtranslation{luego}
\secondaryentry{kukuñuñ}
\secondpartofspeech{imp}
\secondtranslation{¡vete luego!}
\secondaryentry{se'ñuñ}
\secondpartofspeech{imp}
\secondtranslation{¡apúrate!}

\entry{ñuñty'añ}
\relevantdialect{Sab.}
\partofspeech{adj}
\spanishtranslation{desobediente}
\cholexample{Jiñi alob ñoj ñuñty'añ.}
\exampletranslation{Ese joven es muy desobediente.}
\alsosee{xñusaty'añ}

\entry{ñup'}
\defsuperscript{1}
\partofspeech{s}
\spanishtranslation{colindancia, lindero}
\cholexample{Jiñi kñup' mi iñumel tyi' ty'ejl kotyoty.}
\exampletranslation{El lindero pasa cerca de mi casa.}

\entry{ñup'}
\defsuperscript{2}
\partofspeech{vt}
\spanishtranslation{cerrar}
\cholexample{Yom lakñup' lakotyoty che' mi laklok'el.}
\exampletranslation{Debemos cerrar la casa cuando salimos.}

\entry{-ñup'añ}
\nontranslationdef{Sufijo que se presenta con raíces adjetivas que indican color y se refiere a un pedregal o un peñasco.}

\entry{ñup'ip}
\partofspeech{s}
\spanishtranslation{epiglotis}

\entry{ñup'ul}
\partofspeech{adj}
\onedefinition{1}
\spanishtranslation{encerrado}
\cholexample{Ñup'ul jiñi mulaya' tyi potyrero.}
\exampletranslation{La mula está encerrada en el potrero.}
\onedefinition{2}
\spanishtranslation{cerrado}
\cholexample{Ñup'ul jiñi otyoty.}
\exampletranslation{La casa está cerrada.}

\entry{ñuki}
\partofspeech{adv}
\spanishtranslation{comoquiera}
\cholexample{Tsa' jach ñuki mele iyotyoty, mach ity'ojolik.}
\exampletranslation{Su casa no está bonita porque la hizo comoquiera.}

\entry{ñuki ik'}
\partofspeech{s}
\spanishtranslation{viento fuerte}

\entry{ñukiña}
\partofspeech{adv}
\nontranslationdef{Se relaciona con la forma o posición agachada; p. ej.:}
\cholexample{Ñukiña woli tyi xämbal tsa' jk'ele.}
\exampletranslation{Lo vi que estaba caminando medio agachado.}

\entry{ñukityesañ}
\relevantdialect{Sab.}
\partofspeech{vt}
\spanishtranslation{consolar}
\alsosee{ñuk'esäbeñ}

\entry{ñuk'esañ}
\partofspeech{vt}
\onedefinition{1}
\spanishtranslation{ampliar}
\cholexample{Mi kaj kñuk'esañ kotyoty kome kabälix kalobilob.}
\exampletranslation{Ya voy a ampliar mi casa porque ya tengo muchos hijos.}
\onedefinition{2}
\spanishtranslation{engrandecer}
\cholexample{Jiñi komisariado mi iñuk'esañ ibä cha'añ poj añ iye'tyel.}
\exampletranslation{El comisario se engrandece sólo porque tiene ese cargo.}

\entry{ñuk'esäbeñ}
\partofspeech{vt}
\spanishtranslation{consolar}
\dialectvariant{Sab.}
\dialectword{ñukityesañ}

\entry{ñuk'esäñtyel}
\partofspeech{vi}
\onedefinition{1}
\spanishtranslation{ampliarse}
\cholexample{Mi kaj tyi ñuk'esäñtyel jiñi otyoty.}
\exampletranslation{Esa casa va a ampliarse.}
\onedefinition{2}
\spanishtranslation{engrandecerse}
\cholexample{Kabäl mi iñuk'esäñtyel jiñi wiñik.}
\exampletranslation{Ese hombre se engrandece mucho.}

\entry{ñusañ}
\partofspeech{vt}
\onedefinition{1}
\spanishtranslation{pasar}
\cholexample{Kabäl mi lakñusañ wiñikob yik'oty mulatyi bij.}
\exampletranslation{Muchos hombres y mulas pasamos en el camino.}
\onedefinition{2}
\spanishtranslation{desobedecer}
\cholexample{Jiñi alob mi iñusabeñ ity'añ iña'.}
\exampletranslation{Ese niño desobedece a su mamá.}
\onedefinition{3}
\spanishtranslation{abortar}
\cholexample{Jiñi x'ixik tsa' iñusa iyalobil.}
\exampletranslation{Esa mujer abortó a su criatura.}

\entry{ñuty'ul}
\partofspeech{adj}
\spanishtranslation{pegado}
\clarification{una juntura}
\cholexample{Lu' ñuty'ul jiñi tyablaambä tyi bojtye'.}
\exampletranslation{Todas las tablas de la pared están pegadas.}

\entry{ñuts}
\partofspeech{vt}
\spanishtranslation{unir}
\cholexample{Woli iñuts xutyuñtye' cha'añ mi imel k'ajk.}
\exampletranslation{Está uniendo la punta de los tizones para hacer fuego.}

\entry{ñutsul}
\partofspeech{adj}
\spanishtranslation{estrecho}

\entry{ñux}
\relevantdialect{Tila}
\partofspeech{adv}
\spanishtranslation{luego}
\cholexample{Tsa' ñux k'ok'ä.}
\exampletranslation{Sanó luego.}
\alsosee{tyi ora}

\entry{ñuxijel}
\partofspeech{vi}
\spanishtranslation{nadar}

\entry{ñuxukña}
\partofspeech{adv}
\spanishtranslation{nadando}
\cholexample{Ñuxukña mi imajlel tyi ja' jiñi chäy.}
\exampletranslation{Así va nadando el pez.}

\alphaletter{O}

\entry{-ob}
\nontranslationdef{Sufijo que indica el plural.}

\entry{obol}
\partofspeech{adj}
\spanishtranslation{pobre}
\cholexample{¡obol lakbä che' k'amoñla!}
\exampletranslation{¡Pobres de nosotros cuando estamos enfermos!}

\entry{*ok}
\partofspeech{s}
\spanishtranslation{pie}
\secondaryentry{kok}
\secondpartofspeech{s}
\secondtranslation{mi pie}
\secondaryentry{awok}
\secondtranslation{tu pie}
\secondaryentry{iyok}
\secondtranslation{su pie}
\secondaryentry{lakok}
\secondtranslation{nuestro pie}
\secondaryentry{iña' lakok}
\secondtranslation{nuestro dedo grande}
\secondaryentry{iyal lakok}
\secondtranslation{nuestro dedo chico}

\entry{ok'beñ}
\partofspeech{adj}
\spanishtranslation{podrido}
\cholexample{Ok'beñ jiñi tye' tsa'bä yajli.}
\exampletranslation{El árbol que se cayó está podrido.}

\entry{ok'beñal}
\partofspeech{s}
\spanishtranslation{pudrimiento}
\cholexample{Añix iyok'beñal jiñi ch'ujm.}
\exampletranslation{Esa calabaza ya está podrida (lit.: tiene pudrimiento).}

\entry{ok'lel}
\partofspeech{s}
\spanishtranslation{lodazal}
\cholexample{Añix kabäl ok'lel tyi bij cha'añ ja'al.}
\exampletranslation{Ya hay lodazales en el camino por la lluvia.}

\entry{ok'mäl}
\partofspeech{vi}
\spanishtranslation{podrirse}
\cholexample{Woli tyi ok'mäl ilojwel jiñi chityam.}
\exampletranslation{La herida del cerdo se está pudriendo.}

\entry{ok'ol}
\partofspeech{s}
\spanishtranslation{lodo}

\entry{ochel}
\partofspeech{vi}
\onedefinition{1}
\spanishtranslation{entrar}
\cholexample{Mach mik mulañ ochel tyi' yotyoty kyumijel.}
\exampletranslation{No me gusta entrar a la casa de mi tío.}
\onedefinition{2}
\spanishtranslation{recibir el cargo de}
\cholexample{Tyi yambä jabil mi kaj kochel tyi juez.}
\exampletranslation{Para el año próximo voy a recibir el cargo de juez.}
\onedefinition{3}
\spanishtranslation{comenzar a ser}
\cholexample{Tsa'ix ochiyoñ tyi xñopty'añ.}
\exampletranslation{Ya comencé a ser creyente en la Palabra de Dios.}

\entry{ocheñ}
\partofspeech{imp}
\spanishtranslation{¡pase adelante!}
\clarification{dentro de la casa}

\entry{ochix k'iñ}
\spanishtranslation{ya es tarde}
\clarification{en el día}

\entry{ochja'}
\partofspeech{s}
\spanishtranslation{gotera}

\entry{och'ajel k'iñ}
\spanishtranslation{de las tres hasta las cinco de la tarde}

\entry{och'ä k'iñ}
\relevantdialect{Sab.}
\spanishtranslation{de las tres hasta las cinco de la tarde}

\entry{ojbal}
\partofspeech{s}
\spanishtranslation{tos}

\entry{ojlil}
\partofspeech{adj}
\spanishtranslation{mitad}
\dialectvariant{Sab.}
\dialectword{lamityal}

\entry{ojol}
\partofspeech{s}
\spanishtranslation{corcho, jolocín}
\culturalinformation{Información cultural: árbol que se usa para mecapal cuando está chico; cuando está grande se usa para seto.}

\entry{-ol}
\defsuperscript{1}
\nontranslationdef{Sufijo que se presenta con raíces transitivas y neutras para formar otra raíz atributiva que indica posición; p. ej.:}
\cholexample{ñolol}
\exampletranslation{acostado.}
\variation{2*-al, -äl, 2*-el, -ul}

\entry{-ol}
\defsuperscript{2}
\nontranslationdef{Sufijo que se presenta con raíces transitivas para formar una raíz sustantiva que sirve como infinitivo. Se usa con una expresión verbal que significa ‘saber hacer’; p. ej.:}
\cholexample{Yujil ik'ajol}
\exampletranslation{Sabe tapiscar maíz.}

\entry{*olmal}
\partofspeech{s}
\spanishtranslation{hígado}
\cholexample{Añ ik'amäjel iyolmal.}
\exampletranslation{Está malo del hígado.}

\entry{*om}
\partofspeech{vt}
\spanishtranslation{querer}
\clarification{conjugación irregular en ch'ol}
\alsosee{Gram. 6.20}
\secondaryentry{kom}
\secondtranslation{yo quiero}
\secondaryentry{awom}
\secondtranslation{tú quieres}
\secondaryentry{yom}
\secondtranslation{él quiere}

\entry{omos}
\partofspeech{s esp}
\spanishtranslation{humus, hojas podridas}
\cholexample{Kabäl omos ya' tyi akajpe'lel.}
\exampletranslation{En su cafetal hay muchas hojas podridas.}

\entry{oñ}
\partofspeech{adj}
\spanishtranslation{bastante}
\cholexample{Jujump'ejl jab oñ mi jk'ax ixim.}
\exampletranslation{Cada año cosecho bastante maíz.}

\entry{-oñ}
\onedefinition{1}
\nontranslationdef{Sufijo que se presenta con verbos intransitivos en tiempo pasado para indicar la primera persona de singular del sujeto; p. ej.:}
\cholexample{Tsa' majliyoñ.}
\exampletranslation{Yo fui.}
\onedefinition{2}
\nontranslationdef{Sufijo que se presenta con verbos transitivos en tiempo pasado para indicar la primera persona de singular del objeto; p. ej.:}
\cholexample{Tsa' ijats'äyoñ.}
\exampletranslation{Me pegó.}

\entry{-oñib}
\nontranslationdef{Sufijo que se presenta con raíces transitivas y neutras para formar raíces que indican instrumento.}

\entry{oñiyi}
\partofspeech{adv}
\spanishtranslation{hace mucho tiempo}
\cholexample{Mi iyäl ktatuch cha'añ oñiyi tsa' yajli tyiñil k'ajk.}
\exampletranslation{Dice mi abuelo que hace mucho tiempo cayó ceniza.}

\entry{-oñla}
\onedefinition{1}
\nontranslationdef{Sufijo que se presenta con verbos intransitivos en tiempo pasado para indicar la primera persona inclusiva plural del sujeto; p. ej.:}
\cholexample{Tsa' majliyoña.}
\exampletranslation{Fuimos.}
\onedefinition{2}
\nontranslationdef{Sufijo que se presenta con verbos transitivos en tiempo pasado para indicar la primera persona inclusiva plural del objeto; p. ej.:}
\cholexample{Tsa' ijats'äyoñla.}
\exampletranslation{Nos pegó.}

\entry{oñel}
\partofspeech{s}
\spanishtranslation{grito}
\cholexample{Mach mik mejlel tyi oñel kome so' kbik'.}
\exampletranslation{No puedo gritar porque mi garganta está ronca.}

\entry{ok'isañ}
\relevantdialect{Sab.}
\partofspeech{vt}
\spanishtranslation{juntar}
\cholexample{Mi kok'isañlaixim tyi' yotylel.}
\exampletranslation{Juntamos maíz en la troje.}

\entry{ora}
\relevantdialect{Sab.}
\partofspeech{adv}
\onedefinition{1}
\spanishtranslation{luego}
\onedefinition{2}
\spanishtranslation{ligero}
\cholexample{Ora tsa' majli tyi lum.}
\exampletranslation{Se fue ligero al pueblo.}
\alsosee{tyi ora}

\entry{ora jach}
\partofspeech{adv}
\spanishtranslation{de inmediato}
\cholexample{Ora jach yom ma'bäk' k'ux awaj.}
\exampletranslation{Debes comer de inmediato.}

\entry{otyañ}
\partofspeech{vt}
\spanishtranslation{gritar}
\clarification{a una persona}
\cholexample{Otyañ awijts'iñ cha'añ mi ityilel.}
\exampletranslation{Grítale a tu hermanito para que venga.}

\entry{otyoty}
\partofspeech{s}
\spanishtranslation{casa}
\secondaryentry{ityi' otyoty}
\secondtranslation{entrada de una casa}
\secondaryentry{iyotylel ixim}
\secondtranslation{troje}
\secondaryentry{otyoty xux}
\secondtranslation{panal de avispa}

\entry{otsañ}
\partofspeech{vt}
\onedefinition{1}
\spanishtranslation{meter}
\cholexample{Otsañ jiñi we'eläl tyi p'ejty.}
\exampletranslation{Mete la carne en la olla.}
\onedefinition{2}
\spanishtranslation{escribir}
\clarification{nombre}
\cholexample{Otsañ ak'aba' ya' tyi juñ.}
\exampletranslation{Escribe tu nombre en el papel.}

\entry{-ox}
\nontranslationdef{Sufijo que se presenta con raíces atributivas para formar otra raíz atributiva que indica condición defectiva; p. ej.:}
\cholexample{bojlox}
\exampletranslation{suelo disparejo.}
\variation{-ux}

\entry{oy}
\partofspeech{s}
\spanishtranslation{horcón}
\secondaryentry{iyoyel otyoty}
\secondtranslation{horcón de la casa}

\entry{o'chañ}
\partofspeech{s}
\spanishtranslation{camaleón}
\clarification{reptil}

\alphaletter{P}

\entry{pakal}
\partofspeech{adj}
\spanishtranslation{bien cargado}
\clarification{maíz, fruta}
\cholexample{Pakal ichol tsa' jk'ele.}
\exampletranslation{Vi que su milpa está bien cargada (de mazorcas).}

\entry{Paktyuñ}
\partofspeech{s}
\spanishtranslation{Piedras}
\clarification{colonia}

\entry{pak'}
\partofspeech{s}
\spanishtranslation{semilla}
\secondaryentry{cha'leñ pak'}
\secondtranslation{sembrar}

\entry{pak'akña}
\partofspeech{adj}
\spanishtranslation{con neblina baja}
\cholexample{Pak'akña jiñityokal wä' tyi tyumbalá.}
\exampletranslation{Aquí en Tumbalá hay mucha neblina baja.}

\entry{pach'al}
\partofspeech{adj}
\spanishtranslation{ancho y largo}
\clarification{pie, mano}
\cholexample{Pach'al lakok.}
\exampletranslation{Nuestros pies son anchos y largos.}

\entry{paj}
\partofspeech{adj}
\spanishtranslation{agrio}
\cholexample{Weñ paj jiñi sa'.}
\exampletranslation{Ese pozol está muy agrio.}

\entry{pajäl}
\partofspeech{s}
\spanishtranslation{sosa}
\clarification{arbusto}

\entry{pajäy}
\partofspeech{s}
\spanishtranslation{zorrillo}
\clarification{mamífero}

\entry{-pajk}
\nontranslationdef{Sufijo numeral para contar telas; p. ej.:}
\cholexample{Tsak mäñä cha'pajk kpislel.}
\exampletranslation{Compré dos mudas de ropa.}

\entry{pajk'}
\partofspeech{s}
\spanishtranslation{pared de barro}

\entry{pajch'}
\partofspeech{s}
\spanishtranslation{piña}

\entry{pajch'il}
\partofspeech{s}
\spanishtranslation{sembrado de piña}

\entry{pajch'iñ}
\partofspeech{vt}
\onedefinition{1}
\spanishtranslation{revolcarse}
\clarification{de dolor}
\cholexample{Woli ipajch'iñ ibä cha'añ wokol.}
\exampletranslation{Se está revolcando de dolor.}
\onedefinition{2}
\spanishtranslation{sufrir dolores}
\clarification{de muerte}

\entry{-pajl}
\nontranslationdef{Sufijo numeral para contar racimos; p. ej.:}
\cholexample{Jiñi jumpajl ja'as añix ik'äñel.}
\exampletranslation{Ese racimo ya tiene plátanos maduros.}

\entry{pajliñ}
\partofspeech{vt}
\onedefinition{1}
\spanishtranslation{pelar}
\clarification{corteza, cáscara}
\cholexample{Yom mi apajlibeñ ipaty sik'äb cha'añ mi ak'ux.}
\exampletranslation{Hay que pelar la cáscara de la caña para chuparla.}
\onedefinition{2}
\spanishtranslation{hacer punta}
\cholexample{Mi lakpajlibeñ iñi' tye'.}
\exampletranslation{Le hacemos punta al palo.}

\entry{pajlumil}
\partofspeech{s}
\spanishtranslation{tierra agria}
\culturalinformation{Información cultural: Algunas mujeres comen esta tierra para quitarse la sed.}

\entry{pajk'ibil}
\partofspeech{adj}
\spanishtranslation{embarrado}
\cholexample{Pajk'ibil iyotyoty tsa' imele.}
\exampletranslation{Hizo su casa embarrada de lodo.}

\entry{paj sa'}
\partofspeech{s}
\spanishtranslation{pozol (pozole) agrio}

\entry{pajtyo'}
\partofspeech{s}
\spanishtranslation{caña agria}
\clarification{hierba}

\entry{paj'añ}
\partofspeech{vi}
\spanishtranslation{agriar}
\cholexample{Muk'ix ikajel tyi paj'añ jiñi bu'ul.}
\exampletranslation{Ya se va a agriar el frijol.}

\entry{palal}
\partofspeech{adj}
\spanishtranslation{arracimado}
\cholexample{Palal iwuty jiñi ja'as.}
\exampletranslation{La fruta del plátano está arracimada.}

\entry{pale}
\partofspeech{s esp}
\spanishtranslation{padre}
\clarification{religioso}
\spanishtranslation{cura}

\entry{pam}
\partofspeech{s}
\onedefinition{1}
\spanishtranslation{frente}
\clarification{de la cara}
\cholexample{Añ iyejtyal machity tyi' pam jiñi wiñik.}
\exampletranslation{Ese hombre tiene una cicatriz en la frente.}
\onedefinition{2}
\spanishtranslation{cima, vértice}
\cholexample{Ya' ty'uchul jiñi wiñik tyi' pam otyoty.}
\exampletranslation{Ese hombre está parado encima de la casa (lit.: en la cima de la casa).}
\onedefinition{3}
\spanishtranslation{patio}
\cholexample{Wolityo kpäty ipam kotyoty.}
\exampletranslation{Estoy chaporreando todavía el patio de mi casa.}
\secondaryentry{ipamlel}
\secondpartofspeech{adv}
\secondtranslation{encima}

\entry{pamakña}
\partofspeech{adj}
\onedefinition{1}
\spanishtranslation{parejo}
\cholexample{Pamakña jiñi lum ya' tyi joktyäl.}
\exampletranslation{La tierra es pareja en la planada.}
\onedefinition{2}
\spanishtranslation{bien lleno}
\cholexample{Pamakña jiñi ja' tyi p'ejty.}
\exampletranslation{La olla está bien llena de agua.}

\entry{pamal}
\partofspeech{adj}
\spanishtranslation{nivelado}
\cholexample{Pamal tsak melbe imal kotyoty.}
\exampletranslation{El piso de mi casa está nivelado.}

\entry{*pamäk ej}
\spanishtranslation{dientes}

\entry{pambij}
\partofspeech{adv}
\spanishtranslation{primero}
\clarification{en el camino}
\cholexample{Pambij woli imajlel jiñi ambä iye'tyel.}
\exampletranslation{El encargado va primero en el camino.}

\entry{*pamlel}
\partofspeech{s}
\spanishtranslation{superficie}

\entry{pampañ}
\relevantdialect{Sab.}
\partofspeech{adv}
\spanishtranslation{hablar mal (de otro)}
\cholexample{Jiñi wiñik mi ipampañ a'leñ ipi'äl.}
\exampletranslation{Ese hombre habla mal de su prójimo.}

\entry{pamuñ}
\partofspeech{vt}
\spanishtranslation{nivelar}
\cholexample{Mi kajel kpamuñ ilumil kotyoty.}
\exampletranslation{Voy a nivelar el terraplén de mi casa.}

\entry{pañchañ}
\partofspeech{s}
\spanishtranslation{cielo}
\dialectvariant{Tila}
\dialectword{chañ}

\entry{pañchañ wuty}
\spanishtranslation{persona mirando al cielo}
\cholexample{Pañchañ wuty jiñi alob kome tyi chañ jach añ iwuty.}
\exampletranslation{El chamaco siempre anda mirando al cielo (lit.: el muchacho que está mirando al cielo, siempre tiene los ojos levantados hacia arriba).}

\entry{pañch'eñ}
\partofspeech{s}
\spanishtranslation{lugar encima de una cueva, escarpa}

\entry{pañpañichim}
\partofspeech{s}
\spanishtranslation{campana}
\clarification{arbusto que da flores en forma de campana}

\entry{pañtye'}
\partofspeech{s}
\onedefinition{1}
\spanishtranslation{palo atravesado en un arroyo}
\onedefinition{2}
\spanishtranslation{puente}
\cholexample{Tsa'ix k'äski jiñi pañtye' kome okbeñix.}
\exampletranslation{El puente ya se quebró porque estaba podrido.}
\dialectvariant{Sab.}
\dialectword{k'ahtye', k'atye}

\entry{*pañtye'lel}
\partofspeech{s}
\spanishtranslation{tapesco}
\clarification{cama tosca de madera o de carrizo colocada sobre cuatro palos}

\entry{pañtyuñ}
\relevantdialect{Sab.}
\partofspeech{s}
\spanishtranslation{superficie de la piedra}
\cholexample{Ma'añ chu' mi ikolel tyi pañtyuñ.}
\exampletranslation{En la superficie de una piedra no crece nada.}

\entry{pañimil}
\partofspeech{s}
\onedefinition{1}
\spanishtranslation{mundo}
\cholexample{Jiñi lum yik'oty wits yik'otytyokal tyi pejtyelel añ tyi pañimil.}
\exampletranslation{La tierra, los cerros y las nubes están en el mundo.}
\onedefinition{2}
\spanishtranslation{país}
\cholexample{Tsajñi jkäñ yañ tyakbä pañimil.}
\exampletranslation{Fui a conocer otros países.}
\dialectvariant{Sab., Tila}
\dialectword{mulawil}

\entry{pañujl}
\partofspeech{s esp}
\spanishtranslation{pañuelo}

\entry{partye}
\relevantdialect{Sab.}
\partofspeech{s esp}
\spanishtranslation{parte, partido, facción}
\cholexample{Jiñi partyejo'bä wiñikob ma'añ mi ilajob ity'añ yik'oty komisariado.}
\exampletranslation{Una parte no apoya al comisariado.}

\entry{pasaru}
\partofspeech{s esp}
\onedefinition{1}
\nontranslationdef{uno que ha salido con un cargo como presidente, comisariado o sacristán}
\onedefinition{2}
\spanishtranslation{estimado}
\clarification{por su edad}

\entry{pasel}
\partofspeech{vi}
\onedefinition{1}
\spanishtranslation{salir}
\clarification{el sol}
\cholexample{Mux ikajel ipasel k'iñ.}
\exampletranslation{El sol ya va a salir.}
\onedefinition{2}
\spanishtranslation{brotar}
\clarification{una planta}
\cholexample{Tyi' waxäkp'ejlel k'iñ mi ipasel jiñi ixim.}
\exampletranslation{El maíz empieza a brotar a los ocho días.}

\entry{*pasib k'iñ}
\spanishtranslation{Oriente}

\entry{Pasija'}
\relevantdialect{Sab.}
\partofspeech{s}
\spanishtranslation{Agua Brotante}
\clarification{colonia}

\entry{paslam}
\partofspeech{s}
\spanishtranslation{inflamación}
\cholexample{Woli ityi pasel paslam ya' tyi kok.}
\exampletranslation{Me está saliendo una inflamación en el pie.}
\dialectvariant{Sab.}
\dialectword{pots'lom}

\entry{paso' k'iñ}
\relevantdialect{Sab.}
\spanishtranslation{Oriente}

\entry{*paty}
\relevantdialect{Sab.}
\onedefinition{1}
\partofspeech{s}
\spanishtranslation{espalda}
\cholexample{Käläx lojwem ipaty jiñi mula.}
\exampletranslation{La espalda de la mula está muy lastimada.}
\onedefinition{2}
\partofspeech{adv}
\spanishtranslation{atrás, detras}
\cholexample{Wa'al juñtyikil wiñik tyi' paty otyoty.}
\exampletranslation{Un hombre está parado detrás de la casa.}
\secondaryentry{ipaty lakbik'}
\secondtranslation{nuca}
\secondaryentry{ipaty lakok}
\secondtranslation{encima del pie}
\secondaryentry{ipaty laj k'äb}
\secondtranslation{dorso de la mano}
\secondaryentry{ipaty otyoty}
\secondtranslation{patio de la casa}
\secondaryentry{ipaty tye'}
\secondtranslation{cáscara, corteza}
\secondaryentry{ipaty xiñk'iñil}
\secondtranslation{la una de la tarde aproximadamente}

\entry{pats'}
\defsuperscript{1}
\partofspeech{s}
\spanishtranslation{petejul}
\clarification{reg.; tamal hecho de masa y frijol tierno}

\entry{pats'}
\defsuperscript{2}
\relevantdialect{Sab.}
\partofspeech{s}
\spanishtranslation{huarache}
\alsosee{xäñäbäl}

\entry{pats'al}
\partofspeech{adj}
\spanishtranslation{ancho}
\clarification{el pie}
\cholexample{Weñ pats'al iyok jiñi wiñik.}
\exampletranslation{El pie de ese hombre es muy ancho.}
\secondaryentry{pats'albä kok}
\secondtranslation{mi pie}

\entry{*pats'tyilel laj k'äb}
\spanishtranslation{anchura de la mano}

\entry{paxyal}
\partofspeech{s esp}
\onedefinition{1}
\spanishtranslation{paseo}
\cholexample{Mik majlel tyi paxyal tyi tyejklum.}
\exampletranslation{Voy de paseo al pueblo.}
\onedefinition{2}
\spanishtranslation{excusado}
\cholexample{Tsa' majli tyi paxyal.}
\exampletranslation{Fue al excusado.}

\entry{payxo}
\partofspeech{adj}
\spanishtranslation{falso}
\cholexample{Woli ipayxo päs ibä.}
\exampletranslation{Se muestra ser falsa.}
\secondaryentry{payxo tsi'ijba}
\secondtranslation{escritura falsa}

\entry{pa'}
\partofspeech{s}
\spanishtranslation{arroyo}

\entry{Pa'as lum}
\spanishtranslation{Vía Láctea}

\entry{päk}
\partofspeech{vt}
\spanishtranslation{doblar}
\clarification{una milpa}
\cholexample{Ijk'äl mi kaj kpäk kchol.}
\exampletranslation{Mañana voy a doblar mi milpa.}

\entry{päkäkña}
\partofspeech{adv}
\spanishtranslation{agachadamente}
\cholexample{Päkäkña woli ityilel jiñibajlum.}
\exampletranslation{El jaguar viene agachadamente.}

\entry{päkäl}
\partofspeech{adj}
\spanishtranslation{echado}
\clarification{animal}
\cholexample{Päkäl jiñi ts'i' ya' tyi' pam koxtyal.}
\exampletranslation{El perro está echado encima del costal.}

\entry{päkchokoñ}
\partofspeech{vt}
\spanishtranslation{echar}
\clarification{gallina}
\cholexample{Mu'tyo kaj ipäkchokoñ iña' muty.}
\exampletranslation{Todavía va a echarse su gallina.}

\entry{päkleñ}
\partofspeech{vt}
\spanishtranslation{postrar}
\cholexample{Päkleñ ya' tyi mesa.}
\exampletranslation{Póstrate en la mesa.}

\entry{päktyañ}
\partofspeech{vt}
\spanishtranslation{echar sobre}
\clarification{huevos}
\cholexample{Jiñi xña'muty mi ipäktyañ ityuñ pech.}
\exampletranslation{Esa gallina está echada sobre los huevos de la pata.}

\entry{päktyäl}
\partofspeech{vi}
\onedefinition{1}
\spanishtranslation{acostarse boca abajo}
\onedefinition{2}
\spanishtranslation{echarse}
\clarification{sobre huevos}
\cholexample{Mach yomik päktyäl jiñi xña'muty.}
\exampletranslation{La gallina no quiere echarse.}

\entry{päk'}
\defsuperscript{4}
\partofspeech{s}
\spanishtranslation{verruga}
\cholexample{Kabäl ipäk' jiñi alob.}
\exampletranslation{Ese chamaco tiene muchas verrugas.}

\entry{päk'}
\defsuperscript{1}
\partofspeech{vt}
\spanishtranslation{sembrar}
\cholexample{Tyi jilibal abril mi päk' lakchol.}
\exampletranslation{Sembramos nuestra milpa a fines de abril.}

\entry{päk'}
\defsuperscript{2}
\partofspeech{vt}
\spanishtranslation{manchar}
\cholexample{Jiñi ch'ityoñ tsa' ipäk'ä ibä tyi ok'ol.}
\exampletranslation{Ese chamaco se manchó con lodo.}

\entry{päk'}
\defsuperscript{3}
\partofspeech{vt}
\spanishtranslation{culpar}
\cholexample{Tsa' ipäk'äyob jiñi wiñik tyi xujch'.}
\exampletranslation{A ese hombre lo culparon de robo.}

\entry{päk'äbäl}
\partofspeech{s}
\spanishtranslation{hortaliza}

\entry{päk'mäyel}
\relevantdialect{Sab.}
\partofspeech{vi}
\spanishtranslation{podrir}
\cholexample{Mux ikajel tyi päk'mäyel jiñi tye'.}
\exampletranslation{Ya se va a podrir ese árbol.}

\entry{päk'ojib}
\partofspeech{s}
\spanishtranslation{macana}
\clarification{palo con punta para sembrar maíz}

\entry{*pächälel}
\relevantdialect{Sab.}
\partofspeech{s}
\onedefinition{1}
\spanishtranslation{piel}
\cholexample{Pim ipächälel kawayu'.}
\exampletranslation{La piel del caballo es gruesa.}
\onedefinition{2}
\spanishtranslation{cuerpo}
\cholexample{Kolem ipächälel jiñi wakax.}
\exampletranslation{El cuerpo de la vaca es grande.}
\secondaryentry{ipächälel lakej}
\secondpartofspeech{s}
\secondtranslation{labio}
\secondaryentry{ipächälel lakwuty}
\secondpartofspeech{s}
\secondtranslation{párpado}

\entry{pächi}
\partofspeech{s}
\onedefinition{1}
\spanishtranslation{cuero}
\cholexample{Mi laj k'äñ jiñi pächi cha'añ laktyajbal.}
\exampletranslation{Utilizamos el cuero para mecapal.}
\onedefinition{2}
\spanishtranslation{chicote}
\cholexample{Mi laj k'äñ jiñi pächi cha'añ mi lakjats' mula.}
\exampletranslation{Usamos el chicote para pegarle a la mula.}

\entry{päjkem}
\partofspeech{adj}
\spanishtranslation{doblado}
\cholexample{Maxtyo päjkemik ichol.}
\exampletranslation{Su milpa todavía no está doblada.}

\entry{päjk'el}
\defsuperscript{1}
\partofspeech{vi}
\spanishtranslation{sembrar}
\cholexample{Wäle iyorajlelix mi ipäjk'el jiñi cholel.}
\exampletranslation{Ahora es el tiempo para sembrar la milpa.}

\entry{päjk'el}
\defsuperscript{2}
\partofspeech{vi}
\spanishtranslation{mancharse}
\cholexample{Mi lakbäch' lakwex cha'añ ma'añik mi ipäjk'el tyi ok'ol.}
\exampletranslation{Arremangamos nuestro pantalón para que no se manche con el lodo.}

\entry{päl}
\partofspeech{adj}
\spanishtranslation{largo}
\cholexample{Päl jiñi ktyajbal.}
\exampletranslation{Mi mecapal está muy largo.}

\entry{päl'esañ}
\partofspeech{vt}
\spanishtranslation{alargar}
\clarification{mecate, soga}

\entry{päm}
\partofspeech{s}
\spanishtranslation{tucán cuello amarillo}
\clarification{ave}

\entry{päñg}
\partofspeech{adj}
\spanishtranslation{primero}
\clarification{fruta}
\cholexample{Maxtyo añik mi lakchoñ, kome ipäñg wutytyo.}
\exampletranslation{Todavía no la vendemos; apenas es la primera fruta.}

\entry{päñ ñumel}
\spanishtranslation{pasar de largo}
\clarification{sin saludar}
\cholexample{Mi ipäñ ñumel tyi lakty'ejl jiñi x'ixik.}
\exampletranslation{Esa mujer pasa de largo a nuestro lado, sin saludar.}

\entry{päñtyesañ}
\partofspeech{vt}
\spanishtranslation{transformar}
\cholexample{Jiñi xiba mi ipäñtyesañ ibä tyi ts'i'.}
\exampletranslation{El diablo se transforma a sí mismo en perro.}

\entry{päñtyiyel}
\partofspeech{vi}
\spanishtranslation{convertirse en}
\cholexample{Jiñi wiñik mi ipäñtyiyel tyi xiba, mi iyäl.}
\exampletranslation{Se dice que el hombre se convierte en diablo.}

\entry{päñts'uñ}
\partofspeech{vt}
\spanishtranslation{mover}
\clarification{mano, machete o palo}
\cholexample{Jiñi wiñik mi ipäñts'uñ ik'äb che' woli tyi ty'añ.}
\exampletranslation{Ese hombre mueve las manos cuando habla.}

\entry{*päk'il}
\partofspeech{s}
\spanishtranslation{antepasados}

\entry{päk'il i wuty}
\spanishtranslation{lunar del ojo}

\entry{päräñtyuñ}
\partofspeech{s}
\spanishtranslation{tirador, honda}

\entry{päs}
\partofspeech{vt}
\onedefinition{1}
\spanishtranslation{mostrar}
\cholexample{Yom mi apäsbeñ majlel ibijlel tyila.}
\exampletranslation{Hay que mostrarle el camino que va a Tila.}
\onedefinition{2}
\spanishtranslation{enseñar}
\cholexample{Yom maestyro cha'añ mi ipäs juñ.}
\exampletranslation{Quiere un maestro para que le enseñe a leer.}

\entry{*päsbal}
\partofspeech{s}
\spanishtranslation{enseñanza}

\entry{päty}
\defsuperscript{1}
\partofspeech{vt}
\spanishtranslation{chaporrear}
\clarification{camino}
\cholexample{Mik majlel kpäty bij.}
\exampletranslation{Voy para chaporrear el camino.}

\entry{päty}
\defsuperscript{2}
\relevantdialect{Tila}
\partofspeech{vt}
\spanishtranslation{hacer}
\clarification{casa}
\cholexample{Mu'tyo kajel kpäty kotyoty.}
\exampletranslation{Voy a hacer mi casa.}

\entry{pätya}
\partofspeech{s}
\spanishtranslation{guayabo}
\clarification{árbol}

\entry{pätyatye'}
\partofspeech{s}
\spanishtranslation{guayabillo}
\clarification{árbol}

\entry{pätye'}
\partofspeech{s}
\spanishtranslation{corteza de corcho}
\clarification{que se utiliza para cama}

\entry{päy}
\partofspeech{vt}
\onedefinition{1}
\spanishtranslation{llamar}
\cholexample{Woli ipäy iyijts'iñ.}
\exampletranslation{Está llamando a su hermanito.}
\onedefinition{2}
\spanishtranslation{casarse}
\cholexample{Jiñi ch'ityoñ mi kaj ipäy iyijñam.}
\exampletranslation{Ese joven va a casarse.}
\secondaryentry{päy majlel}
\secondpartofspeech{vt}
\secondtranslation{llevar}
\secondaryentry{päy tyilel}
\secondpartofspeech{vt}
\secondtranslation{traer}

\entry{peazul}
\relevantdialect{Tila}
\partofspeech{s}
\spanishtranslation{queisque}
\spanishtranslation{grajo verde}
\clarification{ave}
\alsosee{xkekex}

\entry{pek'}
\partofspeech{adj}
\onedefinition{1}
\spanishtranslation{chaparrito}
\cholexample{Pek' jiñi wiñik.}
\exampletranslation{Ese hombre es chaparrito.}
\onedefinition{2}
\spanishtranslation{bajo}
\cholexample{Pek' tsak mele kotyoty.}
\exampletranslation{Hice mi casa baja.}
\onedefinition{3}
\spanishtranslation{humillado}
\cholexample{Pek' tsa' imele ibä.}
\exampletranslation{Se mostró humillado.}

\entry{pech}
\partofspeech{s}
\spanishtranslation{pato}

\entry{pechañ}
\partofspeech{vt}
\spanishtranslation{hacer tortillas}

\entry{pechäbtye'}
\partofspeech{s}
\spanishtranslation{taburete}
\clarification{reg.}
\spanishtranslation{silla rústica}

\entry{*pechäjib}
\partofspeech{s}
\spanishtranslation{hoja de plátano}
\clarification{donde se echan las tortillas}

\entry{pechekña}
\partofspeech{adj}
\spanishtranslation{ancho}
\cholexample{Pechekña jiñi xajlel ya' tyi bij.}
\exampletranslation{Esa piedra en el camino es ancha.}

\entry{pechel}
\partofspeech{adj}
\spanishtranslation{plano}
\cholexample{Pechel iñi' jiñi pech.}
\exampletranslation{El pico del pato es plano.}

\entry{pechmañ}
\relevantdialect{Sab.}
\partofspeech{vt}
\spanishtranslation{hacer tortillas}

\entry{pechom}
\partofspeech{s}
\spanishtranslation{acción de hacer tortillas}
\cholexample{Añ ipechäbtye'ba' mi icha'leñ pechom.}
\exampletranslation{Tiene un taburete en donde está haciendo las tortillas.}

\entry{*pechkeñ}
\partofspeech{s}
\spanishtranslation{omóplato}

\entry{pechtyäl}
\partofspeech{adj}
\spanishtranslation{así de ancho}
\cholexample{Che'tyo pechtyäl ikukujlel kotyoty.}
\exampletranslation{Así de ancha es la viga de mi casa.}

\entry{pechtyeñ}
\partofspeech{vt}
\spanishtranslation{aplastar}
\cholexample{Jiñi yak mi ipechtyeñ jiñi tye'lal.}
\exampletranslation{La trampa aplasta al tepescuintle.}

\entry{pejkañ}
\defsuperscript{1}
\partofspeech{vt}
\onedefinition{1}
\spanishtranslation{hablar con}
\cholexample{Mik majlel kpejkañ kerañob.}
\exampletranslation{Voy a hablar con mis hermanos.}
\onedefinition{2}
\spanishtranslation{leer en voz alta}
\cholexample{Mik pejkañ majlel juñ.}
\exampletranslation{Leo el libro en voz alta.}

\entry{pejkañ}
\defsuperscript{2}
\partofspeech{vt}
\onedefinition{1}
\spanishtranslation{enamorar a}
\cholexample{Woli imuku pejkañ jiñi xch'ok.}
\exampletranslation{Está enamorando en secreto a la muchacha.}
\onedefinition{2}
\spanishtranslation{copular con}
\cholexample{Tsa' ipejka jiñi x'ixik.}
\exampletranslation{Copuló con la mujer.}

\entry{pejk'}
\partofspeech{s}
\spanishtranslation{mecapal de cuero}

\entry{pejpem}
\partofspeech{s}
\spanishtranslation{mariposa}
\clarification{de cualquier clase}
\secondaryentry{kolem pejpem}
\secondtranslation{mariposa gavilana}
\secondaryentry{pejpem wuty}
\secondtranslation{tiña}
\clarification{roncha}

\entry{pejpeñ tye'}
\spanishtranslation{árbol de mariposa (las hojas son semejantes a las alas de las mariposas)}

\entry{-pejty}
\nontranslationdef{Sufijo numeral para contar partes de cafetal o milpa; p. ej.:}
\cholexample{jumpejty kajpe'lel}
\exampletranslation{parte de un cafetal.}

\entry{pejtyel}
\defsuperscript{1}
\partofspeech{adj}
\spanishtranslation{todo}
\secondaryentry{pejtyel ora}
\secondtranslation{siempre}
\secondaryentry{pejtyel k'iñ}
\secondtranslation{todos los días}
\secondaryentry{pejtyel jiñi}
\secondtranslation{todo ése}
\secondaryentry{pejtyel jabil}
\secondtranslation{todos los años}

\entry{pejtyel}
\defsuperscript{2}
\partofspeech{vi}
\spanishtranslation{sacar}
\clarification{la olla del fuego}
\cholexample{Muk'ix ikaj tyi pejtyel ip'ejtyal we'eläl.}
\exampletranslation{Ya van a sacar la olla de la carne del fuego.}

\entry{pejtyelel}
\partofspeech{adj}
\spanishtranslation{todo}

\entry{peñsaliñ}
\relevantdialect{Sab.}
\partofspeech{vi esp}
\spanishtranslation{estar triste, preocuparse}
\alsosee{k'oj'ojtyañ}

\entry{peñsar}
\partofspeech{s esp}
\spanishtranslation{tristeza}
\cholexample{Woli tyi peñsar cha'añ kabäl ibety.}
\exampletranslation{Tiene tristeza por sus muchas deudas.}

\entry{pepech'ak'}
\partofspeech{s}
\spanishtranslation{bejuco que se tiende en el suelo}

\entry{pepets'}
\partofspeech{adv}
\spanishtranslation{repetidamente}
\cholexample{Woli ipepets' tyek' motso'.}
\exampletranslation{Está pisoteando repetidamente los gusanos.}

\entry{periyal}
\relevantdialect{Sab., Tila}
\partofspeech{s}
\spanishtranslation{pleito}
\alsosee{letyo}

\entry{petyejty}
\partofspeech{s}
\spanishtranslation{huso}
\clarification{palito para enrollar hilo}

\entry{petyem}
\relevantdialect{Sab.}
\partofspeech{s}
\spanishtranslation{laguna}
\alsosee{abañ}

\entry{petyol}
\partofspeech{adj}
\spanishtranslation{todo}
\cholexample{Tyi petyol jiñi cholel tsa' ilaj yäsa ik'.}
\exampletranslation{El viento botó toda la milpa.}

\entry{pets}
\partofspeech{adv}
\nontranslationdef{Concuerda con los pies encogidos; p. ej.:}
\cholexample{Pets buchjul jiñi aläl ya' tyi lum.}
\exampletranslation{Ese niño está sentado en el suelo con los pies encogidos.}

\entry{petsekña}
\partofspeech{adj}
\spanishtranslation{aplanado y ancho}
\cholexample{Petsekña jiñi xajlel ya'ba' mi ijijlelob.}
\exampletranslation{La piedra donde están descansando es aplanada y ancha.}

\entry{petsel}
\partofspeech{adj}
\spanishtranslation{en forma aplanada}
\cholexample{Petselbä ts'ak woli kap.}
\exampletranslation{Estoy tomando medicina en forma de pastilla.}

\entry{*petsetyuñ}
\partofspeech{s}
\spanishtranslation{Piedra de Forma Aplanada}
\clarification{colonia}

\entry{petspetsña}
\partofspeech{adj}
\spanishtranslation{acurrucado}
\cholexample{Petspetsña jaxtyo tyi lum jiñi aläl.}
\exampletranslation{Ese niño está acurrucado en el suelo.}

\entry{petstyäl}
\partofspeech{adj}
\onedefinition{1}
\spanishtranslation{redondo}
\cholexample{Che'tyo petstyäl jiñi uw che' pomol.}
\exampletranslation{Así de redonda es la luna cuando está llena.}
\onedefinition{2}
\spanishtranslation{redondo y sentado}
\cholexample{Petstyäl jach mi imejlel tyi lum.}
\exampletranslation{Sólo puede quedarse sentada en el suelo.}

\entry{pets'}
\partofspeech{vt}
\onedefinition{1}
\spanishtranslation{apretar}
\clarification{con la mano}
\cholexample{Tsäts mi ipets' ye' laj k'äb.}
\exampletranslation{Nos aprieta fuertemente la mano.}
\onedefinition{2}
\spanishtranslation{exprimir}
\cholexample{Woli ipets' alaxax cha'añ mi ijap.}
\exampletranslation{Está exprimiendo el jugo de la naranja para tomarlo.}

\entry{pewal}
\partofspeech{s}
\spanishtranslation{cucaracha}
\clarification{insecto}

\entry{pexel}
\partofspeech{adj}
\spanishtranslation{torcida}
\clarification{la boca}
\cholexample{Pexel iyej jiñi wiñik.}
\exampletranslation{Ese hombre tiene la boca torcida.}

\entry{peya'}
\partofspeech{s}
\spanishtranslation{pea}
\spanishtranslation{papán}
\clarification{ave}

\entry{pi}
\partofspeech{s}
\spanishtranslation{sonzapote, zonzapote, zapote amarillo}
\clarification{árbol}

\entry{pik}
\partofspeech{vt}
\spanishtranslation{cavar}
\cholexample{Mi kaj ipik jiñi lum cha'añ mi ipäk' ikajpe'.}
\exampletranslation{Va a cavar la tierra para sembrar café.}

\entry{-pik}
\nontranslationdef{Sufijo numeral para contar unidades de ocho mil; p. ej.:}
\cholexample{cha'pik}
\exampletranslation{16,000}

\entry{pikoñib}
\partofspeech{s}
\spanishtranslation{coa para cavar tierra}

\entry{pik'os}
\partofspeech{adv}
\spanishtranslation{repetidas veces}
\clarification{amarrar}
\cholexample{Jiñi alob woli ipik'os käch iyälas.}
\exampletranslation{El niño está amarrando su juguete repetidas veces.}

\entry{pik'xuñ}
\partofspeech{vt}
\spanishtranslation{manosear}
\cholexample{Jiñi alob woli jach ipik'xuñ we'eläl.}
\exampletranslation{Ese niño nada más está manoseando la carne.}

\entry{pich}
\partofspeech{s}
\spanishtranslation{orina}
\secondaryentry{cha'leñ pich}
\secondpartofspeech{vt}
\secondtranslation{orinar}

\entry{pichoñ}
\partofspeech{s}
\spanishtranslation{paloma}

\entry{pijtyañ}
\partofspeech{vt}
\spanishtranslation{esperar}

\entry{pijtyäbil}
\partofspeech{adj}
\spanishtranslation{esperado}
\cholexample{Pijtyäbil jiñi jefe de zoña kome tyal.}
\exampletranslation{El jefe de zona es esperado porque ha de venir.}

\entry{pim}
\partofspeech{adj}
\onedefinition{1}
\spanishtranslation{grueso}
\cholexample{Weñ pim mi imel iwaj.}
\exampletranslation{Hace sus tortillas muy gruesas.}
\onedefinition{2}
\spanishtranslation{tupido}
\cholexample{Käläx pim jiñi cholel.}
\exampletranslation{La milpa está demasiado tupida.}

\entry{pimel}
\partofspeech{s}
\spanishtranslation{hierba}

\entry{piñka}
\partofspeech{s esp}
\spanishtranslation{finca}

\entry{piñtsik' päm}
\partofspeech{s}
\spanishtranslation{pico de garza}

\entry{piñxik'päm}
\partofspeech{s}
\spanishtranslation{pico de hacha}
\spanishtranslation{tucán}
\clarification{ave}

\entry{pipäl}
\partofspeech{s}
\onedefinition{1}
\spanishtranslation{guaje blanco}
\clarification{árbol}
\onedefinition{2}
\spanishtranslation{tepeguaje}
\clarification{árbol}

\entry{pipity}
\partofspeech{adv}
\nontranslationdef{Se relaciona con objetos redondos; p. ej.:}
\cholexample{Mu' jach ipipity k'ech tyi' kejlab jiñi xajlel.}
\exampletranslation{Carga en su hombro la piedra.}

\entry{pisil}
\partofspeech{s}
\onedefinition{1}
\spanishtranslation{tela}
\onedefinition{2}
\spanishtranslation{ropa}
\secondaryentry{ipislel}
\secondtranslation{su ropa}

\entry{pityil}
\partofspeech{adj}
\spanishtranslation{piedra grande y redonda}
\cholexample{Ya' pityil jiñi xajlel tyi' yojlil cholel.}
\exampletranslation{Ahí está esa roca redonda y grande en medio de la milpa.}

\entry{pitytyäl}
\partofspeech{adj}
\spanishtranslation{grande}
\cholexample{Che'tyo pitytyäl jiñi xajlel ya'bä tyi bij.}
\exampletranslation{Así de grande es la piedra que está a la orilla del camino.}

\entry{*pitytyälel}
\partofspeech{s}
\spanishtranslation{altura}
\clarification{de animales}

\entry{pits'}
\partofspeech{vt}
\spanishtranslation{escaldar}
\cholexample{Tsa' ujtyi ipits' tyi tyikäw ja' iyalobil.}
\exampletranslation{Acaba de escaldar con agua caliente a su hijo.}

\entry{pits'chokoñ}
\partofspeech{vt}
\spanishtranslation{desnudar}
\cholexample{Woli ipits'chokoñ iyalobil.}
\exampletranslation{Está desnudando a su hijo.}

\entry{pits'il}
\partofspeech{adj}
\spanishtranslation{desnudo}
\cholexample{Pits'il jiñi alob kome mi kaj icha'leñ ñuxijel.}
\exampletranslation{Ese chamaco está desnudo porque va a nadar.}
\dialectvariant{Tila}
\dialectword{chakal}

\entry{pix}
\defsuperscript{1}
\partofspeech{vt}
\spanishtranslation{envolver}
\cholexample{Mi laj k'äñ juñ cha'añ mi lakpix asukal.}
\exampletranslation{Usamos papel para envolver el azúcar.}

\entry{pix}
\defsuperscript{2}
\partofspeech{s}
\spanishtranslation{rodilla}

\entry{pixil}
\partofspeech{adj}
\spanishtranslation{envuelto}
\cholexample{Pixil ijol tyi pisil jiñi x'ixik.}
\exampletranslation{La cabeza de esa mujer está envuelta con un trapo.}

\entry{*pixiñtyib}
\partofspeech{s}
\spanishtranslation{material que se usa para envolver}

\entry{pixoläl}
\partofspeech{s}
\spanishtranslation{sombrero}
\secondaryentry{kpixol}
\secondtranslation{mi sombrero}

\entry{*pixol xiba}
\partofspeech{s}
\spanishtranslation{flor de pato, flor de pelícano}
\clarification{bejuco}

\entry{pixoñib}
\partofspeech{s}
\spanishtranslation{tela que se usa para envolver}

\entry{-piyañ}
\nontranslationdef{Sufijo que se presenta con raíces adjetivas que indican color y se refiere al aspecto del cielo.}

\entry{piyikña}
\partofspeech{adj}
\spanishtranslation{liso y brilloso}
\cholexample{Piyikña jax ipaty jiñi chityam kome weñ jujp'em.}
\exampletranslation{La espalda de ese puerco está lisa y brillosa porque está bien gordo.}

\entry{pi'äl}
\partofspeech{s}
\onedefinition{1}
\spanishtranslation{compañero, amigo, vecino}
\cholexample{Jiñäch juñtyikil kpi'äl tyi e'tyel.}
\exampletranslation{Es uno de mis compañeros de trabajo.}
\onedefinition{2}
\spanishtranslation{esposo, esposa}
\cholexample{Jiñäch kpi'äl.}
\exampletranslation{Ella es mi esposa.}
\onedefinition{3 pariente}
\cholexample{Mero lakpi'äl lakbä.}
\exampletranslation{De veras somos parientes.}
\secondaryentry{pi'äl tyi chumtyäl}
\secondpartofspeech{s}
\secondtranslation{vecino}

\entry{pi'leñ}
\partofspeech{vt}
\onedefinition{1}
\spanishtranslation{acompañar}
\cholexample{Woli ipi'leñ majlel iyijñam tyi tyejklum.}
\exampletranslation{Va a acompañar a su esposa al pueblo.}
\onedefinition{2}
\spanishtranslation{tener relación sexual}

\entry{pok}
\partofspeech{vt}
\spanishtranslation{lavar}
\clarification{manos, cara, trastos}
\cholexample{La'i pok ik'äb jiñi ch'ityoñ.}
\exampletranslation{Que el joven se lave las manos.}

\entry{pokok}
\relevantdialect{Sab.}
\partofspeech{s}
\spanishtranslation{sapo}
\variation{popok}
\alsosee{xpokok}

\entry{poko'}
\partofspeech{s}
\spanishtranslation{hoja de quequexte, mafafa}

\entry{pok'}
\partofspeech{s}
\spanishtranslation{jicalpeste}

\entry{poch}
\onedefinition{1}
\partofspeech{vt}
\spanishtranslation{pelar}
\cholexample{Wolik poch kch'ajañ.}
\exampletranslation{Estoy pelando mi mecapal (cáscara).}
\onedefinition{2}
\partofspeech{adv}
\nontranslationdef{La manera en que queda tirada una tela; p. ej.:}
\cholexample{Tsa' poch yajli jiñi pisil tyi lum.}
\exampletranslation{Se cayó el trapo al suelo.}

\entry{-pochañ}
\nontranslationdef{Sufijo que se presenta con raíces adjetivas que indican color y se refiere a una reflexión como de una camisa o un machete.}

\entry{pochityok'}
\partofspeech{s esp}
\spanishtranslation{pochitoque, casquito}
\clarification{tortuga pequeña}

\entry{pochob}
\defsuperscript{1}
\partofspeech{s}
\onedefinition{1}
\spanishtranslation{carrizo}
\onedefinition{2}
\spanishtranslation{pito}

\entry{pochob}
\defsuperscript{2}
\partofspeech{s}
\nontranslationdef{Personas que participan en la fiesta de carnaval disfrazadas con cueros de tigres y enaguas negras.}

\entry{pochol bä k'uts}
\spanishtranslation{hoja de tabaco}

\entry{pochtyäl}
\partofspeech{adj}
\spanishtranslation{así de ancho}
\clarification{machete, hacha, cinturón}
\cholexample{Che'tyo pochtyäl ikajchiñäk'.}
\exampletranslation{Así de ancho es su cinturón.}

\entry{poch'}
\partofspeech{s}
\spanishtranslation{chinche}
\clarification{insecto}
\secondaryentry{*poch'il}
\secondpartofspeech{s}
\secondtranslation{chinches}

\entry{poch'iñ}
\partofspeech{vt}
\spanishtranslation{abofetear}
\cholexample{Tsa' ipoch'i iyijts'iñ.}
\exampletranslation{Abofeteó a su hermanito.}

\entry{poj}
\defsuperscript{1}
\partofspeech{adv}
\spanishtranslation{por favor}
\clarification{por tiempo limitado}
\cholexample{¿mu'ba apoj ak'eñoñ tyi majañ ajuloñib?}
\exampletranslation{¿Me prestas un rato tu escopeta, por favor?}

\entry{poj}
\defsuperscript{2}
\partofspeech{vt}
\onedefinition{1}
\spanishtranslation{desarmar, desocupar}
\cholexample{Woli ipoj ikarro.}
\exampletranslation{Está desarmando su carro.}
\onedefinition{2}
\relevantdialect{Sab.}
\spanishtranslation{quitar}
\cholexample{Tsi pojbeyoñ kmachity.}
\exampletranslation{Me quitó mi machete.}

\entry{pojkäm}
\partofspeech{s}
\spanishtranslation{frijol botil}
\clarification{frijolillo}

\entry{-pojch}
\nontranslationdef{Sufijo numeral para contar ropa; p. ej.:}
\cholexample{Ma'añik jaypojch ibujk.}
\exampletranslation{No tiene muchas camisas.}

\entry{pojleñ}
\relevantdialect{Sab., Tila}
\partofspeech{vt}
\spanishtranslation{conocer}
\cholexample{Mi ipojleñ majlel jiñi bij.}
\exampletranslation{Él va conociendo el camino.}
\alsosee{käñ}

\entry{pojob}
\relevantdialect{Sab.}
\partofspeech{adj}
\spanishtranslation{en balde}
\cholexample{Pojobix ktyroñel.}
\exampletranslation{Mi trabajo es en balde.}

\entry{pojokña}
\partofspeech{adj}
\spanishtranslation{flojo}
\cholexample{Pojokña tsa' käle ityorñilojlel karro.}
\exampletranslation{Los tornillos del carro quedaron medio flojos.}

\entry{Pojol}
\relevantdialect{Tila}
\partofspeech{s}
\spanishtranslation{Pojol}
\clarification{colonia}

\entry{pojol}
\defsuperscript{3}
\relevantdialect{Tila}
\partofspeech{adj}
\spanishtranslation{de balde}

\entry{pojol}
\defsuperscript{1}
\partofspeech{adj}
\spanishtranslation{desocupado}
\clarification{casa}

\entry{pojol}
\defsuperscript{2}
\relevantdialect{Sab., Tila}
\partofspeech{adj}
\spanishtranslation{conocido}
\clarification{camino}

\entry{pojp}
\partofspeech{s}
\spanishtranslation{petate}

\entry{pojpobil}
\partofspeech{adj}
\spanishtranslation{asado}
\cholexample{Pojpobil jiñi we'eläl tsa'bä iyäk'eyoñ.}
\exampletranslation{La carne que me dio está asada.}

\entry{pojpoñ}
\partofspeech{vt}
\spanishtranslation{asar}
\cholexample{Woli ipojpoñ we'eläl.}
\exampletranslation{Está asando carne.}

\entry{pojts'emal}
\partofspeech{s}
\spanishtranslation{vano}
\clarification{cuando una fruta no produce y está seca}
\cholexample{Añ kabäl ipojts'emal jiñi kajpe'.}
\exampletranslation{El café tiene mucho vano.}

\entry{pom}
\partofspeech{s}
\spanishtranslation{incienso}

\entry{*pomlel}
\partofspeech{s}
\spanishtranslation{cantidad de agua}
\cholexample{Ts'itya' jach ipomlel ja' añ tyi pok'.}
\exampletranslation{Es muy poca el agua que hay en el jicalpeste.}

\entry{pomol}
\partofspeech{adj}
\spanishtranslation{poco líquido}
\clarification{en su envase}
\cholexample{Pomol jiñi ja' ya' tyi pok'.}
\exampletranslation{En el jicalpeste hay un poco de agua.}

\entry{pomol uw}
\spanishtranslation{luna llena}
\cholexample{Pomol jiñi uw.}
\exampletranslation{La luna está llena.}

\entry{pomoy}
\partofspeech{s}
\spanishtranslation{capulín cimarrón}
\clarification{árbol}

\entry{pomtyäl}
\partofspeech{adv}
\spanishtranslation{poquito}
\clarification{líquido}
\cholexample{Che' pomtyäl ja' tsak ch'ämä tyilel.}
\exampletranslation{Traje un poquito de agua.}

\entry{poñch'ox}
\relevantdialect{Sab., Tila}
\partofspeech{s}
\spanishtranslation{piñanona}
\clarification{bejuco}
\alsosee{juk'utyuñ}

\entry{popok}
\relevantdialect{Sab.}
\conjugationtense{variante}
\conjugationverb{pokok}
\spanishtranslation{sapo}

\entry{pokik'äbäl}
\partofspeech{s}
\spanishtranslation{vasija}

\entry{poki'ejäl}
\partofspeech{s}
\spanishtranslation{cepillo de dientes}

\entry{porajiñ}
\partofspeech{vt}
\spanishtranslation{podar}
\cholexample{Mu'tyo kajel kporajiñ jkajpe'lel.}
\exampletranslation{Voy a podar mi cafetal.}

\entry{porokña}
\partofspeech{adv}
\nontranslationdef{Se relaciona con la forma de respirar cuando hay una obstrucción.}

\entry{potsokña}
\partofspeech{adj}
\spanishtranslation{muy espumoso}
\cholexample{Potsokña jachix ilojk jiñi ñoja'.}
\exampletranslation{El río está muy espumoso.}

\entry{potsol}
\partofspeech{adj}
\spanishtranslation{espumoso}
\cholexample{Ya' potsol ilojk ja' tyi' jajp xajlel.}
\exampletranslation{El agua está espumosa en la rendija de la piedra.}

\entry{*potsots}
\partofspeech{s}
\spanishtranslation{pulmones}

\entry{potspotsña}
\partofspeech{adj}
\spanishtranslation{espumándose}
\cholexample{Potspotsña ilojk xapom.}
\exampletranslation{El jabón está espumándose.}

\entry{pots'}
\partofspeech{adj}
\spanishtranslation{ciego}
\cholexample{Pots' jiñi wiñik.}
\exampletranslation{Ese hombre es ciego.}

\entry{pots'lom}
\relevantdialect{Sab.}
\partofspeech{s}
\spanishtranslation{inflamación}
\alsosee{paslam}

\entry{pox}
\partofspeech{s}
\spanishtranslation{chincuya, anona morada}
\clarification{árbol}

\entry{poy}
\partofspeech{s}
\spanishtranslation{balsa}

\entry{poytye'}
\partofspeech{s}
\spanishtranslation{corcho, jonote}
\clarification{árbol}

\entry{po'om}
\partofspeech{s}
\spanishtranslation{ocote agrio}
\clarification{árbol}

\entry{preñtya}
\partofspeech{s esp}
\spanishtranslation{prenda}
\cholexample{Tsa' jkäyä tyi preñtya jiñi kuloñib cha'añ tsa' jk'ajtyi tyak'iñ.}
\exampletranslation{Dejé mi escopeta como prenda por el dinero que pedí.}

\entry{prowaliñ}
\relevantdialect{Sab.}
\partofspeech{vt}
\spanishtranslation{probar}
\cholexample{Yom mi aprowaliñ jiñi pats'.}
\exampletranslation{Hay que probrarse los huaraches.}

\entry{puk}
\partofspeech{vt}
\spanishtranslation{repartir}
\cholexample{Woli ipuk ñumel juñ jiñi komityé.}
\exampletranslation{El comité está repartiendo papeles.}

\entry{puk ty'añ}
\spanishtranslation{anunciar}
\cholexample{Woli ipuk majlel ty'añ cha'añ tyal diputyado.}
\exampletranslation{Están anunciando que viene el diputado.}

\entry{puk'}
\partofspeech{vt}
\spanishtranslation{batir}
\cholexample{Yom mi apuk'beñoñ ksa'.}
\exampletranslation{Hay que batir mi pozol.}

\entry{puk'tya'}
\partofspeech{s}
\spanishtranslation{huevos que no brotan}
\cholexample{Kabäl tsa' lok'i ipuk'tya'lel kmuty.}
\exampletranslation{Quedaron muchos huevos de mi gallina sin brotar.}

\entry{puk'tyäl}
\partofspeech{adj}
\spanishtranslation{así de gordo}
\cholexample{Chetyo puk'tyäl jiñi ch'ityoñ.}
\exampletranslation{Así de grueso es ese joven.}

\entry{puch'}
\partofspeech{vt}
\spanishtranslation{apachurrar}
\cholexample{Tsa' ujtyi ipuch' tyeñ iyok yik'oty xajlel.}
\exampletranslation{Acaba de apachurrar su pie con una piedra.}

\entry{pujbañ}
\partofspeech{vt}
\spanishtranslation{rociar}
\clarification{líquido con la mano}
\cholexample{Jiñi x'ixik mi ipujbañ tyi ja' jiñi pisil che' mi ijuk'.}
\exampletranslation{La mujer rocía la ropa con agua cuando la va a planchar.}

\entry{pujch'el}
\partofspeech{vi}
\onedefinition{1}
\spanishtranslation{gastarse la punta o filo}
\cholexample{Wolix ipujch'el machity tyi xajlel.}
\exampletranslation{Se está gastando el filo del machete en la piedra.}
\onedefinition{2}
\spanishtranslation{echarse a perder}
\cholexample{Mi kaj ipujch'el jiñi lum cha'añ woli isek'ob tye'.}
\exampletranslation{Se va a echar a perder ese terreno porque están tumbando los árboles.}

\entry{*pujil}
\partofspeech{s}
\spanishtranslation{pus}
\cholexample{Kabäl mi ityempañ ipujil laklojwel che' ma'añik mi laksujkuñ.}
\exampletranslation{Cuando no limpiamos una herida, se junta mucha pus.}
\variation{*pujwil}

\entry{pujiña}
\partofspeech{adj}
\spanishtranslation{resoplando}
\cholexample{Pujiña iñi' jiñi kawayu' cha'añ ajñel tsa' tyili.}
\exampletranslation{El caballo está resoplando porque vino corriendo.}

\entry{pujmäyel}
\partofspeech{vi}
\spanishtranslation{formar pus}
\cholexample{Woli ipujmäyel ilojwel.}
\exampletranslation{Está formándose pus en su herida.}

\entry{pujkel}
\partofspeech{vi}
\onedefinition{1}
\spanishtranslation{distribuirse}
\cholexample{Mi kaj ipujkel pak' cha'añ ch'ujm.}
\exampletranslation{Se va a distribuir la semilla de calabaza.}
\onedefinition{2}
\spanishtranslation{divulgarse}
\cholexample{Wolix tyi pujkel tyi alol jiñi tsa'bä imele jiñi wiñik.}
\exampletranslation{Ya se está divulgando lo que hizo ese hombre.}

\entry{pujkiktyik}
\partofspeech{adj}
\spanishtranslation{regado}
\clarification{piedra, semilla}
\cholexample{Pujkiktyik añ xajlel ya' tyi' pam kotyoty.}
\exampletranslation{Las piedras están regadas en el patio de mi casa.}

\entry{*pujwil}
\conjugationtense{variante}
\conjugationverb{*pujil}
\spanishtranslation{pus}

\entry{pujyu'}
\partofspeech{s}
\onedefinition{1}
\spanishtranslation{tecolote}
\clarification{ave; medio grande, canta a medianoche}
\onedefinition{2}
\spanishtranslation{caballero, chotacabra}
\clarification{ave nocturna}

\entry{pul}
\partofspeech{vt}
\spanishtranslation{quemar}
\cholexample{Mach mi lakpul jiñi cholel kome mi iyäsiyel jiñi lum.}
\exampletranslation{No sirve quemar la rozadura porque se descompone el terreno.}

\entry{pulel}
\partofspeech{vi}
\onedefinition{1}
\spanishtranslation{quemarse}
\onedefinition{2}
\relevantdialect{Sab.}
\spanishtranslation{haber eclipse}
\cholexample{Chäñkol tyi pulel lakch'ujtyaty.}
\exampletranslation{Hay un eclipse de sol (lit.: nuestro Padre Santo se está quemando).}

\entry{pulem}
\partofspeech{adj}
\spanishtranslation{quemado}

\entry{pulibäl}
\partofspeech{s}
\spanishtranslation{sarampión}

\entry{pupuy}
\partofspeech{s}
\spanishtranslation{concha vacía de caracol}

\entry{pus}
\partofspeech{s}
\spanishtranslation{baño de vapor}

\entry{pusik'al}
\partofspeech{s}
\onedefinition{1}
\spanishtranslation{corazón}
\cholexample{Woli ityi k'ux ipusik'al.}
\exampletranslation{Le está doliendo el corazón.}
\onedefinition{2}
\spanishtranslation{región central del cuerpo}
\cholexample{Añ ik'amäjel ipusik'al.}
\exampletranslation{Tiene un mal en su abdomen.}
\secondaryentry{añix ipusik'al}
\secondtranslation{ya comprende}
\secondaryentry{k'uñ ipusik'al}
\secondtranslation{es tratable}
\secondaryentry{k'ux kpusik'al}
\secondtranslation{me duele el corazón}
\secondaryentry{k'ux tsa' k'otyi tyi pusik'kal, k'ux tsa' k'otyi pusik'al}
\secondtranslation{se ofendió}
\secondaryentry{cha'chajp ipusik'al}
\secondtranslation{tiene dos caras; es hipócrita}
\secondaryentry{ch'ejl ipusik'al}
\secondtranslation{es valiente}
\secondaryentry{ch'ijiyem ipusik'al}
\secondtranslation{está triste}
\secondaryentry{jump'ejl ipusik'al}
\secondtranslation{un solo corazón; es fiel o recto}
\secondaryentry{leko ipusik'al}
\secondtranslation{piensa mal}
\secondaryentry{machtyojik ipusik'al}
\secondtranslation{sus intenciones no son buenas}
\secondaryentry{mi imel ipusik'al}
\secondtranslation{está preocupado}
\secondaryentry{tyi pejtyelel ipusik'al}
\secondtranslation{con todo su corazón}
\secondaryentry{tyijikña ipusik'al}
\secondtranslation{está feliz}
\secondaryentry{tyoj ipusik'al}
\secondtranslation{es recto de corazón}
\clarification{actitud}
\secondaryentry{tsäts ipusik'al}
\secondtranslation{es duro su carácter}
\secondaryentry{tsa' k'otyi tyi' pusik'al}
\secondtranslation{llegó a su corazón, comprendió}
\secondaryentry{uts ipusik'al}
\secondtranslation{es muy amable}
\secondaryentry{woli tyi ty'añ ipusik'al}
\secondtranslation{está pensando algo}

\entry{putyuñ}
\partofspeech{adv}
\onedefinition{1}
\spanishtranslation{únicamente}
\spanishtranslation{sólo}
\cholexample{Putyuñ k'ay awom.}
\exampletranslation{Únicamente tú quieres cantar.}
\onedefinition{2}
\spanishtranslation{mucho}
\cholexample{Woli ityi putyuñ ty'añ.}
\exampletranslation{Está hablando mucho.}

\entry{puts'el}
\partofspeech{vi}
\spanishtranslation{huir}
\cholexample{Woli tyi puts'el kawayu'.}
\exampletranslation{Ese caballo está huyendo.}

\entry{puts'ibäl}
\partofspeech{s}
\spanishtranslation{refugio}

\entry{puts'tyañ}
\partofspeech{vt}
\spanishtranslation{esconder}
\cholexample{Woli iputs'tyañ ibä jiñi alob.}
\exampletranslation{Ese chamaco se está escondiendo.}

\entry{puy}
\partofspeech{s}
\onedefinition{1}
\spanishtranslation{hilo}
\cholexample{Jiñi aläl tsa' isoko jiñi puy.}
\exampletranslation{El niño enredó su hilo.}
\onedefinition{2}
\spanishtranslation{caracol}
\cholexample{Kabäl mi ip'ojlel puy ya' tyi pa'.}
\exampletranslation{Los caracoles se multiplican en el arroyo.}

\entry{Puypa'}
\partofspeech{s}
\spanishtranslation{Arroyo con Caracoles}
\clarification{lugar}

\alphaletter{P'}

\entry{p'aj}
\partofspeech{vt}
\spanishtranslation{maldecir}
\cholexample{Mach weñik mi lakp'aj lakpi'älob.}
\exampletranslation{No es bueno maldecir a nuestros semejantes.}

\entry{-p'ajk}
\nontranslationdef{Sufijo numeral para contar cañutos de algo: p. ej.:}
\cholexample{jump'ajk}
\partofspeech{adj}
\exampletranslation{cañuto de algo,}
\cholexample{cha'p'ajk}
\partofspeech{adj}
\exampletranslation{dos cañutos de algo.}

\entry{*p'ajomal}
\partofspeech{s}
\spanishtranslation{lo que no sirve}
\clarification{frijol, maíz, animales}
\cholexample{Kabäl tsa' lok'i ip'ajomal kixim.}
\exampletranslation{Salió mucho maíz que no sirve.}

\entry{p'ajoñel}
\partofspeech{s}
\spanishtranslation{grosería}
\cholexample{Kabäl woli tyi p'ajoñel.}
\exampletranslation{Está diciendo muchas groserías.}

\entry{p'ajtyel}
\partofspeech{vi}
\spanishtranslation{caer}
\clarification{fruta, carne, dinero, tortillas}
\cholexample{Tsa' p'ajtyi tyi lum alaxax.}
\exampletranslation{La naranja se cayó al suelo.}

\entry{p'akel}
\partofspeech{s}
\spanishtranslation{coyuntura}
\secondaryentry{*p'akel laj k'äb}
\secondtranslation{coyuntura de la mano}

\entry{p'asuñ}
\partofspeech{vt}
\spanishtranslation{cortar}
\clarification{en trozos}
\cholexample{Mi kajel kp'asuñ jiñi tye'.}
\exampletranslation{Voy a cortar ese palo en trozos.}

\entry{p'äklaw}
\partofspeech{adv}
\spanishtranslation{así suena}
\clarification{sonido de la lluvia}
\cholexample{P'äklaw woli tyi tyejchel ja'al tyi pam otyoty.}
\exampletranslation{Así suena la lluvia goteando encima de la casa.}

\entry{p'äk' oka}
\partofspeech{vt}
\spanishtranslation{tropezar}
\cholexample{Jiñi ch'ityoñ tsa' ipäk oka tye'.}
\exampletranslation{Ese muchacho tropezó con un palo.}

\entry{p'äjkel k'iñ}
\relevantdialect{Sab.}
\partofspeech{vi}
\spanishtranslation{ponerse el sol}

\entry{p'äty}
\partofspeech{vt}
\spanishtranslation{amarrar carga}
\cholexample{Mu'tyo kajel ip'äty isi'.}
\exampletranslation{Todavía va a amarrar su leña.}

\entry{p'ätyäl}
\partofspeech{adj}
\spanishtranslation{fuerte}
\cholexample{P'ätyäl jiñi mulakome mi ikuch jo'k'al kilo.}
\exampletranslation{La mula es fuerte porque carga cien kilos.}

\entry{p'äty'añ}
\partofspeech{vi}
\spanishtranslation{hacerse fuerte}
\cholexample{Wolix tyi p'äty'añ jiñi alob.}
\exampletranslation{Ese niño ya se está poniendo fuerte.}

\entry{p'ätsañ}
\partofspeech{vt}
\spanishtranslation{dejar caer}
\clarification{alimento, dinero}
\cholexample{Tsa' ip'ätsa iwaj tyi lum.}
\exampletranslation{Dejó caer su tortilla al suelo.}

\entry{-p'ejl}
\nontranslationdef{Sufijo numeral para contar cosas en general; p. ej.:}
\cholexample{Añix jump'ejl jab jiñi kotyoty.}
\exampletranslation{Mi casa ya tiene un año.}

\entry{p'ejty}
\partofspeech{s}
\spanishtranslation{olla}

\entry{*p'ejtyal}
\partofspeech{s}
\spanishtranslation{olla}
\cholexample{Tsa' ijetye ip'ejtyal bu'ul tyi k'ajk.}
\exampletranslation{Puso la olla de frijol en el fuego.}

\entry{p'ejw}
\partofspeech{vt}
\spanishtranslation{regalar}
\clarification{en abundancia}
\cholexample{Jiñi weñ chumulbä yom ip'ejw ichubä'añ.}
\exampletranslation{El rico debe dar algo de su abundancia.}

\entry{*p'ejwlel}
\partofspeech{s}
\spanishtranslation{abundancia}
\cholexample{Dios mi iyäk'eñoñlalakp'ejwlel tyi lakchubä'añ.}
\exampletranslation{Dios nos da abundancia en nuestros bienes.}

\entry{p'el}
\partofspeech{vt}
\spanishtranslation{aserrar}
\cholexample{Mi kaj kp'el itye'el kotyoty.}
\exampletranslation{Voy a aserrar la madera de mi casa.}

\entry{*p'eñel}
\relevantdialect{Sab.}
\partofspeech{s}
\onedefinition{1}
\spanishtranslation{hijo}
\clarification{del padre}
\cholexample{¿amba ip'eñel jiñi wiñik?}
\exampletranslation{¿Tiene un hijo ese hombre?}
\onedefinition{2}
\spanishtranslation{esperma}
\clarification{del hombre}

\entry{p'ew}
\partofspeech{vt}
\spanishtranslation{aumentar}
\cholexample{Dios woli ip'ewbeñ ichubä'añ jiñi wiñik.}
\exampletranslation{Dios le está aumentando los bienes a ese hombre.}

\entry{p'e'}
\partofspeech{adj}
\spanishtranslation{roto y abierto}
\cholexample{P'e' ts'ijlem iwex.}
\exampletranslation{Está roto y abierto su pantalón.}

\entry{p'ik}
\partofspeech{vt}
\spanishtranslation{sacar, escarbar}
\clarification{con aguja, palillo}
\cholexample{Wolik p'ik lok'el ch'ix tyi kok.}
\exampletranslation{Estoy sacándome una espina del pie.}

\entry{p'iko'ch'ix}
\partofspeech{s}
\spanishtranslation{aguja para quitar espinas}

\entry{p'ich}
\partofspeech{vt}
\spanishtranslation{hacer tacos}
\cholexample{Woli ip'ich bu'ul.}
\exampletranslation{Está haciendo tacos de frijol.}

\entry{-p'ijch}
\nontranslationdef{Sufijo numeral para contar tacos; p. ej.:}
\cholexample{Ak'bi tsa' k'uxu jump'ejch bu'ul.}
\exampletranslation{Ayer comí un taquito de frijol.}

\entry{p'ijlistyik}
\partofspeech{adj}
\spanishtranslation{pinto}
\cholexample{P'ijlistyik jiñi xña'muty.}
\exampletranslation{La gallina es pinta.}

\entry{p'ijtyel}
\partofspeech{vi}
\spanishtranslation{quebrarse}
\cholexample{Mi ip'ijtyel ixim che' mi ichok jiñi ik'.}
\exampletranslation{El tallo se quiebra cuando azota el viento.}

\entry{p'ip'}
\partofspeech{adj}
\onedefinition{1}
\spanishtranslation{arisco}
\cholexample{P'ip' jiñi ma'tye'muty.}
\exampletranslation{Los pájaros son ariscos.}
\onedefinition{2}
\spanishtranslation{inteligente}
\cholexample{P'ip' ijol jiñi alob.}
\exampletranslation{Ese chamaco es inteligente.}

\entry{p'ip'añ}
\partofspeech{vi}
\spanishtranslation{ser inteligente}
\cholexample{Tsa' ochiyoñ tyi eskuelacha'añ mik p'ip'añ.}
\exampletranslation{Ingresé a la escuela para ser inteligente.}

\entry{p'ip'iktyäl}
\partofspeech{adj}
\spanishtranslation{así de pequeño}
\cholexample{Che' jach ya p'ip'iktyäl ibäk' jiñi ich.}
\exampletranslation{Así de pequeña es la pepita del chile.}

\entry{p'is}
\partofspeech{vt}
\spanishtranslation{medir}
\cholexample{Yom mi ap'is akajpe' che' mi amejlel achoñ.}
\exampletranslation{Hay que medir tu café antes de ir a venderlo.}

\entry{-p'is}
\nontranslationdef{Sufijo numeral para contar tazas; p. ej.:}
\cholexample{Mi lakjap jump'is sa'.}
\exampletranslation{Tomamos una taza de pozole.}

\entry{p'isbeñtyel}
\partofspeech{vi}
\spanishtranslation{medir}
\cholexample{Maxtyo añik mi kaj ip'isbeñtyel ilum.}
\exampletranslation{Todavía no se le va a medir su terreno.}

\entry{p'isbil}
\partofspeech{adj}
\spanishtranslation{medido}
\cholexample{P'isbil jiñi ats'am che' mi lakmäñ.}
\exampletranslation{La sal ya está medida cuando nosotros la compramos.}

\entry{*p'isol}
\partofspeech{s}
\spanishtranslation{medida, talla}
\cholexample{Mach ip'isolik ibujk tsa'bä imäñä.}
\exampletranslation{No es de su talla la camisa que se compró.}

\entry{*p'isoñib}
\partofspeech{s}
\spanishtranslation{medidor, metro, regla, balanza}

\entry{p'ity}
\partofspeech{adv}
\spanishtranslation{así de chico}
\cholexample{Tsa' p'ity k'äski iyok.}
\exampletranslation{Así se le dislocaron los huesos chicos del pie.}

\entry{p'ixel}
\partofspeech{vi}
\spanishtranslation{despertarse}
\cholexample{Se'el mik p'ixel.}
\exampletranslation{Me despierto temprano.}

\entry{p'ixil}
\relevantdialect{Sab.}
\partofspeech{adj}
\spanishtranslation{despierto}
\alsosee{kañal lakwuty}

\entry{p'ok}
\partofspeech{s}
\spanishtranslation{lagarto}

\entry{p'oklaw}
\partofspeech{adv}
\nontranslationdef{Se relaciona con el sonido de gotas de agua al caer; p. ej.:}
\cholexample{P'oklaw woli tyi yajlel ja'.}
\exampletranslation{Están cayendo unas gotas de agua (dando sonido).}

\entry{p'ojlel}
\partofspeech{vi}
\spanishtranslation{reproducirse}
\cholexample{Kabäl woli tyi p'ojlel imuty.}
\exampletranslation{Sus pollos se están reproduciendo mucho.}

\entry{p'ojlesañ}
\partofspeech{vt}
\spanishtranslation{aumentar}
\cholexample{Ibajñel jach woli ip'ojlesañ iye'tyel.}
\exampletranslation{Él solo está aumentando su trabajo.}

\entry{p'ojp'ostyäl}
\partofspeech{adj}
\spanishtranslation{grueso}
\clarification{frijol}
\cholexample{Che'ix p'ojp'ostyäl jiñi bu'ul kome ñejep'ix.}
\exampletranslation{Los frijoles ya están gruesos porque ya están sazonados.}

\entry{p'ojkiñ}
\partofspeech{vt}
\spanishtranslation{tropezar}
\cholexample{Tsa' ujtyi kp'ojkiñ kok tyi xajlel.}
\exampletranslation{Mi pie acaba de tropezar con un piedra.}

\entry{p'ol}
\partofspeech{vt}
\onedefinition{1}
\spanishtranslation{engendrar}
\onedefinition{2}
\spanishtranslation{producir}
\clarification{animales}
\cholexample{Mi kaj kp'ol muty.}
\exampletranslation{Voy a dedicarme a la producción de pollos.}

\entry{*p'olbal}
\partofspeech{s}
\onedefinition{1}
\spanishtranslation{hijos}
\cholexample{Kabälix ip'olbal jiñi wiñik.}
\exampletranslation{Ese hombre ya tiene muchos hijos.}
\onedefinition{2}
\spanishtranslation{descendientes}
\cholexample{Ip'oljbaloñlajiñi mayajob.}
\exampletranslation{Somos descendientes de los mayas.}

\entry{p'olmal}
\relevantdialect{Tila}
\partofspeech{s}
\spanishtranslation{tienda}
\alsosee{choñoñibäl}

\entry{*p'olmäjel}
\partofspeech{s}
\spanishtranslation{mercancía}
\cholexample{Ñoj kabäl ap'olmäjel.}
\exampletranslation{Tienes mucha mercancía.}
\alsosee{p'olmulel}

\entry{p'osiñ}
\partofspeech{vt}
\spanishtranslation{tropezar}

\entry{p'ots}
\partofspeech{adj}
\spanishtranslation{corto}
\cholexample{P'ots ityajbal imorral.}
\exampletranslation{El mecapal de su morral está corto.}

\entry{p'o'}
\defsuperscript{1}
\partofspeech{vt}
\onedefinition{1}
\spanishtranslation{sacar tripa}
\cholexample{Woli ip'o'ob jiñi wakax.}
\exampletranslation{Están sacando la tripa de la vaca.}
\onedefinition{2}
\spanishtranslation{operar}
\cholexample{Jiñi doktyor wolityo ip'o' jiñi wiñik.}
\exampletranslation{El doctor todavía está operando a ese hombre.}

\entry{p'o'}
\defsuperscript{2}
\relevantdialect{Tila}
\partofspeech{s}
\spanishtranslation{vestido}

\entry{p'o'tyo'}
\partofspeech{s}
\spanishtranslation{platanillo}
\clarification{heliconia ‘sp.’; planta}

\entry{p'uchul}
\partofspeech{adj}
\onedefinition{1}
\spanishtranslation{alomado}
\cholexample{P'uchul jiñi lum.}
\exampletranslation{El terreno está alomado.}
\onedefinition{2}
\spanishtranslation{amontonado}
\cholexample{P'uchul jiñi ixim ya' tyi' yotylel.}
\exampletranslation{El maíz está amontonado en la troje.}

\entry{-p'ujl}
\nontranslationdef{Sufijo numeral para contar montones; p. ej.:}
\cholexample{Wä'añ jump'ujl jiñi tyañ.}
\exampletranslation{Aquí hay un montón de cal.}

\entry{p'ujp'ubil}
\partofspeech{adj}
\spanishtranslation{regado}
\cholexample{P'ujp'ubil ibäk' ich ya' tyi mal cholel.}
\exampletranslation{Hay semilla de chile regada en medio de la milpa.}

\entry{p'ujp'uñ}
\partofspeech{vt}
\spanishtranslation{regar}
\clarification{semilla}
\cholexample{Mi lakp'ujp'uñ ich.}
\exampletranslation{Regamos semilla de chile.}

\entry{p'ul}
\partofspeech{adj}
\spanishtranslation{amontonado}
\cholexample{Ya' jach tsa' ip'ul käyä jiñi xajlel.}
\exampletranslation{Ahí dejó nada más amontonadas las piedras.}

\entry{-p'ulañ}
\nontranslationdef{Sufijo que se presenta con raíces adjetivas que indican color, y se refiere a animales o piedras amontonadas.}

\entry{p'ulbeñ}
\partofspeech{vt}
\spanishtranslation{apurar}
\cholexample{Yom mi ap'ulbeñ xämbal cha'añ sebtyo mi ak'otyel tyi awotyoty.}
\exampletranslation{Debes apurarte para llegar temprano a tu casa.}

\entry{p'ulchokoñ}
\partofspeech{vt}
\spanishtranslation{amontonar}
\clarification{maíz, frijol, piedra}
\cholexample{Mi kaj kp'ulchokoñ jiñi ixim ya' tyi mal otyoty.}
\exampletranslation{Voy a amontonar el maíz en medio de la casa.}

\entry{p'ultyäl}
\partofspeech{adv}
\spanishtranslation{así}
\clarification{amontonado}
\cholexample{Che'tyo p'ultyäl xajlel añ tyi pamtyo iyotyoty.}
\exampletranslation{Las piedras están amontonadas así en el patio de su casa.}

\entry{p'ulukña}
\onedefinition{1}
\partofspeech{adv}
\nontranslationdef{Se relaciona con el movimiento de una muchedumbre (personas o animales); p. ej.:}
\cholexample{P'ulukña woli ik'ux jam wakax tyi potyrero.}
\exampletranslation{Mucho ganado va comiendo en el potrero.}
\onedefinition{2}
\partofspeech{adj}
\spanishtranslation{amontonado}
\cholexample{P'ulukña wiñikob ya'ba woli iñujpuñel.}
\exampletranslation{La gente está amontonada donde se están casando.}

\entry{p'ulul}
\partofspeech{adj}
\spanishtranslation{amontonado}
\cholexample{P'ulul tsa' käle tyi lum jiñi xajlel.}
\exampletranslation{La piedra quedó amontonada en el suelo.}

\entry{p'ump'uñ}
\partofspeech{adj}
\onedefinition{1}
\spanishtranslation{pobre}
\cholexample{Weñ p'ump'uñ jiñi wiñik.}
\exampletranslation{Ese hombre es muy pobre.}
\onedefinition{2}
\spanishtranslation{lastimoso}
\cholexample{P'ump'uñjax jiñi x'ixik.}
\exampletranslation{Esa mujer es lastimosa.}

\entry{p'ump'uña}
\partofspeech{adv}
\spanishtranslation{palpitando}
\cholexample{Tsäts p'ump'uña kpusik'al che' mik cha'leñ ajñel.}
\exampletranslation{Mi corazón va palpitando mucho cuando corro.}

\entry{p'uñtyañ}
\partofspeech{vt}
\spanishtranslation{tener lástima de}
\cholexample{Mi ip'uñtyañ iyalobil jiñi wiñik.}
\exampletranslation{Ese hombre tiene lástima de su hijo.}

\entry{p'uñtyäñtyel}
\partofspeech{vi}
\spanishtranslation{compadecerse}
\cholexample{Ma'añik mi ip'uñtyäñtyel jiñi alob.}
\exampletranslation{No se compadecen de ese chamaco.}

\alphaletter{Q}

\entry{keb}
\partofspeech{s}
\spanishtranslation{eructo}
\cholexample{Tsa' icha'le keb.}
\exampletranslation{Eructó (lit.: dio un eructo).}

\entry{kech'ekña}
\partofspeech{adv}
\spanishtranslation{rechinando}
\cholexample{Kech'ekña woli imel ibäkel iyej.}
\exampletranslation{Está rechinando los dientes.}

\entry{kej}
\partofspeech{adv}
\nontranslationdef{Se relaciona con la forma de caer, ayudándose con una mano; p. ej.:}
\cholexample{Tsa' kej yajli wiñik.}
\exampletranslation{El hombre se cayó, salvándose con una mano.}

\entry{kejkañ}
\partofspeech{vt}
\spanishtranslation{cortar}
\cholexample{Woli ikejkañ jubel ik'äb tye'.}
\exampletranslation{Está cortando los gajos del árbol.}

\entry{*kejlab}
\relevantdialect{Sab.}
\partofspeech{s}
\spanishtranslation{hombro}

\entry{kejlo'}
\partofspeech{s}
\spanishtranslation{hombro}

\entry{-kejp}
\nontranslationdef{Sufijo numeral para contar pencas; p. ej.:}
\cholexample{Ya' tyi kotyoty añ jo'kejp ja'as.}
\exampletranslation{En mi casa tengo cinco pencas de plátano.}

\entry{*kejp}
\partofspeech{s}
\spanishtranslation{escalón}
\cholexample{Ya' tyi ruiña añ kabäl ikejp tyak jiñi ts'ajk.}
\exampletranslation{En las ruinas hay muchos escalones.}

\entry{kejpuktyik}
\partofspeech{adj}
\spanishtranslation{gradado}
\cholexample{Kejpuktyik jiñi witsba' tsa' letsiyoñ majlel.}
\exampletranslation{El cerro que subí está gradado.}

\entry{kelekña}
\partofspeech{adv}
\spanishtranslation{por filas}
\cholexample{Kelekña woli tyi ñumel wiñikob.}
\exampletranslation{La gente está pasando por filas.}

\entry{kelkelña}
\partofspeech{adj}
\spanishtranslation{chorreando}
\clarification{sangre, agua}
\cholexample{Kelkelña woli tyi lok'el jiñi ch'ich'ba' tsa' tsejpi.}
\exampletranslation{La sangre está chorreando de donde se hirió.}

\entry{kep}
\partofspeech{adv}
\spanishtranslation{pendiente}
\cholexample{Mu' jach ikep käy ichobal.}
\exampletranslation{Únicamente deja pendiente su rozadura.}

\entry{kepekña}
\partofspeech{adv}
\spanishtranslation{por tramos}
\cholexample{Kepekña woli tyi ujtyel majlel jiñi e'tyel.}
\exampletranslation{El trabajo se va terminando por tramos.}

\entry{kepel}
\partofspeech{adv}
\spanishtranslation{pendiente}
\cholexample{Kepel ichobal tsa' käle.}
\exampletranslation{Su rozadura quedó pendiente.}

\entry{kepkepña}
\partofspeech{adv}
\spanishtranslation{por pedazos}
\cholexample{Kepkepña woli ipulel majlel jiñi cholel.}
\exampletranslation{La rozadura se va quemando por pedazos.}

\entry{keptyäl}
\partofspeech{vi}
\spanishtranslation{quedar pendiente}
\cholexample{Ya' jach mi ikeptyäl ichobalba' jaxäl.}
\exampletranslation{La rozadura queda pendiente allí en donde termina su terreno}

\entry{kepuñ}
\partofspeech{vt}
\spanishtranslation{hacer muescas}
\cholexample{Tsa' ujtyi ikepuñ ityek'oñib.}
\exampletranslation{Acaba de hacer muescas en su escalera.}

\entry{kes}
\partofspeech{vt}
\spanishtranslation{moler}
\clarification{la primera pasada}
\cholexample{Wolityo ikes iwaj.}
\exampletranslation{Apenas está dando la primera molida a su maíz.}

\entry{kestyo}
\partofspeech{adj}
\spanishtranslation{nixtamal entero}
\cholexample{Kestyo jiñi waj.}
\exampletranslation{Todavía está entero el nixtamal.}

\entry{kets'}
\partofspeech{adj}
\spanishtranslation{atorado, trabado}
\cholexample{Kets'el tsa' käle jiñi lum tyi bej yok ja'.}
\exampletranslation{La tierra quedó atorada en la pendiente de la zanja.}

\entry{kets'el}
\partofspeech{adj}
\spanishtranslation{inconcluso}
\clarification{trabajo}

\entry{ke'ñañ}
\partofspeech{vt}
\spanishtranslation{entregar para cuidar}
\clarification{taburete}
\cholexample{Yom mi ake'ñañ jiñi we'ibäl jiñtyo mik cha' sujtyel.}
\exampletranslation{Debes entregar el taburete para que lo cuiden hasta que regreses.}

\entry{kich'}
\partofspeech{vt}
\spanishtranslation{lazar}
\cholexample{Wokol mi lakich' jiñi wakax.}
\exampletranslation{Es difícil lazar la vaca.}

\entry{-kich'añ}
\nontranslationdef{Sufijo que se presenta con raíces adjetivas que indican color, y se refiere a una superficie áspera.}

\entry{kilikña}
\partofspeech{adv}
\nontranslationdef{Se relaciona con el sonido de una cadena que se arrastra; p. ej.:}
\cholexample{Kilikña mi laktyujk'añ tyilel kadeña.}
\exampletranslation{La cadena viene sonando cuando la traemos jalando.}

\entry{kiliñ wuluj}
\partofspeech{adv}
\nontranslationdef{Se relaciona con el ruido de un objeto al caer; p. ej.:}
\cholexample{Kiliñ wuluj tsa' yajli jiñi kaxatye'.}
\exampletranslation{El cofre se cayó (haciendo sonido).}

\entry{kiñtyäl}
\partofspeech{adv}
\spanishtranslation{así}
\cholexample{Che'tyo kiñtyäl jiñi otyoty.}
\exampletranslation{La casa es así de larga.}

\entry{*kiñtyälel}
\partofspeech{s}
\spanishtranslation{lo largo}
\cholexample{Lujump'ejl metyro ikiñtyälel jiñi otyoty.}
\exampletranslation{La casa tiene diez metros de largo.}

\entry{kis}
\partofspeech{adv}
\nontranslationdef{Se relaciona con el olor de pescado; p. ej.:}
\cholexample{Kis ik'äb cha'añ iyujts'il chäy.}
\exampletranslation{La mano apesta a pescado.}

\entry{kisiñ}
\partofspeech{s}
\spanishtranslation{vergüenza}
\cholexample{Mu' tyi kisiñ jiñi x'ixik.}
\exampletranslation{Esa mujer tiene vergüenza.}

\entry{kisiñtyik}
\partofspeech{adv}
\spanishtranslation{vergonzosamente}

\entry{kisñil}
\relevantdialect{Tila}
\partofspeech{s}
\spanishtranslation{vergüenza}
\cholexample{Mi iyäk'oñ tyi kisñil.}
\exampletranslation{Me da vergüenza.}

\entry{kisñiñ}
\partofspeech{vt}
\spanishtranslation{avergonzar}
\cholexample{Mi jkisñiñ jiñi kaxlañ kome mikbäk'ñañ kpejkañ.}
\exampletranslation{Me avergüenzo con el latino porque tengo miedo de hablarle.}

\entry{kitsil}
\partofspeech{adv}
\nontranslationdef{Se relaciona con un objeto largo colgado; p. ej.:}
\cholexample{Kitsil tsa' käle jiñi lukum tyi ñi' tye'.}
\exampletranslation{La culebra quedó colgada en la punta del palo.}

\entry{kitstyäl}
\partofspeech{adv}
\spanishtranslation{así de largo}
\clarification{y flexible}
\cholexample{Che'tyo kitstyäl jiñi lukum tsa'bä ktsäñsa.}
\exampletranslation{Así de larga es la culebra que maté.}

\entry{kixtyañu}
\relevantdialect{Tila}
\partofspeech{s}
\spanishtranslation{persona, gente}

\entry{kixtyä}
\relevantdialect{Tila}
\partofspeech{adv}
\spanishtranslation{negativo}
\cholexample{Kixtyä komox.}
\exampletranslation{No quiero.}

\alphaletter{Q'}

\entry{k'ebäch}
\partofspeech{part}
\spanishtranslation{no hay de qué}
\clarification{respuesta al dar lo que se pide}

\entry{k'ebel}
\partofspeech{adv}
\spanishtranslation{apenas alcanza}
\cholexample{Mach yomik ijisañ ts'itya' ityak'iñ kome k'ebel cha'añ mi imäñ iyäts'am.}
\exampletranslation{No quiero gastar mucho dinero, porque apenas me alcanza para comprar sal.}

\entry{k'ebiñ}
\partofspeech{vt}
\spanishtranslation{no prestar por ser tacaño}
\cholexample{Ma'añik mi ik'ebiñ imula.}
\exampletranslation{No presta su mula porque es tacaño.}

\entry{k'ebtyo}
\partofspeech{part}
\spanishtranslation{de nada}
\clarification{respuesta después a alguien que dice “gracias”}
\dialectvariant{Sab.}
\dialectword{k'etyo sajl}

\entry{k'ech}
\partofspeech{vt}
\spanishtranslation{cargar}
\clarification{en el hombro}
\cholexample{Woli ik'ech majlel tye' tyi' kejlab.}
\exampletranslation{Está llevando un palo en su hombro.}

\entry{k'echel}
\partofspeech{adj}
\spanishtranslation{cargado}
\clarification{en el hombro}
\cholexample{K'echel tyi' kejlab jiñi kajpe'.}
\exampletranslation{Tiene cargado un bulto de café en el hombro.}

\entry{k'echulañ}
\partofspeech{vt}
\spanishtranslation{levantar}
\clarification{el pie}
\cholexample{Woli jach ik'echulañ letsel iyok.}
\exampletranslation{Sólo está levantando su pie hacia arriba.}

\entry{k'ej}
\relevantdialect{Tila}
\partofspeech{vt}
\spanishtranslation{hacer a un lado}

\entry{k'ejcheñ k'ejcheñ}
\partofspeech{adv}
\spanishtranslation{cojeando}
\cholexample{K'ejcheñ k'ejcheñ che' mi icha'leñ xämbal.}
\exampletranslation{Camina cojeando.}

\entry{*k'ejchil}
\partofspeech{s}
\spanishtranslation{camilla de un muerto}

\entry{-k'ejl}
\nontranslationdef{Sufijo numeral para contar objetos planos; p. ej.:}
\cholexample{Ya' tyi tyejklum tsak mäñä jo'k'ejl tyabla.}
\exampletranslation{Compré cinco tablas en el pueblo.}

\entry{k'ejlel}
\partofspeech{vi}
\spanishtranslation{ser visto}

\entry{k'el}
\partofspeech{vt}
\spanishtranslation{ver}

\entry{k'elekña}
\partofspeech{adj}
\spanishtranslation{despejado}
\cholexample{K'elekña pañimil wäle.}
\exampletranslation{Ahora está bien despejado.}

\entry{k'eloñib}
\partofspeech{s}
\onedefinition{1}
\spanishtranslation{lugar alto de donde se pueden ver los valles}
\onedefinition{2}
\spanishtranslation{ventana}
\cholexample{Ch'och'ok jax tsa' imele ik'eloñib.}
\exampletranslation{Hizo su ventana muy chica.}

\entry{k'elo'pañimil}
\partofspeech{s}
\spanishtranslation{lugar para ver un paisaje}

\entry{k'elo'k'iñ}
\partofspeech{s}
\spanishtranslation{reloj}

\entry{k'el pañimil}
\partofspeech{vt}
\spanishtranslation{nacer}
\dialectvariant{Sab., Tila}
\dialectword{ch'ok'añ}

\entry{k'em}
\partofspeech{s}
\spanishtranslation{jabalí}
\clarification{mamífero}

\entry{k'e tyo sajl}
\relevantdialect{Sab.}
\spanishtranslation{de nada (respuesta después que se dice “gracias”)}

\entry{k'ewex}
\partofspeech{s}
\spanishtranslation{anona colorada}
\clarification{árbol}
\alsosee{k'ätsats}

\entry{k'ex}
\partofspeech{vt}
\spanishtranslation{cambiar}
\cholexample{Tsa' ujtyi ik'ex ijuloñib.}
\exampletranslation{Acaba de cambiar su escopeta.}

\entry{*k'exol}
\partofspeech{s}
\onedefinition{1}
\spanishtranslation{trueque}
\cholexample{Mach yomix iyäk' ik'exol ktyak'iñ.}
\exampletranslation{Ya no quiero dar el trueque de mi dinero.}
\onedefinition{2}
\spanishtranslation{tocayo}
\cholexample{Jiñi alob ik'exol ityaty.}
\exampletranslation{Ese muchacho es tocayo de su padre.}

\entry{k'exolañ}
\partofspeech{vt}
\spanishtranslation{sustituir}
\clarification{cargo, nombre}
\cholexample{Che' mach ya'añ jiñi komisariado mi ik'exolañ jiñi supleñtye.}
\exampletranslation{Cuando el comisario no está lo sustituye el suplente.}

\entry{k'exoñel}
\partofspeech{s}
\onedefinition{1}
\spanishtranslation{acción de mudar}
\cholexample{Woli tyi k'exoñel itsutsel jiñi kawayu'.}
\exampletranslation{El caballo está mudando de pelo.}
\onedefinition{2}
\spanishtranslation{acción de cambiar}
\cholexample{Woli tyi k'exoñel jiñi año'bä tyi ye'tyel.}
\exampletranslation{Están cambiando a las autoridades.}

\entry{k'extyiyel}
\partofspeech{vi}
\spanishtranslation{cambiarse}
\cholexample{Tsi'ityajax yom cha'añ mi ik'extyiyel jabil.}
\exampletranslation{Ya falta poco para que el año cambie.}

\entry{k'ichil}
\partofspeech{adj}
\spanishtranslation{levantado}
\clarification{el pie por estar lastimado}
\cholexample{K'ichil iyok cha'añ k'ux.}
\exampletranslation{Tiene el pie levantado porque le duele.}

\entry{k'ichk'ichña}
\partofspeech{adv}
\spanishtranslation{cojeando}
\cholexample{K'ichk'ichña mu' tyi xämbal.}
\exampletranslation{Anda cojeando.}

\entry{k'ijchiñ}
\partofspeech{vt}
\spanishtranslation{cojear}

\entry{k'iñ}
\partofspeech{s}
\spanishtranslation{día}

\entry{k'iñi ja'lel}
\partofspeech{adj}
\spanishtranslation{temporada de lluvia}

\entry{k'iñijel}
\partofspeech{s}
\spanishtranslation{fiesta}

\entry{k'iñil}
\partofspeech{adv}
\spanishtranslation{de día}
\cholexample{Tyi k'iñil mi lakcha'leñ e'tyel.}
\exampletranslation{Trabajamos de día.}

\entry{*k'iñilel}
\partofspeech{s}
\spanishtranslation{día especial}
\cholexample{Wolix iläk'tyiyel ik'iñilel paxku.}
\exampletranslation{Se está acercando la fiesta de la pascua.}

\entry{k'iñlaw}
\partofspeech{adj}
\spanishtranslation{clara}
\clarification{la noche}
\cholexample{K'iñlaw ak'älel cha'añ añix uw.}
\exampletranslation{La noche está clara porque ya hay luna.}

\entry{k'iñ tyuñil}
\spanishtranslation{tiempo de seca}

\entry{k'iñ tyuñimuty}
\spanishtranslation{cuco bobo, cuclillo chiflador}
\clarification{ave}

\entry{k'ix}
\partofspeech{s}
\spanishtranslation{acción de calentarse}
\cholexample{Woli tyi k'ix tyi k'ajk.}
\exampletranslation{Está calentándose en el fuego.}

\entry{k'ixiñ}
\defsuperscript{1}
\relevantdialect{Sab.}
\partofspeech{adj}
\spanishtranslation{borracho}
\alsosee{yäk}

\entry{k'ixiñ}
\defsuperscript{2}
\partofspeech{adj}
\spanishtranslation{caliente}
\clarification{poco}
\cholexample{K'ixiñ jiñi ja' wolibä ik'äñ cha'añ mi ipits' chityam.}
\exampletranslation{El agua para chamuscar el cerdo está caliente.}

\entry{k'ixiñ pañimil}
\spanishtranslation{hacer calor}
\cholexample{Wolix ichäk'añ kajpe' cha'añ k'ixiñ pañimil.}
\exampletranslation{El café está madurándose porque hace calor.}

\entry{*k'ixñilel}
\partofspeech{s}
\spanishtranslation{tierra caliente}

\entry{k'ixñesañ}
\partofspeech{vt}
\spanishtranslation{calentar}
\clarification{poco}
\cholexample{Woli ik'ixñesañ ja'.}
\exampletranslation{Está calentando el agua.}

\entry{k'ix k'iñ}
\partofspeech{vi}
\spanishtranslation{asolearse}
\cholexample{Jiñi xñox woli tyi k'ix k'iñ.}
\exampletranslation{El hombre viejo está asoleándose.}

\entry{k'iyikña}
\partofspeech{adj}
\spanishtranslation{parejo}
\cholexample{K'iyikña jiñi lum tyi pam otyoty.}
\exampletranslation{La tierra del patio de la casa está pareja.}

\entry{k'iyil}
\partofspeech{adj}
\spanishtranslation{tendido}
\cholexample{K'iyil tyi k'iñ jiñi kajpe'.}
\exampletranslation{El café está tendido al sol.}

\entry{k'iyiñ}
\partofspeech{vt}
\spanishtranslation{tender al sol}
\cholexample{Mu'tyo kajel jk'iyiñ jkajpe'.}
\exampletranslation{Voy a tender mi café al sol.}

\entry{k'iyiñtyib}
\partofspeech{s}
\spanishtranslation{tapesco}

\entry{K'iyo'ich}
\partofspeech{s}
\spanishtranslation{Tapesco de Chile}
\clarification{colonia}

\entry{k'iytyäl}
\partofspeech{vi}
\spanishtranslation{tenderse}
\clarification{café, chile, frijol, ropa}

\alphaletter{R}

\entry{rebus}
\partofspeech{s esp}
\spanishtranslation{rebozo}
\dialectvariant{Sab.}
\dialectword{mujch'äl}

\entry{rejerol}
\partofspeech{s esp}
\spanishtranslation{regidor, suplente}

\entry{resal}
\partofspeech{s esp}
\spanishtranslation{rezo}
\cholexample{Maxtyo weñ kujilik resal.}
\exampletranslation{Todavía no sé rezar (lit.: no sé cómo hacer el rezo).}

\entry{*reyil}
\partofspeech{s}
\spanishtranslation{reino}

\entry{rima}
\partofspeech{s esp}
\spanishtranslation{lima}
\clarification{para afilar}

\entry{rok}
\partofspeech{s}
\spanishtranslation{garabato de hierro o palo}

\entry{rus}
\partofspeech{s esp}
\onedefinition{1}
\spanishtranslation{cruz}
\cholexample{Tsa' imele juñts'ijty irus.}
\exampletranslation{Hizo una cruz.}
\onedefinition{2}
\spanishtranslation{ídolo}
\cholexample{Mi ich'ujutyesañ jiñi rus.}
\exampletranslation{Adora al ídolo.}
\secondaryentry{rus ek'}
\secondtranslation{Orión}
\secondaryentry{rusibal bij}
\secondtranslation{crucero}

\entry{rusiñ}
\partofspeech{vt esp}
\spanishtranslation{persignarse}
\cholexample{Jiñi wiñik woli irusiñ iwuty.}
\exampletranslation{Ese hombre está persignándose.}

\alphaletter{S}

\entry{-s}
\conjugationtense{variante}
\conjugationverb{-es-}
\nontranslationdef{Sufijo que se presenta con raíces transitivas y atributivas para formar una raíz causativa: <tsu'san> |rdarle de mamar|r|i.}

\entry{sakaty}
\partofspeech{s}
\spanishtranslation{piojo}

\entry{sakol}
\partofspeech{s}
\onedefinition{1}
\spanishtranslation{grisón}
\clarification{mamífero}
\onedefinition{2}
\spanishtranslation{viejo de monte}
\clarification{mamífero}

\entry{sak'}
\partofspeech{adj}
\spanishtranslation{escocido}
\cholexample{Sak' ik'äb cha'añ tsa' ik'uxu sip.}
\exampletranslation{Tiene la mano escocida por el piquete de una garrapata.}

\entry{sajbiñ}
\partofspeech{s}
\spanishtranslation{comadreja}
\clarification{mamífero}

\entry{sajkañ}
\partofspeech{vt}
\spanishtranslation{buscar}
\cholexample{Woli isajkañ e'tyel jiñi wiñik.}
\exampletranslation{Ese hombre está buscando trabajo.}

\entry{sajk'}
\partofspeech{s}
\spanishtranslation{chapulín}
\clarification{insecto}

\entry{*sajl}
\partofspeech{s}
\spanishtranslation{migaja}
\cholexample{Isajl jach waj tsa' iyäk'eyoñ.}
\exampletranslation{Solamente me dio las migajas de las tortillas.}
\secondaryentry{*sajl tye'}
\secondtranslation{astilla de palo}

\entry{sajmäx}
\partofspeech{adv}
\spanishtranslation{desde hoy}
\cholexample{Sajmäx tsa' majli.}
\exampletranslation{Se fue desde hoy.}

\entry{saj ora}
\relevantdialect{Sab.}
\spanishtranslation{luego}
\cholexample{Saj ora jach tsajñiyob tyi tyejklum.}
\exampletranslation{Luego fueron al pueblo.}
\alsosee{tyi ora}

\entry{sajp'el}
\partofspeech{vi}
\spanishtranslation{bajar}
\clarification{agua}
\cholexample{Muk'ix ikajel tyi sajp'el jiñi ñoja'.}
\exampletranslation{El río ya va a bajar.}

\entry{sajtyel}
\relevantdialect{Tila}
\partofspeech{vi}
\onedefinition{1}
\spanishtranslation{perderse}
\cholexample{Wolik sajtyel tyi bij.}
\exampletranslation{Estoy perdiéndome en el camino.}
\onedefinition{2}
\spanishtranslation{morir}
\cholexample{Tsa' ujtyi tyi sajtyel kalobil.}
\exampletranslation{Mi hijo acaba de morir.}

\entry{sajtyem}
\relevantdialect{Tila}
\partofspeech{adj}
\onedefinition{1}
\spanishtranslation{perdido}
\cholexample{Añix ora sajtyem kmula.}
\exampletranslation{Ya hace tiempo que mi mula está perdida.}
\onedefinition{2}
\spanishtranslation{muerto}
\cholexample{Sajtyem kalobil, ma'ix mi kaj kchäñ k'el.}
\exampletranslation{Mi hijo está muerto. Ya no lo voy a ver.}

\entry{sal}
\partofspeech{s}
\spanishtranslation{roncha, sarna}

\entry{saliyel}
\partofspeech{vi}
\spanishtranslation{padecer sarna}
\cholexample{Woli ityi saliyel ik'äb.}
\exampletranslation{Tiene sarna en la mano.}

\entry{sami}
\partofspeech{vi}
\spanishtranslation{ir}
\cholexample{Samiyoñ tyi tyejklum.}
\exampletranslation{Voy a ir al pueblo.}
\alsosee{Gram. 6.16.}

\entry{sañtyo}
\partofspeech{s esp}
\onedefinition{1}
\spanishtranslation{santo}
\cholexample{Jiñi fotyo jiñäch sañtyo icha'añ.}
\exampletranslation{Esa foto es su santo.}
\onedefinition{2}
\spanishtranslation{ídolo}
\cholexample{Mi ich'ujutyesañ jiñi sañtyo.}
\exampletranslation{Él adora ese ídolo.}

\entry{*sak'el}
\partofspeech{s}
\spanishtranslation{escozor}
\cholexample{Kabäl mi ip'ol isak'el lakxix jol.}
\exampletranslation{La caspa produce mucho escozor.}

\entry{sa'}
\partofspeech{s}
\spanishtranslation{masa}

\entry{säk}
\partofspeech{adj}
\spanishtranslation{limpio}
\cholexample{Säk ipislel.}
\exampletranslation{Su ropa está limpia.}

\entry{säkbajlumtye'}
\partofspeech{s}
\spanishtranslation{árbol grande}
\clarification{de madera dura y blanca, y de savia amarilla}

\entry{säkkolañ}
\partofspeech{adj}
\spanishtranslation{ya aclarando}
\clarification{de mañana}
\cholexample{Yomix ch'ojyikoñlakome säkkolañix pañimil.}
\exampletranslation{Levantémonos porque ya está aclarando.}

\entry{säkchaxañ}
\partofspeech{adj}
\spanishtranslation{blanco}
\cholexample{Ibäkel jach jiñi kawayu' säkchaxañ tsa' käle.}
\exampletranslation{Solamente los huesos del caballo quedaron blancos.}

\entry{säkjamañ}
\partofspeech{adj}
\spanishtranslation{claro}
\clarification{el tiempo}
\cholexample{Säkjamañ jiñi pañchañ; ma'añiktyokal.}
\exampletranslation{El cielo está claro; no hay nubes.}

\entry{säkjamtyäl}
\partofspeech{s}
\spanishtranslation{claridad}
\cholexample{Koñlaya'ba'añ säkjamtyäl.}
\exampletranslation{Vamos a donde haya claridad (cuando está en oscuridad una casa).}

\entry{säk jilel}
\partofspeech{vt}
\spanishtranslation{desaparecer}
\cholexample{Wolix isäk jilel its'ijbal ibujk.}
\exampletranslation{Ya está despareciendo el color de su ropa.}

\entry{säklañ}
\partofspeech{vt}
\spanishtranslation{buscar}
\cholexample{Tsa' kaji ksäklañ majlel bij.}
\exampletranslation{Fui a buscar el camino.}
\alsosee{sajkañ}

\entry{*säklel}
\partofspeech{s}
\spanishtranslation{luz, claridad}
\cholexample{Wolix itsiktyiyel isäklel pañimil.}
\exampletranslation{Está apareciendo la claridad del día.}

\entry{säklemañ}
\partofspeech{adj}
\spanishtranslation{brilloso}
\cholexample{Säklemañ jiñi lámiña che' tsijibtyo.}
\exampletranslation{La lámina es brillosa cuando está nueva.}

\entry{säklibañ}
\partofspeech{adj}
\spanishtranslation{pálido}
\cholexample{Säklibañ ipächilel lakej cha'añbäk'eñ.}
\exampletranslation{Nuestros labios están pálidos de miedo.}

\entry{*säklojk}
\partofspeech{s}
\spanishtranslation{espuma}

\entry{Säklumpa'}
\partofspeech{s}
\spanishtranslation{Arroyo de Tierra Blanca}
\clarification{colonia}

\entry{säkluts'añ}
\partofspeech{adj}
\spanishtranslation{blanco}
\clarification{piel}
\cholexample{Säkluts'añ ik'äb, cha'añ mach mi ipul tyi k'iñ.}
\exampletranslation{Tiene la mano blanca porque no le quema el sol.}

\entry{säkme'}
\relevantdialect{Tila}
\partofspeech{s}
\spanishtranslation{venado común}
\alsosee{chijmay}

\entry{säkmotyañ}
\partofspeech{adj}
\spanishtranslation{blancas}
\clarification{conjunto de piedras}
\cholexample{Ya' mi apijtyañoñba' säkmotyañ xajlel.}
\exampletranslation{Ahí me esperas donde está el conjunto de piedras blancas.}

\entry{säkñup'añ}
\partofspeech{adj}
\spanishtranslation{blanco}
\clarification{como piedra}
\cholexample{Ñajtyi säkñup'añ jiñi kolem xajlel.}
\exampletranslation{De lejos esa roca se ve blanca.}

\entry{säkpochañ}
\defsuperscript{1}
\partofspeech{adj}
\spanishtranslation{bien afilado}
\cholexample{Säkpochañ iyej imachity.}
\exampletranslation{El machete está bien afilado.}

\entry{säkpochañ}
\defsuperscript{2}
\partofspeech{adj}
\spanishtranslation{limpio}
\cholexample{Säkpochañ ibujk.}
\exampletranslation{Está blanca y limpia su camisa.}

\entry{säkpojañ}
\partofspeech{adj}
\onedefinition{1}
\spanishtranslation{cara medio blanca}
\onedefinition{2}
\spanishtranslation{limpio}
\clarification{río, cara}

\entry{säkpoyañ}
\relevantdialect{Tila}
\partofspeech{adj}
\spanishtranslation{limpio}
\clarification{casa o cuarto}
\cholexample{Säkpoyañ lakotyoty.}
\exampletranslation{Está bien limpia nuestra casa.}

\entry{säkp'ilañ pañchañ}
\partofspeech{s}
\spanishtranslation{nubes altas y delgadas a distintos niveles}

\entry{säktye'}
\partofspeech{s}
\spanishtranslation{árbol de madera blanca}
\culturalinformation{Información cultural: Se usa para hacer las paredes de una casa.}

\entry{*säktye'lel}
\partofspeech{s}
\spanishtranslation{varillas del techo}
\cholexample{Tsa' jk'äñä amäy cha'añ isäktye'lel kotyoty.}
\exampletranslation{Utilicé carrizo para las varillas del techo de mi casa.}

\entry{säktyijañ}
\partofspeech{adj}
\onedefinition{1}
\spanishtranslation{blanco}
\cholexample{Che' mi iyajlel iña'al tsäñal säktyijañ mi ikäytyäl jiñi jam.}
\exampletranslation{Todo se pone blanco cuando la helada cae sobre el zacate.}
\onedefinition{2}
\spanishtranslation{blanco}
\clarification{cabello, trenzas}
\cholexample{Säktyijañ tyi tyañ ijol jiñi wiñik.}
\exampletranslation{Ese hombre tiene el cabello blanco de cal.}

\entry{säktyilañ}
\partofspeech{adj}
\spanishtranslation{brilloso y blanco}
\cholexample{Säktyilañ ek' tyi pañchañ che' tyi ak'älel.}
\exampletranslation{Las estrellas en el cielo son blancas y brillosas por la noche.}

\entry{säktyo}
\onedefinition{1}
\spanishtranslation{hay luz todavía}
\onedefinition{2}
\spanishtranslation{limpio todavía}
\cholexample{Säktyo jiñi kpislel.}
\exampletranslation{Mi ropa está limpia todavía.}

\entry{säktyojañ}
\partofspeech{adj}
\spanishtranslation{blanco oscuro}
\clarification{como de nubes amenazantes}
\cholexample{Ixix säktyojañ tyilel ja'al.}
\exampletranslation{Allá se ve que viene la lluvia (por la apariencia de las nubes).}

\entry{säktyo'}
\partofspeech{s}
\spanishtranslation{hoja grande para envolver pozol}

\entry{säkts'ijañ}
\partofspeech{adj}
\spanishtranslation{blanco}
\clarification{como piedra}
\cholexample{Ñajtyi säkts'ijañ jiñi xajlel.}
\exampletranslation{Esa roca blanca se ve desde lejos.}

\entry{säk waj}
\spanishtranslation{maíz blanco}

\entry{säkwa'añ}
\partofspeech{adj}
\spanishtranslation{blanco}
\cholexample{Tsikil säkwa'añ jiñi tye' ya' tyi wits.}
\exampletranslation{El palo que está en la cumbre se ve blanco desde lejos.}

\entry{säkwelañ}
\partofspeech{adj}
\spanishtranslation{blanca}
\clarification{tela}
\cholexample{Säkwelañ jiñi pisil wolibä ichoñ.}
\exampletranslation{La tela que está vendiendo es blanca.}

\entry{säkwolañ}
\partofspeech{adj}
\spanishtranslation{blanco}
\clarification{pelo}
\cholexample{Säkwolañ ijol jiñi xñox.}
\exampletranslation{La cabeza de ese anciano es blanca.}

\entry{säkwutyiyel}
\partofspeech{vi}
\spanishtranslation{quedarse blanco}
\cholexample{Mi isäkwutyiyel jiñi waj che' ma'añik mi laksutyk'iñ.}
\exampletranslation{Un lado de la tortilla queda blanco cuando no se voltea.}

\entry{säkxojañ}
\partofspeech{adj}
\spanishtranslation{se ve claro}
\cholexample{Säkxojañ ik'ak'al lámpara.}
\exampletranslation{La luz de la lámpara se ve clara.}

\entry{säk'}
\partofspeech{vt}
\spanishtranslation{lavar}
\clarification{café, maíz}
\cholexample{Mi laksäk' jiñi kajpe' che' pajix.}
\exampletranslation{Cuando el café está fermentado, lo lavamos.}

\entry{säk'ajel}
\partofspeech{adv}
\spanishtranslation{al amanecer}
\cholexample{Tyi säk'ajel tsa' majli.}
\exampletranslation{Al amanacer se fue.}

\entry{säk'oñel}
\partofspeech{s}
\spanishtranslation{acción de lavar}
\clarification{maíz}
\cholexample{Mach yujilik säk'oñel.}
\exampletranslation{No sabe lavar maíz.}

\entry{säjlel}
\partofspeech{vi}
\spanishtranslation{agotarse}
\cholexample{Mux kaj tyi säjlel kixim.}
\exampletranslation{Ya se va a agotar mi maíz.}

\entry{säl}
\partofspeech{vt}
\spanishtranslation{ensartar}
\cholexample{Mi laksäl we'eläl che' mi lakpojpoñ.}
\exampletranslation{Ensartamos la carne al asarla.}

\entry{sälem}
\partofspeech{adj}
\spanishtranslation{jaspeado}
\cholexample{Sälem jiñi xña'muty.}
\exampletranslation{La gallina es jaspeada.}

\entry{sämäkña}
\partofspeech{adj}
\spanishtranslation{ondeando}
\cholexample{Juñsujm jachbä' sämäkña woli iñijkañ ibä jiñi ja'.}
\exampletranslation{El agua está ondeando tranquilamente en un sólo lugar.}

\entry{sämäl}
\partofspeech{adj}
\spanishtranslation{quieto, tranquilo}
\clarification{líquido}
\cholexample{Sämäl jiñi ja'.}
\exampletranslation{El agua está quieta.}

\entry{sämlaw i yik'añ}
\spanishtranslation{oscuro}
\cholexample{Sämlaw iyik'añ tsa' juliyoñ tyi kotyoty.}
\exampletranslation{Llegué a mi casa cuando ya estaba oscuro.}

\entry{sämtyäl}
\partofspeech{vi}
\spanishtranslation{bajar y quedar}
\clarification{neblina}
\cholexample{Pejtyelel ora ya' mi isämtyäl jiñityokal tyi ñoja'.}
\exampletranslation{La neblina siempre baja y se queda sobre el río.}

\entry{säp}
\partofspeech{adj}
\nontranslationdef{Se relaciona con el sonido y la forma en que va una flecha; p. ej.:}
\cholexample{Tsak säp julu muty yik'oty kjaläjp.}
\exampletranslation{Le tiré mi flecha a un pájaro.}

\entry{säpäl}
\partofspeech{adj}
\spanishtranslation{estirado}
\clarification{alambre, soga}

\entry{säpäty yopom}
\spanishtranslation{hoja de pozol}

\entry{säpla}
\partofspeech{adv}
\spanishtranslation{zumbando}
\cholexample{Säplatyak tsa' ñumibalatyik chikiñ.}
\exampletranslation{La bala pasó zumbando por mi oído.}

\entry{säp'äl}
\partofspeech{adj}
\spanishtranslation{bajo}
\clarification{agua}
\spanishtranslation{vadoso}
\cholexample{Säp'äl jiñi ñoja'.}
\exampletranslation{El río ya está bajo.}

\entry{*säkel tyumuty}
\spanishtranslation{clara de huevo}

\entry{säkix i jol}
\spanishtranslation{cabeza blanca}

\entry{säsäk}
\partofspeech{adj}
\spanishtranslation{blanco}

\entry{säsäk k'ätsats}
\spanishtranslation{papausa, ilama}
\clarification{árbol frutal}

\entry{säsäk tye'}
\spanishtranslation{árbol de madera blanca}
\clarification{se utiliza para muebles y leña}

\entry{säsäk uch'}
\spanishtranslation{piojos del cuerpo}

\entry{säty}
\partofspeech{vt}
\spanishtranslation{echar a perder}
\cholexample{Tsa' ujtyi isäty ityak'iñ.}
\exampletranslation{Acaba de perder su dinero.}
\secondaryentry{säty k'iñ}
\secondtranslation{perder tiempo}

\entry{säts'}
\defsuperscript{1}
\partofspeech{vt}
\spanishtranslation{estirar}
\cholexample{Woli isäts' ik'äb cha'añ tsa' xujli.}
\exampletranslation{Está estirando la mano porque se la zafó.}

\entry{säts'}
\defsuperscript{2}
\partofspeech{s}
\spanishtranslation{gusano comestible}

\entry{säy}
\partofspeech{vt}
\spanishtranslation{sacar}
\clarification{del agua con la mano}

\entry{seb}
\partofspeech{adv}
\onedefinition{1}
\spanishtranslation{temprano}
\cholexample{Seb mik ch'ojyel cha'añ mik majlel tyi e'tyel.}
\exampletranslation{Me levanto temprano para ir a trabajar.}
\onedefinition{2}
\spanishtranslation{rápido}
\cholexample{Seb mi ipechañ iwaj jiñi x'ixik.}
\exampletranslation{Esa mujer hace rápido sus tortillas.}
\secondaryentry{seboñ}
\secondpartofspeech{adv}
\secondtranslation{rápido (yo)}
\secondaryentry{sebety}
\secondpartofspeech{adv}
\secondtranslation{rápido (tú)}

\entry{sebuñ}
\partofspeech{vt}
\spanishtranslation{apurarse}
\cholexample{Sebuñ abä cha'añ mi amel awe'tyel.}
\exampletranslation{Apúrate para que hagas tu trabajo.}

\entry{sek'}
\partofspeech{vt}
\spanishtranslation{tumbar}

\entry{sejb}
\partofspeech{adj}
\spanishtranslation{liviano}
\cholexample{Sejb jach ikuch kmula.}
\exampletranslation{La carga de mi mula está liviana.}

\entry{sejb'añ}
\partofspeech{vi}
\spanishtranslation{hacerse liviano}
\cholexample{Mi kaj kty'ox lok'sañ jiñi kuchäl cha'añ mi isejbañ.}
\exampletranslation{Voy a quitar un poco de carga para hacerla más liviana.}

\entry{sejel}
\partofspeech{adj}
\spanishtranslation{abundante}
\cholexample{Sejel jiñi xiñich' tyij kajpe'lel.}
\exampletranslation{Las hormigas son abundantes en mi cafetal.}

\entry{sejluñ}
\partofspeech{vt}
\spanishtranslation{rebanar}
\cholexample{Mi kaj isejluñ ochel seboria ya tyi' p'ejtyal we'eläl.}
\exampletranslation{Vamos a rebanar cebolla en la olla de carne.}

\entry{-sejm}
\nontranslationdef{Sufijo numeral para contar manojos; p. ej.:}
\cholexample{Wä'añ cha'sejm jiñi jam.}
\exampletranslation{Aquí están dos manojos de zacate.}

\entry{-sejty}
\nontranslationdef{Sufijo numeral para contar rollos de algo; p. ej.:}
\cholexample{Añ cha'sejty jiñi tye'.}
\exampletranslation{Hay dos rollos de palos.}

\entry{sel}
\onedefinition{1}
\partofspeech{adv}
\nontranslationdef{Concuerda con un objeto rollizo ; p. ej.}
\cholexample{Tsa' isel ty'ojo ch'ujm.}
\exampletranslation{Ella partió la calabaza.}
\onedefinition{2}
\partofspeech{vi}
\spanishtranslation{enrollarse}
\cholexample{Ya' tyi' yebal xajlel mi isel ibä jiñi lukum.}
\exampletranslation{La culebra se enrolla en la rendija de la piedra.}

\entry{selel}
\partofspeech{adj}
\spanishtranslation{redondo}
\cholexample{Selel tsa' imele isemejty.}
\exampletranslation{Hizo su comal redondo.}

\entry{selelob}
\partofspeech{adj}
\onedefinition{1}
\spanishtranslation{reunidos}
\cholexample{Selelob jiñi wiñikob wolibä ik'elob jiñi yumäl.}
\exampletranslation{Todos los hombres que están reunidos miran al jefe.}
\onedefinition{2}
\spanishtranslation{gente en un círculo}

\entry{selol}
\partofspeech{adj}
\spanishtranslation{alrededor}
\cholexample{Tyi selol jiñi otyoty ya'añ jiñi wiñikob.}
\exampletranslation{La gente está alrededor de la casa.}

\entry{selulañ}
\partofspeech{vt}
\spanishtranslation{rodar}
\cholexample{Mi kaj kselulañ jubel jiñi selelbä xajlel.}
\exampletranslation{Voy a rodar esa piedra redonda.}

\entry{selulañtyel}
\partofspeech{vi}
\spanishtranslation{enrollarse}
\cholexample{Tsa'ix ujtyi tyi selulañtyel jiñi laso.}
\exampletranslation{Se acaba de enrollar la soga.}

\entry{selu'}
\partofspeech{s}
\spanishtranslation{tipo de pájaro}

\entry{semejty}
\partofspeech{s}
\spanishtranslation{comal}

\entry{señ}
\partofspeech{adj}
\spanishtranslation{entumido}
\cholexample{Señix ibäkel iyej.}
\exampletranslation{Ya tiene el diente entumido.}

\entry{señ'añ}
\partofspeech{vi}
\spanishtranslation{entumirse}

\entry{sep'}
\partofspeech{vt}
\spanishtranslation{pellizcar}
\cholexample{Mi imulañ sep' jiñi alob.}
\exampletranslation{Al muchacho le gusta pellizcar.}

\entry{sek'el}
\partofspeech{adj}
\spanishtranslation{tumbado}
\cholexample{Wajalix sek'el jiñi tye'.}
\exampletranslation{Ese árbol ya tiene tiempo de estar tumbado.}

\entry{serbatyañ}
\partofspeech{s}
\spanishtranslation{zoyate, coyolito}
\clarification{arbusto}

\entry{sesety}
\partofspeech{adv}
\nontranslationdef{Se relaciona con la idea de manojos; p. ej.}
\cholexample{Sesety kächbil tyak jiñi jam.}
\exampletranslation{El zacate está amarrado en manojos.}

\entry{ses p'ok}
\spanishtranslation{salamandra}

\entry{setychokoñ}
\partofspeech{vt}
\spanishtranslation{dejar}
\clarification{rollo de palos}
\cholexample{Wä' mi kaj ksetychokoñ jiñi si'.}
\exampletranslation{Voy a dejar aquí un rollo de leña.}

\entry{setyejty}
\partofspeech{s}
\spanishtranslation{trompo}

\entry{setyuñ}
\partofspeech{vt}
\spanishtranslation{cortar}
\clarification{palo, tabla}
\cholexample{Woli setyuñ iñi' kukujl.}
\exampletranslation{Estoy cortando la punta de la viga.}

\entry{sety'}
\partofspeech{vt}
\spanishtranslation{cortar}
\clarification{papel, tela, pelo}
\cholexample{Uts'aty mi isety' tyak joläl jiñi wiñik.}
\exampletranslation{Ese hombre corta bien el pelo.}

\entry{sety'el}
\partofspeech{adj}
\spanishtranslation{completo}
\clarification{cantidad necesaria para ajustar}
\cholexample{Sety'el tsa' iyäk'ä jump'ejl kilo ixim.}
\exampletranslation{Dio un kilo de maíz completo para ajustar lo que faltaba.}

\entry{sewal}
\partofspeech{s}
\spanishtranslation{red chica}

\entry{sewel}
\partofspeech{adj}
\spanishtranslation{torcido}
\cholexample{Sewel tsa' käle ityi' jiñi tyabla.}
\exampletranslation{La orilla de la tabla quedó torcida.}

\entry{se'ñuñ}
\partofspeech{imp}
\spanishtranslation{¡apúrate!}

\entry{se'sebety}
\partofspeech{imp}
\spanishtranslation{¡debes apurarte!}

\entry{sibal}
\partofspeech{s}
\spanishtranslation{acto de cortar leña}
\cholexample{Tsa' majli tyi sibal.}
\exampletranslation{Fue a cortar leña (lit.: donde está haciendo el acto de cortar leña).}

\entry{sibik}
\partofspeech{s}
\onedefinition{1}
\spanishtranslation{cohete}
\onedefinition{2}
\spanishtranslation{pólvora de armas}

\entry{sibikchuch}
\partofspeech{s}
\spanishtranslation{un bejuco}
\culturalinformation{Información cultural: se encuentra en tierra caliente; la flor se usa para fermentar}

\entry{sikojk}
\partofspeech{s}
\spanishtranslation{seso}

\entry{sik'}
\partofspeech{vt}
\spanishtranslation{oler, olfatear}
\cholexample{Jiñi ts'i' woli isik'beñ iyujts'il tye'lal.}
\exampletranslation{El perro está olfateando el tufo de tepezcuintle.}

\entry{sik'äb}
\partofspeech{s}
\spanishtranslation{caña}

\entry{sik'bal}
\partofspeech{s}
\spanishtranslation{cañal}
\cholexample{Tsa' ujtyi iyäk'ñañ isik'bal.}
\exampletranslation{Terminó de limpiar su cañal.}

\entry{*sik'bal ixim}
\spanishtranslation{palo o tallo (de maíz)}

\entry{sijiñ}
\partofspeech{vt}
\spanishtranslation{dar en casamiento}
\cholexample{Mi kaj isijiñ xch'okbä iyalobil.}
\exampletranslation{Va a dar en casamiento a su hija.}

\entry{sijmal}
\partofspeech{s}
\spanishtranslation{catarro}

\entry{sijmañ}
\partofspeech{vt}
\spanishtranslation{sonar}
\clarification{nariz}
\cholexample{Woli isijmañ iñi' cha'añ sijmal.}
\exampletranslation{Se está sonando la nariz por el catarro.}

\entry{sijom}
\partofspeech{s}
\spanishtranslation{tornamilpa}
\culturalinformation{Información cultural: Se siembra después de la cosecha de la milpa del año. No se quema.}
\variation{sijomal}

\entry{sijpel}
\partofspeech{vi}
\onedefinition{1}
\spanishtranslation{desprenderse de}
\cholexample{Mi kaj tyi sijpel jiñi kukujl ya' tyi' yoyel.}
\exampletranslation{Se va a desprender la viga del horcón.}
\onedefinition{2}
\spanishtranslation{caer}
\clarification{trampa}
\cholexample{Tsa' loñ sijpi yak.}
\exampletranslation{Cayó de balde la trampa.}

\entry{sijty'em}
\partofspeech{adj}
\spanishtranslation{hinchado}
\cholexample{Sijty'em iyokba' tsa' ijats'ä tyi xajlel.}
\exampletranslation{Su pie está hinchado en donde se golpeó con la piedra.}

\entry{sil}
\partofspeech{vt}
\spanishtranslation{rajar}
\cholexample{Mi laksil tyi ojlil jiñi ch'ajañ.}
\exampletranslation{Rajamos el mecate en medio.}

\entry{silaj}
\partofspeech{s esp}
\spanishtranslation{sidra}

\entry{silikña}
\defsuperscript{1}
\partofspeech{adj}
\spanishtranslation{angosto}
\cholexample{Silikña jach jiñi tyabla.}
\exampletranslation{Esta tabla está angosta.}

\entry{silikña}
\defsuperscript{2}
\partofspeech{adv}
\spanishtranslation{ruidosamente}
\clarification{sonido que da la madera al rajarse}
\cholexample{Silikña jiñi tye' che' mi laksil.}
\exampletranslation{La madera se raja ruidosamente.}

\entry{simaroñ}
\partofspeech{adj esp}
\onedefinition{1}
\relevantdialect{Tila}
\spanishtranslation{malo}
\cholexample{Weñ simaroñ jiñi wiñik.}
\exampletranslation{Ese hombre es muy malo.}
\onedefinition{2}
\spanishtranslation{violento, peligroso}
\cholexample{Mach añochtyañ jiñi simaroñbä wiñik ame mi itsäñsañety.}
\exampletranslation{No te acerques a ese hombre peligroso; no sea que te mate.}

\entry{*simaroñiyel}
\relevantdialect{Sab.}
\partofspeech{s esp}
\spanishtranslation{maldad}
\cholexample{Mach yom ikäy isimaroñiyel.}
\exampletranslation{No quiere dejar su maldad.}
\alsosee{mulil}

\entry{siñcho}
\partofspeech{s esp}
\spanishtranslation{cinturón}

\entry{síñtyiko}
\partofspeech{s esp}
\spanishtranslation{síndico}
\culturalinformation{Información cultural: la autoridad que es superior al <mayor.> Recibe órdenes del <ajcal> y lo pasa al <mayor>, y el <mayor> se lo comunica al <wasil> para llevar a cabo tal orden. En asuntos graves es necesario la intervención del síndico para que los tres resuelvan el caso.}

\entry{siñañ}
\partofspeech{s}
\spanishtranslation{alacrán}

\entry{siñañ ak'}
\partofspeech{s}
\spanishtranslation{bejuco de alacrán}
\culturalinformation{Información cultural: Si uno está sudando y toca el bejuco, la piel arde mucho.}

\entry{sip}
\partofspeech{s}
\spanishtranslation{garrapata}
\clarification{insecto}

\entry{sity'kuyel}
\partofspeech{vi}
\spanishtranslation{hincharse}
\cholexample{Wolix isity'kuyel jiñi wiñik cha'añ jilemix ich'ich'el.}
\exampletranslation{Ese hombre está hinchándose porque está anémico.}

\entry{sity'il}
\partofspeech{adj}
\spanishtranslation{hinchado}
\cholexample{Sity'il ik'äb jiñi alob.}
\exampletranslation{La mano de ese muchacho está hinchada.}

\entry{sity'olajel}
\partofspeech{vi}
\spanishtranslation{hincharse}
\cholexample{Mi isity'olajel jiñi bu'ul yik'oty ixim che' mi lakotsañ tyi ja'.}
\exampletranslation{El frijol y el maíz se hinchan al remojarlos en agua.}

\entry{*sits'}
\partofspeech{s}
\onedefinition{1}
\spanishtranslation{ganas de comer}
\clarification{carne}
\cholexample{Kabäl isits' cha'añ yom we'eläl.}
\exampletranslation{Tiene ganas de comer carne.}
\onedefinition{2}
\spanishtranslation{saliva}
\cholexample{Woli imäsañ isits'.}
\exampletranslation{Se está tragando la saliva, porque tiene ganas de comer.}

\entry{*sits'lel}
\partofspeech{s}
\spanishtranslation{glotonería}
\cholexample{Kabäl isits'lel jiñi wiñik.}
\exampletranslation{Ese hombre es muy glotón (lit.: tiene glotonería).}

\entry{*siyajlel}
\partofspeech{s esp}
\spanishtranslation{montura}
\cholexample{Tsukulix isiyajlel jkawayu'.}
\exampletranslation{La montura de mi caballo ya está vieja.}

\entry{si'}
\partofspeech{s}
\spanishtranslation{leña}

\entry{si'ilañ}
\partofspeech{vt}
\spanishtranslation{usar}
\clarification{como leña}
\cholexample{Woli isi'ilañ ibojtye'lel iyotyoty.}
\exampletranslation{Está usando la cerca de su casa como leña.}

\entry{si'im}
\partofspeech{s}
\onedefinition{1}
\spanishtranslation{esposa del hermano de mi madre}
\onedefinition{2}
\spanishtranslation{tía política}

\entry{sobrajiyel}
\partofspeech{vi esp}
\spanishtranslation{sobrar}
\clarification{comida}
\cholexample{Kabäl mi isobrajiyel jiñi waj.}
\exampletranslation{Sobra mucha tortilla.}

\entry{sok}
\defsuperscript{1}
\partofspeech{vt}
\spanishtranslation{enredar}
\cholexample{Jiñi aläl woli isokbeñ ipuy iña'.}
\exampletranslation{El niño está enredando el hilo de su mamá.}

\entry{sok}
\defsuperscript{2}
\partofspeech{s}
\spanishtranslation{guabina}
\clarification{pez}

\entry{sokokña}
\partofspeech{adv}
\spanishtranslation{ruidosamente}
\clarification{animal pasando por la hierba}
\cholexample{Jiñi lukum sokokña mi iñijkañ majlel tyikiñ pimel.}
\exampletranslation{La culebra va por las hojas secas ruidosamente.}

\entry{sojlel}
\partofspeech{vi}
\onedefinition{1}
\spanishtranslation{penetrar}
\clarification{frío, agua}
\cholexample{Ma'añik mi isojlel tsäñal che' cha'lajm lakmosil.}
\exampletranslation{El frío no penetra cuando tenemos dos cobijas.}
\onedefinition{2}
\spanishtranslation{aprender}
\cholexample{Wolix isojlel majlel tyi k'el juñ.}
\exampletranslation{Está aprendiendo mucho en sus estudios.}

\entry{sojlem}
\partofspeech{adj}
\spanishtranslation{capacitado}
\cholexample{Maxtyo sojlemoñik tyi k'el juñ.}
\exampletranslation{Todavía no estoy capacitado para leer.}

\entry{sojkel}
\partofspeech{vi}
\onedefinition{1}
\spanishtranslation{enredarse}
\cholexample{Mi isojkel jiñi chij.}
\exampletranslation{Se enreda el ixtle.}
\onedefinition{2}
\spanishtranslation{trastornarse}

\entry{sojkem}
\partofspeech{adj}
\onedefinition{1}
\spanishtranslation{enredado}
\cholexample{Sojkem jiñi laso.}
\exampletranslation{La soga está enredada.}
\onedefinition{2}
\spanishtranslation{desorientado}
\cholexample{Sojkem ijol che' tyi vierñes.}
\exampletranslation{El viernes estuvo desorientado.}

\entry{sojwik'tyik}
\partofspeech{adj}
\spanishtranslation{enredado}
\cholexample{Sojwik'tyik jiñi puy.}
\exampletranslation{Ese hilo está todo enredado.}

\entry{solk'iñ}
\relevantdialect{Tila}
\partofspeech{s}
\nontranslationdef{Períodos de 52 años que están relacionados con el sol.}

\entry{sombäl}
\partofspeech{vi}
\spanishtranslation{encogerse}
\cholexample{Woli isombäl kwex.}
\exampletranslation{Mi pantalón se está encogiendo.}

\entry{sombreróñ}
\partofspeech{s esp}
\spanishtranslation{fantasma}
\culturalinformation{Información cultural: Se dice que tiene sombrero grande y que es de un metro de altura. También se dice que anda en la noche, y nos echa a la tierra para que nos enfermemos.}

\entry{soñ}
\partofspeech{s}
\spanishtranslation{baile}
\cholexample{Mi mulañ soñ jiñi xyäk'äjel.}
\exampletranslation{A ese borracho le gusta el baile.}

\entry{soñso}
\partofspeech{adj esp}
\spanishtranslation{zonzo}
\cholexample{Weñ soñsojety; ma'añik mi aña'tyañ pañimil.}
\exampletranslation{Eres muy zonzo; no entiendes.}

\entry{sop'}
\partofspeech{adj}
\spanishtranslation{cansado}
\clarification{los pies, las piernas}
\cholexample{Sop' jiñi kok cha'añ kabäl xämbal.}
\exampletranslation{Tengo los pies muy cansados por caminar tanto.}

\entry{soraru}
\partofspeech{s esp}
\spanishtranslation{soldado}

\entry{sorarujiñtyel}
\partofspeech{vi esp}
\spanishtranslation{ser soldado}
\cholexample{Jiñi ch'ityoñ tsa' majli tyi sorarujiñtyel.}
\exampletranslation{El muchacho iba a ser soldado.}

\entry{sorokña}
\partofspeech{adj}
\spanishtranslation{burbujeante}
\cholexample{Sorokña ip'ejtyal bu'ul cha'añ woli tyi lojk.}
\exampletranslation{La olla de frijoles está burbujeante porque está hirviendo.}

\entry{*sos muty}
\spanishtranslation{molleja}

\entry{*soty'oty'}
\partofspeech{s}
\spanishtranslation{hígado}

\entry{sowilañ}
\defsuperscript{1}
\partofspeech{vt}
\spanishtranslation{arrugar}
\clarification{trapo}
\cholexample{Tsa' isowilatyi' k'äb.}
\exampletranslation{Arrugó el trapo con la mano.}

\entry{sowilañ}
\defsuperscript{2}
\partofspeech{vt}
\spanishtranslation{enredar}
\clarification{hilo}
\cholexample{Tsa' isowilajiñi puy.}
\exampletranslation{Enredó el hilo.}

\entry{sowol}
\partofspeech{adj}
\spanishtranslation{tirado}
\clarification{trapo}
\cholexample{Sowol jiñi pisil ya' tyi lum.}
\exampletranslation{El trapo está tirado en el suelo.}

\entry{soytya'}
\partofspeech{s}
\spanishtranslation{tripa}

\entry{stsaja tyuñ}
\spanishtranslation{chaya pica}
\clarification{hierba que produce ronchas ardientes}

\entry{stsats}
\partofspeech{s}
\spanishtranslation{sardina plateada}
\clarification{pez}

\entry{stselel}
\partofspeech{s}
\spanishtranslation{carpintero real, picotero}
\clarification{ave}

\entry{stsijb}
\partofspeech{s}
\spanishtranslation{helecho}

\entry{stsijk}
\partofspeech{s}
\spanishtranslation{pico de oro, cerquero pico dorado}
\clarification{ave}

\entry{stsijtye'}
\partofspeech{s}
\spanishtranslation{temperante}
\clarification{árbol}

\entry{stsimajtye'}
\partofspeech{s}
\spanishtranslation{morro, cuautecomate}
\clarification{árbol}

\entry{stsuk bajlum}
\partofspeech{s}
\spanishtranslation{leoncillo, jaguarundi}
\clarification{mamífero}

\entry{stsuk'ñichim}
\partofspeech{s}
\spanishtranslation{encendedor de las velas en la iglesia}

\entry{stsuy}
\partofspeech{s}
\spanishtranslation{árbol de hojas comestibles}

\entry{sts'äkaya}
\partofspeech{s}
\spanishtranslation{curandero, hierbatero, brujo}
\culturalinformation{Información cultural: El curandero frota el cuerpo del enfermo con agua y hierbas. Sopla el epazote. Mata un puerco, ave o pollo. Usa el caldo para soplarle al enfermo.}

\entry{sts'äptye' bu'ul}
\spanishtranslation{frijol de vara}

\entry{sts'ijb}
\partofspeech{s}
\spanishtranslation{escriba}
\dialectvariant{Sab.}
\dialectword{ajts'ijb}

\entry{sts'ijbaya}
\partofspeech{s}
\spanishtranslation{escritor, secretario}

\entry{sts'ijbujel}
\partofspeech{s}
\spanishtranslation{escritor}

\entry{sts'isoñel}
\partofspeech{s}
\spanishtranslation{sastre}

\entry{sts'ots'}
\partofspeech{s}
\spanishtranslation{cría de puerco}

\entry{sts'u'lel}
\partofspeech{s}
\spanishtranslation{haragán}
\cholexample{Sts'u'lel jiñi ch'ityoñ.}
\exampletranslation{Ese joven es un haragán.}
\alsosee{ts'u'lel}

\entry{sub}
\partofspeech{vt}
\onedefinition{1}
\spanishtranslation{decir}
\cholexample{Che'äch yom mi asubbajche' tsa' awälä.}
\exampletranslation{Así debes decirlo, como lo oíste.}
\onedefinition{2}
\spanishtranslation{avisar}
\cholexample{Tsa'tyo majli iñaxañ sub.}
\exampletranslation{Primero fue a avisar.}
\onedefinition{3}
\spanishtranslation{confesar}
\cholexample{Tsa' isubu ibä tyi'tyojlel jiñi ambä iye'tyel.}
\exampletranslation{Confesó a la autoridad lo que hizo.}
\secondaryentry{sub ty'añ}
\secondtranslation{predicar}
\secondaryentry{subty'añ}
\secondpartofspeech{s}
\secondtranslation{conferencia}

\entry{subil}
\partofspeech{adj}
\spanishtranslation{dicho}
\cholexample{Maxtyo añik subilbaki ora tyal.}
\exampletranslation{No se ha dicho cuándo viene.}

\entry{suboñel}
\partofspeech{s}
\spanishtranslation{mensajero}

\entry{sujbem}
\partofspeech{adj}
\spanishtranslation{dicho}
\cholexample{Che' yom mi lakmelbajche' sujbem.}
\exampletranslation{Así lo debemos hacer, como está dicho.}

\entry{sujkuñ}
\partofspeech{vt}
\spanishtranslation{limpiar}
\cholexample{Tsa' isujku mesa.}
\exampletranslation{Limpió la mesa.}

\entry{-sujl}
\nontranslationdef{Sufijo numeral para contar zambullidas; p. ej.:}
\cholexample{Cha'sujl tsa' its'ajä ibä tyi ja'.}
\exampletranslation{Se zambulló dos veces en el agua.}

\entry{*sujl}
\partofspeech{s}
\onedefinition{1}
\spanishtranslation{cáscara}
\clarification{de arroz, café, frijol, maíz}
\cholexample{Tsa' jach käle isujl bu'ul.}
\exampletranslation{Solamente quedó la cáscara del frijol.}
\onedefinition{2}
\spanishtranslation{escama}
\secondaryentry{isujlil}
\secondtranslation{su cáscara, su escama}

\entry{sujlum}
\partofspeech{s}
\spanishtranslation{musgo que brota de la tierra}

\entry{-sujm}
\nontranslationdef{Sufijo numeral para contar cosas de la misma sustancia o pieza; p. ej.:}
\cholexample{Juñsujm jach lakpislel.}
\exampletranslation{Nuestras telas son de la misma pieza.}

\entry{sujmityesañ}
\partofspeech{vt}
\spanishtranslation{aclarar}
\cholexample{Mi kaj ksujmityesañ tsa'bä iyäläyob.}
\exampletranslation{Voy a aclarar lo que dijeron.}

\entry{*sujmlel}
\partofspeech{s}
\spanishtranslation{verdad}
\cholexample{Tsa' isubu isujmlel ity'añ dios.}
\exampletranslation{Habló la verdad de la Palabra de Dios.}
\alsosee{isujm}

\entry{sujp'el}
\partofspeech{vi}
\spanishtranslation{sumergir}
\cholexample{Mi laksujp'el ochel tyi ja' che' mach lakujilik ñuxijel.}
\exampletranslation{Nos sumergimos en el agua cuando no sabemos nadar.}

\entry{sujtyel}
\partofspeech{vi}
\spanishtranslation{regresar}
\cholexample{Samiyoñix tyi sujtyel.}
\exampletranslation{Ya me voy a regresar.}

\entry{-sujtyel}
\nontranslationdef{Sufijo numeral para contar vueltas; p. ej.:}
\cholexample{Cha'sujtyel tsajñiyoñ tyi tyejklum sajmäl.}
\exampletranslation{Hoy di dos vueltas al pueblo.}

\entry{sujtyib}
\partofspeech{s}
\spanishtranslation{vuelto, cambio}
\clarification{de dinero}
\cholexample{Ma'añik tsa' iyäk'eyoñ isujtyib ktyak'iñ.}
\exampletranslation{No me dio el vuelto de mi dinero.}

\entry{sul}
\partofspeech{vt}
\onedefinition{1}
\spanishtranslation{zambullir}
\cholexample{Tsa' isulu ibä tyi ja'.}
\exampletranslation{Se zambulló en el agua.}
\onedefinition{2}
\spanishtranslation{remojar}
\cholexample{Tsa' ujtyi ksul kbä tyi ja'.}
\exampletranslation{Acabo de remojarme en el agua.}

\entry{sulul}
\partofspeech{adj}
\spanishtranslation{remojado}
\cholexample{Sulul tyi ja' jiñi pisil.}
\exampletranslation{El trapo está remojado en el agua.}

\entry{sulup}
\partofspeech{s}
\spanishtranslation{mariposa del comején}

\entry{sumuk}
\partofspeech{adj}
\spanishtranslation{sabroso}
\cholexample{Sumuk jiñi we'eläl che' añ its'äkal.}
\exampletranslation{La carne está sabrosa cuando está condimentada.}
\secondaryentry{isumuklel}
\secondpartofspeech{s}
\secondtranslation{su sabor}

\entry{sup}
\partofspeech{adj}
\spanishtranslation{sin sal}
\cholexample{Sup jiñi bu'ul cha'añ ma'añik iyäts'mil.}
\exampletranslation{El frijol está insípido porque no tiene sal (lit.: es sin sal).}

\entry{susubil}
\partofspeech{adj}
\spanishtranslation{raspado}
\cholexample{Susubil ipaty jiñi tye'.}
\exampletranslation{La cáscara de ese palo está raspada.}

\entry{susuñ}
\partofspeech{vt}
\spanishtranslation{raspar}
\cholexample{Wolix susubeñ ipaty jiñi ñi'uk'.}
\exampletranslation{Estoy raspando la cáscara del chayote.}

\entry{*sutyolel}
\partofspeech{s}
\spanishtranslation{alrededor}
\cholexample{Tyi' sutyolel iyotyoty tsa' ilaj mäkä tyi ch'ix tyak'iñ.}
\exampletranslation{Cercó alrededor de su casa con alambre.}

\entry{sutyk'iñ}
\partofspeech{vt}
\onedefinition{1}
\spanishtranslation{voltear}
\cholexample{Maxtyo mejlik isutyk'iñ jiñi waj tyi semejty.}
\exampletranslation{Todavía no puede voltear la tortilla en el comal.}
\onedefinition{2}
\spanishtranslation{devolver}
\cholexample{Mach yomix isutyk'iñ ktyak'iñ.}
\exampletranslation{Ya no quiere devolverme mi dinero.}

\entry{sutyujty}
\partofspeech{s}
\spanishtranslation{vueltas}
\cholexample{Woli tyi sutyujty alob.}
\exampletranslation{El chamaco está dando vueltas.}

\entry{sutyuty ik'}
\partofspeech{s}
\spanishtranslation{remolino de viento}

\entry{suty' kächbil}
\partofspeech{adj}
\spanishtranslation{amarrado}
\clarification{un rollo}
\cholexample{Suty' kächbil icha'añ juñsejm jam.}
\exampletranslation{Tiene amarrado un rollo de zacate.}

\entry{suty'chokoñ}
\partofspeech{vt}
\spanishtranslation{colocar}
\clarification{un rollo}
\cholexample{Mi kaj ksuty'chokoñ ili juñsejty si'.}
\exampletranslation{Voy a colocar este rollo de leña en el suelo.}

\entry{suts'}
\partofspeech{s}
\spanishtranslation{murciélago}

\entry{suts'atyax i wuty}
\spanishtranslation{tiene sueño}
\clarification{lit.: tiene cara de murciélago; porque el murciélago duerme con la cabeza para abajo}

\entry{suts'tye'}
\partofspeech{s}
\spanishtranslation{liquidámbar}
\clarification{árbol de madera maciza}

\entry{suts'tye'ol}
\partofspeech{s}
\spanishtranslation{arboleda de liquidámbar}

\entry{suts'ul}
\partofspeech{s}
\spanishtranslation{caoba}
\clarification{árbol}

\alphaletter{Ty}

\entry{tyak'}
\partofspeech{s}
\spanishtranslation{menor}
\cholexample{Jiñi askuñ jiñäch ityak' jiñi xñox.}
\exampletranslation{Ese joven es el menor.}

\entry{tyak'äl}
\partofspeech{adj}
\spanishtranslation{próximo en edad}
\cholexample{Ili ch'ityoñ tyak'äl ibä yik'oty ili alob.}
\exampletranslation{Este chamaco le sigue en edad a este niño.}

\entry{tyaj}
\defsuperscript{1}
\partofspeech{vt}
\onedefinition{1}
\spanishtranslation{encontrar}
\cholexample{Ma'añik woli ityaj e'tyel.}
\exampletranslation{No está encontrando trabajo.}
\onedefinition{2}
\spanishtranslation{alcanzar}
\cholexample{Tsa' majli ityaj tyilel ityaty.}
\exampletranslation{Fue a alcanzar a su papá.}

\entry{tyaj}
\defsuperscript{2}
\partofspeech{s}
\spanishtranslation{ocote}

\entry{tyajbachim}
\partofspeech{s}
\nontranslationdef{Criatura del agua con muchos pies, alcanza hasta treinta centímetros.}

\entry{tyajbal}
\partofspeech{s}
\spanishtranslation{mecapal}

\entry{tyajm}
\relevantdialect{Sab.}
\partofspeech{s}
\spanishtranslation{mecapal}

\entry{tyajñ}
\partofspeech{s}
\spanishtranslation{pecho, pechuga}
\clarification{de aves}

\entry{tyajñañ}
\partofspeech{vt}
\spanishtranslation{golpear una herida}
\cholexample{Tsa' ujtyi ktyajñañ jk'äbba' lojwem.}
\exampletranslation{Me acabo de golpear la mano en donde está la herida.}

\entry{tyajol}
\defsuperscript{1}
\partofspeech{adv}
\spanishtranslation{a veces}
\cholexample{Añ ityajol ma'añik mi iweñ mejlel cholel.}
\exampletranslation{A veces la milpa no se da bien.}

\entry{tyajol}
\defsuperscript{2}
\partofspeech{s}
\spanishtranslation{ocotal}

\entry{tyajpuñ}
\partofspeech{vt}
\spanishtranslation{afilar}
\cholexample{Mu'tyo kaj kñaxañ tyajpuñ kmachity.}
\exampletranslation{Primero voy a afilar mi machete.}

\entry{tyal}
\conjugationtense{3ª pers. sing.}
\conjugationverb{tyilel}
\spanishtranslation{viene}
\cholexample{Tyal lakjula'wäle.}
\exampletranslation{Ya viene nuestra visita.}

\entry{tyaloñ}
\conjugationtense{1ª pers. sing.}
\conjugationverb{tyilel}
\spanishtranslation{vengo}

\entry{tyal'a}
\relevantdialect{Tila}
\partofspeech{s}
\spanishtranslation{cura}

\entry{tyam}
\partofspeech{adj}
\onedefinition{1}
\spanishtranslation{largo}
\cholexample{Tyam tsa' iyäk'ä tyi melol iwex.}
\exampletranslation{Mandó a hacer largo su pantalón.}
\onedefinition{2}
\spanishtranslation{hondo}
\cholexample{Tyam iyebal jiñi ja'.}
\exampletranslation{El agua está honda.}

\entry{Tyambäpa'}
\partofspeech{s}
\spanishtranslation{Arroyo Hondo}
\clarification{colonia}

\entry{tyamboril}
\partofspeech{s esp}
\spanishtranslation{tambor}

\entry{tyame ch'ijty}
\partofspeech{s}
\spanishtranslation{tzumi}
\clarification{reg.; árbol}

\entry{*tyamlel}
\partofspeech{s}
\onedefinition{1}
\spanishtranslation{largo}
\cholexample{Ityamlel jiñi otyoty jiäch lujump'ejl metyro.}
\exampletranslation{El largo de la casa es de diez metros.}
\onedefinition{2}
\spanishtranslation{hondo}
\cholexample{Ityamlel jiñi ja' jiñäch uxp'ejl metyro.}
\exampletranslation{El hondo del río es de tres metros.}

\entry{tyañ}
\partofspeech{s}
\spanishtranslation{cal}
\secondaryentry{ityäñil k'ajk}
\secondtranslation{cenizas}

\entry{Tyañija'}
\partofspeech{s}
\spanishtranslation{Agua de Cal}
\clarification{colonia}

\entry{tyañal}
\relevantdialect{Sab.}
\partofspeech{adj}
\spanishtranslation{desnudo}
\cholexample{Tyañal jiñi aläl.}
\exampletranslation{La criatura está desnuda.}

\entry{tyak'iñ}
\partofspeech{s}
\spanishtranslation{moneda, dinero}
\secondaryentry{tsukul tyak'iñ}
\secondtranslation{metal}

\entry{tyak'iñ ch'ijty}
\spanishtranslation{caimito}
\clarification{árbol}

\entry{tyak'iñ pimel}
\spanishtranslation{planta que se tiende en el suelo}
\culturalinformation{Información cultural: Sirve para compresas en heridas. Se dice que cuando se crían tumores en el cuerpo, se puede despedazar en una jícara con agua. Luego se envuelve la herida y el hierbatero lo amarra, y así se cura.}

\entry{tyasa}
\partofspeech{s esp}
\spanishtranslation{pedazo de vidrio}
\cholexample{Tsa' ilowo iyok tyi tyasa.}
\exampletranslation{Se lastimó el pie con vidrio.}

\entry{*tyasil}
\partofspeech{s}
\spanishtranslation{mantel, cobertor, pañal, sudadero}

\entry{tyasiñ}
\partofspeech{vt}
\onedefinition{1}
\spanishtranslation{poner}
\clarification{mantel en la mesa}
\onedefinition{2}
\spanishtranslation{poner}
\clarification{sudadero sobre la espalda de una mula}

\entry{*tyaty}
\partofspeech{s}
\spanishtranslation{padre}

\entry{tyatya}
\partofspeech{s}
\spanishtranslation{papá}

\entry{tyatyäl}
\partofspeech{s}
\spanishtranslation{padre}
\secondaryentry{ityaty}
\secondtranslation{su padre, macho de animales}

\entry{tyatye}
\relevantdialect{Sab.}
\partofspeech{voc}
\spanishtranslation{¡anciano!}
\clarification{saludo a un anciano por un joven}

\entry{tyatyiñ}
\partofspeech{vt}
\spanishtranslation{reconocer como padre}
\cholexample{Mi kaj ktyatyiñ.}
\exampletranslation{Le voy a reconocer como padre.}

\entry{*tatuch}
\partofspeech{s}
\spanishtranslation{abuelo paterno}

\entry{tya'}
\partofspeech{s}
\spanishtranslation{excremento, estiércol, óxido, herrumbre}

\entry{tya'chäb}
\partofspeech{s}
\spanishtranslation{cera}

\entry{tya'ek'}
\partofspeech{s}
\spanishtranslation{meteoro}
\clarification{lit.: excremento de estrella}

\entry{tya'jibäl}
\partofspeech{s}
\spanishtranslation{excusado}
\clarification{lugar en el campo que se usa como excusado}

\entry{tya'jol}
\partofspeech{s}
\spanishtranslation{zopilote}
\clarification{ave}

\entry{tya'ñi'}
\partofspeech{s}
\spanishtranslation{moco}

\entry{tya'ok}
\partofspeech{s}
\spanishtranslation{pantorrilla}

\entry{tya'tya'}
\partofspeech{adj}
\onedefinition{1}
\spanishtranslation{sucio}
\spanishtranslation{manchado}
\cholexample{Tya'tya'tyik iwuty jiñi alob.}
\exampletranslation{Ese chamaco tiene toda la cara sucia.}
\onedefinition{2}
\spanishtranslation{oxidado}
\cholexample{Tya'tya' tyikix jiñi kjuloñib.}
\exampletranslation{Mi escopeta está muy oxidada.}

\entry{*tya' tyokal}
\partofspeech{s}
\spanishtranslation{neblina}

\entry{tyäkäch}
\partofspeech{s}
\spanishtranslation{cosquilla}

\entry{tyäkäl}
\partofspeech{adj}
\spanishtranslation{inclinado}
\cholexample{Tyäkäl añ jiñi postye.}
\exampletranslation{El poste está inclinado.}

\entry{tyäkchuñ}
\partofspeech{vt}
\spanishtranslation{hacer cosquillas}

\entry{tyäklaw}
\partofspeech{adv}
\spanishtranslation{tatarateando}
\cholexample{Tyäklaw mi imajlel jiñi xyäk'ajel.}
\exampletranslation{Va tatarateando ese borracho.}

\entry{tyäk'}
\onedefinition{1}
\partofspeech{vt}
\spanishtranslation{añadir}
\cholexample{Mi kaj kcha' tyäk' yambä juñlajm kotyoty.}
\exampletranslation{Voy a añadir otra división a mi casa.}
\onedefinition{2}
\partofspeech{adj}
\spanishtranslation{pegajoso}
\cholexample{Tyäk' iyetsel jiñi ñi'uk'.}
\exampletranslation{Es pegajosa la trementina del chayote.}

\entry{tyäk'äkña}
\partofspeech{adj}
\spanishtranslation{doloroso}
\cholexample{Tyäk'äkña mi iyubiñ cha'añ ilojwel.}
\exampletranslation{Su herida es muy dolorosa.}

\entry{tyäjk'}
\partofspeech{s}
\spanishtranslation{quemadura aguda}
\cholexample{Tyäjk' tsa' kubi che'bä tsa' ipuluyoñ ñich k'ajk.}
\exampletranslation{Sentí una quemadura aguda cuando me quemó la chispa.}

\entry{-tyäjk'}
\nontranslationdef{Sufijo numeral para contar añadiduras; p. ej.:}
\cholexample{Cha'tyäjk' tsa kotsäbe ilamiñajlel otyoty.}
\exampletranslation{Puse dos añadiduras de lámina en el techo de la casa.}

\entry{tyäjts'el}
\partofspeech{vi}
\spanishtranslation{resbalar}
\cholexample{Mi ityäjts'el lakok che' bojy jiñi bij.}
\exampletranslation{Se resbala el pie por estar lodoso el camino.}

\entry{tyäl}
\partofspeech{vt}
\spanishtranslation{tocar}
\cholexample{Mach atyäl awuty che' bibi' ak'äb.}
\exampletranslation{No toques tus ojos cuando están sucias tus manos.}

\entry{-tyäl}
\nontranslationdef{Sufijo que se presenta con raíces atributivas para formar otra raíz atributiva que indica cantidad; p. ej.:}
\cholexample{jomtyäl}
\exampletranslation{mucha (gente).}

\entry{tyälel}
\relevantdialect{Sab.}
\partofspeech{vi}
\spanishtranslation{venir}
\alsosee{tyilel}

\entry{tyäñäl k'ajk}
\spanishtranslation{ceniza}

\entry{tyäñäm}
\partofspeech{s}
\spanishtranslation{algodón}
\alsosee{tyiñäm}

\entry{tyäñäme'}
\partofspeech{s}
\spanishtranslation{oveja}
\clarification{mamífero}

\entry{tyäp'leñ}
\partofspeech{vt}
\spanishtranslation{seguir}
\cholexample{Yom mi atyäp'leñ majlel jiñi wiñikob tyi bij.}
\exampletranslation{Debes ir siguiendo a los hombres en el camino.}

\entry{tyäkiñ}
\relevantdialect{Sab.}
\partofspeech{adj}
\spanishtranslation{seco}
\alsosee{tyikiñ}

\entry{tyäs}
\partofspeech{vt}
\spanishtranslation{tender}
\cholexample{Mi ityäs iwäyib tyi lum.}
\exampletranslation{Tiende su cama en el suelo.}

\entry{tyäsäkña}
\partofspeech{adj}
\spanishtranslation{nivelado}
\clarification{terreno}
\cholexample{Tyäsäkña jiñi lum tyi pam iyotyoty.}
\exampletranslation{Está nivelada la tierra en el patio de su casa.}

\entry{tyätyäk}
\partofspeech{adv}
\spanishtranslation{tambaleante}
\cholexample{Tsa' majli ityätyäk k'el.}
\exampletranslation{Tambaleante se fue a verlo.}

\entry{tyäts'}
\partofspeech{vt}
\spanishtranslation{retirar}
\cholexample{Yom mi atyäts' abäba'añ jiñi xyäk'äjelob.}
\exampletranslation{Retírese de los borrachos.}

\entry{tyäwäl}
\partofspeech{adj}
\spanishtranslation{pobre}
\clarification{da lástima porque va a ser castigado}
\cholexample{Tyäwäl jiñi alob, ¿chukoch tsa' ijats'ä ipi'äl?}
\exampletranslation{¡Pobre del niño!; ¿Por qué le pegó a su compañero?}

\entry{tyek'}
\partofspeech{vt}
\spanishtranslation{pisar}
\cholexample{Mi laktyek' ch'ix che' ma'añik lakwarach.}
\exampletranslation{Cuando no tenemos huaraches pisamos las espinas.}

\entry{tyek'jol}
\partofspeech{s}
\spanishtranslation{caballete}
\clarification{de casa}

\entry{tyek'oñib}
\partofspeech{s}
\onedefinition{1}
\spanishtranslation{palo recortado para subir}
\onedefinition{2}
\spanishtranslation{escalera}

\entry{tyech}
\partofspeech{vt}
\onedefinition{1}
\spanishtranslation{empezar}
\cholexample{Ijk'äl mik tyech kchobal.}
\exampletranslation{Mañana empezaré a rozar.}
\onedefinition{2}
\spanishtranslation{levantar}
\cholexample{Woli ityech iwäyib.}
\exampletranslation{Está levantando su cama.}

\entry{tyech ty'añ}
\relevantdialect{Sab.}
\partofspeech{vt}
\spanishtranslation{buscar pleito}
\cholexample{Jiñi aj'e'tyelob pätyuñ tyech ty'añ yomob.}
\exampletranslation{Las autoridades andan buscando pleitos.}

\entry{-tyejk}
\nontranslationdef{Sufijo numeral para contar árboles; p. ej.:}
\cholexample{Tyij kajpelel añ jo'tyejk alaxax.}
\exampletranslation{En mi cafetal tengo cinco matas de naranjas.}

\entry{tyejklum}
\partofspeech{s}
\spanishtranslation{pueblo}
\cholexample{Tsa' majliyob tyi tyejklum.}
\exampletranslation{Se fueron al pueblo.}
\dialectvariant{Sab.}
\dialectword{lum}

\entry{tyejchel}
\partofspeech{vi}
\onedefinition{1}
\spanishtranslation{levantarse}
\onedefinition{2}
\spanishtranslation{despertarse}

\entry{*tyejchibal}
\partofspeech{s}
\spanishtranslation{comienzo, principio}
\cholexample{Ili ora ityejchibaltyo tyuk' kajpe'.}
\exampletranslation{Ahora apenas es el comienzo del corte de café.}

\entry{*tyejchilañ}
\relevantdialect{Sab.}
\partofspeech{s}
\spanishtranslation{principio}
\alsosee{*yajñelañ}

\entry{-tyejm}
\nontranslationdef{Sufijo numeral para contar rollos; p. ej.:}
\cholexample{Yom mi ach'äm tyilel jumtyejm si'.}
\exampletranslation{Debes traer un rollo de leña.}

\entry{tyejñañ}
\partofspeech{vt}
\spanishtranslation{afirmar}
\clarification{la tierra}

\entry{tyem}
\partofspeech{s}
\spanishtranslation{banca}

\entry{*tyem cha'añ}
\spanishtranslation{colectivo}
\cholexample{Jiñi kajpe'lel ityem cha'añ wiñikob.}
\exampletranslation{Este es un cafetal colectivo.}

\entry{tyemel}
\partofspeech{adv}
\spanishtranslation{junto}
\cholexample{Tyemel tsa' majliyob.}
\exampletranslation{Se fueron juntos.}

\entry{tyempañ}
\partofspeech{vt}
\spanishtranslation{juntar}
\cholexample{Wolik tyempañ ktyak'iñ cha'añ mik mel kotyoty.}
\exampletranslation{Estoy juntando dinero para hacer mi casa.}

\entry{tyempäbil}
\partofspeech{adj}
\spanishtranslation{recogido}
\cholexample{Tyempäbil jiñi xajlel cha'añ mi imejlel otyoty.}
\exampletranslation{Esas piedras fueron recogidas para hacer una casa.}

\entry{tyempäyel}
\partofspeech{vi}
\spanishtranslation{juntarse}
\cholexample{Mi kaj tyi tyempäyel ibäl ñäk'äl cha'añ mi ik'uxob xk'iñijelob.}
\exampletranslation{Se van a reunir víveres para que coman los que vienen a la fiesta.}

\entry{tyeñ}
\partofspeech{vt}
\spanishtranslation{aplastar}
\cholexample{Tsa' ujtyi ktyeñ tsuk tyi xajlel.}
\exampletranslation{Aplasté un ratón con una piedra.}

\entry{tyeñkech}
\partofspeech{s}
\spanishtranslation{hongo}
\secondaryentry{ityeñkechlel tye'}
\secondtranslation{hongo que crece en los árboles}

\entry{tyeñtsuñ}
\partofspeech{s}
\spanishtranslation{diablo, fantasma}
\culturalinformation{Información cultural: Se dice que este espíritu malo es el compañero del brujo. Se convierte en chivo. No tiene ojos, pero tiene cuatro patas. Se cree que es de un metro de altura, que apesta, que no se puede matar y que produce enfermedades.}

\entry{tyeñe}
\partofspeech{adv}
\spanishtranslation{a menudo}
\cholexample{Mi ityeñe tyilel ja'al.}
\exampletranslation{Llueve a menudo.}

\entry{tyep'}
\partofspeech{vt}
\spanishtranslation{envolver}
\cholexample{Mi laktyep' tyi juñ jiñi ats'am.}
\exampletranslation{Envolvemos la sal con papel.}

\entry{tyep'bil}
\partofspeech{adj}
\spanishtranslation{envuelto}
\cholexample{Tyep'bil jiñi juñk'ojl sa' yik'oty yopom.}
\exampletranslation{El pozol está envuelto en una hoja.}

\entry{tyekel}
\partofspeech{adj}
\spanishtranslation{parado}
\cholexample{Juñtyejk tye' ya' tyekel tyi yojlil kchol.}
\exampletranslation{Un árbol está (parado) en medio de mi milpa.}

\entry{-tyes}
\conjugationtense{variante}
\conjugationverb{-es-}
\nontranslationdef{Sufijo que se presenta con raíces transitivas y atributivas para formar una raíz causativa: p. ej.:}
\cholexample{käñtyesañ}
\exampletranslation{enseñar.}

\entry{tyestyiku}
\partofspeech{s esp}
\spanishtranslation{testigo}

\entry{tyexelex}
\partofspeech{s esp}
\spanishtranslation{tijeras}

\entry{tye'}
\partofspeech{s}
\spanishtranslation{árbol, palo, madera}

\entry{tye' bu'ul}
\partofspeech{s}
\spanishtranslation{varita para que suba el frijol}

\entry{tye'el}
\partofspeech{s}
\spanishtranslation{bosque}
\clarification{pequeño}
\dialectvariant{Sab.}
\dialectword{ñajtye'el}
\secondaryentry{itye'el kajpe'}
\secondpartofspeech{s}
\secondtranslation{mata de café}
\secondaryentry{tye'e pimel}
\secondpartofspeech{s}
\secondtranslation{acahual}

\entry{tye'jam}
\partofspeech{s}
\spanishtranslation{zacate duro como caña, pajón}

\entry{tye'lal}
\partofspeech{s}
\spanishtranslation{agutí}
\clarification{mamífero}

\entry{tye'lemuty}
\partofspeech{s}
\spanishtranslation{pájaro}

\entry{tye'ok}
\partofspeech{s}
\spanishtranslation{pierna de palo}

\entry{tyi}
\nontranslationdef{palabra de enlace}

\entry{tyik}
\partofspeech{vt}
\spanishtranslation{desatar}
\cholexample{Mi kaj ktyik kawayu'.}
\exampletranslation{Voy a desatar el caballo.}

\entry{-tyik}
\nontranslationdef{Sufijo que se presenta con raíces atributivas para formar otra raíz atributiva que indica cantidad o grado; p. ej.:}
\cholexample{kisiñtyik}
\exampletranslation{vergonzoso.}

\entry{tyikäw}
\partofspeech{adj}
\spanishtranslation{caliente}
\clarification{muy}
\cholexample{Tyikäw ja' tsa' ik'äñä ipuk' isa'.}
\exampletranslation{Usó agua caliente para batir su pozol.}

\entry{tyikäw pañimil}
\spanishtranslation{hace calor}

\entry{-tyiklel}
\nontranslationdef{Sufijo numeral para contar personas con números ordinales; p. ej.:}
\cholexample{Tsa' k'otyi icha'tyiklelbä.}
\exampletranslation{Llegó la segunda persona.}

\entry{tyikwal}
\partofspeech{s}
\spanishtranslation{calor}
\cholexample{Kabäl tyikwal.}
\exampletranslation{Hace mucho calor.}

\entry{tyikwak'iñ}
\partofspeech{s}
\spanishtranslation{reyezuelo}
\clarification{pajarito}

\entry{tyikwälel}
\partofspeech{s}
\spanishtranslation{tierra caliente}
\cholexample{Tyikwälel yabaki samiyoñ.}
\exampletranslation{A donde voy es tierra caliente.}

\entry{tyikwesañ}
\partofspeech{vt}
\spanishtranslation{calentar}
\clarification{mucho}
\cholexample{Mi kaj ktyikwesañ ja'.}
\exampletranslation{Voy a calentar agua.}

\entry{tyik'}
\partofspeech{vt}
\spanishtranslation{prohibir}
\cholexample{Jiñi forestyal mi ityik' cha'añ mach mi laksek' tye'.}
\exampletranslation{La forestal prohíbe que tumbemos árboles.}

\entry{tyik'añ}
\partofspeech{vi}
\spanishtranslation{cocer}
\cholexample{Wolix ityik'añ jiñi bu'ul.}
\exampletranslation{Ya se está cociendo el frijol.}

\entry{tyik'äjib}
\partofspeech{s}
\spanishtranslation{lumbre, estufa}

\entry{tyik'äl}
\partofspeech{adv}
\spanishtranslation{posiblemente, tal vez}
\cholexample{Tyik'äl ijk'äl mik majlel tyi tyejklum.}
\exampletranslation{Posiblemente mañana vaya (yo) al pueblo.}

\entry{tyik'lañ}
\partofspeech{vt}
\spanishtranslation{molestar}
\cholexample{Mach atyik'lañ awijts'iñ.}
\exampletranslation{No molestes a tu hermanito.}

\entry{*tyik'ol}
\partofspeech{s}
\spanishtranslation{amonestación}
\cholexample{Jiñi tyatyäl mi iyäk'eñ ityik'ol iyalobil.}
\exampletranslation{El padre amonesta (lit.: da amonestación) a su hijo}

\entry{tyik'oñel}
\partofspeech{s}
\spanishtranslation{prohibición}
\cholexample{Woli tyi tyik'oñel jiñi wiñik.}
\exampletranslation{Ese hombre está anunciando una prohibición.}

\entry{tyich'}
\partofspeech{vt}
\spanishtranslation{tender}
\cholexample{Yom mi atyich' amosil.}
\exampletranslation{Debes tender su cobija.}

\entry{tyich'ikña}
\partofspeech{adv}
\spanishtranslation{con la mano extendida}
\cholexample{Tyich'ikña ik'äb jiñi wiñik woli ik'ajtyiñ tyak'iñ.}
\exampletranslation{El hombre está pidiendo dinero con la mano extendida.}

\entry{tyich'il}
\partofspeech{adj}
\spanishtranslation{extendido}

\entry{*tyiempojlel}
\relevantdialect{Sab.}
\partofspeech{s esp}
\spanishtranslation{hora, tiempo}
\cholexample{Tsa'ix k'otyi ityiempojlel chobal.}
\exampletranslation{Ya llegó el tiempo de rozar.}

\entry{-tyijañ}
\nontranslationdef{Sufijo que se presenta con raíces adjetivas que indican color y se refiere a puntas de pelo.}

\entry{tyijkañ}
\partofspeech{vt}
\spanishtranslation{sacudir}
\cholexample{Woli ityijkañ its'ubeñal ibujk.}
\exampletranslation{Está sacudiendo el polvo de su camisa.}

\entry{tyijikña}
\partofspeech{adj}
\spanishtranslation{contento, satisfecho}
\cholexample{Pejtyelel ora añ ityijikñäyel ipusik'al.}
\exampletranslation{Siempre está contento.}

\entry{tyijil}
\partofspeech{s}
\spanishtranslation{rocío, sereno}

\entry{tyijlel}
\partofspeech{vi}
\spanishtranslation{deshilacharse}

\entry{-tyijp'}
\nontranslationdef{Sufijo numeral para contar brincos.}

\entry{tyijp'ejl}
\partofspeech{vi}
\spanishtranslation{brincar}
\cholexample{Chañ mi ityijp'el jiñi alob.}
\exampletranslation{Ese chamaco brinca alto.}

\entry{tyijtyiñ}
\partofspeech{vt}
\spanishtranslation{sacudir}
\cholexample{Tyijtyiñ jiñi tsuts cha'añ mi ilok'el ityus'ubeñal.}
\exampletranslation{Sacude la cobija para quitarle el polvo.}

\entry{tyil}
\partofspeech{adv}
\spanishtranslation{desatar}
\clarification{a manera de}
\cholexample{Tsa' ityil jityi ich'ajñal ab.}
\exampletranslation{Desató la soga de la hamaca.}

\entry{tyilel}
\partofspeech{vi}
\spanishtranslation{venir}
\cholexample{Mi kaj ityilel lakmaestyro.}
\exampletranslation{Viene nuestro maestro.}
\conjugationtense{3ª pers. sing. pret.:}
\otherconjugation{tsajni}
\alsosee{Gram. 6.18}
\dialectvariant{Sab.}
\dialectword{tyälel}

\entry{tyilel bixel lak pusik'al}
\spanishtranslation{inconstante}
\cholexample{Mach yomik tyilel bixel lakpusik'al.}
\exampletranslation{No debemos ser inconstantes.}

\entry{*tyilib}
\partofspeech{s}
\spanishtranslation{tubería, canoa}
\cholexample{Muk'ix ikajel iyots'äñtyel ityilib ja'.}
\exampletranslation{Ya van a poner la tubería para el agua.}

\entry{tyiñäm}
\partofspeech{s}
\onedefinition{1}
\nontranslationdef{Fibra dentro de las capas de plátanos que sirve para mechas e hilos.}
\onedefinition{2}
\spanishtranslation{algodón}
\alsosee{tyäñäm}

\entry{Tyiñeral}
\partofspeech{s esp}
\spanishtranslation{Trinidad}
\clarification{colonia}

\entry{tyiñtyäl}
\relevantdialect{Sab.}
\partofspeech{vi}
\spanishtranslation{agacharse, inclinarse}

\entry{tyi ora}
\partofspeech{adv}
\spanishtranslation{luego, de inmediato}
\cholexample{Tyi ora mi icha' sujtyel.}
\exampletranslation{Regresas de inmediato.}
\dialectvariant{Tila}
\dialectword{ñux;}
\dialectvariant{Sab.}
\dialectword{ora}

\entry{tyip'tyip'ña}
\partofspeech{adj}
\onedefinition{1}
\spanishtranslation{palpitando}
\cholexample{Pejtyelel ora tyip'tyip'ña kpusik'al.}
\exampletranslation{Mi corazón siempre está palpitando.}
\onedefinition{2}
\spanishtranslation{rebotando}
\cholexample{Tyip'tyip'ña mi imajlel pelotya.}
\exampletranslation{La pelota va rebotando.}

\entry{tyikesañ}
\partofspeech{vt}
\spanishtranslation{secar}
\cholexample{Tsa'ix ujtyi ktyikesañ jkajpe'.}
\exampletranslation{Ya terminé de secar mi café.}

\entry{-tyikil}
\nontranslationdef{Sufijo numeral para contar personas; p. ej.:}
\cholexample{Che' añ jo'tyikil lakpi'äl tyi chobal mi iñuk'añ jiñi lakchol.}
\exampletranslation{Cuando hay cinco compañeros que nos ayudan a rozar, se agranda la rozadura.}

\entry{tyikiñ}
\partofspeech{adj}
\spanishtranslation{seco}
\clarification{camino, madera, café, arroyo}
\dialectvariant{Sab.}
\dialectword{tyäkiñ}

\entry{tyikiñjopol}
\partofspeech{adj}
\spanishtranslation{sin caldo}
\cholexample{Tyikiñjopol jiñi bu'ul.}
\exampletranslation{El frijol ya está sin caldo.}

\entry{tyis}
\partofspeech{s}
\spanishtranslation{pedo}

\entry{tyiskok}
\partofspeech{s}
\spanishtranslation{una planta medicinal}
\culturalinformation{Información cultural: Crece en las piedras; y se usa para las heridas.}

\entry{tyity}
\partofspeech{adj}
\onedefinition{1}
\spanishtranslation{espeso}
\cholexample{Tyity mi ijap isa'.}
\exampletranslation{Toma su pozol espeso.}
\onedefinition{2}
\spanishtranslation{sucio}
\clarification{líquido}
\cholexample{Tyity jiñi ja'.}
\exampletranslation{El agua está sucia.}

\entry{tyi'}
\partofspeech{s}
\spanishtranslation{orilla}
\cholexample{Mik pijtyañety ya' tyi' tyi' bij.}
\exampletranslation{Te espero a la orilla del camino.}
\secondaryentry{ityi'il}
\secondpartofspeech{s}
\secondtranslation{su orilla}

\entry{tyok}
\partofspeech{vt}
\spanishtranslation{picar}
\clarification{tierra}
\cholexample{Woliktyok lum.}
\exampletranslation{Estoy picando la tierra.}

\entry{tyokal}
\partofspeech{s}
\spanishtranslation{nube}
\secondaryentry{tyokal wuty}
\secondtranslation{ciego con los ojos abiertos}

\entry{tyokbil}
\partofspeech{adj}
\spanishtranslation{hollado}
\cholexample{Tyokbil jiñi lumba' mi its'äjpel jiñi oy.}
\exampletranslation{La tierra donde se siembra el horcón está hollada.}

\entry{Tyokob}
\partofspeech{s}
\onedefinition{1}
\spanishtranslation{nombre de una colonia}
\onedefinition{2}
\spanishtranslation{árbol}
\clarification{un tipo cuya fruta se come el tepescuintle}

\entry{tyokol}
\partofspeech{adj}
\onedefinition{1}
\spanishtranslation{abierto}
\cholexample{Tyokol ityi' otyoty.}
\exampletranslation{La entrada de la casa está abierta.}
\onedefinition{2}
\spanishtranslation{roto}
\cholexample{Tyokol jiñi koxtyal.}
\exampletranslation{El costal está roto.}
\secondaryentry{tyokolbä lum}
\secondtranslation{grieta}

\entry{Tyokol ja'}
\spanishtranslation{Vertiente de la Apertura}
\clarification{nombre de lugar}

\entry{tyoktyäl}
\partofspeech{adv}
\spanishtranslation{así de agujerado}

\entry{tyoktyok}
\partofspeech{s}
\spanishtranslation{un pájaro}
\clarification{amarillo en la parte de abajo}

\entry{tyok'}
\partofspeech{s}
\spanishtranslation{piedra de chispa}
\clarification{pedernal}
\secondaryentry{tyok'bä tyuñ}
\secondtranslation{piedras chicas que sacan chispas}

\entry{tyoj}
\onedefinition{1}
\partofspeech{adj}
\spanishtranslation{recto}
\cholexample{Weñtyoj jiñi tye'.}
\exampletranslation{El palo es muy recto.}
\onedefinition{2}
\partofspeech{adj}
\spanishtranslation{justo}
\cholexample{Ma'añik mi juñjtyikiltyojbä wiñik.}
\exampletranslation{No hay ni un hombre justo.}
\onedefinition{3}
\partofspeech{vt}
\spanishtranslation{pagar}
\cholexample{Tsa' majli ityoj ibety.}
\exampletranslation{Fue a pagar su cuenta.}

\entry{tyojbiñ}
\partofspeech{vt}
\spanishtranslation{apuntar}
\cholexample{Tyoj mi laktyojbiñ jiñibä'tye'el che' mi lakjul.}
\exampletranslation{Al tirarle al animal apuntamos derecho.}

\entry{-tyojk}
\nontranslationdef{Sufijo numeral para contar rendijas; p. ej.:}
\cholexample{Ya' tyi pam jiñi kotyoty añ cha'tyojk jiñi lum.}
\exampletranslation{En el patio de mi casa hay dos rendijas.}

\entry{tyojklel}
\partofspeech{s}
\spanishtranslation{quemazón de monte, incendio}

\entry{tyojk'ay}
\partofspeech{s}
\spanishtranslation{cuiche, codorniz común}
\clarification{ave}

\entry{tyojmel}
\partofspeech{vi}
\onedefinition{1}
\spanishtranslation{tronar}
\cholexample{Woli tyityojmel chajk.}
\exampletranslation{El rayo está tronando.}
\onedefinition{2}
\spanishtranslation{explotar}
\cholexample{Tsa'tyojmi diñamitya tyi bij.}
\exampletranslation{La dinamita explotó en el camino.}
\onedefinition{3}
\spanishtranslation{reventar}
\cholexample{Mi kaj tyityojmel ichäkajl.}
\exampletranslation{Se le va a reventar el tumor.}

\entry{tyojmesañ}
\partofspeech{vt}
\onedefinition{1}
\spanishtranslation{disparar}
\clarification{arma}
\onedefinition{2}
\spanishtranslation{tronar}
\clarification{cohete}

\entry{tyojmulil}
\partofspeech{s}
\spanishtranslation{castigo, infierno}
\clarification{lit.: pagar pecado}

\entry{tyojokña}
\partofspeech{adv}
\nontranslationdef{Manera en que vienen las nubes; p. ej.:}
\cholexample{Tyojokña tyilel ja'al ya' tyi wits.}
\exampletranslation{Viene un montón de nubes con agua.}

\entry{*tyojol}
\partofspeech{s}
\spanishtranslation{precio, valor, paga}
\cholexample{Ma'añik tsa' isubu jayk'al añ ityojol kajpe'.}
\exampletranslation{No dijo qué precio tiene el café.}

\entry{tyojolañ}
\partofspeech{vt}
\spanishtranslation{pagar}
\cholexample{Mach kajik ilajtyojolañ.}
\exampletranslation{No alcanzará con eso para pagarlo.}

\entry{tyojoñel}
\partofspeech{s}
\spanishtranslation{impuesto}
\cholexample{Mi kaj ik'ajtyiñtyojoñel.}
\exampletranslation{Va a pedir impuestos.}

\entry{tyojsiñ}
\partofspeech{vt}
\spanishtranslation{remoler}
\cholexample{Woli'tyojsiñ iwaj tyi ña'atyuñ.}
\exampletranslation{Está remoliendo su nixtamal en el metate.}

\entry{tyojty}
\partofspeech{s}
\spanishtranslation{calandria}
\clarification{ave}

\entry{tyojtyojil}
\partofspeech{vi}
\spanishtranslation{agotarse}
\cholexample{Tyojtyojil jach iyixim jujump'ejl k'iñ.}
\exampletranslation{Cada día se le agota su maíz.}

\entry{tyojtyoñ}
\partofspeech{vt}
\spanishtranslation{sacudir}
\clarification{tierra de una canasta}

\entry{tyoj'esañ}
\partofspeech{vt}
\onedefinition{1}
\spanishtranslation{enderezar}
\cholexample{Mi kaj ktyoj'esañ ikuch jiñi mula.}
\exampletranslation{Voy a enderezar la carga de esa mula.}
\onedefinition{2}
\spanishtranslation{guiar}
\cholexample{Ma'añik majch mi'tyoj'esañ majlel tyi bij.}
\exampletranslation{No hay quien lo guíe en el camino.}

\entry{tyoj'esaty'añ}
\partofspeech{s}
\spanishtranslation{corrección}
\clarification{de acta o carta}
\cholexample{Jiñi sekretyario mi icha'leñtyoj'esaty'añ.}
\exampletranslation{El secretario hace correcciones.}
\secondaryentry{xtyojesaty'añ}
\secondpartofspeech{s}
\spanishtranslation{corregidor}

\entry{*tyoj'ijib}
\partofspeech{s}
\spanishtranslation{enseñanza, instrucción}

\entry{tyom}
\partofspeech{s}
\spanishtranslation{raíz de plátano}
\culturalinformation{Información cultural: Se les da a comer a los pavos, pollos y cerdos.}

\entry{*tyomel}
\partofspeech{s}
\onedefinition{1}
\spanishtranslation{muslo}
\cholexample{K'ux ityomel iya'.}
\exampletranslation{Le duelen los muslos.}
\onedefinition{2}
\spanishtranslation{raíz}
\cholexample{Ñukix ityomel jiñi juk'.}
\exampletranslation{La raíz del quequeste es grande.}
\secondaryentry{*tyomel lakya'}
\secondtranslation{muslo}
\secondaryentry{*tyomel laj k'äb}
\secondtranslation{músculo del brazo}

\entry{tyomokña}
\partofspeech{adv}
\nontranslationdef{Se relaciona con la forma redonda; p. ej.:}
\cholexample{Tyomokña woli iletsel buts'.}
\exampletranslation{El humo está subiendo en forma de bola.}

\entry{tyomtye'}
\partofspeech{s}
\spanishtranslation{árbol}
\clarification{de madera negra que sirve para muebles}

\entry{tyomtyom}
\partofspeech{adv}
\nontranslationdef{Se relaciona con la manera de disparar repetidas veces sin apuntar; p. ej.:}
\cholexample{Tsi'tyomtyom julu majlel.}
\exampletranslation{Disparó repetidamente sin apuntar.}

\entry{tyoñkots}
\partofspeech{adj}
\spanishtranslation{atrasado, ignorante}
\cholexample{Tyoñkots che'bä tsa' iyilapañimil jiñi wiñik.}
\exampletranslation{Ese hombre es atrasado desde que nació.}

\entry{tyoñtyek'}
\partofspeech{vt}
\spanishtranslation{dar patada}
\cholexample{Tyi'tyoñtyek'eyoñ jiñi kabawu'.}
\exampletranslation{El caballo me dio de patadas.}

\entry{tyoñtye'}
\partofspeech{s}
\spanishtranslation{nogal}
\clarification{árbol de madera negra y maciza}

\entry{*tyoñtyojlel}
\partofspeech{s}
\onedefinition{1}
\spanishtranslation{estupidez}
\cholexample{Cha'añ ityoñtyojlel ma'añik tsa' mejli iweñ cha'leñ iye'tyel.}
\exampletranslation{Por su estupidez no puede hacer bien su trabajo.}
\onedefinition{2}
\spanishtranslation{mala vida}
\cholexample{Tsa' tyejchi wokol tyi' yotyoty cha'añ ityoñtyojlel.}
\exampletranslation{Hubo dificultades en su casa por su mala vida.}

\entry{tyoñel}
\relevantdialect{Tila}
\partofspeech{s}
\spanishtranslation{trabajo pequeño}
\alsosee{e'tyel}

\entry{tyop'}
\partofspeech{vt}
\spanishtranslation{quebrar}
\clarification{piedra, tierra, vidrio}
\cholexample{Woli'tyop' lumba' mi kaj imel ipäk'äb.}
\exampletranslation{Está quebrando la tierra donde va a hacer su hortaliza.}

\entry{tyok'e}
\relevantdialect{Tila}
\partofspeech{vi}
\spanishtranslation{poder}
\cholexample{Mu'tyotyok'e tyi e'tyel jiñi ñox.}
\exampletranslation{Ese anciano todavía puede trabajar.}
\alsosee{mejlel}

\entry{Tyok'ipa'}
\partofspeech{s}
\spanishtranslation{Piedras Chicas en el Arroyo}
\clarification{colonia}

\entry{tyots'chokoñ}
\partofspeech{vt}
\spanishtranslation{acostar}
\cholexample{Tyots'chokoñ jiñi aläl.}
\exampletranslation{Acuesta a la criatura en la cama.}

\entry{tyots'ol}
\partofspeech{adj}
\spanishtranslation{acostado}
\clarification{boca arriba}
\cholexample{Tyots'ol aläl tyi ab.}
\exampletranslation{La criatura está acostada boca arriba en la hamaca.}

\entry{tyots'tyäl}
\partofspeech{vi}
\spanishtranslation{acostarse}
\cholexample{Yom ma'tyots'tyäl ya' tyi wäyibäl.}
\exampletranslation{Acuéstate en la cama.}

\entry{tyow}
\partofspeech{s}
\spanishtranslation{silbo}
\clarification{producido con los dedos en la boca}

\entry{tyowiñ}
\partofspeech{vt}
\spanishtranslation{silbar}
\clarification{con los dedos en la boca}
\cholexample{Mi kaj ktyowiñ cha'añ mi ityilel.}
\exampletranslation{Voy a silbar para que venga.}

\entry{tyoy}
\partofspeech{vt}
\spanishtranslation{contagiar}
\cholexample{Tsa' ujtyi ityoy ik'amäjel.}
\exampletranslation{Acaba de contagiarse de la enfermedad.}

\entry{tyoyol}
\partofspeech{adj}
\spanishtranslation{pegado}
\cholexample{Jiñi juñ ya' tyi pajk'tyoyol tyi tyachäb.}
\exampletranslation{El papel que está pegado en la pared está pegado con cera.}

\entry{tyo'ña'}
\partofspeech{s}
\spanishtranslation{un tipo de oruga}

\entry{tyo'ol jach}
\spanishtranslation{en vano}
\cholexample{Tyo'ol jach tsa' imele iyotyoty mi ma'añik woli ik'äñ.}
\exampletranslation{En vano hizo su casa si no la está usando.}

\entry{tyratyo}
\partofspeech{s esp}
\onedefinition{1}
\spanishtranslation{contrato}
\cholexample{Tsa' imele tyratyo yik'oty iyum fiñka cha'añ mi imämbeñ ilum.}
\exampletranslation{Hizo un contrato con el dueño de la finca para comprar su terreno.}
\onedefinition{2}
\spanishtranslation{acuerdo}
\cholexample{Tsa' ich'ämäyob tyratyo ya' tyi juñtya.}
\exampletranslation{Tomaron un acuerdo en la junta.}

\entry{tyremeñtyiñatye'}
\relevantdialect{Tila}
\partofspeech{s}
\spanishtranslation{aguacatillo}
\clarification{árbol}
\alsosee{xpomtye'}

\entry{tyreñkipal}
\partofspeech{s esp}
\spanishtranslation{principal}
\cholexample{Jiñi tyreñkipal mi icha'leñ mañdar ya tyi melo'bäjäl.}
\exampletranslation{El principal da órdenes allá en el juzgado.}

\entry{tyroñel}
\relevantdialect{Tila}
\partofspeech{s}
\spanishtranslation{trabajo grande}
\alsosee{e'tyel}

\entry{tyroñeläl}
\relevantdialect{Sab.}
\spanishtranslation{s cosecha}

\entry{tyuk}
\partofspeech{s}
\spanishtranslation{chapay amargo sin espina}
\clarification{vegetal comestible de una palmera}

\entry{tyukñäk'}
\partofspeech{s}
\spanishtranslation{diarrea}

\entry{tyuksiñ}
\partofspeech{vt}
\spanishtranslation{escarbar}
\cholexample{Kabäl tsa' ujtyi ityuksiñ lum jiñi chityam.}
\exampletranslation{El cerdo escarbó mucha tierra.}

\entry{tyuk'}
\partofspeech{vt}
\spanishtranslation{cortar}
\clarification{fruta}
\cholexample{Mu'tyo kaj ktyuk' jkajpe'.}
\exampletranslation{Todavía voy a cortar mi café.}

\entry{Tyuk'oñichim}
\relevantdialect{Tila}
\partofspeech{s}
\spanishtranslation{Lugar Donde se Cortan Flores}
\clarification{colonia}

\entry{tyuk'o' bij}
\partofspeech{s}
\spanishtranslation{vereda}

\entry{tyuk'ul}
\partofspeech{s}
\spanishtranslation{patzagua, guamúchil}
\clarification{árbol}

\entry{tyuk'uy}
\partofspeech{s}
\spanishtranslation{árbol de espinas venenosas}
\culturalinformation{Información cultural: Produce hinchazón y calentura; se encuentra en tierra caliente.}

\entry{tyuch pajch'}
\partofspeech{s}
\spanishtranslation{piña chica}

\entry{tyuchul}
\partofspeech{adj}
\spanishtranslation{nudoso}
\clarification{cara o rama}
\cholexample{Tyuchul ik'äb tye'.}
\exampletranslation{La rama está nudosa.}

\entry{tyuch'}
\partofspeech{vt}
\spanishtranslation{señalar}
\cholexample{Woli ityuch' yik'oty iyal ik'äb.}
\exampletranslation{Está señalando con su dedo.}

\entry{tyuch'beñ}
\partofspeech{vt}
\spanishtranslation{señalar}

\entry{*tyuch'oñib}
\partofspeech{s}
\spanishtranslation{dedo índice}

\entry{tyuch'k'iñ}
\partofspeech{s}
\spanishtranslation{campamocha}
\clarification{insecto}

\entry{*tyujb}
\partofspeech{s}
\spanishtranslation{saliva}

\entry{tyujbañ}
\partofspeech{vt}
\spanishtranslation{escupir}
\cholexample{Woli ityujbañ iyijts'iñ.}
\exampletranslation{Está escupiendo a su hermanito.}

\entry{tyujk'añ}
\partofspeech{vt}
\spanishtranslation{jalar}
\cholexample{Mi kaj ktyujk'añ majlel kmula.}
\exampletranslation{Me voy a llevar a la mula jalando.}

\entry{tyujlañ}
\partofspeech{vt}
\spanishtranslation{forzar para soltarse}
\cholexample{Mi laktyujlañ lakbä cha'añ mi ikoloñla.}
\exampletranslation{Nos forzamos para que nos suelte.}

\entry{tyujlux}
\partofspeech{s}
\spanishtranslation{libélula}
\spanishtranslation{caballito del diablo}
\clarification{insecto}

\entry{tyujk'el}
\partofspeech{vi}
\spanishtranslation{reventarse}
\cholexample{Mux ityujk'el ktyajbal.}
\exampletranslation{Ya se va a reventar mi mecapal.}

\entry{tyujtyuñ}
\partofspeech{vt}
\spanishtranslation{desplumar}
\cholexample{Mi laktyujtyuñ lok'el itsutsel xña'muty che' mi lakchajpañ.}
\exampletranslation{Desplumamos la gallina cuando la preparamos.}

\entry{tyujts'}
\partofspeech{s}
\spanishtranslation{rana}
\clarification{reptil}

\entry{tyul}
\partofspeech{adv}
\spanishtranslation{repentinamente}
\cholexample{Tsäts tsa' ityul ye'e ik'äb.}
\exampletranslation{Le agarró la mano fuerte y repentinamente.}

\entry{Tyulija'}
\partofspeech{s}
\spanishtranslation{nombre de un río}

\entry{*tyul mal tye'}
\spanishtranslation{corazón del árbol}

\entry{tyul mek'}
\partofspeech{vt}
\spanishtranslation{abrazar}
\cholexample{Jiñi kaxlañob mi ityul mek'ob ibä.}
\exampletranslation{Los que no son indígenas se abrazan.}

\entry{tyulul}
\partofspeech{adj}
\spanishtranslation{fijos}
\clarification{ojos}
\cholexample{Tyulul iwuty mi ik'eloñlawakax.}
\exampletranslation{El ganado nos está mirando con los ojos fijos.}

\entry{tyulum}
\partofspeech{s}
\onedefinition{1}
\spanishtranslation{corazón de árbol}
\onedefinition{2}
\spanishtranslation{chicle}
\clarification{árbol}

\entry{Tyumbalá}
\partofspeech{s}
\spanishtranslation{Lugar de la Piedra Aquí}
\clarification{pueblo}

\entry{tyumbeñ}
\partofspeech{vt}
\spanishtranslation{hacer}
\cholexample{¿chuki woli atyumbeñ awerañ?}
\exampletranslation{¿Qué le estás haciendo a tu hermano?}

\entry{tyumbiñ}
\relevantdialect{Sab.}
\partofspeech{vt}
\spanishtranslation{orientar, aconsejar}
\cholexample{Mi kajel ityumbiñoñob.}
\exampletranslation{Van a orientarme.}

\entry{tyumtyumña}
\partofspeech{adj}
\spanishtranslation{palpitando}
\cholexample{Tyi pejtyel ora tyumtyumña kpusik'al.}
\exampletranslation{Mi corazón siempre está palpitando.}

\entry{tyumuty}
\partofspeech{s}
\spanishtranslation{huevo}

\entry{tyuñ}
\relevantdialect{Tila}
\partofspeech{s}
\onedefinition{1}
\spanishtranslation{piedra}
\cholexample{Añ kabäl tyuñ tyik chol.}
\exampletranslation{Hay muchas piedras en mi milpa.}
\onedefinition{2}
\spanishtranslation{huevo}
\clarification{de pájaro, pez, tortuga, gusano}
\cholexample{Jiñi chäy mi iyäk' ityuñ tyi' tyi' ja'.}
\exampletranslation{El pez pone sus huevos en la orilla del río.}
\alsosee{xajlel}

\entry{tyuñija'}
\partofspeech{s}
\spanishtranslation{granizo, hielo}

\entry{*tyuñil la kok}
\spanishtranslation{talón}

\entry{tyuñtyo'}
\partofspeech{s}
\spanishtranslation{hoja que se usa para envolver pozol}

\entry{tyuñ'aty}
\partofspeech{s}
\spanishtranslation{testículos}

\entry{*tyuñ'ok}
\partofspeech{s}
\spanishtranslation{talón}

\entry{tyup}
\partofspeech{vi}
\spanishtranslation{apachurrarse}
\cholexample{Mi ityup ibä jiñi lámiña che' jay.}
\exampletranslation{La lámina se apachurra cuando es delgada.}

\entry{tyutsul}
\partofspeech{adj}
\spanishtranslation{corto}
\cholexample{Tyutsul jax imachity.}
\exampletranslation{Es muy corto su machete.}

\entry{tyuts'}
\partofspeech{s}
\spanishtranslation{pataste}
\culturalinformation{Información cultural: Árbol que da fruta no comestible; la madera se usa para hacer cucharas.}

\entry{-tyuts'}
\nontranslationdef{Sufijo numeral para contar cucharadas de algo; p. ej.:}
\cholexample{juñtyuts' adj}
\exampletranslation{una cucharada de algo.}

\entry{tyuw}
\partofspeech{adj}
\spanishtranslation{apestoso}
\cholexample{Weñ tyuw jiñi chämeñ ts'i'.}
\exampletranslation{Ya está apestoso ese perro muerto.}
\variation{tyuweñ}

\entry{*tyuwel}
\partofspeech{s}
\spanishtranslation{fetidez}

\entry{tyuxbañ}
\partofspeech{vt}
\spanishtranslation{llamar}
\clarification{aves de corral}
\cholexample{Jiñi x'ixik mi ityuxbañ muty cha'añ mi ibuk'sañ.}
\exampletranslation{La mujer llama a sus pollos para darles de comer.}

\entry{tyuyub}
\partofspeech{s}
\spanishtranslation{periquito de aliamarillo}
\clarification{ave}

\alphaletter{Ty'}

\entry{ty'ajchel}
\partofspeech{vi}
\spanishtranslation{despegarse}
\clarification{una parte}
\cholexample{Wolix ity'ajchel ipajk'il iyotyoty.}
\exampletranslation{Se le está despegando la pared de su casa.}

\entry{*ty'añ}
\partofspeech{s}
\onedefinition{1}
\spanishtranslation{palabra}
\cholexample{Pejtyel ora jiñi jachbä ty'añ mi iyäl.}
\exampletranslation{Siempre dice la misma palabra.}
\onedefinition{2}
\spanishtranslation{idioma}
\cholexample{Jiñi ch'ol jiñäch ity'añ jiñi año'bä tyi tyumbalá.}
\exampletranslation{Chol es el idioma de los habitantes de Tumbalá.}

\entry{ty'äläkña}
\partofspeech{adv}
\spanishtranslation{estruendosamente}
\clarification{agua}
\cholexample{Ty'äläkña jiñi ja' ya' tyi aguazul.}
\exampletranslation{El agua cae estruendosamente en Agua Azul.}

\entry{ty'älty'älña}
\onedefinition{1}
\partofspeech{adj}
\spanishtranslation{temblando la tierra}
\clarification{por terremoto o rayo}
\cholexample{Ty'älty'älña jiñi lum cha'añ yujkel.}
\exampletranslation{La tierra está temblando por el terremoto.}
\onedefinition{2}
\partofspeech{adv}
\nontranslationdef{Se relaciona con la manera de temblar; p. ej.:}
\cholexample{Ty'älty'älña tyibäk'eñ cha'añ mi kaj ikäjchel.}
\exampletranslation{Está temblando de miedo porque va a estar encarcelado.}

\entry{ty'äsläk}
\partofspeech{s}
\spanishtranslation{sapillo}
\clarification{grano de carne del cerdo}

\entry{ty'äty'äña}
\relevantdialect{Tila}
\partofspeech{adj}
\spanishtranslation{temblando (de miedo)}
\cholexample{Ty'äty'äñayoñ tyibäk'eñ.}
\exampletranslation{Yo estaba temblando de miedo.}

\entry{*ty'ejl}
\partofspeech{s}
\spanishtranslation{lado}
\cholexample{Ya jach chumul tyi' ty'ejl kotyoty.}
\exampletranslation{Él vive al lado de mi casa.}

\entry{ty'ejty'ej}
\partofspeech{adv}
\spanishtranslation{palmeando}
\cholexample{Woli ity'ejty'ej jats'beñ ipaty iyerañ.}
\exampletranslation{Está palmeando la espalda de su hermano.}

\entry{ty'es}
\partofspeech{vt}
\spanishtranslation{sonar}
\clarification{los dedos}
\cholexample{Mi ity'es tyak ik'äb.}
\exampletranslation{Hace sonar sus dedos.}

\entry{ty'ichty'ichña}
\partofspeech{adv}
\spanishtranslation{de puntillas}
\cholexample{Ty'ichty'ichña iyok woli tyi xämbal.}
\exampletranslation{Está caminando de puntillas.}

\entry{ty'ijchiñ}
\partofspeech{vt}
\spanishtranslation{brincar con un pie}

\entry{ty'ity'ichñayok}
\partofspeech{adv}
\spanishtranslation{caminando de puntillas}
\cholexample{Ty'ity'ichñayok mi icha'leñ xämbal.}
\exampletranslation{De puntillas va caminando.}

\entry{Ty'obojuñ}
\partofspeech{s}
\spanishtranslation{Amate Hollado por Dentro}
\clarification{colonia}

\entry{ty'oj}
\partofspeech{vt}
\spanishtranslation{cortar}
\clarification{carne, piedra, madera}
\cholexample{Tsa' ity'ojo we'eläl.}
\exampletranslation{Cortó la carne.}

\entry{ty'ojchiñ}
\partofspeech{vt}
\spanishtranslation{desgranar}
\clarification{con el pulgar}
\cholexample{Woli ity'ojchiñ ixim yik'oty ik'äb.}
\exampletranslation{Está desgranando maíz con la uña del dedo pulgar.}

\entry{ty'ojläwel}
\partofspeech{vi}
\spanishtranslation{mejorarse}
\cholexample{Yom mi aweñ ak'ñañ akajpe'lel cha'añ mi ity'ojläwel.}
\exampletranslation{Debes limpiar bien tu cafetal para que se mejore.}

\entry{*ty'ojläwib}
\partofspeech{s}
\spanishtranslation{belleza}
\cholexample{Cha'añ ity'ojläwib jiñi kotyoty mi kaj kboñ.}
\exampletranslation{Voy a pintar mi casa para que esté bonita (lit.: para que tenga belleza).}

\entry{ty'ojlox}
\partofspeech{adj}
\spanishtranslation{disparejo}
\clarification{pared o suelo}
\cholexample{Ty'ojloxtyik jiñi lum tyi' mal iyotyoty.}
\exampletranslation{Está disparejo el piso de su casa.}

\entry{*ty'ojol}
\partofspeech{adj}
\onedefinition{1}
\spanishtranslation{bonito}
\cholexample{Ity'ojol jiñi tyejklum.}
\exampletranslation{Es bonito el pueblo.}
\onedefinition{2}
\spanishtranslation{recomendable}
\cholexample{Ity'ojolbajche' tsa' mele awe'tyel tyi presideñtye.}
\exampletranslation{Es recomendable la forma en que hiciste tu trabajo como presidente.}
\dialectvariant{Sab., Tila}
\dialectword{k'otya}

\entry{ty'ojtyi'iñ}
\partofspeech{vt}
\spanishtranslation{maldecir}
\cholexample{Jiñi xwujty mi kajel ity'ojtyi'iñ ipi'äl.}
\exampletranslation{El brujo va a maldecir a su compañero.}

\entry{ty'olokña}
\partofspeech{adj}
\spanishtranslation{sonando fuerte}
\clarification{lluvia}
\cholexample{Ty'olokña ja'al tyi' pam otyoty.}
\exampletranslation{Está sonando fuerte la lluvia en el techo.}

\entry{ty'olol}
\partofspeech{adj}
\spanishtranslation{disparejo}
\cholexample{Ty'olol bexel jiñi lum.}
\exampletranslation{La tierra está dispareja.}

\entry{ty'orjol}
\relevantdialect{Sab.}
\partofspeech{s}
\spanishtranslation{toloque}
\spanishtranslation{basilisco}
\spanishtranslation{pasarríos}
\clarification{reptil}

\entry{ty'os}
\partofspeech{part}
\nontranslationdef{Onomatopeya que indica el sonido al abrir un refresco; p. ej.:}
\cholexample{Che' mi lakjam refresko “ty'os”, che'eñ.}
\exampletranslation{Cuando abrimos un refresco se oye el sonido “t'os”.}

\entry{ty'oty'}
\partofspeech{s}
\spanishtranslation{caracol del monte}

\entry{ty'ox}
\partofspeech{vt}
\spanishtranslation{dividir}
\cholexample{Mi kaj ity'ox ilum.}
\exampletranslation{Va a dividir su terreno.}

\entry{ty'ox ja'}
\partofspeech{s}
\spanishtranslation{arco iris}

\entry{ty'oxlem}
\partofspeech{adj}
\spanishtranslation{dividido}
\cholexample{Ty'oxlem cha'añ komisariado ikajpe'lel.}
\exampletranslation{Su cafetal fue dividido por el comisariado.}

\entry{*ty'uchlib}
\partofspeech{s}
\spanishtranslation{tapesco, percha}
\clarification{lugar para pararse}
\cholexample{Mi kaj kmel ity'uchlib kmuty.}
\exampletranslation{Voy a hacer el tapesco para mis gallinas.}

\entry{ty'uchtyañ}
\partofspeech{vt}
\spanishtranslation{pisar}
\cholexample{Tsa' ujtyi kty'uchtyañ ch'ix.}
\exampletranslation{Acabo de pisar una espina.}

\entry{ty'uchtyäl}
\partofspeech{vi}
\spanishtranslation{pararse}
\cholexample{Jiñi muty mi ity'uchtyäl tyi tye'.}
\exampletranslation{El pájaro se para en el árbol.}

\entry{ty'uchul}
\partofspeech{adj}
\spanishtranslation{trepado}
\cholexample{Ty'uchul jiñi ch'ityoñ tyi' ñi' tye'.}
\exampletranslation{El chamaco está trepado encima del árbol.}

\entry{-ty'ujm}
\nontranslationdef{Sufijo numeral para contar hilos; p. ej.:}
\cholexample{Che' mi lakts'is jiñi pimbä pisil wersa yom cha'ty'ujum jiñi puy.}
\exampletranslation{Cuando cosemos una tela muy gruesa es conveniente usar hilo doble.}

\entry{ty'ul}
\partofspeech{s}
\spanishtranslation{conejo}
\clarification{mamífero}

\entry{ty'um}
\partofspeech{vt}
\spanishtranslation{seguir}
\cholexample{Tyoj mi aty'um majlel jiñi ñoj bij.}
\exampletranslation{Vas a seguir derecho por el camino real.}

\entry{ty'uñul}
\partofspeech{adj}
\onedefinition{1}
\spanishtranslation{grande}
\cholexample{Ty'uñul itye'elal jkajpe'lel.}
\exampletranslation{Ya están grandes los árboles de mi cafetal.}
\onedefinition{2}
\spanishtranslation{grueso}
\cholexample{Ty'uñulix jiñi ixim.}
\exampletranslation{Ya están gruesas las mazorcas.}

\entry{ty'us}
\partofspeech{adv}
\nontranslationdef{Se relaciona con la forma gruesa (de algo que está echado); p. ej.:}
\cholexample{Ty'us ñolol tyi lum jiñi chityam.}
\exampletranslation{Está echado en el suelo ese cerdo gordo.}

\entry{ty'ustyäl}
\partofspeech{adj}
\spanishtranslation{así de grueso}
\cholexample{Che'tyo ty'ustyäl jiñi tye' tsa'bä ksek'e.}
\exampletranslation{Así de grueso es el árbol que tumbé.}

\entry{*ty'ustyälel}
\partofspeech{s}
\spanishtranslation{tamaño}
\clarification{de trozo, de cerdo}
\cholexample{Ma'añik tsak p'isi ity'ustyälel jiñi tye'.}
\exampletranslation{No medí el grueso del palo.}

\entry{ty'usul}
\partofspeech{adj}
\spanishtranslation{botado}
\clarification{una cosa gruesa}
\cholexample{Ya' ty'usul jiñi tye, tyi' mal kchol.}
\exampletranslation{Ahí está botado el trozo grueso de un árbol en medio de mi milpa.}

\entry{ty'utspajch'}
\relevantdialect{Tila}
\partofspeech{s}
\spanishtranslation{piñuela}
\alsosee{xch'ix pajch'}

\alphaletter{Ts}

\entry{tsaj}
\partofspeech{adj}
\spanishtranslation{dulce}
\cholexample{Tsa' jjiñi kajpe'.}
\exampletranslation{Es dulce el café.}

\entry{tsaja bul'ich}
\partofspeech{s}
\spanishtranslation{salpullido}
\clarification{Erupción cutánea que se produce en tiempo de sequía y calor, cuando no se baña uno diario.}

\entry{tsajkañ}
\partofspeech{vt}
\spanishtranslation{seguir}
\cholexample{Mach yomik mi atsajkañ jiñi joñtyolbä wiñik.}
\exampletranslation{No debes seguir a ese hombre malo.}

\entry{tsajilety}
\partofspeech{vt irr}
\spanishtranslation{cuidado}
\cholexample{Yom tsajiletybajche' mi acha'leñ ty'añ.}
\exampletranslation{Debes tener cuidado cuando hablas.}

\entry{tsajiñ}
\partofspeech{vt}
\onedefinition{1}
\spanishtranslation{examinar}
\cholexample{Yom mi atsajiñ jiñi lum mu'bä amäñ.}
\exampletranslation{Debes examinar el terreno que vas a comprar.}
\onedefinition{2}
\spanishtranslation{cuidarse}
\cholexample{Yom mi atsajiñ abä che' mi amajlel tyi kolem tyejklum.}
\exampletranslation{Debes cuidarte cuando vayas al pueblo grande.}

\entry{tsajñi}
\defsuperscript{1}
\conjugationtense{3ª pers. sing. pret.}
\conjugationverb{majlel}
\spanishtranslation{fue}
\cholexample{Tsajñi imel ichol.}
\exampletranslation{Fue a hacer su milpa.}

\entry{tsajñi}
\defsuperscript{2}
\conjugationtense{3ª pers. sing. pret.}
\conjugationverb{tyilel}
\spanishtranslation{vino}
\cholexample{Tsajñi ijula'tyañoñ.}
\exampletranslation{Vino a visitarme.}

\entry{tsal}
\partofspeech{adv}
\spanishtranslation{apenas}
\cholexample{Tsa' jaxtyo ktsal tyeche kchobal.}
\exampletranslation{Apenas he comenzado a rozar mi milpa.}

\entry{tsaläl}
\partofspeech{s esp}
\spanishtranslation{cuarto}
\secondaryentry{itsal tye'}
\secondtranslation{división, pared}

\entry{tsak'iñ}
\partofspeech{s}
\spanishtranslation{chicle}
\clarification{goma de mascar}

\entry{tsa'}
\partofspeech{part}
\nontranslationdef{Palabra que indica el aspecto de tiempo pasado; p. ej.:}
\cholexample{Ñikolás tsa' majli tyi tyejklum.}
\exampletranslation{Nicolás fue al pueblo.}

\entry{tsa'ix}
\partofspeech{adv}
\spanishtranslation{ya}
\cholexample{Tsa'ix ujtyi kpäk' kchol.}
\exampletranslation{Ya terminé de sembrar mi milpa.}

\entry{tsäkleñ}
\partofspeech{vt}
\spanishtranslation{seguir}
\cholexample{Jiñi ts'i' mi itsäkleñ majlel iyum.}
\exampletranslation{El perro va siguiendo a su amo.}

\entry{tsäktsäkña}
\partofspeech{adv}
\spanishtranslation{después}
\clarification{por detrás}
\cholexample{Tsäktsäkña majlel jiñi ts'i' tyi' paty iyum.}
\exampletranslation{El perro va caminando detrás (después) de su dueño.}

\entry{tsäk'mäl}
\partofspeech{vi}
\spanishtranslation{consumirse}
\clarification{en fuego}
\cholexample{Mi itsäk'mäl majlel jiñi ja' che' mi lakotsañ tyi k'ajk.}
\exampletranslation{El agua se consume cuando la ponemos en el fuego.}

\entry{tsäk'mesañ}
\partofspeech{vt}
\spanishtranslation{consumir por hervir}
\cholexample{Mi kaj itsäk'mesañ iya'lel bu'ul.}
\exampletranslation{Se le va a consumir el caldo del frijol.}

\entry{tsäk'ojm}
\partofspeech{adj}
\spanishtranslation{seco}
\clarification{frijol}
\cholexample{Tsäk'ojm bu'ul woli ik'ux.}
\exampletranslation{Está comiendo frijol seco.}

\entry{tsäk'om}
\partofspeech{adj}
\spanishtranslation{seco}
\clarification{frijol}
\cholexample{Tsäjk'om bu'ul che' mi iweñ tyik'añ.}
\exampletranslation{El frijol queda sin caldo (seco) cuando se cuece mucho.}

\entry{tsäjyuñ}
\partofspeech{vt}
\spanishtranslation{rehusar}
\clarification{tortilla cuando está torteando}
\cholexample{Jiñi x'ixik mi itsäjyuñ iwaj che' mi ipechañ.}
\exampletranslation{Esa mujer rehúsa dar tortillas cuando está torteando.}

\entry{tsäñsa}
\partofspeech{s}
\spanishtranslation{homicidio}
\cholexample{Tsa' käjchi cha'añ tsa' icha'le tsäñsa.}
\exampletranslation{Está encarcelado porque cometió un homicidio.}

\entry{tsäñsañ}
\partofspeech{vt}
\spanishtranslation{matar}
\cholexample{Sebtyo tsa' ch'ojyi itsäñsañ ichityam.}
\exampletranslation{Se levantó temprano para matar su puerco.}

\entry{tsäñsäñtyel}
\partofspeech{vi}
\spanishtranslation{matarse}
\cholexample{Mi kaj itsäñsäñtyel jiñi wiñik.}
\exampletranslation{Ese hombre se va a matar.}

\entry{tsäñtsäña}
\partofspeech{adj}
\spanishtranslation{bonito}
\clarification{sonido}
\cholexample{Tsäñtsäña jiñi guityarra.}
\exampletranslation{La guitarra se oye bonito.}

\entry{tsäñal}
\partofspeech{s}
\spanishtranslation{frío}
\cholexample{Tyalix iyorajlel tsäñal.}
\exampletranslation{Ya se aproxima el tiempo de frío.}

\entry{tsäñä}
\partofspeech{adj}
\spanishtranslation{frío}
\cholexample{Yom mi awotsañ tsäñä ja' tyi sa'.}
\exampletranslation{Debes ponerle agua fría al pozol.}

\entry{tsäñä wäyel}
\partofspeech{vi}
\spanishtranslation{dormir sin cobija}
\clarification{sufriendo el frío}
\cholexample{Che' tsajñi tyi tyejklum tsa' icha'le tsäñä wäyel.}
\exampletranslation{Durmió sin cobija cuando fue al pueblo.}

\entry{tsäñesañ}
\partofspeech{vt}
\spanishtranslation{enfriar}
\cholexample{Yom mi atsäñesañ jiñi sa' yik'oty tsäñä ja'.}
\exampletranslation{Hay que enfriar el pozol con agua fría.}

\entry{tsäts}
\partofspeech{adj}
\onedefinition{1}
\spanishtranslation{duro}
\cholexample{Tsäts jiñi lumba' tsak päk'ä kixim.}
\exampletranslation{La tierra donde sembré mi maíz es dura.}
\onedefinition{2}
\spanishtranslation{áspero}
\cholexample{Tsäts tsa' icha'le ty'añ jiñi komisariado.}
\exampletranslation{El comisario habló con palabras ásperas.}
\onedefinition{3}
\spanishtranslation{responsable}
\cholexample{Tsäts iye'tyel tsa' ityaja jiñi kerañ.}
\exampletranslation{A mi hermano se le dio un trabajo responsable.}

\entry{tsätsä bichil}
\partofspeech{adj}
\spanishtranslation{tieso}
\cholexample{Tsätsä bichil ik'äb.}
\exampletranslation{Tiene tiesa la mano.}

\entry{tsätsä ñu'ty'ul}
\partofspeech{adj}
\spanishtranslation{reducido}
\cholexample{Tsäts ñuty'ul jax jiñi kotyoty.}
\exampletranslation{Mi casa está muy reducida.}

\entry{*tsätslel}
\partofspeech{s}
\spanishtranslation{dureza}
\cholexample{Kabäl itsätslel jiñi lum.}
\exampletranslation{Esa tierra es muy dura (lit.: tiene dureza).}

\entry{tsäwañ pañimil}
\spanishtranslation{tierra fría}

\entry{tsäy}
\defsuperscript{1}
\partofspeech{adv}
\spanishtranslation{de momento, por un rato}
\cholexample{Tsa' itsäy kächä tyi tye' imula.}
\exampletranslation{Amarró su mula en un palo durante un rato.}

\entry{tsäy}
\defsuperscript{2}
\partofspeech{vt}
\spanishtranslation{encoger}
\cholexample{Mi itsäy ibä lakñäk' che' k'amoñla.}
\exampletranslation{Cuando estamos enfermos, el estómago se nos encoge.}

\entry{tsäyäkña}
\partofspeech{adj}
\spanishtranslation{pegajoso}
\cholexample{Tsäyäkña jiñi tya'chäb tyi laj k'äb.}
\exampletranslation{La cera es pegajosa en la mano.}

\entry{tsäyäl}
\partofspeech{adj}
\spanishtranslation{angosto}
\cholexample{Tsäyäl iñäk'ba' mi ikäch ibä.}
\exampletranslation{Él tiene la cintura angosta.}

\entry{tsäytsäyña}
\partofspeech{adj}
\spanishtranslation{tacaño}
\cholexample{Tsäytsäyña ipusik'al cha'añ mach yomik iyäk' tyi majañ imula.}
\exampletranslation{Su corazón es tacaño por no querer prestar su mula.}

\entry{tsejluk ak'}
\spanishtranslation{bejuco de uva}

\entry{*tsejpel}
\partofspeech{s}
\spanishtranslation{herida}
\clarification{cortada}

\entry{tselekña}
\partofspeech{adv}
\nontranslationdef{Se relaciona con el movimiento de alguien o algo que corre; p. ej.:}
\cholexample{Tselekña ñumel jiñi ts'i'.}
\exampletranslation{El perro va corriendo.}

\entry{tseljol}
\partofspeech{s}
\spanishtranslation{basilisco, iguana}
\clarification{reptil}

\entry{*tsel muty}
\partofspeech{s}
\spanishtranslation{cresta de pájaro o gallo}

\entry{*tseñek}
\partofspeech{s}
\spanishtranslation{canilla de la pierna}

\entry{tsep}
\partofspeech{vt}
\spanishtranslation{cortar}
\cholexample{Tsa' majli itsep jam.}
\exampletranslation{Se fue a cortar zacate.}

\entry{tse'ñal}
\partofspeech{s}
\spanishtranslation{risa, sonrisa}
\cholexample{Mach yomik mi acha'leñ tse'ñal tyi juñtya.}
\exampletranslation{No debes reírte (lit.: dar risa) en la junta}

\entry{tse'tyañ}
\partofspeech{vt}
\spanishtranslation{reírse de}
\cholexample{Woli itse'tyañ iyijts'iñ cha'añ tsa' yajli.}
\exampletranslation{Está riéndose de su hermanito porque se cayó.}

\entry{tse'tyeñtyik}
\partofspeech{adv}
\spanishtranslation{chistosamente}
\cholexample{Tse'tyeñtyik mi imel jiñi wiñik.}
\exampletranslation{Ese hombre habla chistosamente.}

\entry{tsik}
\onedefinition{1}
\partofspeech{vt}
\spanishtranslation{contar}
\cholexample{Mach weñ yulilik tsik.}
\exampletranslation{No puede contar muy bien.}
\onedefinition{2}
\partofspeech{s}
\spanishtranslation{mes}
\cholexample{Ora jach woli iñumel tyak majlel jiñi tsik.}
\exampletranslation{Muy pronto están pasando los meses.}
\secondaryentry{tsik pusik'al}
\secondtranslation{preocupación}

\entry{tsiktyesañ}
\partofspeech{vt}
\onedefinition{1}
\spanishtranslation{declarar}
\cholexample{Che'äch yom mi atsiktyesañbajche' tsa' k'ele.}
\exampletranslation{Debes declarar tal como lo viste.}
\onedefinition{2}
\spanishtranslation{dar a saber}
\cholexample{Tsa' tyili itsiktyesañ cha'añ tyal jiñi yumäl.}
\exampletranslation{Vino para dar a saber que vendrá un gobernante.}

\entry{tsiktyesäbil}
\partofspeech{adj}
\spanishtranslation{anunciado}
\cholexample{Tsiktyesäbil tyi pejtyelel koloñia tyak cha'añ tyal jiñi delegado.}
\exampletranslation{Está anunciado en todas las colonias que viene el delegado.}

\entry{tsiktyiyel}
\partofspeech{vi}
\onedefinition{1}
\spanishtranslation{descubrirse}
\cholexample{Ma'añik woli itsiktyiyel majki tsa' itsäñsa jiñi wiñik.}
\exampletranslation{No se ha descubierto quién fue el que mató a ese hombre.}
\onedefinition{2}
\spanishtranslation{aparecer}
\cholexample{Mi kaj itsiktyiyel wits che' mi ilajmel buts'.}
\exampletranslation{Los cerros aparecerán cuando desaparezca el humo.}

\entry{tsij}
\partofspeech{adj}
\spanishtranslation{crudo}
\cholexample{Tsijtyo jiñi bu'ul.}
\exampletranslation{El frijol está crudo todavía.}

\entry{tsijkañ}
\partofspeech{vt}
\spanishtranslation{rociar}
\cholexample{Jiñi x'ixik mi itsijkañ ja' tyi pisil cha'añ mi cha'leñ juk'oñel.}
\exampletranslation{La mujer rocía la ropa con agua para plancharla.}

\entry{tsijkotso'}
\partofspeech{s}
\spanishtranslation{chicantor}
\clarification{ave}

\entry{tsijib}
\partofspeech{adj}
\spanishtranslation{nuevo}
\cholexample{Tsijib ibujk tsa' ixojo tyilel.}
\exampletranslation{Era nueva la camisa que trajo puesta.}

\entry{tsijibtyesañ}
\partofspeech{vt}
\spanishtranslation{hacer de nuevo, renovar}
\cholexample{Woli itsijibtyesañ iyotyoty.}
\exampletranslation{Están renovando su casa.}

\entry{tsijlem}
\partofspeech{adj}
\spanishtranslation{roto}
\clarification{ropa, costal}
\cholexample{Tsijlem ibujk jiñi ch'ityoñ.}
\exampletranslation{Está rota la camisa de ese chamaco.}

\entry{*tsijlemal}
\partofspeech{s}
\spanishtranslation{rasgadura}
\cholexample{Ñuk itsijlemal iwex.}
\exampletranslation{Es grande la rasgadura de su pantalón.}

\entry{tsil}
\partofspeech{vt}
\spanishtranslation{romper}
\cholexample{Tsa' ujtyi itsil iwex yik'oty ch'ix.}
\exampletranslation{Acaba de romper su pantalón con una espina.}

\entry{tsiltsilñiyel}
\partofspeech{vi}
\spanishtranslation{temblar}
\clarification{de frío o miedo}
\cholexample{Woli tyi tsiltsilñiyel cha'añbäk'eñ.}
\exampletranslation{Está temblando de miedo.}

\entry{tsiltsilña}
\partofspeech{adj}
\spanishtranslation{temblando de miedo}
\cholexample{Tsiltsilña jiñi wiñik cha'añ añ imul.}
\exampletranslation{Ese hombre está temblando de miedo porque cometió un delito.}

\entry{tsima}
\partofspeech{s}
\spanishtranslation{jícara}

\entry{tsimajil}
\partofspeech{s}
\spanishtranslation{arboleda de huacales}
\clarification{guacal}

\entry{tsimiñ}
\partofspeech{s}
\spanishtranslation{tapir}
\clarification{mamífero}
\dialectvariant{Tila}
\dialectword{jamoñ}

\entry{tsiñ}
\partofspeech{vt}
\spanishtranslation{disminuir}
\clarification{como meteoro}
\cholexample{Jiñi itya' ek' mi itsiñ ibä.}
\exampletranslation{Va disminuyéndose el tamaño del meteorito.}

\entry{tsiñtyäl}
\partofspeech{adv}
\spanishtranslation{disminuyendo}
\cholexample{Che' ya tsiñtyäl tsa' jili majlel itya' ek'.}
\exampletranslation{El meteoro se fue disminuyendo.}

\entry{tsiñtsiña}
\partofspeech{adj}
\onedefinition{1}
\spanishtranslation{retumbante}
\cholexample{Tsiñtsiña jiñi kampaña che' mi lakjats.}
\exampletranslation{La campana es retumbante al golpearla.}
\onedefinition{2}
\spanishtranslation{sonoro}
\cholexample{Tsiñtsiña woli ijats' guityarra.}
\exampletranslation{La guitarra es sonora cuando la toca.}

\entry{tsikil}
\partofspeech{adj}
\onedefinition{1}
\spanishtranslation{visible}
\cholexample{Che' mi isäjp'el jubel ja' tsikil xajlelal tyak.}
\exampletranslation{Cuando el río está bajo, las piedras del fondo están visibles.}
\onedefinition{2}
\spanishtranslation{comprensible}
\cholexample{Weñ tsikil isujmlel.}
\exampletranslation{El significado es comprensible.}

\entry{tsoklaw}
\partofspeech{adv}
\nontranslationdef{Se relaciona con el ruido que hacen los palitos y las hojas secas cuando pasan muchos aimales o pájaros; p. ej.:}
\cholexample{Tsoklaw tsa' majli kabäl matye' chityam.}
\exampletranslation{Los marranos del monte se fueron haciendo mucho ruido con las hojas y palitos secos.}

\entry{tsoktsokña}
\partofspeech{adv}
\nontranslationdef{Manera en que hacen sonido las hojas y palitos secos; p. ej.:}
\cholexample{Tsoktsokña mi imajlelbäktye'el tyi ma'tye'el.}
\exampletranslation{El animal va por el bosque haciendo ruido con las hojas y palitos secos.}
\alsosee{tsoklaw}

\entry{tsok'ol}
\partofspeech{adj}
\spanishtranslation{guindado}
\cholexample{Ya'tyo tsok'ol jiñi jumpajl ja'as.}
\exampletranslation{Ahí está guindado un racimo de plátanos.}

\entry{-tsojk'}
\nontranslationdef{Sufijo numeral para contar racimos de plátanos o uvas; p. ej.:}
\cholexample{Tsa' yajli juñtsojk' ja'as.}
\exampletranslation{Se cayó un racimo de mi plátano.}

\entry{tsojtyel}
\partofspeech{vi}
\spanishtranslation{gatear}
\clarification{niño}
\cholexample{Woli'tyo iñop tsojtyel jiñi aläl.}
\exampletranslation{Ese niño apenas está aprendiendo a gatear.}

\entry{tsol}
\partofspeech{vt}
\spanishtranslation{alinear, poner en fila}
\cholexample{Mi kaj ktsol ak' xajlel tyi' paty kotyoty.}
\exampletranslation{Voy a alinear las piedras atrás de mi casa.}

\entry{tsolokña}
\partofspeech{adv}
\spanishtranslation{en filas}
\cholexample{Tsolokña woli imajlel wiñikob tyi bij.}
\exampletranslation{Los hombres van en filas por el camino.}

\entry{tsolol}
\partofspeech{adj}
\spanishtranslation{en filas}
\cholexample{Tsolol tyak otyoty ya' tyi tyejklum.}
\exampletranslation{Las casas del pueblo están en filas.}

\entry{-tsolom}
\nontranslationdef{Sufijo numeral para contar filas; p. ej.:}
\cholexample{Che' mi lakcha'leñ ak'iñ, ñaxañ mi laklok' jujuñtsolom.}
\exampletranslation{Cuando limpiamos el cafetal, primero sacamos la jornada de cada fila.}

\entry{tsoltsolñiyel}
\partofspeech{vi}
\spanishtranslation{pasar de fila en fila}
\cholexample{Mi kaj icha'leñob tsoltsolñiyel ilatyi kale.}
\exampletranslation{Van a caminar en filas aquí en la calle.}

\entry{tsostyäl}
\partofspeech{adv}
\spanishtranslation{así de alto}
\clarification{de animal}
\cholexample{Che' ya' tsostyäl jiñi ts'i'.}
\exampletranslation{El perro es así de alto.}

\entry{tsotytyäl}
\partofspeech{vi}
\spanishtranslation{sentarse}
\clarification{agachado}
\cholexample{Tsotytyäl jaxtyo mi icha'leñ jiñi aläl.}
\exampletranslation{Ese niño apenas se sienta agachado.}

\entry{tsots}
\partofspeech{s}
\spanishtranslation{fruta de un bejuco que es comestible}

\entry{tsoy}
\partofspeech{s}
\spanishtranslation{llaga}

\entry{tsoy'ajel}
\partofspeech{vi}
\spanishtranslation{llagarse}
\cholexample{Wolix tyi weñ tsoy'ajel iyok.}
\exampletranslation{Se le está llagando el pie.}

\entry{tsuk}
\partofspeech{s}
\spanishtranslation{ratón}
\clarification{mamífero}

\entry{tsuk bajlum}
\spanishtranslation{ocelote, tigrillo}
\clarification{mamífero}

\entry{*tsuktyi'}
\partofspeech{s}
\spanishtranslation{barba}

\entry{tsuktsukñiyel}
\partofspeech{vi}
\spanishtranslation{pasar}
\clarification{de casa en casa}
\cholexample{Tsuktsukñiyel jach mi icha'leñ jiñi x'ixik.}
\exampletranslation{A esa mujer solo le gusta pasar de casa en casa.}

\entry{tsuktsukña}
\partofspeech{adv}
\spanishtranslation{rondando}
\clarification{en busca de su presa}
\cholexample{Tsuktsukña woli iñumel jiñi ts'i' cha'añ mi isäklañ iwe'el.}
\exampletranslation{Ese perro anda rondando en busca de su presa.}

\entry{tsukul}
\partofspeech{adj}
\onedefinition{1}
\spanishtranslation{gastado}
\cholexample{Maxtyo tsukulik jiñi machity.}
\exampletranslation{Todavía no está gastado el machete.}
\onedefinition{2}
\spanishtranslation{viejo}
\cholexample{Maxtyo añik tsukul jiñi juloñib.}
\exampletranslation{Todavía no está vieja mi escopeta.}
\secondaryentry{tsukul pisil}
\secondtranslation{ropa vieja}

\entry{*tsukulel}
\partofspeech{s}
\onedefinition{1}
\spanishtranslation{lo gastado}
\cholexample{Tsikilix itsukulel jiñi moñtyura.}
\exampletranslation{Ya se ve lo gastado de la montura.}
\onedefinition{2}
\spanishtranslation{herrumbre}
\cholexample{Kabäl itsukulel ijuloñib.}
\exampletranslation{Su escopeta tiene mucha herrumbre.}
\onedefinition{3}
\spanishtranslation{grosería}
\cholexample{Añ itsukulel ity'añ.}
\exampletranslation{Él habla con groserías.}

\entry{tsukutyak'iñ}
\partofspeech{s}
\spanishtranslation{metal}

\entry{tsuk'}
\defsuperscript{1}
\partofspeech{adv}
\nontranslationdef{Se relaciona con la manera de introducir una cosa delgada; p. ej.:}
\cholexample{Tsa' jach itsuk' ts'aja ik'äb tyi tyikäw ja'.}
\exampletranslation{Sólo mojó la punta de su dedo en el agua caliente.}

\entry{tsuk'}
\defsuperscript{2}
\partofspeech{vt}
\spanishtranslation{encender}
\cholexample{Mi laktsuk' jiñi k'ajk che' wolix iyik'añ.}
\exampletranslation{Cuando está oscuro encendemos el candil.}

\entry{tsujk'em}
\partofspeech{adj}
\spanishtranslation{encendido}
\cholexample{Maxtyo añik tsujk'em jiñi kas.}
\exampletranslation{El candil todavía no está encendido.}

\entry{tsuñkay}
\partofspeech{s}
\spanishtranslation{tipo de ave}
\clarification{de tamaño regular; de cola corta; anda en el suelo; sirve de alimento}

\entry{tsuñtye'chañ}
\partofspeech{s}
\spanishtranslation{nauyaca cornuda}
\clarification{víbora verde que es muy venenosa}

\entry{*tsuñtye'lel}
\partofspeech{s}
\spanishtranslation{tipo de musgo}
\clarification{verde oscuro que se cría en la corteza de algunos árboles}

\entry{*tsukil ixim}
\spanishtranslation{sobras del ratón}
\cholexample{Jiñi jachix itsukil ixim tsa käle.}
\exampletranslation{Del maíz solamente quedaron las sobras del ratón.}

\entry{tsuts}
\defsuperscript{1}
\partofspeech{vt}
\spanishtranslation{resembrar}
\cholexample{Yom lakcha' tsuts jiñi cholelba' ma'añik tsa' pasi ixim.}
\exampletranslation{Resembramos la milpa donde no nació el maíz.}

\entry{tsuts}
\defsuperscript{2}
\partofspeech{s}
\onedefinition{1}
\spanishtranslation{lana}
\cholexample{Jiñi tyäñäme' mi iyäq'keñoñlatsuts.}
\exampletranslation{La oveja nos da lana.}
\onedefinition{2}
\spanishtranslation{cobija}
\cholexample{Yom cha'p'ejl tsuts che' tsäwañ pañimil.}
\exampletranslation{Se necesitan dos cobijas cuando hace frío.}
\onedefinition{3}
\spanishtranslation{chamarra}
\cholexample{Yom añ atsuts che' mi amajlel tyi jobel.}
\exampletranslation{Quiere que lleves una chamarra cuando vayas a Las Casas.}

\entry{tsuts chikiñ}
\partofspeech{s}
\spanishtranslation{tipo de hongo}
\clarification{de color café arriba y blanco abajo, que se cría en los árboles; se come cuando está brotando}

\entry{tsutschoj}
\partofspeech{s}
\spanishtranslation{hombre con mucha barba}

\entry{*tsutsel}
\partofspeech{s}
\onedefinition{1}
\spanishtranslation{pluma de pájaro}
\onedefinition{2}
\spanishtranslation{pelo}
\onedefinition{3}
\spanishtranslation{vello de hombre}
\secondaryentry{*tsutsel lakol}
\secondtranslation{cabellos}
\secondaryentry{*tsutsel laj k'äb}
\secondtranslation{vello}
\secondaryentry{*tsutsel lakchoj}
\secondtranslation{barba}
\secondaryentry{*tsutsel lakwuty}
\secondtranslation{pestaña}
\secondaryentry{*tsutsel muty}
\secondtranslation{plumas}
\secondaryentry{*tsutsel tyiñäme'}
\secondtranslation{lana}

\entry{tsutsob}
\partofspeech{s}
\spanishtranslation{gente de Tenejapa}

\entry{tsutspuy}
\relevantdialect{Sab.}
\partofspeech{s}
\spanishtranslation{esponja}

\entry{tsuy}
\defsuperscript{1}
\partofspeech{s}
\spanishtranslation{hierba laxante que se come}
\secondaryentry{*tsuy me'}
\secondtranslation{planta}

\entry{tsuy}
\defsuperscript{2}
\partofspeech{vt}
\onedefinition{1}
\spanishtranslation{pegar}
\clarification{papel, tela}
\cholexample{Mi laktsuy juñ tyi tyabla.}
\exampletranslation{Pegamos papel en la tabla.}
\onedefinition{2}
\spanishtranslation{extender}
\cholexample{Yom laktsuy lakotyoty kome ch'och'ok.}
\exampletranslation{Quiere extender la casa porque está chica.}
\onedefinition{3}
\spanishtranslation{llevar (fuego)}
\cholexample{Mi itsuy tyilel k'ajk tyi xutyuñtye'.}
\exampletranslation{Trae lumbre con tizón.}

\entry{tsu'sañ}
\partofspeech{vt}
\spanishtranslation{dar de mamar}
\cholexample{Jiñi x'ixik woli itsu'sañ iyalobil.}
\exampletranslation{La mujer está dando de mamar a su criatura.}

\entry{tsu'um}
\partofspeech{s}
\spanishtranslation{amate, matapalo}
\clarification{planta parásita, sube por cualquier árbol, enrollándolo; puede tener una circunferencia hasta de diez metros en la base.}

\alphaletter{Ts'}

\entry{ts'ak}
\partofspeech{s}
\spanishtranslation{medicina, remedio}

\entry{ts'aj}
\partofspeech{vt}
\spanishtranslation{remojar}
\cholexample{Mi lakts'aj jiñi pisil tyi ja'.}
\exampletranslation{Remojamos la ropa en el agua.}

\entry{ts'ajk}
\partofspeech{s}
\spanishtranslation{pared de piedra, banqueta, pared de cemento}

\entry{-ts'ajl}
\nontranslationdef{Sufijo numeral para contar hacinadas (de leña, maíz); p. ej.:}
\cholexample{Tyi ili jabil tsak lok'o cha'ts'ajl jiñi ixim.}
\exampletranslation{Este año coseché dos hacinadas de maíz.}

\entry{ts'ajmel}
\partofspeech{vi}
\spanishtranslation{entrar}
\clarification{agua en casa}
\cholexample{Wolix its'ajmel ochel jiñi bej yok ja' tyi mal otyoty.}
\exampletranslation{El corrental está entrando dentro de la casa.}

\entry{ts'ajkiñ}
\partofspeech{vt}
\spanishtranslation{poner piedras en su lugar para hacer una casa}

\entry{ts'ajts'añ}
\partofspeech{vt}
\spanishtranslation{remojar}
\clarification{la cabeza}
\cholexample{Woli jach its'ajts'añ ijol.}
\exampletranslation{Sólo se está remojando la cabeza.}

\entry{ts'ayakña}
\partofspeech{adj}
\spanishtranslation{liso}
\cholexample{Ts'ayakña tsa' käle kmesa.}
\exampletranslation{Mi mesa quedó lisa.}

\entry{ts'a'}
\partofspeech{adv}
\spanishtranslation{odiosamente}
\cholexample{Ts'a' mi ik'elob ibä jiñi wiñikob.}
\exampletranslation{Esos hombres se ven odiosamente.}

\entry{ts'a'añ}
\relevantdialect{Sab.}
\onedefinition{1}
\partofspeech{adj}
\spanishtranslation{picante}
\cholexample{Ma'añik ts'a'añ jiñi ich.}
\exampletranslation{Ese chile no es picante.}
\onedefinition{2}
\partofspeech{s}
\spanishtranslation{aguardiente}
\cholexample{Mi imulañ ijap ts'a'añ.}
\exampletranslation{Le gusta tomar aguardiente.}

\entry{ts'a'lebil}
\partofspeech{adj}
\onedefinition{1}
\spanishtranslation{despreciado}
\cholexample{Ts'a'lebil cha'añ ipi'älob jiñi x'ixik.}
\exampletranslation{Esa mujer es despreciada por sus hermanos.}
\onedefinition{2}
\spanishtranslation{aborrecido}
\cholexample{Ts'a'lebil jiñi wiñik cha'añ tsa' tyajle imul.}
\exampletranslation{Ese hombre es aborrecido porque se descubrió su maldad.}

\entry{ts'a'leñ}
\partofspeech{vt}
\spanishtranslation{aborrecer}
\cholexample{Jiñi wiñik mi its'a'leñ iyijñam.}
\exampletranslation{Ese hombre aborrece a su mujer.}

\entry{*ts'a'ñal}
\partofspeech{s}
\spanishtranslation{fuerza}
\clarification{de la cal}
\cholexample{Weñ añtyo its'añal jiñi tyañ che' ujtyeltyo tyi pulel.}
\exampletranslation{La cal tiene mucha fuerza cuando se acaba de quemar.}

\entry{ts'äb}
\partofspeech{vt}
\spanishtranslation{encender}
\cholexample{Ak'älelix, yom mi ats'äb kas.}
\exampletranslation{Ya está oscuro, hay que encender el candil.}

\entry{ts'äbab}
\partofspeech{s}
\spanishtranslation{clase de bejuco chico y espinoso}

\entry{ts'äbäkña}
\partofspeech{adj}
\spanishtranslation{molido}
\clarification{granos finos o remolidos de arena, azúcar}

\entry{*ts'äboñib}
\partofspeech{s}
\spanishtranslation{encendedor}
\cholexample{Añ its'äboñib cha'añ ik'ujts.}
\exampletranslation{Lleva encendedor para su cigarro.}

\entry{*ts'äkal}
\partofspeech{s}
\onedefinition{1}
\spanishtranslation{recaudo}
\cholexample{Jiñäch its'äkal we'eläl.}
\exampletranslation{Es el recaudo de la comida.}
\onedefinition{2}
\spanishtranslation{medicina}
\cholexample{Jiñäch its'äkal ak'amäjel.}
\exampletranslation{Es la medicina para tu enfermedad.}

\entry{ts'äkañ}
\partofspeech{vt}
\spanishtranslation{curar}
\cholexample{Yom mi ats'äkañ jiñi awojbal.}
\exampletranslation{Debes curarte la tos.}

\entry{ts'äkäbil}
\partofspeech{adj}
\spanishtranslation{tratado}
\clarification{con medicina o químicos}
\cholexample{Ts'äkäbil jiñi xkamatyilo tsa'bä ipäk'ä.}
\exampletranslation{Las papas que sembró fueron tratadas con insecticida.}

\entry{ts'äkäl}
\partofspeech{adj}
\spanishtranslation{completo}
\cholexample{Ts'äkäl jiñi kajpe' tsa'bä kchoño.}
\exampletranslation{El café que vendí estuvo completo.}

\entry{ts'äkäñtyel}
\partofspeech{vi}
\spanishtranslation{curarse}
\cholexample{Tsa' majli tyi ts'äkäñtyel tyi tyuxtyla.}
\exampletranslation{Fue a Tuxtla para curarse.}

\entry{ts'äktyesañ}
\partofspeech{vt}
\spanishtranslation{completar}
\cholexample{Tsajñi kts'äktyesañ iwäkp'ejlel k'iñ ke'tyel tyi karretyera.}
\exampletranslation{Fui a completar mis seis días de trabajo en la carretera.}

\entry{ts'äktyesäñtyel}
\partofspeech{vi}
\spanishtranslation{cumplirse}
\cholexample{Che'äch mi kaj tyi ts'äktyesäñtyelbajche' tsa' ajli.}
\exampletranslation{Así se va a cumplir, como se dijo.}

\entry{ts'äl}
\partofspeech{vt}
\spanishtranslation{hacinar}
\cholexample{Woli its'äl jump'ejl tyarea si'.}
\exampletranslation{Está hacinando una tarea de leña.}

\entry{ts'äläl}
\partofspeech{adj}
\spanishtranslation{hacinado}
\cholexample{Tyi' yojlil iyotyoty ts'äläl icha'añ ikajpe'.}
\exampletranslation{En medio de su casa tiene sus bultos de café hacinados.}

\entry{ts'älbil}
\partofspeech{adj}
\spanishtranslation{hacinado}
\cholexample{Ts'älbil tyak jiñi tyablacha'añ mi ityikiñ.}
\exampletranslation{Las tablas están hacinadas para que se sequen.}

\entry{ts'ältye'ebil}
\partofspeech{adj}
\spanishtranslation{entarimado}
\clarification{con palos}
\cholexample{Ts'ältye'ebil jiñi bijba' weñ kabäl ok'ol.}
\exampletranslation{El camino donde hay mucho lodo está entarimado.}

\entry{ts'ämäl}
\partofspeech{adj}
\spanishtranslation{tranquilo}
\clarification{agua, río}
\cholexample{Ts'ämäl jach añ jiñi ja'.}
\exampletranslation{El agua está tranquila.}

\entry{ts'ämel}
\partofspeech{s}
\spanishtranslation{baño}
\cholexample{Tsäwañ ja' mi jk'äñ tyi ts'ämel.}
\exampletranslation{Uso agua fría para mi baño.}

\entry{ts'ämi' chityam}
\spanishtranslation{bañadero de puerco}

\entry{ts'äñsañ}
\partofspeech{vt}
\spanishtranslation{bañar}
\cholexample{Wersa yom mi lakts'äñsañ jiñi kawayu'.}
\exampletranslation{Es necesario bañar al caballo.}

\entry{ts'äñsäbil}
\partofspeech{adj}
\spanishtranslation{bañado}
\cholexample{Maxtyo añik ts'äñsäbil jiñi alob.}
\exampletranslation{El niño todavía no está bañado.}

\entry{ts'äñsäñtyel}
\partofspeech{vi}
\spanishtranslation{bañarse}
\cholexample{Mi iñaxañ ts'äñsäñtyel jiñi ch'ujleläl cha'añ mi'bäjk'el.}
\exampletranslation{Primero se baña el cadáver para envolverlo.}

\entry{ts'äp}
\partofspeech{vt}
\spanishtranslation{sembrar}
\clarification{poste}
\cholexample{Tsa' its'äpä postye tyi potyrero.}
\exampletranslation{Sembró postes en el potrero.}

\entry{ts'äpäl}
\partofspeech{adj}
\spanishtranslation{enterrado}
\clarification{poste}
\cholexample{Tyam ts'äpäl jiñi postye.}
\exampletranslation{El poste está enterrado hondo.}

\entry{ts'äplaw}
\partofspeech{adv}
\spanishtranslation{cayendo}
\clarification{unas cuantas gotas}
\cholexample{Ts'äplaw tsa' ñumi ja'al.}
\exampletranslation{Unas cuantas gotas de agua cayeron al pasar la lluvia (lit.: la lluvia pasó cayendo unas cuantas gotas).}

\entry{ts'äyäkña}
\partofspeech{adv}
\nontranslationdef{Se relaciona con la manera de alumbrar; p. ej.:}
\cholexample{Ts'äyäkña jiñi k'iñ che' mi ipoj tyäts' ibä jiñityokal.}
\exampletranslation{El sol sale alumbrando cuando la nube se retira.}

\entry{ts'äyäl}
\partofspeech{adj}
\spanishtranslation{encendido}
\cholexample{Ts'äyäl jiñi k'ajk che' tyi ak'älel.}
\exampletranslation{La luz queda encendida por la noche.}

\entry{ts'äylaw}
\partofspeech{adj}
\spanishtranslation{chispeante}
\cholexample{Ts'äylaw ik'äk'al xajlel che' mi laktsep yik'oty machity.}
\exampletranslation{La piedra da chispas (lit.: está chispeante) cuando le pegamos con el machete.}

\entry{ts'ej}
\partofspeech{adv}
\spanishtranslation{de lado}
\cholexample{Tsa' ts'ej ñole tyi wäyib.}
\exampletranslation{Se acostó de lado en su cama.}

\entry{*ts'ej}
\partofspeech{s}
\spanishtranslation{mano izquierda}
\cholexample{Mi ik'äñ its'ej tyi e'tyel.}
\exampletranslation{Usa su mano izquierda para trabajar.}

\entry{ts'ejchokoñ}
\partofspeech{vt}
\spanishtranslation{colocar de lado}
\cholexample{Ts'ejchokoñ jiñi jobeñ ya' tyi lum.}
\exampletranslation{Coloca de lado ese tablero en el suelo.}

\entry{ts'ejel}
\partofspeech{adv}
\spanishtranslation{de lado}
\cholexample{Ts'ejel tsa' yajliyoñ.}
\exampletranslation{Me caí de lado.}

\entry{ts'ej ñolol}
\spanishtranslation{acostado de lado}
\cholexample{Ts'ej ñolol tyi' wäyib jiñi ch'ityoñ.}
\exampletranslation{Ese muchacho está acostado de lado en su cama.}

\entry{ts'ej ñoltyäl}
\spanishtranslation{acostarse de lado}
\cholexample{Mi kaj kts'ej ñoltyäl ilatyi wäyibäl.}
\exampletranslation{Me voy a acostar de lado en esta cama.}

\entry{*ts'ejtyäl}
\partofspeech{s}
\spanishtranslation{lado}
\cholexample{Jiñi kchol ya' jach tsak mele tyi' ts'ejtyäl jkajpe'lel.}
\exampletranslation{Hice mi milpa al lado de mi cafetal.}

\entry{ts'ejuña}
\partofspeech{adv}
\spanishtranslation{de lado a lado}
\cholexample{Ts'ejuña tsa' majli jiñi chäy, ma'añik tsa' chämi.}
\exampletranslation{El pez se fue de lado a lado, pero no se murió.}

\entry{ts'elekña}
\partofspeech{adv}
\spanishtranslation{a gritos}
\cholexample{Ts'elekña tyi uk'el jiñi ch'ityoñ.}
\exampletranslation{Ese niño está llorando a gritos.}

\entry{ts'ij}
\defsuperscript{1}
\partofspeech{adv}
\spanishtranslation{ruidosamente}
\cholexample{Tsa' its'ij ty'ojo xajlel.}
\exampletranslation{Ruidosamente cortó la piedra con su machete.}

\entry{ts'ij}
\defsuperscript{2}
\partofspeech{vt}
\spanishtranslation{quebrar}
\clarification{leña, piedra, calabaza}
\cholexample{Woli its'ij xajlel yik'oty marro.}
\exampletranslation{Está quebrando piedra con marro.}

\entry{ts'ij}
\defsuperscript{3}
\partofspeech{vt}
\spanishtranslation{rajar}
\cholexample{Mi kaj kñaxañ ts'ij ksi'.}
\exampletranslation{Primero voy a rajar mi leña.}

\entry{-ts'ijañ}
\nontranslationdef{Sufijo que se presenta con raíces adjetivas que indican color y se refiere a un cerro o a una peña.}

\entry{ts'ijb}
\partofspeech{s}
\onedefinition{1}
\spanishtranslation{letra}
\cholexample{Kolem mi imel its'ijb.}
\exampletranslation{Hace sus letras grandes.}
\onedefinition{2}
\spanishtranslation{escritura}
\cholexample{Ma'añik mi lakch'ämbeñ isujm its'ijb.}
\exampletranslation{No entendemos su escritura.}

\entry{*ts'ijbal}
\partofspeech{s}
\onedefinition{1}
\spanishtranslation{dibujo}
\cholexample{Ity'ojoljax its'ijbal ibujk.}
\exampletranslation{Es muy bonito el dibujo de su camisa.}
\onedefinition{2}
\spanishtranslation{color}
\cholexample{K'äñk'äñ its'ijbal jiñi muty.}
\exampletranslation{El color de ese pájaro es amarillo.}

\entry{ts'ijbañ}
\partofspeech{vt}
\spanishtranslation{escribir}
\cholexample{Yom mi ats'ijbañ ak'aba' ya' tyi' juñilel alum.}
\exampletranslation{Debes escribir tu nombre en el certificado de tu terreno.}
\variation{ts'ijbuñ}

\entry{ts'ijbaya}
\partofspeech{s}
\spanishtranslation{el que escribe}

\entry{ts'ijbujel}
\onedefinition{1}
\partofspeech{vi}
\spanishtranslation{escribir}
\cholexample{Weñ yujil ts'ijbujel tyi mákiña.}
\exampletranslation{Sabe escribir bien a máquina.}
\onedefinition{2}
\partofspeech{s}
\spanishtranslation{escrito}
\cholexample{Jiñi aktya jiñäch its'ijbujel sekretyario.}
\exampletranslation{La acta es el escrito del secretario.}
\alsosee{sts'ijbujel}

\entry{ts'ijbuñ}
\conjugationtense{variante}
\conjugationverb{ts'ijbañ}
\spanishtranslation{escribir}

\entry{ts'ijñ}
\partofspeech{s}
\spanishtranslation{yuca}
\clarification{planta}

\entry{ts'ijkityiñ}
\partofspeech{s}
\spanishtranslation{chicharra}
\clarification{insecto}

\entry{-ts'ijty}
\nontranslationdef{Sufijo numeral para contar lápices; p. ej.:}
\cholexample{Añ kcha'añ tyi kotyoty cha'ts'ijty lápiz.}
\exampletranslation{Tengo dos lápices en la casa.}

\entry{ts'ilikña}
\relevantdialect{Sab.}
\partofspeech{adv}
\spanishtranslation{fuertemente}
\clarification{llorando}
\cholexample{Ts'ilikña tyi uk'el jiñi ch'ityoñ.}
\exampletranslation{Ese chamaco está llorando fuertemente.}
\alsosee{wo'okña}

\entry{ts'iñ}
\relevantdialect{Sab.}
\partofspeech{adv}
\spanishtranslation{poquito}
\cholexample{Mi ts'iñ sajlik tyik japä.}
\exampletranslation{No tomé ni otro poquito.}

\entry{*ts'iñsaj}
\relevantdialect{Sab.}
\partofspeech{s}
\spanishtranslation{poquito}
\cholexample{Abeñoñtyo yambä its'iñsaj.}
\exampletranslation{Dame otro poquito.}

\entry{ts'iñtyäl}
\partofspeech{adv}
\spanishtranslation{así de delgado}
\cholexample{Che' jax ya' ts'iñtyäl jiñi x'ixik kome ma'añix woli tyi we'el.}
\exampletranslation{Esa mujer está así de delgada porque no come.}

\entry{ts'iplaw}
\partofspeech{adv}
\spanishtranslation{echando chispas}
\cholexample{Ts'iplaw tyak woli imel jiñi k'ajk.}
\exampletranslation{El fuego arde echando chispas.}

\entry{ts'ikijk}
\partofspeech{s}
\spanishtranslation{chiquitín}

\entry{ts'ir bu'ul}
\spanishtranslation{tipo de frijol chico}

\entry{ts'is}
\partofspeech{vt}
\spanishtranslation{costurar}
\cholexample{Mi kaj its'is kbujk.}
\exampletranslation{Va a costurar mi vestido.}

\entry{ts'islum}
\partofspeech{s}
\spanishtranslation{comején}
\clarification{hormiga blanca}

\entry{ts'itya'}
\onedefinition{1}
\partofspeech{adj}
\spanishtranslation{poco}
\cholexample{Kojach ts'itya' kom bu'ul.}
\exampletranslation{Sólo quiero pocos frijoles.}
\onedefinition{2}
\partofspeech{adv}
\spanishtranslation{casi}
\clarification{falta poco}
\cholexample{Ts'itya' yom cha'añ mi laj k'otyel tyi tyejklum.}
\exampletranslation{Ya casi llegamos al pueblo.}

\entry{ts'iwil}
\partofspeech{adj}
\spanishtranslation{mucho, bastante}
\cholexample{Ts'iwil lächix wiñikob che' tyi k'iñ.}
\exampletranslation{Hay mucha gente en la fiesta.}

\entry{ts'iwi'}
\partofspeech{s}
\spanishtranslation{chuy}
\clarification{hierba}

\entry{ts'i'}
\partofspeech{s}
\spanishtranslation{perro}

\entry{*ts'i'lel}
\partofspeech{s}
\onedefinition{1}
\spanishtranslation{fornicación}
\cholexample{Wajali tsa'tyo iweñ cha'le its'i'lel.}
\exampletranslation{Anteriormente cometió mucha fornicación.}
\onedefinition{2}
\spanishtranslation{adulterio}
\cholexample{Tsa' ityaja its'i'lel yik'oty yambä x'ixik.}
\exampletranslation{Cometió adulterio con otra mujer.}

\entry{ts'i'ts'i'ña}
\partofspeech{adj}
\spanishtranslation{piando}
\cholexample{Ts'i'ts'i'ña jiñi alä muty cha'añ añix iwi'ñal.}
\exampletranslation{Los pollitos están piando porque ya tienen hambre.}

\entry{ts'obokña}
\partofspeech{adj}
\spanishtranslation{humeante}
\cholexample{Ts'obokña jach ibuts'il jiñi cholel; ma'añik woli tyi pulel.}
\exampletranslation{La milpa sólo está humeante; no está quemándose.}

\entry{ts'ok}
\partofspeech{vt}
\spanishtranslation{reventar}
\clarification{lazo, hilo}
\cholexample{Tsa' ujtyi its'ok ilasojlel jiñi mula.}
\exampletranslation{Esa mula acaba de reventar la soga.}

\entry{ts'ojkel}
\partofspeech{vi}
\spanishtranslation{reventarse}
\clarification{alambre, cordón, hilo}
\cholexample{Muk'ix ikajel tyi ts'ojkel jiñi chij.}
\exampletranslation{El cordón ya se va a reventar.}
\alsosee{tyujk'el}

\entry{ts'omtye'}
\partofspeech{s}
\spanishtranslation{mástil}
\culturalinformation{Información cultural: Se coloca atravesado entre los padrones para amarrar el seto en una casa.}

\entry{*ts'omtye'lel}
\partofspeech{s}
\spanishtranslation{cinta de casa}
\cholexample{Its'omtye'lel jiñi otyoty jiñäch mu'bä laj k'äty käch tyi' chumtye'lelba' mi laj kächbeñ ibojtye'lel.}
\exampletranslation{La cinta es un palo que se amarra atravesado en el padrón donde se amarra el seto.}

\entry{ts'opts'opña}
\partofspeech{adv}
\spanishtranslation{chapoteando}
\cholexample{Ts'opts'opña woli ity'um ok'ol.}
\exampletranslation{Pasa chapoteando por el lodo.}

\entry{ts'oty}
\relevantdialect{Sab.}
\partofspeech{vt}
\onedefinition{1}
\spanishtranslation{torcer}
\cholexample{Woli its'oty ich'ajañ cha'añ mi ik'äñ ikäch ibojtye'lel iyotyoty.}
\exampletranslation{Está torciendo su mecate para usarlo al amarrar el seto de su casa.}
\onedefinition{2}
\spanishtranslation{cerrar con llave}
\cholexample{Tyi' tyäli its'oty iyotyoty.}
\exampletranslation{Vino a cerrar su casa con llave.}
\alsosee{k'ach}

\entry{ts'otyiñ}
\partofspeech{vt}
\spanishtranslation{retorcer}
\cholexample{Yom mi ats'otyiñ jiñi ch'ajañ cha'añ ma'añik mi its'ojkel.}
\exampletranslation{Hay que retorcer el mecate para que no se reviente.}

\entry{ts'otyol}
\partofspeech{adj}
\spanishtranslation{torcido}
\cholexample{Ts'otyol jiñi tye'.}
\exampletranslation{El árbol está torcido.}

\entry{ts'otyol metyel}
\spanishtranslation{torcido}
\cholexample{Ts'otyol metyel jiñi bij.}
\exampletranslation{Ese camino está muy torcido.}

\entry{ts'o'}
\defsuperscript{1}
\partofspeech{adv}
\nontranslationdef{Se relaciona con la forma redonda; p. ej.:}
\cholexample{Wolix its'o' ochel iwuty jiñi xñox.}
\exampletranslation{A ese anciano ya se le están sumiendo los ojos.}

\entry{ts'o'}
\defsuperscript{2}
\partofspeech{adv}
\nontranslationdef{Se relaciona con la manera de pisar una cosa blanda; p. ej.:}
\cholexample{Tsak ts'o' tyek'e tsuk.}
\exampletranslation{Pisé un ratón (cosa blanda).}

\entry{ts'ub}
\partofspeech{adj}
\spanishtranslation{flojo, perezoso}
\cholexample{Ts'ub jiñi ch'ityoñ, mach yomik e'tyel.}
\exampletranslation{Ese chamaco es perezoso; no quiere trabajar.}

\entry{ts'ubejñ}
\partofspeech{s}
\spanishtranslation{polvo}
\secondaryentry{*ts'ubejñal}
\secondpartofspeech{s}
\secondtranslation{su polvo}

\entry{*ts'ubeñal}
\partofspeech{s}
\spanishtranslation{polvo}
\clarification{que tiene un objeto}
\cholexample{Añ kabäl its'ubeñal jiñi ch'ak.}
\exampletranslation{La cama tiene mucho polvo.}

\entry{*ts'ubil}
\partofspeech{s}
\spanishtranslation{moronas de galletas}

\entry{ts'ubukña}
\partofspeech{adj}
\spanishtranslation{molido, fino}
\cholexample{Ts'u'bukña jiñi kajpe' che' chämeñ juch'bil.}
\exampletranslation{Cuando el café se remuele bien, sale muy fino.}

\entry{ts'ukul}
\partofspeech{adj}
\spanishtranslation{puntiagudo}
\cholexample{Ts'ukul iñi' tsa' imele imachity.}
\exampletranslation{Hizo la punta de su machete puntiaguda.}

\entry{ts'uj}
\partofspeech{vt}
\spanishtranslation{derramar un poco}
\cholexample{Tsa' ujtyi tyi ts'uj yajlel kas tyi lum.}
\exampletranslation{Se acaba de derramar un poco de petróleo en el suelo.}

\entry{ts'ujk}
\partofspeech{s}
\spanishtranslation{úvula}
\secondaryentry{*ts'ujkil lakbik'}
\secondpartofspeech{s}
\secondtranslation{úvula}

\entry{ts'ujlaw}
\partofspeech{adv}
\spanishtranslation{gota por gota}
\cholexample{Wolix tyi tyikiñ jiñi ja', ya ts'ujlaw jax woli tyi yajlel.}
\exampletranslation{Ya se va a acabar el agua, nada más está cayendo gota por gota.}

\entry{ts'ujlel}
\partofspeech{vi}
\spanishtranslation{pelarse}
\cholexample{Mi its'ujlel ipächälel laj k'äb yik'oty tyikäw ja'.}
\exampletranslation{La piel de la mano se pela con agua caliente.}

\entry{ts'ujlem}
\partofspeech{adj}
\spanishtranslation{pelado}
\clarification{piel, cáscara}

\entry{ts'ujtyäl}
\partofspeech{adj}
\spanishtranslation{poquito}
\clarification{líquido}
\cholexample{Che' jach ya ts'ujtyäl mi ijap sa'.}
\exampletranslation{Toma sólo un poquito de pozol.}

\entry{ts'ujts'uñ}
\partofspeech{vt}
\spanishtranslation{besar}
\cholexample{Jiñi x'ixik woli its'ujts'uñ iyalobil.}
\exampletranslation{Esa mujer está besando a su hijo.}

\entry{ts'ujy}
\partofspeech{adj}
\spanishtranslation{duro}
\clarification{cáscara de fruta}
\cholexample{Ts'ujyatyax ipaty jiñi alaxax che' tyikiñ.}
\exampletranslation{La cáscara de la naranja es muy dura cuando se seca.}

\entry{ts'ujyel}
\partofspeech{vi}
\spanishtranslation{colgarse}
\clarification{sube y baja}
\cholexample{Woli tyi ts'ujyel tyi' ñi' tye'.}
\exampletranslation{Está colgándose en la punta del árbol.}

\entry{ts'ul}
\onedefinition{1}
\partofspeech{adj}
\spanishtranslation{desnudo}
\clarification{de pelo}
\cholexample{Ts'ul jiñi chityam.}
\exampletranslation{Ese puerco está pelón (lit.: sin pelo).}
\onedefinition{2}
\partofspeech{adv}
\spanishtranslation{despegando}
\cholexample{Mi its'ul letsel ik' jiñi lámiña.}
\exampletranslation{El viento levanta la lámina, despegándola.}
\onedefinition{3}
\partofspeech{vt}
\spanishtranslation{pelar}
\cholexample{Mi lakts'ul ipächälel akax.}
\exampletranslation{Pelamos la piel de la res.}

\entry{ts'uñuñ}
\partofspeech{s}
\spanishtranslation{colibrí, chupaflor}
\clarification{ave}

\entry{ts'usub}
\partofspeech{s}
\spanishtranslation{uva}

\entry{ts'usubil}
\partofspeech{s}
\spanishtranslation{viña}

\entry{ts'usu'ol}
\relevantdialect{Sab.}
\partofspeech{s}
\spanishtranslation{viña}

\entry{ts'uts'ub}
\partofspeech{s}
\spanishtranslation{tejón}
\spanishtranslation{coatí}
\clarification{mamífero}

\entry{ts'uy}
\partofspeech{adv}
\nontranslationdef{Se relaciona con la idea de algo guindado; p. ej.:}
\cholexample{Ts'uy kächbil we'eläl tyi' mal otyoty.}
\exampletranslation{La carne está amarrada y guindada adentro de la casa.}

\entry{ts'uychokoñ}
\partofspeech{vt}
\spanishtranslation{colgar}
\cholexample{Yom mi ats'uychokoñ jiñi p'isoñib che' mi ap'is jiñi kajpe'.}
\exampletranslation{Tienes que colgar la romana cuando peses el café.}

\entry{ts'uye'el}
\partofspeech{adj}
\spanishtranslation{colgado de la mano}
\cholexample{Ts'uye'el icha'añ jump'ejl ch'ujm.}
\exampletranslation{Lleva una calabaza colgada de la mano.}

\entry{ts'uyiña}
\partofspeech{adv}
\nontranslationdef{Se relaciona con la manera de colgarse; p. ej.:}
\cholexample{Ts'uyiña jiñi max tyi ak'.}
\exampletranslation{El mono está colgándose en los bejucos.}

\entry{ts'u'}
\partofspeech{vt}
\spanishtranslation{chupar}
\cholexample{Che' jach yik'oty ipaty woli its'u' jiñi alaxax.}
\exampletranslation{Está chupando la naranja con la cáscara.}

\entry{ts'u' chab}
\spanishtranslation{oso hormiguero, chupamiel}
\clarification{mamífero}

\entry{*ts'u'lel}
\partofspeech{s}
\spanishtranslation{flojera}
\cholexample{Cha'añ its'u'lel ma'añik tsa' majli tyi e'tyel.}
\exampletranslation{No se fue a trabajar por flojera.}

\entry{ts'u' lukum}
\spanishtranslation{tipo de pájaro con pico largo}
\clarification{Tiene la cabeza blanca y amarilla, el cuerpo negro y amarillo arriba, y blanco abajo. Las alas son negras y amarillas, la cola verde y negra, y los pies blancos. Se encuentra en tierra fría.}

\entry{ts'u'ts'uñ}
\partofspeech{vt}
\spanishtranslation{chupar}
\clarification{jugo de fruta}
\cholexample{Woli its'i'ts'uñ lok'el iya'lel alaxax.}
\exampletranslation{Está chupando el jugo de la naranja.}

\entry{ts'u'umiyel}
\partofspeech{vi}
\spanishtranslation{secarse}
\cholexample{Wolix its'u'umiyel jiñi kajpe'.}
\exampletranslation{El café se está secando.}

\alphaletter{U}

\entry{ubiñ}
\partofspeech{vt}
\onedefinition{1}
\spanishtranslation{escuchar}
\cholexample{Ubiñbajche' woli isubeñety.}
\exampletranslation{Escucha lo que te está diciendo.}
\onedefinition{2}
\spanishtranslation{sentir}
\cholexample{Ma'añik k'ux mi iyubiñ.}
\exampletranslation{No siente si le duele.}

\entry{ubiñtyel}
\partofspeech{vi}
\onedefinition{1}
\spanishtranslation{oír}
\cholexample{Ma'añik tsa' ubiñtyibaki ora tsa' tyili.}
\exampletranslation{No se oyó cuando vino.}
\onedefinition{2}
\spanishtranslation{saber}
\cholexample{Ma'añik woli tyi ubiñtyel tyi'tyojlel jiñi yumulobbajche' wolik tyik'läñtyel.}
\exampletranslation{Las autoridades no saben que soy explotado.}
\onedefinition{3}
\spanishtranslation{sentir}
\cholexample{Ch'ijiyem tsa' ubiñtyi cha'añ jiñi tsa'bä subeñtyi.}
\exampletranslation{Está sentido por lo que le fue dicho.}

\entry{uk'tyañ}
\partofspeech{vt}
\spanishtranslation{lamentar}
\cholexample{Woli iyuk'tyañ iyalobil tsa'bä chämi.}
\exampletranslation{Está lamentándose por su hijo que murió.}

\entry{uk'um}
\partofspeech{s}
\spanishtranslation{cántaro}

\entry{uch}
\partofspeech{s}
\spanishtranslation{tlacuache}
\clarification{mamífero}

\entry{uchchañ}
\partofspeech{s}
\spanishtranslation{boa}
\spanishtranslation{mazacoatl}
\clarification{reptil}

\entry{uch'}
\partofspeech{s}
\spanishtranslation{piojo}
\clarification{insecto}

\entry{uch'el}
\partofspeech{s}
\onedefinition{1}
\spanishtranslation{acto de tomar o beber}
\cholexample{Yom lakcha'leñ uch'el ya' tyi ja'.}
\exampletranslation{Vamos a tomar (lit.: hacemos el acto de beber) algo allá en el arroyo}
\onedefinition{2}
\spanishtranslation{acto de comer}
\cholexample{Iyorajlelix mi lakmajlel tyi uch'el.}
\exampletranslation{Ya es hora de ir a comer (lit.: para hacer el acto de comer).}

\entry{uch'eñ}
\partofspeech{vt}
\onedefinition{1}
\spanishtranslation{tomar}
\clarification{agua, café, pozol, refresco, aguardiente}
\cholexample{Yom mi kuch'eñlasa'.}
\exampletranslation{Vamos a tomar pozol.}
\onedefinition{2}
\spanishtranslation{comer}
\cholexample{Maxtyo añik uch'em.}
\exampletranslation{Todavía no he comido.}

\entry{uch'ibäl}
\partofspeech{s}
\spanishtranslation{taza}

\entry{uch'ijibäl}
\partofspeech{s}
\spanishtranslation{trastos}

\entry{uch'ja'}
\partofspeech{s}
\spanishtranslation{zancudo}
\clarification{insecto}

\entry{uch'uñtye'}
\partofspeech{s}
\spanishtranslation{papaya}

\entry{ujañ}
\partofspeech{s}
\spanishtranslation{gargantillas}
\cholexample{Jiñi xch'ok woli iyujañ ich'äjlil ibik'.}
\exampletranslation{Esa muchacha está añadiendo gargantillas a su collar.}

\entry{ujax}
\partofspeech{part}
\spanishtranslation{mira}
\clarification{no más}
\cholexample{Ujax kilal cha'añ tyi amul.}
\exampletranslation{Mira cómo estoy por tu culpa.}

\entry{ujäl}
\partofspeech{s}
\spanishtranslation{gargantilla}

\entry{ujkuts}
\partofspeech{s}
\spanishtranslation{paloma}

\entry{ujchib}
\partofspeech{s}
\spanishtranslation{uco, agutí}
\clarification{mamífero}

\entry{*ujil}
\partofspeech{vt}
\spanishtranslation{saber}
\cholexample{Joñoñ kujil e'tyel.}
\exampletranslation{Yo sé trabajar.}
\cholexample{Jatyety awujil jats' kityara.}
\exampletranslation{Tú sabes tocar la guitarra.}
\cholexample{Jiñi wiñik yujil imelol rebosäl.}
\exampletranslation{Ese hombre sabe hacer rebozos.}

\entry{ujrich'}
\partofspeech{s}
\spanishtranslation{loro verde}
\clarification{ave}

\entry{ujtyel}
\partofspeech{vi}
\spanishtranslation{terminarse}
\cholexample{Muk'ix ikajel tyi ujtyel jiñi e'tyel.}
\exampletranslation{El trabajo ya se va a terminar.}

\entry{ujtyesañ}
\partofspeech{vt}
\spanishtranslation{terminar}
\cholexample{Mik majlel kujtyesañ kchobal.}
\exampletranslation{Voy a terminar mi rozadura.}

\entry{ujtyuy}
\partofspeech{s}
\spanishtranslation{tipo de árbol}
\clarification{no se seca, de madera colorada; sirve para postes}

\entry{ujts'il}
\partofspeech{s}
\spanishtranslation{olor}
\cholexample{Leko iyuts'il jiñi pajäy.}
\exampletranslation{El zorrillo tiene mal olor.}

\entry{ujts'iñ}
\partofspeech{vt}
\spanishtranslation{saludar}
\clarification{besando la mano o tocando la frente}
\cholexample{Jiñi xch'ok mi iyujts'ibeñ ik'äb jiñi x'ixik.}
\exampletranslation{La muchacha besa la mano de la mujer.}

\entry{ujux chaxtyäl k'iñ}
\spanishtranslation{¡mira qué alto está ya el sol!}

\entry{ul}
\partofspeech{s}
\spanishtranslation{atole}
\clarification{de masa}

\entry{-ul}
\nontranslationdef{Sufijo que se presenta con raíces transitivas y neutras para formar una raíz atributiva que indica posición; p. ej.:}
\cholexample{ñup'ul}
\exampletranslation{cerrado.}
\variation{2*-al, -äl, 2*-el, 1*-ol}

\entry{-ulañ}
\nontranslationdef{Sufijo que se presenta con raíces neutras para formar una raíz transitiva que indica movimiento; p. ej.:}
\cholexample{lämulañ}
\exampletranslation{agitar.}

\entry{ulej}
\partofspeech{s esp}
\spanishtranslation{tirador}

\entry{ulmäl}
\partofspeech{vi}
\spanishtranslation{derretirse}
\cholexample{Jiñi tyuñija' mi iyulmäl tyi ora.}
\exampletranslation{El hielo se derrite luego.}

\entry{ulmesañ}
\partofspeech{vt}
\spanishtranslation{derretir}
\cholexample{Jiñi k'ajk mi iyulmesañ jiñi ñichim.}
\exampletranslation{El fuego derrite la vela.}

\entry{ulukña}
\partofspeech{adj}
\spanishtranslation{liso}
\cholexample{Ulukña jiñi tye' che' weñ juk'bil.}
\exampletranslation{La madera es lisa cuando está bien cepillada.}

\entry{ul'ulña}
\partofspeech{adv}
\nontranslationdef{Se relaciona con la forma de derretirse; p. ej.:}
\cholexample{Ul'ulña jiñi ñichim cha'añ ts'äbil.}
\exampletranslation{La vela se agota derritiéndose cuando está encendida.}

\entry{um}
\conjugationtense{variante}
\conjugationverb{äj}
\spanishtranslation{aquí}

\entry{uma'}
\partofspeech{s}
\spanishtranslation{mudo}

\entry{um ba'añ}
\spanishtranslation{aquí está}
\cholexample{Umba'añ awaj.}
\exampletranslation{Aquí están tus tortillas.}

\entry{*um ja'}
\partofspeech{vt}
\spanishtranslation{enjuagar}
\clarification{la boca}
\cholexample{Yom mi lakum ja' che' mi isäk'añ.}
\exampletranslation{Debemos enjuagarnos la boca al amanecer.}

\entry{-uñ}
\nontranslationdef{Sufijo que se presenta con raíces transitivas y neutras para formar una raíz transitiva que indica movimiento; p. ej.:}
\cholexample{jaxuñ}
\exampletranslation{enrollar.}

\entry{uñix}
\partofspeech{s}
\spanishtranslation{perico}
\clarification{ave}

\entry{uñtye'}
\partofspeech{s}
\spanishtranslation{palo de humo}
\spanishtranslation{laurel}
\clarification{árbol}

\entry{uñ tsa'}
\spanishtranslation{aquí está, mira}
\cholexample{Uñ tsa' mi atsa' awomik.}
\exampletranslation{¡Mira!, ya ves que no quieres.}

\entry{uñx}
\partofspeech{part}
\spanishtranslation{ves}
\cholexample{Uñx awilal.}
\exampletranslation{Ves cómo estás.}

\entry{uñe'}
\partofspeech{s}
\spanishtranslation{criatura chiquita que llora}

\entry{uk'el}
\onedefinition{1}
\partofspeech{s}
\spanishtranslation{llanto}
\onedefinition{2}
\partofspeech{s}
\spanishtranslation{canto}
\clarification{radio, tocadisco, gallo}
\onedefinition{3}
\partofspeech{vi}
\spanishtranslation{llorar}
\cholexample{Woli tyi uk'el cha'añ tsa' ijats'äyob.}
\exampletranslation{Está llorando porque le pegaron.}

\entry{us}
\partofspeech{s}
\spanishtranslation{mosca}

\entry{uts}
\partofspeech{adj}
\onedefinition{1}
\spanishtranslation{bueno}
\cholexample{Uts jiñi wiñik.}
\exampletranslation{Ese hombre es bueno.}
\onedefinition{2}
\spanishtranslation{contento}
\cholexample{Utsix jiñi alob.}
\exampletranslation{Ese chamaco ya está contento.}
\dialectvariant{Sab.}
\dialectword{utyesañ}

\entry{utsbiñ}
\partofspeech{vt}
\spanishtranslation{amansar}
\cholexample{Woli iyutsbiñ imula.}
\exampletranslation{Está amansando su mula.}

\entry{utsesañ}
\relevantdialect{Sab.}
\partofspeech{vt}
\spanishtranslation{contentar}
\cholexample{Mach yom mich'ety, yom mi autsesañ abä.}
\exampletranslation{No estés enojado, debes contentarte.}

\entry{utsi}
\partofspeech{adv}
\spanishtranslation{cada vez más}
\cholexample{Utsi woli ik'am'añ.}
\exampletranslation{Se pone más grave cada vez.}

\entry{utsil}
\relevantdialect{Tila}
\partofspeech{adj}
\spanishtranslation{despejado, abierto}
\clarification{tiempo}
\cholexample{Utsilix pañimil.}
\exampletranslation{Ya está despejado.}
\alsosee{jamäl}

\entry{uts'aty}
\onedefinition{1}
\partofspeech{adv}
\spanishtranslation{bien}
\cholexample{Uts'atybajche' tsa' mele.}
\exampletranslation{Está bien como lo hiciste.}
\onedefinition{2}
\partofspeech{adj}
\spanishtranslation{bueno, sí (una repuesta)}
\cholexample{Uts'aty, mi kaj kmajlel awik'oty.}
\exampletranslation{Bueno, voy contigo.}

\entry{uts'atyiyel}
\relevantdialect{Sab.}
\partofspeech{vi}
\onedefinition{1}
\spanishtranslation{recuperarse}
\cholexample{Jiñi wiñik tyi' cha' uty'atyiyi.}
\exampletranslation{Ese hombre ya se recuperó.}
\onedefinition{2}
\spanishtranslation{componerse}
\clarification{el tiempo}
\cholexample{Tsa'ix uts'atyiyi, ma'añix mi icha'leñ ja'al.}
\exampletranslation{Ya se compuso el tiempo; ya no va a llover.}

\entry{uw}
\partofspeech{s}
\onedefinition{1}
\spanishtranslation{luna}
\cholexample{Yom laktsep tye' cha'añ oy che' tyi' chämel uw.}
\exampletranslation{Hay que cortar los palos para horcones cuando la luna se muere.}
\onedefinition{2}
\spanishtranslation{mes}
\cholexample{Jiñi cholel mi ikolel tyi cha'p'ejl jach uw.}
\exampletranslation{La milpa crece en sólo dos meses.}
\secondaryentry{iyuwil}
\secondpartofspeech{s}
\secondtranslation{su mes}

\entry{-ux}
\conjugationtense{variante}
\conjugationverb{-ox}
\nontranslationdef{Sufijo que se presenta con raíces atributivas para formar otra raíz atributiva que indica condición defectiva; p. ej.:}
\cholexample{bujlux}
\exampletranslation{maíz con pocos granos.}

\entry{*uxchajplel}
\partofspeech{s}
\spanishtranslation{tercer tipo}
\cholexample{Iyuxchajplel e'tyel cha'añ cholel jiñäch ak'iñ.}
\exampletranslation{El tercer tipo de trabajo en la milpa es la limpia.}

\entry{uxi}
\partofspeech{adv}
\spanishtranslation{de aquí a tres días}

\entry{uxix}
\relevantdialect{Tila}
\partofspeech{s}
\spanishtranslation{pajarera}
\clarification{culebra arbórea}

\entry{uxlajm}
\partofspeech{adj}
\spanishtranslation{de tres pisos}
\cholexample{Uxlajm iyotyoty tsa' imele.}
\exampletranslation{Hizo su casa de tres pisos.}

\entry{uxlujump'ejl}
\partofspeech{adj}
\spanishtranslation{trece}

\entry{uxp'ejl}
\partofspeech{adj}
\spanishtranslation{tres}

\entry{uxp'ejk'iñ}
\partofspeech{s}
\spanishtranslation{miércoles, media semana}

\entry{uxwäyel}
\partofspeech{adv}
\spanishtranslation{tres noches}
\clarification{de dormir}
\cholexample{Uxwäyel tsak ñusa ya tyi matye'el.}
\exampletranslation{Dormí tres noches en el monte.}

\entry{uxyajl}
\partofspeech{adv}
\spanishtranslation{tres veces}
\cholexample{Uxyajl mi ik'ajtyiñtyojoñel tyi jump'ejl jab.}
\exampletranslation{Tres veces al año pide impuestos.}

\entry{uxyi}
\relevantdialect{Sab.}
\partofspeech{adv}
\spanishtranslation{en tres días}
\cholexample{Uxyityo mik majlel tyi lum.}
\exampletranslation{De aquí a tres días iré al pueblo.}

\entry{uya'}
\partofspeech{s}
\spanishtranslation{arete}

\entry{uyuj}
\partofspeech{s}
\spanishtranslation{mico de noche}
\clarification{mamífero}

\entry{u'yaj}
\partofspeech{s}
\spanishtranslation{chisme}
\cholexample{Mi imulañ u'yaj jiñi x'ixik.}
\exampletranslation{A esa mujer le gusta decir chismes.}
\dialectvariant{Tila}
\dialectword{choñtyi'}

\alphaletter{W}

\entry{wakal}
\partofspeech{adj}
\spanishtranslation{sin pantalón}
\cholexample{Wakal jiñi ch'ityoñ.}
\exampletranslation{Ese chamaco anda sin pantalones.}

\entry{wakax}
\partofspeech{s}
\spanishtranslation{toro, vaca}

\entry{waj}
\partofspeech{s}
\spanishtranslation{tortilla}
\dialectvariant{Sab.}
\dialectword{k'äñwaj}
\clarification{amarilla}

\entry{wajal}
\partofspeech{s}
\spanishtranslation{burla}
\cholexample{Mi icha'leñ wajal che' mach weñik lakotyoty.}
\exampletranslation{Él nos hace burla cuando nuestra casa no está arreglada.}

\entry{wajalix}
\partofspeech{adv}
\spanishtranslation{hace tiempo, anteriormente}
\cholexample{Wajalix tsa' cha' juliyoñ.}
\exampletranslation{Ya hace tiempo que regresé.}

\entry{wajawajal}
\relevantdialect{Sab.}
\partofspeech{adv}
\spanishtranslation{orgullosamente}
\cholexample{Wajawajal mi isutyk'isañ ibä.}
\exampletranslation{Orgullosamente vuelve la espalda.}

\entry{wajleñ}
\partofspeech{vt}
\spanishtranslation{burlar}
\cholexample{Mi iwajleñoñlache' ma'añik laktyak'iñ.}
\exampletranslation{Se burlan de nosotros cuando no tenemos dinero.}

\entry{wajmäl}
\partofspeech{s}
\spanishtranslation{compañero}
\clarification{expresión ofensiva}
\cholexample{Ma'añik chuki yujil awajmäl.}
\exampletranslation{Tu compañero no sabe nada.}

\entry{wajpam}
\partofspeech{s}
\spanishtranslation{barro}
\clarification{en la cara}
\cholexample{Kabäl jax awajam.}
\exampletranslation{Tienes mucho barro en la cara.}

\entry{wajtyañ}
\partofspeech{s}
\spanishtranslation{elote}

\entry{wajyuñ}
\partofspeech{vt}
\spanishtranslation{lavarse}
\clarification{la cara}
\cholexample{Yom mi awajyuñ awuty.}
\exampletranslation{Debes lavarte la cara.}

\entry{waj'um}
\relevantdialect{Tila}
\partofspeech{s}
\spanishtranslation{planta o caldo que tiene <ña'al>}
\clarification{el espíritu de la abundancia}
\culturalinformation{Información cultural: Se dice que estos espíritus tienen distintos poderes. Pueden producir bastante maíz en una parte de una milpa. Los más poderosos son de color verde-gris y habitan en las cuevas.}

\entry{walk'uñ}
\partofspeech{vt}
\spanishtranslation{mezclar}
\clarification{maíz con frijol}
\cholexample{Mi lakwalk'uñ ixim yik'oty bu'ul.}
\exampletranslation{Mezclamos maíz con frijol.}

\entry{walts'uñ}
\partofspeech{vt}
\spanishtranslation{mezclar}
\clarification{con condimento}
\cholexample{Yom mi awalts'uñ we'eläl tyi ich.}
\exampletranslation{Hay que mezclar el chile con la carne.}

\entry{wamal}
\partofspeech{adj}
\spanishtranslation{amontonado}
\clarification{muchas cosas}
\cholexample{Wamal jiñi pisil tyi otyoty.}
\exampletranslation{La ropa está amontonada en la casa.}

\entry{wamlaw}
\partofspeech{adj}
\spanishtranslation{turbulento}
\cholexample{Ya'baki wamlaw jiñi ja' mach mejlik lakñumel.}
\exampletranslation{No podemos pasar donde el agua está turbulenta.}
\variation{watylaw}

\entry{wamtyäl}
\partofspeech{adv}
\onedefinition{1}
\spanishtranslation{así de volumen}
\cholexample{Che' wamtyäl tsa' ich'ämä majlel pisil.}
\exampletranslation{Así era la cantidad de ropa que llevó.}
\onedefinition{2}
\spanishtranslation{así de largo}
\clarification{el pelo}
\cholexample{Che' wamtyäl ijol.}
\exampletranslation{Así de largo es su pelo.}

\entry{wañku}
\relevantdialect{Sab.}
\partofspeech{s esp}
\spanishtranslation{banca}

\entry{warach}
\partofspeech{s esp}
\spanishtranslation{huarache}

\entry{wasil}
\partofspeech{s esp}
\onedefinition{1}
\spanishtranslation{alguacil}
\culturalinformation{Información cultural: Es el encargado de entregar cartas y citatorios a las personas que cometen algún delito, y también de detener a las personas por cualquier delito.}
\onedefinition{2}
\spanishtranslation{auxiliar de ayuntamiento}

\entry{watyax}
\partofspeech{v irr}
\spanishtranslation{revolcarse}
\cholexample{Wolil tyi watyax jiñi alob.}
\exampletranslation{Ese niño se está revolcando.}

\entry{watylaw}
\conjugationtense{variante}
\conjugationverb{wamlaw}
\spanishtranslation{turbulento}

\entry{wats}
\partofspeech{s}
\spanishtranslation{plátano endosado}

\entry{-watsañ}
\nontranslationdef{Sufijo que se presenta con raíces adjetivas que indican color.}

\entry{watstyäl}
\partofspeech{adv}
\spanishtranslation{así de cantidad de broza o pelo}
\cholexample{Che'tyo watstyäl ik'u' cholel.}
\exampletranslation{Así es el montón de broza de la milpa.}

\entry{waw}
\partofspeech{s esp}
\spanishtranslation{guao}
\spanishtranslation{tres lomas}
\clarification{tortuga}

\entry{wax}
\partofspeech{s}
\spanishtranslation{zorra gris}
\clarification{mamífero}

\entry{-waxañ}
\nontranslationdef{Sufijo que se presenta con raíces adjetivas que indican color y se refiere a la cara.}

\entry{waxäklujump'ejl}
\partofspeech{adj}
\spanishtranslation{dieciocho}

\entry{waxäkñij}
\partofspeech{adv}
\spanishtranslation{octavo día}

\entry{waxäkp'ejl}
\partofspeech{adj}
\spanishtranslation{ocho}

\entry{waxk'uñ}
\partofspeech{vt}
\spanishtranslation{rodar}
\clarification{piedra, bola}
\cholexample{Woli iwaxk'uñ jubel xajlel ya' tyi wits.}
\exampletranslation{Está rodando una piedra del cerro.}

\entry{way ja'as}
\partofspeech{s}
\spanishtranslation{zapote}
\clarification{árbol}

\entry{wa'}
\relevantdialect{Sab.}
\partofspeech{adv}
\spanishtranslation{luego}
\cholexample{Mi iwa' tyälel yujkel che' mach ña'tyäbil lakcha'añ.}
\exampletranslation{Luego el temblor viene cuando menos lo pensamos.}
\alsosee{tyi ora}

\entry{wa'akñiyel}
\partofspeech{vi}
\spanishtranslation{pasear}
\cholexample{Woli ityi wa'akñiyel ya' tyi tyejklum.}
\exampletranslation{Está paseándose en el pueblo.}
\dialectvariant{Sab.}
\dialectword{chätyañ}

\entry{wa'al}
\partofspeech{adj}
\spanishtranslation{parado}
\cholexample{Ibajñel jach ya' wa'al.}
\exampletranslation{Él solo está parado.}

\entry{-wa'añ}
\nontranslationdef{Sufijo que se presenta con raíces adjetivas que indican color, y se aplica a toda la ropa que lleva una persona.}

\entry{wa'chokoñ}
\partofspeech{vt}
\onedefinition{1}
\spanishtranslation{edificar, levantar}
\clarification{casa}
\cholexample{Woli iwa'chokoñ iyotyoty.}
\exampletranslation{Está levantando su casa.}
\onedefinition{2}
\spanishtranslation{nombrar}
\clarification{a un puesto}
\cholexample{Tsa' iwa'chokoyob komityé.}
\exampletranslation{El comité fue nombrado.}

\entry{*wa'lib}
\partofspeech{s}
\onedefinition{1}
\spanishtranslation{andamiada, miradero}
\clarification{lugar donde el cazador espera la caza}
\onedefinition{2}
\spanishtranslation{andamiaje}
\cholexample{Jiñi karpiñtyero ñaxañ tsa' imele iwa'lib otyoty.}
\exampletranslation{El carpintero hizo primero el andamiaje de la casa.}

\entry{wa'tyäl}
\partofspeech{vi}
\spanishtranslation{pararse}
\cholexample{Jiñi aläl maxtyo añik mi imejlel tyi wa'tyäl.}
\exampletranslation{La criatura todavía no puede pararse.}

\entry{*wa'tyilel}
\partofspeech{s}
\onedefinition{1}
\spanishtranslation{altura}
\cholexample{Iwa'tyilel jiñi wiñik jiñäch cha'p'ejl metyro.}
\exampletranslation{Ese hombre tiene dos metros de altura.}
\onedefinition{2}
\spanishtranslation{tiro}
\clarification{de pantalón}
\cholexample{Tyam iwa'tyilel jiñik wex.}
\exampletranslation{El tiro de mi pantalón es largo.}

\entry{wa'wa'ña}
\partofspeech{adv}
\spanishtranslation{paseando tranquilamente}
\cholexample{Wa'wa'ña jiñi wiñik tyi mal cholel.}
\exampletranslation{El hombre está paseando tranquilamente en su milpa.}

\entry{wäklujump'ejl}
\partofspeech{adj}
\spanishtranslation{dieciséis}

\entry{wäkñij}
\partofspeech{adv}
\spanishtranslation{de hoy en seis días}

\entry{wäkp'ejl}
\partofspeech{adj}
\spanishtranslation{seis}

\entry{wäch'}
\partofspeech{s}
\spanishtranslation{guapaque}
\clarification{árbol}

\entry{wäläk}
\partofspeech{adj}
\spanishtranslation{mismo}
\cholexample{Tsa' iwäläk tsepe ibä.}
\exampletranslation{Él mismo se cortó.}

\entry{wäläk paty ok}
\spanishtranslation{espíritu malo}
\culturalinformation{Información cultural: Se dice que es peligroso quedarse solo en la casa o en el camino, porque el <wʌlʌc pat oc> viene hablando y muerde la punta de nuestra lengua de la cual emana cantidad de sangre. Después nos deja muertos. Sucede que el mismo espíritu aparece en la selva como perdiz, chachalaca, etc. Entonces la persona va siguiendo al animal con el fin de cazarlo, pero no se imagina que al estar siguiéndolo queda desorientado, sin saber por dónde seguir.}

\entry{wäle}
\partofspeech{adv}
\spanishtranslation{ahora, hoy}
\cholexample{Wäle samiyoñ tyi cholel.}
\exampletranslation{Hoy voy a mi milpa.}

\entry{wälilañ}
\partofspeech{vt}
\spanishtranslation{criticar}
\cholexample{Lakpi'älob mi iwälilañoñla.}
\exampletranslation{Nuestros compañeros nos critican.}

\entry{wälwäl ty'añ}
\spanishtranslation{murmuración}

\entry{wäñ}
\partofspeech{adv}
\spanishtranslation{de antemano}
\cholexample{Mi iwäñ k'ajtyiñbajche' añ ityojol.}
\exampletranslation{De antemano pregunta cuál es el precio.}

\entry{wäts'}
\partofspeech{vt}
\spanishtranslation{voltear}
\clarification{objeto}
\cholexample{Tsa' ujtyi kwäts' kbujk.}
\exampletranslation{Acabo de voltear mi camisa.}

\entry{wäts'äl}
\partofspeech{adj}
\spanishtranslation{volteado}
\cholexample{Wäts'äl jiñi koxtyal.}
\exampletranslation{El costal está volteado.}

\entry{wäy}
\partofspeech{s}
\spanishtranslation{compañero}
\clarification{un espíritu}
\cholexample{Iwäy ktatuch jiñäch juñkojtybajlum.}
\exampletranslation{El compañero de mi abuelo es un jaguar.}
\culturalinformation{Información cultural: Esta creencia es semejante a la creencia de la tona que hay en Oaxaca.}

\entry{wäyäl}
\partofspeech{adj}
\spanishtranslation{dormido}

\entry{wäyel}
\partofspeech{vi}
\spanishtranslation{dormir}

\entry{wäyib}
\partofspeech{s}
\spanishtranslation{cama}
\clarification{movible}
\secondaryentry{wäyib aläl}
\secondtranslation{cuna}

\entry{wäyibäl}
\partofspeech{s}
\spanishtranslation{cama}
\clarification{movible}

\entry{wäysañ}
\partofspeech{vt}
\spanishtranslation{adormecer}

\entry{wäytyañ}
\partofspeech{vt}
\spanishtranslation{usar para dormir}
\cholexample{Yom mi awäytyañ koxtyal.}
\exampletranslation{Debes usar el costal para dormir.}

\entry{wäytyäbil}
\partofspeech{adj}
\spanishtranslation{usado}
\clarification{para dormir}
\cholexample{Wäytyäbil jiñi koxtyal.}
\exampletranslation{Ese costal es usado para dormir.}

\entry{wäytyesañ}
\partofspeech{vt}
\spanishtranslation{adormecer}

\entry{wä'}
\partofspeech{adv}
\spanishtranslation{aquí}
\cholexample{Wä' mi acha' tyilel.}
\exampletranslation{Vas a volver a venir aquí.}

\entry{wä'añ}
\partofspeech{adv}
\spanishtranslation{aquí está}
\cholexample{Añix jump'ejl jab wä'añ kik'oty.}
\exampletranslation{Ya tiene un año que está aquí conmigo.}

\entry{wä'i}
\partofspeech{adv}
\spanishtranslation{aquí}
\cholexample{Yom mi ach'ämbeñoñ tyilel postye wä'i.}
\exampletranslation{Debes traerme aquí el poste.}

\entry{wek'uña}
\partofspeech{adv}
\spanishtranslation{de corriente rápida}
\cholexample{Wek'uña mi ik'otyel jiñi ja' ya' tyi' tyi' ja'.}
\exampletranslation{De corriente rápida llega el torrente a la orilla del río.}

\entry{wech}
\defsuperscript{1}
\partofspeech{adv}
\nontranslationdef{Se relaciona con la forma de tomar una cosa plana; p. ej.:}
\cholexample{Jiñi ts'i' tsa' iwech k'uxu majlel waj.}
\exampletranslation{El perro se llevó la tortilla para comérsela.}

\entry{wech}
\defsuperscript{2}
\partofspeech{s}
\spanishtranslation{armadillo}
\clarification{mamífero}
\variation{xwech}
\dialectvariant{Tila}
\dialectword{ib}

\entry{wechekña}
\partofspeech{adj}
\spanishtranslation{bien plano}
\cholexample{Wechekña jiñi tyabla.}
\exampletranslation{Esa tabla está bien plana.}

\entry{wechel}
\partofspeech{adj}
\spanishtranslation{plano}
\clarification{libro, papel, tortilla, tabla}
\cholexample{Wechel jiñi waj.}
\exampletranslation{La tortilla es plana.}

\entry{wechtyäl}
\partofspeech{adv}
\spanishtranslation{así de ancho}
\cholexample{Che'tyo' wechtyäl ipuertyajlel kotyoty.}
\exampletranslation{Así de ancha es la puerta de mi casa.}

\entry{wech wech}
\spanishtranslation{ancho}
\cholexample{Wech wech lamiña.}
\exampletranslation{Esa lámina es muy ancha.}

\entry{wejk'añ}
\partofspeech{vt}
\spanishtranslation{sacar}
\clarification{agua}
\cholexample{Yom mi awejk'añ lok'el ja' ya'ba' tsa' ochi.}
\exampletranslation{Hay que sacar el agua de donde entró.}

\entry{wejch'uñ}
\partofspeech{vt}
\spanishtranslation{regar}
\clarification{semilla}
\cholexample{Mi kajel kwejch'uñ ibäk' ich ya'ba' ak'ñäbil kcha'añ.}
\exampletranslation{Voy a regar semilla de chile en donde tengo limpiado.}

\entry{-wejl}
\nontranslationdef{Sufijo numeral para contar lados; p. ej.:}
\cholexample{Tyi juñwejl jiñi tyablamach wersajik mi lakjuk'.}
\exampletranslation{Del otro lado de la tabla no se necesita cepillar.}

\entry{wejlañ}
\partofspeech{vt}
\spanishtranslation{soplar}
\clarification{con abanico}
\cholexample{Yom mi awejlañ k'ajk cha'añ mi ilejmel.}
\exampletranslation{Debes soplar el fuego para que arda.}

\entry{wejlel}
\partofspeech{vi}
\spanishtranslation{volar}
\cholexample{Mach mejlix tyi wejlel jiñi muty kome tsa' xujli iwich'.}
\exampletranslation{El pájaro ya no puede volar porque se le quebró el ala.}

\entry{*wejlib ja'}
\spanishtranslation{caída de agua}

\entry{wejlujel}
\partofspeech{vi}
\spanishtranslation{chaporrear}
\cholexample{Jiñi wiñik tsa' majli tyi wejlujel.}
\exampletranslation{Ese hombre fue a chaporrear.}

\entry{wejluñ}
\defsuperscript{1}
\partofspeech{vt}
\spanishtranslation{mecer}
\cholexample{Woli iwejluñ majlel imachity jiñi xyäk'äjel.}
\exampletranslation{Ese borracho va meciendo su machete.}

\entry{wejluñ}
\defsuperscript{2}
\partofspeech{vt}
\spanishtranslation{chaporrear}
\cholexample{Woli iwejluñ pimel tyi yebal cholel.}
\exampletranslation{Está chaporreando la hierba en su milpa.}

\entry{-wejty}
\nontranslationdef{Sufijo numeral para contar tazas; p. ej.:}
\cholexample{Tsak ch'ämä juñxejty ktyaza.}
\exampletranslation{Compré una taza.}

\entry{wejtyuñ}
\partofspeech{vt}
\spanishtranslation{regar, sembrar}
\clarification{regado}
\cholexample{Mi lakwejtyuñ iwuty jam.}
\exampletranslation{Regamos la semilla de zacate.}

\entry{wejtyuyel}
\partofspeech{vi}
\spanishtranslation{caerse}
\clarification{arroz, frijol, maíz}
\cholexample{Woli iwejtyuyel lok'el bu'ul tyi koxtyal kome buty'ul.}
\exampletranslation{Está cayéndose el frijol del costal porque está lleno.}

\entry{wel}
\partofspeech{adv}
\nontranslationdef{Se relaciona con la forma de tirar un objeto plano; p. ej.:}
\cholexample{Tsa' iwel choko juñ tyi lum.}
\exampletranslation{Tiró un papel en el suelo.}

\entry{-welañ}
\nontranslationdef{Sufijo que se presenta con raíces adjetivas que indican color y se refiere a papel o tela.}

\entry{welekña}
\partofspeech{adv}
\onedefinition{1}
\spanishtranslation{volando}
\cholexample{Welekña tsa' majli jiñi xäye'.}
\exampletranslation{El águila se fue volando.}
\onedefinition{2}
\spanishtranslation{yendo}

\entry{welel}
\partofspeech{adj}
\spanishtranslation{tendido, plano}
\cholexample{Ya'añ juñ welel tyi pam mexa.}
\exampletranslation{Allí sobre la mesa está extendido el papel.}

\entry{welñäk'}
\partofspeech{s}
\spanishtranslation{delantal}

\entry{weltyäl}
\partofspeech{adv}
\nontranslationdef{Se relaciona con la forma de un objeto plano y ancho; p. ej.:}
\cholexample{Che' weltyäl yom mi ach'ämbeñoñ tyilel jiñi tyabla.}
\exampletranslation{Me traes la tabla así de ancha.}

\entry{*weltyiklel}
\partofspeech{s}
\spanishtranslation{anchura}

\entry{weluña}
\partofspeech{ve}
\spanishtranslation{meneando}
\cholexample{Weluña jiñi pisil cha'añ ik'.}
\exampletranslation{La ropa se está meneando por el viento.}

\entry{welux}
\partofspeech{s esp}
\spanishtranslation{puerro}
\clarification{planta semejante a la cebolla}

\entry{welwelña}
\partofspeech{ve}
\spanishtranslation{volando}
\cholexample{Welwelña woli ityilel jiñi xäye'.}
\exampletranslation{Volando está viniendo ese gavilán.}

\entry{weñ}
\partofspeech{adv}
\spanishtranslation{bien}

\entry{*weñlel}
\partofspeech{s esp}
\spanishtranslation{bienestar}

\entry{weñtya}
\partofspeech{s esp}
\onedefinition{1}
\spanishtranslation{responsabilidad}
\cholexample{Mux lakäk'eñ tyi' weñtya.}
\exampletranslation{Vamos a dejarlo como su responsabilidad.}
\onedefinition{2}
\spanishtranslation{símbolo}
\cholexample{Tsa' ak'eñtyibastyóñ iweñtyajlel iye'tyel.}
\exampletranslation{Le dio un bastón como símbolo de su autoridad.}

\entry{wersa}
\defsuperscript{2}
\partofspeech{adj}
\spanishtranslation{necesario, preciso}
\cholexample{Wersa mi amajlel.}
\exampletranslation{Es necesario que vayas.}

\entry{wersa}
\defsuperscript{1}
\partofspeech{s esp}
\spanishtranslation{fuerza}
\cholexample{Weñ añ iwersa jiñi wiñik.}
\exampletranslation{Ese hombre tiene mucha fuerza.}

\entry{wertya}
\partofspeech{s esp}
\spanishtranslation{puerta}
\secondaryentry{iwertyajlel ityi'}
\secondtranslation{su puerta}

\entry{weswesña}
\partofspeech{adj}
\spanishtranslation{lluvioso}
\cholexample{Weswesña jiñi ja'al che' tyi' yorajlel tsäñal.}
\exampletranslation{En el invierno el tiempo es lluvioso.}

\entry{wets'}
\partofspeech{vt}
\onedefinition{1}
\spanishtranslation{arrear, corretear}
\cholexample{Yom mi awets' lok'el kawayu' tyi cholel.}
\exampletranslation{Hay que arrear al caballo de la milpa.}
\onedefinition{2}
\spanishtranslation{enviar, traer}
\cholexample{Yom mi awets' tyilel wiñikob.}
\exampletranslation{Hay que traer a los hombres.}
\onedefinition{3}
\spanishtranslation{sacar}
\cholexample{Yom mi awets' lok'el ts'ubejñ tyi otyoty.}
\exampletranslation{Debes sacar el polvo de la casa.}

\entry{wets'ekña}
\partofspeech{adj}
\spanishtranslation{bastante}
\clarification{hombres, animales, nubes}
\cholexample{Wets'ekñaba' mi ik'ux jam jiñi wakax.}
\exampletranslation{Bastantes vacas están comiendo zacate.}

\entry{*wex}
\partofspeech{s}
\spanishtranslation{pantalón, calzoncillo}

\entry{wexäl}
\partofspeech{s}
\spanishtranslation{pantalón, calzoncillo}

\entry{Wexib}
\partofspeech{s}
\spanishtranslation{nombre de una ranchería}

\entry{we'ekña}
\conjugationtense{variante}
\conjugationverb{wo'okña}
\spanishtranslation{ruidosamente}
\clarification{llanto}
\cholexample{We'ekña tyi uk'el jiñi x'ixik.}
\exampletranslation{Esa mujer está llorando ruidosamente.}
\dialectvariant{Sab.}
\dialectword{ts'ilikña}

\entry{*we'el}
\partofspeech{s}
\spanishtranslation{carne, alimento}
\secondaryentry{cha'leñ we'el}
\secondpartofspeech{vt}
\secondtranslation{comer}

\entry{we'eläl}
\partofspeech{s}
\spanishtranslation{carne}
\spanishtranslation{alimento}

\entry{*we'el chajk}
\spanishtranslation{tipo de gusano}
\clarification{de aproximadamente doce centímetros de largo y con muchos pies}

\entry{we'ibañ}
\partofspeech{vt}
\spanishtranslation{usar}
\clarification{para comer}
\cholexample{Yom mi awe'ibañ mesajtye'.}
\exampletranslation{Debes usar la mesa para la comida.}

\entry{we'sañ}
\partofspeech{vt}
\spanishtranslation{dar alimento}
\cholexample{Yom mi awe'sañ lakjula'.}
\exampletranslation{Debes dar alimento a nuestra visita.}

\entry{we'tye'}
\partofspeech{s}
\spanishtranslation{mesita (para comer), tortillador}

\entry{wicheñop}
\relevantdialect{Tila}
\partofspeech{s}
\spanishtranslation{fantasmas}
\culturalinformation{Información cultural: espíritus que son dueños de <ña'al>. Según la creencia, pueden ser niños en la mañana. Al mediodía pueden ser jóvenes; y en la noche pueden ser ancianos. Los rayos defienden las colonias de estos espíritus.}

\entry{wich'}
\partofspeech{s}
\spanishtranslation{ala}

\entry{wijlel}
\partofspeech{vi}
\spanishtranslation{dar vueltas}
\cholexample{Woli tyi wijlel jiñi mákiña jucho' kajpe'.}
\exampletranslation{El despulpador está dando vueltas.}

\entry{wijts'añ}
\partofspeech{vt}
\spanishtranslation{mojar}
\cholexample{Jiñi x'ixik mi iwijts'añ ja' ya' tyi lum.}
\exampletranslation{La mujer moja el piso con agua.}

\entry{wijwis}
\partofspeech{adv}
\spanishtranslation{en pedacitos}
\cholexample{Mu' jach iwijwis k'ux iwaj.}
\exampletranslation{Solamente come tortilla en pedacitos.}

\entry{wijwistyäl}
\partofspeech{adj}
\spanishtranslation{chiquito}

\entry{wilijtyañ}
\partofspeech{vt}
\spanishtranslation{rodear}
\cholexample{Jiñi ts'i' mi iwilijtyañ matye'chityam.}
\exampletranslation{El perro rodea al puerco de monte.}

\entry{wilts'uñ}
\partofspeech{vt}
\spanishtranslation{dar vuelta, dar cuerda}
\cholexample{Yom mi awilts'uñ jiñi músika.}
\exampletranslation{Debes dar cuerda a la vitrola.}

\entry{wiñik}
\partofspeech{s}
\onedefinition{1}
\spanishtranslation{hombre}
\onedefinition{2}
\spanishtranslation{empleado}
\secondaryentry{wiñikbä *alobil}
\secondtranslation{hijo adulto}

\entry{wiñikañ}
\partofspeech{vt}
\spanishtranslation{emplear para trabajo}
\cholexample{Mi awom awiñikañ, mik päy tyilel.}
\exampletranslation{Si quieres emplearlo, voy a traerlo.}

\entry{*wiñiklel}
\partofspeech{s}
\spanishtranslation{empleado}

\entry{*wiñkilel}
\relevantdialect{Sab.}
\partofspeech{s}
\spanishtranslation{empleado}

\entry{wirischañ}
\partofspeech{s}
\spanishtranslation{golondrina}
\clarification{ave}

\entry{wis}
\partofspeech{adv}
\spanishtranslation{poquito}
\cholexample{Ma'añik mi iwis cha'leñ e'tyel.}
\exampletranslation{No trabaja ni un poquito.}

\entry{wischokoñ}
\partofspeech{vt}
\spanishtranslation{poner un poco}
\clarification{café, pozol}

\entry{wisil}
\partofspeech{adj}
\spanishtranslation{chiquito}
\cholexample{Bik'tyi wisil tsak tsepe jiñi lukum.}
\exampletranslation{Corté esa culebra en pedazos muy chiquitos.}

\entry{wisik'iñ}
\partofspeech{s}
\spanishtranslation{tipo de planta con espinas}

\entry{wity}
\partofspeech{adv}
\spanishtranslation{manera de caer}
\cholexample{Tsa' awity yajli tyi' jol otyoty.}
\exampletranslation{Se cayó del techo de la casa.}

\entry{wity'}
\partofspeech{vt}
\spanishtranslation{apretar}
\clarification{cincha de mula, rollo de leña}
\cholexample{Yom mi awity' siñcha.}
\exampletranslation{Debes apretar la cincha.}

\entry{wits}
\onedefinition{1}
\partofspeech{s}
\spanishtranslation{cerro, serranía}
\cholexample{Kolem jiñi wits ambä tyi paty yajalóñ.}
\exampletranslation{Es grande el cerro que está atrás de Yajalón.}
\onedefinition{2}
\partofspeech{adj}
\spanishtranslation{inclinado}
\cholexample{Weñ wits bijlel cholel.}
\exampletranslation{El camino a la milpa está muy inclinado.}
\dialectvariant{Sab.}
\dialectword{boltyäl}

\entry{witsikña}
\partofspeech{adj}
\spanishtranslation{amontonado}
\clarification{maíz, café, frijol}
\cholexample{Witsikña jach woli tyi p'ajtyel ixim tyi lum.}
\exampletranslation{Amontonado se cae el maíz.}

\entry{*witsilel}
\partofspeech{s}
\spanishtranslation{región montañosa}

\entry{wits'law}
\partofspeech{adv}
\nontranslationdef{Se relaciona con la forma en que sale el agua en chorros pequeños.}

\entry{*wi'}
\partofspeech{s}
\spanishtranslation{raíz}

\entry{wi'il}
\partofspeech{adv}
\onedefinition{1}
\spanishtranslation{atrás}
\cholexample{Ñaxañ mi imajlel wiñik tyi bij; wi'il mi imajlel iyijñam.}
\exampletranslation{Primero va el hombre en el camino. Atrás va su mujer.}
\onedefinition{2}
\spanishtranslation{después}
\cholexample{Tyi wi'il yom mi atyilel ja'el.}
\exampletranslation{También debes venir después.}

\entry{wi'ilix}
\partofspeech{adj}
\spanishtranslation{último}

\entry{wi'ñal}
\partofspeech{s}
\spanishtranslation{hambre}

\entry{*wi'ts'ijñ}
\partofspeech{s}
\spanishtranslation{raíz de yuca}

\entry{wokol}
\defsuperscript{1}
\onedefinition{1}
\partofspeech{adj}
\spanishtranslation{difícil, duro}
\cholexample{Weñ wokol tyi sek'ol tye'.}
\exampletranslation{Es muy difícil tumbar un árbol.}
\onedefinition{2}
\partofspeech{s}
\spanishtranslation{dificultad}
\cholexample{Añ iwokol ktyaty kome ma'añix mi imel ichol.}
\exampletranslation{Mi papá tiene dificultad porque ya no hace su milpa.}

\entry{wokol}
\defsuperscript{2}
\partofspeech{s}
\spanishtranslation{enfermedad}
\cholexample{Añ iwokol, woli tyi weñ ojbal.}
\exampletranslation{Está enfermo (lit.: tiene enfermedad), pues está tosiendo mucho.}

\entry{wokolix a wälä}
\spanishtranslation{gracias}
\secondaryentry{awokolik}
\secondpartofspeech{part}
\secondtranslation{por favor}
\secondaryentry{wokol awälä}
\secondtranslation{gracias}
\secondaryentry{wokol ty'añ}
\secondtranslation{por favor}

\entry{woch'}
\partofspeech{s}
\spanishtranslation{tostada}
\cholexample{Yom mi amelbeñoñ kwoch'.}
\exampletranslation{Hágame mis tostadas, por favor.}

\entry{woch'esañ}
\partofspeech{vt}
\spanishtranslation{tostar}
\cholexample{Woch'esañ lakwaj.}
\exampletranslation{Debes tostar nuestras tortillas.}

\entry{woch'esäbil}
\partofspeech{adj}
\spanishtranslation{tostada}
\cholexample{Woch'esäbilix lakwaj cha'añ mi lakch'äm majlel tyi bij.}
\exampletranslation{Las tortillas ya están tostadas para llevarlas en el camino.}

\entry{woch'law}
\partofspeech{adv}
\nontranslationdef{Se relaciona con el ruido que producen los palitos y los papeles secos; p. ej.:}
\cholexample{Woch'law tsa' majli wiñik tyi tye'el.}
\exampletranslation{El hombre se fue por el bosque haciendo ruido con los palitos secos al quebrarlos.}

\entry{woch'okña}
\partofspeech{adj}
\spanishtranslation{crujiente}
\cholexample{Woch'okña iyopol tye' che' añ majki mi iñumel tyi tye'el.}
\exampletranslation{Las hojas dan un sonido crujiente cuando alguno pasa por el bosque.}

\entry{woj}
\partofspeech{s}
\spanishtranslation{ladrido}

\entry{wojiñ}
\partofspeech{vt}
\spanishtranslation{ladrar}
\cholexample{Joñtyol jiñi ts'i'; mi iwojiñoñla.}
\exampletranslation{El perro es bravo; nos ladra.}

\entry{-wojl}
\nontranslationdef{Sufijo numeral para contar botellas; p. ej.:}
\cholexample{Ya' tyi tyieñda mi lakmäñ juñwojl jiñi akeitye tyriuñfo.}
\exampletranslation{En la tienda compramos una botella de Aceite Triunfo.}

\entry{wojlel jach}
\spanishtranslation{de vez en cuando}
\cholexample{Wojlel jach mi ityilel ktatuch.}
\exampletranslation{De vez en cuando viene mi abuelo.}

\entry{wojsiña}
\partofspeech{ve}
\spanishtranslation{respirando}
\cholexample{K'ele, wojsiñatyo jiñi me'.}
\exampletranslation{¡Mira, todavía el venado está respirando!}

\entry{wojts}
\partofspeech{adj}
\spanishtranslation{bofa, inútil}
\clarification{madera, tierra}
\cholexample{Wojts jiñi tye', kome mach mejlik laktsep.}
\exampletranslation{La madera está bofa, porque no podemos cortarla.}

\entry{wojtsik'tyik}
\partofspeech{adj}
\spanishtranslation{arrugado}
\cholexample{Wojtsik'tyik lakpisil.}
\exampletranslation{Nuestra ropa está arrugada.}

\entry{wojwojña}
\partofspeech{adv}
\spanishtranslation{ladrando}
\cholexample{Wojwojña mi imajlel jiñi ts'i'. woli iyajñesañ me'.}
\exampletranslation{El perro va ladrando al cazar un venado.}

\entry{-wolañ}
\nontranslationdef{Sufijo que se presenta con raíces adjetivas que indican color y se refiere a objetos redondos.}

\entry{woli}
\partofspeech{part}
\nontranslationdef{Una palabra que indica el aspecto continuativo; p. ej.:}
\cholexample{Woli tyi e'tyel jiñi wiñik.}
\exampletranslation{El hombre está trabajando.}
\dialectvariant{Tila}
\dialectword{choñkol}
\dialectvariant{Sab.}
\dialectword{yäkel}

\entry{wolix i bäjlel k'iñ}
\spanishtranslation{ya se está poniendo el sol}

\entry{wolix tyi pasel k'iñ}
\spanishtranslation{ya está saliendo el sol}

\entry{wolol}
\partofspeech{adj}
\spanishtranslation{esférico}
\cholexample{¿ma'añik tsa' k'ele pelotya wolol tyi lum?}
\exampletranslation{¿No viste la pelota esférica en el suelo?}

\entry{wolts'iñ}
\partofspeech{vt}
\spanishtranslation{molestar}
\clarification{con palabras}
\cholexample{Jatyety jach woli awolts'iñoñ tyi ty'añ.}
\exampletranslation{Tú me estás molestando con palabras.}

\entry{wom}
\partofspeech{adv}
\nontranslationdef{Se relaciona con un cuerpo redondo o una bola de hojas, pelo o trapo; p. ej.:}
\cholexample{Tsa' iwom tyek'e chämeñ ts'i'.}
\exampletranslation{Pisoteó un perro muerto.}

\entry{wosilañ}
\partofspeech{vt}
\spanishtranslation{oprimir e inflar}
\cholexample{Jiñi ch'ityoñ woli iwosilañ pelotya.}
\exampletranslation{Ese chamaco está oprimiendo e inflando la pelota.}

\entry{*wosmäjel}
\partofspeech{s}
\spanishtranslation{ampolla, vejiga}

\entry{wosmäl}
\partofspeech{vi}
\spanishtranslation{ampollarse}
\cholexample{Tsa' awosmi ipächälel cha'añ pulibäl.}
\exampletranslation{Se ampolló su piel por la viruela.}

\entry{wosol}
\partofspeech{adj}
\spanishtranslation{ampollado}
\cholexample{Wosol ipächälel cha'añ pulibäl.}
\exampletranslation{Su piel está ampollada por la viruela.}

\entry{woswosña}
\partofspeech{adj}
\spanishtranslation{jadeante}
\cholexample{Woswosña jiñi ch'ityoñ cha'añ woli tyi ajñel.}
\exampletranslation{Ese niño está jadeante por correr.}

\entry{wotyokña}
\partofspeech{adv}
\nontranslationdef{Se relaciona con la manera en que se mueve un ramo de flores por el viento.}

\entry{wotyol}
\partofspeech{adj}
\spanishtranslation{esférico}
\cholexample{Wotyol mi ikolel jiñi ñichim tyak.}
\exampletranslation{La flor crece esférica.}

\entry{wotsilañ}
\partofspeech{vt}
\spanishtranslation{mullir}
\cholexample{Woli iwotsilañ ipaty bu'ul.}
\exampletranslation{Está mulliendo la cáscara del frijol.}

\entry{wotsokña}
\partofspeech{adv}
\nontranslationdef{Se relaciona con la manera de mover un bulto pesado de un lugar a otro.}

\entry{wotsol}
\partofspeech{adj}
\onedefinition{1}
\spanishtranslation{espumoso}
\cholexample{Wotsol woli ikäy ilojk ja'.}
\exampletranslation{El agua se queda espumosa.}
\onedefinition{2}
\spanishtranslation{amontonado}
\clarification{arriera o avispa ya muerta}
\cholexample{Ya wotsol ichäñäl xux tsa' chämi.}
\exampletranslation{Allí están amontonadas las avispas que se murieron.}
\onedefinition{3}
\spanishtranslation{amontonado}
\clarification{ropa}
\cholexample{Ya' wotsol jiñi pisil tyi ch'ak.}
\exampletranslation{La ropa está amontonada allí, en la cama.}
\onedefinition{4}
\spanishtranslation{mullido}
\cholexample{Wotsolix ipaty bu'ul.}
\exampletranslation{La cáscara del frijol ya está mullida.}

\entry{wotswotsña}
\partofspeech{adj}
\onedefinition{1}
\spanishtranslation{espumajoso}
\cholexample{Wotswotsña ilojk ya'ba' woli iyajlel ja'.}
\exampletranslation{El agua está espumajosa donde está cayendo.}
\onedefinition{2}
\spanishtranslation{enjambrado}
\cholexample{Wotswotsña jiñi xux woli isajkañbaki mi imel iyotyoty.}
\exampletranslation{Las avispas enjambradas están buscando dónde hacer su nido.}

\entry{wox}
\defsuperscript{1}
\partofspeech{adv}
\nontranslationdef{Se relaciona con un objeto esférico; p. ej.:}
\cholexample{Jiñi ch'ityoñ tsa' iwox ch'ämä alaxax.}
\exampletranslation{El muchacho tomó una naranja.}

\entry{wox}
\defsuperscript{2}
\partofspeech{vt}
\spanishtranslation{enroscar}
\cholexample{Jiñi lukum mi iwox ibä.}
\exampletranslation{La culebra se enrosca en sí misma.}

\entry{-wox}
\nontranslationdef{Sufijo numeral para contar objetos esféricos (huevos, piedras); p. ej.:}
\cholexample{juñwox}
\exampletranslation{una bola}

\entry{woxbil}
\partofspeech{adj}
\spanishtranslation{empelotado}
\cholexample{Woxbil jiñi puy cha'añ mi ityempañ.}
\exampletranslation{El hilo está empelotado por estar guardado.}

\entry{woxilañ}
\partofspeech{vt}
\spanishtranslation{hacer una bola}
\clarification{masa, tierra}
\cholexample{Mi iwoxilañ sa' tyi' k'äb.}
\exampletranslation{Hace la masa en una bola.}

\entry{woxokña}
\partofspeech{adv}
\nontranslationdef{Se relaciona con el movimiento de un objeto esférico; p. ej.:}
\cholexample{Woxokña mi imajlel pelotya.}
\exampletranslation{La pelota se va rodando.}

\entry{woxol}
\partofspeech{adj}
\spanishtranslation{esférico}
\cholexample{Woxol mi lakmel jiñi tyulum cha'añ kujoñib.}
\exampletranslation{Hacemos la madera de chicle en forma esférica para un pisón.}

\entry{wo'atyax}
\partofspeech{adj}
\spanishtranslation{glotón}
\cholexample{Wo'atyax jiñi wiñik. tyi pejtyelel ora yom dulke.}
\exampletranslation{Ese hombre es un glotón; siempre quiere dulce.}

\entry{wo'okña}
\partofspeech{adv}
\spanishtranslation{recio}
\clarification{llorando}
\cholexample{Wo'okña tyi uk'el jiñi alob.}
\exampletranslation{Ese niño está llorando recio.}
\variation{we'ekña}
\dialectvariant{Sab.}
\dialectword{ts'ilikña}

\entry{wo'wo}
\partofspeech{adv}
\spanishtranslation{a cada rato}
\cholexample{Woli tyi wo'wo uk'el aläl.}
\exampletranslation{A cada rato llora el niño.}

\entry{wuklujump'ejl}
\partofspeech{adj}
\spanishtranslation{diecisiete}

\entry{wukñij}
\partofspeech{adv}
\spanishtranslation{séptimo día}

\entry{wukpik}
\partofspeech{s}
\spanishtranslation{pájaro}
\clarification{de cabeza azul, patas cortas y cola larga}

\entry{wukp'ejl}
\partofspeech{adj}
\spanishtranslation{siete}

\entry{wuj}
\partofspeech{vt}
\spanishtranslation{vaciar}
\clarification{maíz, frijol, café, arroz}
\cholexample{Wujbeñ ixim lakchityam.}
\exampletranslation{Vacía el maíz para nuestros puercos.}

\entry{wujty}
\partofspeech{s}
\spanishtranslation{hechicería}

\entry{wujtyañ}
\partofspeech{vt}
\spanishtranslation{soplar}
\cholexample{Tsa' iwujtya lok'el its'ubejñalbañka cha'añ mi ibuchtyañ.}
\exampletranslation{Sopló el polvo de la banca para sentarse.}

\entry{wujtyiñtyel}
\partofspeech{vi}
\spanishtranslation{curar}
\clarification{por brujo}
\cholexample{Mi kaj iyäk' ibä tyi wujtyiñtyel.}
\exampletranslation{Él va a dejarse curar por el brujo.}
\alsosee{xwujty}

\entry{wulwulña}
\partofspeech{adv}
\onedefinition{1}
\spanishtranslation{brotando lentamente}
\cholexample{Wulwulña woli ilok'el ja' tyi lum.}
\exampletranslation{El agua sale brotando lentamente del suelo.}
\onedefinition{2}
\spanishtranslation{en murmullo}
\cholexample{Wulwulña woliyob tyi ty'añ jiñi wiñikob.}
\exampletranslation{Esos hombres están hablando en murmullo.}

\entry{wulwul ty'añ}
\spanishtranslation{regañada}
\cholexample{Woli tyi chäñ wulwul ty'añ tyi' koñtyra ipi'äl.}
\exampletranslation{Está dándole una regañada a su compañero.}

\entry{wumälel}
\partofspeech{s}
\spanishtranslation{acahual}
\clarification{la milpa después de una cosecha}

\entry{wus}
\onedefinition{1}
\partofspeech{vt}
\spanishtranslation{tocar}
\clarification{caracol, pito, trompeta}
\onedefinition{2}
\partofspeech{vt}
\spanishtranslation{soplar}
\clarification{hule}
\onedefinition{3}
\partofspeech{adv}
\spanishtranslation{con soplo}
\cholexample{Wus jek'e jpelotya.}
\exampletranslation{Penetra la pelota con soplido (con instrumento).}

\entry{*wuty}
\defsuperscript{2}
\partofspeech{s}
\spanishtranslation{fruta}
\cholexample{Kabäl iwuty alaxax.}
\exampletranslation{El naranjo tiene mucha fruta.}
\secondaryentry{*wuty tyaj}
\secondtranslation{piña de pino}
\secondaryentry{*wuty tye'}
\secondtranslation{fruta del árbol}

\entry{*wuty}
\defsuperscript{1}
\partofspeech{s}
\onedefinition{1}
\spanishtranslation{cara}
\cholexample{Käñäl iwuty kome añ iyejtyal machity.}
\exampletranslation{Su cara es conocida porque tiene la marca de un machetazo.}
\onedefinition{2}
\spanishtranslation{ojo}
\cholexample{Mach weñik jump'ejl iwuty.}
\exampletranslation{No tiene bueno un ojo.}
\secondaryentry{*wuty lakok}
\secondtranslation{tobillo}
\clarification{lit.: el ojo de nuestro pie}

\entry{wutschokoñ}
\partofspeech{vt}
\spanishtranslation{poner agachado}
\cholexample{Yom mi awutschokoñ aläl.}
\exampletranslation{Debes poner al niño agachado.}

\entry{wutstyäl}
\partofspeech{vi}
\spanishtranslation{agacharse}
\cholexample{Jiñi wiñik woli iwutstyäl tyi lum.}
\exampletranslation{Aquel hombre está agachándose en el suelo.}

\entry{wutsukña}
\partofspeech{adv}
\spanishtranslation{agachadamente}
\cholexample{Wutsukña mi imajlel jiñi wiñik.}
\exampletranslation{Ese hombre va agachadamente.}

\entry{wutsul}
\partofspeech{adj}
\spanishtranslation{agachado}
\cholexample{Wutsul jiñi wiñik tyi lum.}
\exampletranslation{Ese hombre está agachado en el suelo.}

\entry{wutswutsña}
\partofspeech{adv}
\spanishtranslation{en posición agachada}
\cholexample{Wutswutsña mi imajlel jiñi wiñik ya' tyi mal cholel cha'añ mi ijul me'.}
\exampletranslation{Ese hombre avanza en posición agachada para tirarle a un venado.}

\entry{wuts'}
\partofspeech{vt}
\spanishtranslation{lavar}
\clarification{ropa}
\cholexample{Kña' woli iwuts' pisil.}
\exampletranslation{Mi mamá está lavando la ropa.}

\alphaletter{X}

\entry{x-}
\nontranslationdef{Prefijo que se presenta con sustantivos para indicar que es persona; p. ej.:}
\cholexample{xkoltyaya}
\exampletranslation{ayudante.}
\dialectvariant{Sab., Tila}
\dialectword{aj-}

\entry{xakal}
\partofspeech{adj}
\spanishtranslation{ahorquillado}
\cholexample{Xakal ik'äb tye'.}
\exampletranslation{Las ramas del árbol está ahorquilladas.}
\variation{xäk'äl}

\entry{xak'}
\partofspeech{adj}
\spanishtranslation{a horcajadas}
\clarification{sobre un palo}
\cholexample{Xak' buchul tyi pañtye' jiñi wiñik.}
\exampletranslation{Ese hombre está a horcajadas sobre un palo.}

\entry{xajb}
\partofspeech{s}
\spanishtranslation{hierba para condimentar carne}

\entry{xajlel}
\partofspeech{s}
\spanishtranslation{piedra}
\dialectvariant{Tila}
\dialectword{tyuñ}

\entry{xajlelal pa'}
\spanishtranslation{piedras de arroyo}

\entry{xajlelol}
\partofspeech{s}
\spanishtranslation{pedregal}

\entry{xalal}
\partofspeech{adj}
\onedefinition{1}
\spanishtranslation{rajado}
\cholexample{Xalal jiñi tyabla.}
\exampletranslation{La tabla está rajada.}
\onedefinition{2}
\spanishtranslation{abierto}
\clarification{corral}
\cholexample{Xalal jiñi korral.}
\exampletranslation{El corral está abierto.}
\onedefinition{3}
\spanishtranslation{astillado}
\cholexample{Xalal jiñi latyu.}
\exampletranslation{El plato está astillado.}

\entry{xañ}
\partofspeech{s}
\spanishtranslation{palma real}
\clarification{árbol}

\entry{xapom}
\partofspeech{s}
\spanishtranslation{jabón}

\entry{xatyal}
\partofspeech{adj}
\spanishtranslation{separado}
\clarification{pies, patas}
\cholexample{Xatyal iyok jiñi wakax.}
\exampletranslation{Las patas de la vaca están separadas.}

\entry{xaty wa'al}
\spanishtranslation{parado con los pies separados}
\cholexample{Xaty wa'al jiñi wiñik tyi pam otyoty.}
\exampletranslation{El hombre está parado afuera de la casa con los pies separados.}

\entry{xaxañ}
\partofspeech{vt}
\onedefinition{1}
\spanishtranslation{contaminar}
\cholexample{Jiñi ts'ak mi ixaxañ lakbäk'tyal.}
\exampletranslation{Esa medicina contamina todo el cuerpo.}
\onedefinition{2}
\spanishtranslation{manchar}
\clarification{extendiéndose}
\cholexample{Jiñi ts'ak mi ixaxañ lakbäk'tyal.}
\exampletranslation{La medicina mancha nuestro cuerpo.}

\entry{xäb}
\partofspeech{vt}
\spanishtranslation{mezclar}
\clarification{frijol con maíz}

\entry{xäk'}
\partofspeech{vt}
\spanishtranslation{mezclar}
\clarification{arena, cal, cemento}

\entry{*xäk'}
\partofspeech{s}
\spanishtranslation{donde sale el gajo}
\secondaryentry{*xäk' bij}
\secondtranslation{tijera del camino (reg.)}
\secondtranslation{bifurcación, desviación, entronque}

\entry{xäk'äl}
\conjugationtense{variante}
\conjugationverb{xakal}
\spanishtranslation{ahorquillado}
\cholexample{Xäk'äl ik'äb tye'.}
\exampletranslation{Las ramas del árbol están ahorquilladas.}

\entry{xäk'bil}
\partofspeech{adj}
\spanishtranslation{revuelto}
\cholexample{Xäk'bil tyi ich jiñi bu'ul.}
\exampletranslation{El frijol está revuelto con chile.}

\entry{xäk'tye'}
\spanishtranslation{horqueta}

\entry{*xäk'tyi'}
\partofspeech{s}
\onedefinition{1}
\spanishtranslation{quijada}
\onedefinition{2}
\spanishtranslation{barbilla, mentón}

\entry{xäl'ity}
\partofspeech{s}
\spanishtranslation{tijerilla}
\clarification{insecto de cola horqueta}

\entry{xämbal}
\partofspeech{s}
\spanishtranslation{paseo}
\secondaryentry{cha'leñ xämbal}
\secondtranslation{andar}

\entry{xäñ}
\partofspeech{vi}
\spanishtranslation{andar}
\secondaryentry{xäñ pañimil}
\secondtranslation{el que siempre anda paseando}

\entry{xäñtyesañ}
\partofspeech{vt}
\spanishtranslation{hacer caminar}
\cholexample{Woli ixäñtyesañ iyalobil.}
\exampletranslation{Está haciendo caminar a su hijo.}

\entry{xäñwi}
\relevantdialect{Sab.}
\partofspeech{adv}
\spanishtranslation{caminando}
\cholexample{Tyi' xäñwi majlel.}
\exampletranslation{Se fue caminado.}

\entry{xäñwibäl}
\relevantdialect{Sab.}
\partofspeech{s}
\spanishtranslation{barco}

\entry{Xäñäb}
\partofspeech{s}
\spanishtranslation{Pléyades}

\entry{xäñäbäl}
\partofspeech{s}
\spanishtranslation{huarache, zapato}
\dialectvariant{Sab.}
\dialectword{\textsuperscript{2}pats'}

\entry{*xäp'äkñäyel}
\partofspeech{s}
\spanishtranslation{tristeza}

\entry{xäye'}
\partofspeech{s}
\spanishtranslation{gavilán}
\clarification{ave grande}

\entry{xä'}
\partofspeech{adv}
\spanishtranslation{de vez en cuando}
\cholexample{Mi ixä' mel kaxlañ waj.}
\exampletranslation{Hace pan de vez en cuando.}

\entry{xbakñej}
\partofspeech{s}
\onedefinition{1}
\spanishtranslation{cola que es puro hueso}
\onedefinition{2}
\spanishtranslation{nombre del diablo}

\entry{xba'}
\partofspeech{s}
\spanishtranslation{criatura}

\entry{xbäk ch'ip}
\spanishtranslation{tipo de palma sin espinas}
\clarification{chocón, chapalla}

\entry{xbäläl}
\partofspeech{s}
\spanishtranslation{tipo de langosta de tierra caliente}

\entry{xbijlum}
\partofspeech{s}
\spanishtranslation{insecto parecido a la avispa}
\clarification{de color negro; hace su nido de tierra}

\entry{xbijmuty}
\partofspeech{s}
\spanishtranslation{carpintero}
\clarification{ave}

\entry{xbits}
\partofspeech{s}
\spanishtranslation{cola larga de gallina o gallo}

\entry{xboch jol}
\spanishtranslation{capulinero}
\clarification{ave}

\entry{xborox}
\partofspeech{s}
\spanishtranslation{gallina sin plumas}

\entry{xbuk'utsu'}
\partofspeech{s}
\spanishtranslation{nauyaca saltadora}
\clarification{reptil}

\entry{xbujb}
\partofspeech{s}
\spanishtranslation{renacuajos}

\entry{xbulubik'}
\partofspeech{s}
\spanishtranslation{hombre con lobanillo}

\entry{xkajka}
\partofspeech{s}
\spanishtranslation{hombre que anda confuso}
\alsosee{kajkaña}

\entry{xkañso pech}
\partofspeech{s esp}
\spanishtranslation{ganso de collar}
\clarification{ave}

\entry{xkastyaña}
\partofspeech{s}
\spanishtranslation{árbol del pan}

\entry{xkäjchel}
\partofspeech{s}
\spanishtranslation{preso}
\dialectvariant{Sab.}
\dialectword{ajkächol}

\entry{xkäñtyaya}
\partofspeech{s}
\spanishtranslation{cuidador}

\entry{xkäñañ}
\partofspeech{s}
\spanishtranslation{cuidador}

\entry{xkäñä aläl}
\spanishtranslation{la que cuida una criatura}

\entry{xkäñtyesa}
\partofspeech{s}
\spanishtranslation{maestro}
\secondaryentry{käñtyesa}
\secondpartofspeech{s}
\secondtranslation{enseñanza}

\entry{xkojk}
\partofspeech{s}
\spanishtranslation{sordo}

\entry{xkoltyaya}
\partofspeech{s}
\onedefinition{1}
\spanishtranslation{ayudante}
\cholexample{Jiñi xch'okbä kalobil jiñi jach juñtyikil xkoltyaya kcha'añ.}
\exampletranslation{Mi ayudante es mi hija.}
\onedefinition{2}
\spanishtranslation{salvador}
\cholexample{Ma'añik tsa' chämi jiñi wiñik tsa'bä yajli tyi ja' kome ora jach tsa' tyili jiñi xkoltyaya ilok'sañ.}
\exampletranslation{Ese hombre que se cayó en el río no murió porque vino uno que lo salvó.}
\dialectvariant{Sab.}
\dialectword{ajkoltyaya}

\entry{xko'siñ}
\partofspeech{s}
\spanishtranslation{cueza}
\clarification{reg.; raíz de chayote}

\entry{xku}
\partofspeech{s}
\spanishtranslation{lechuza}
\clarification{tipo de tecolote; ave}

\entry{xkukluñtya'}
\partofspeech{s}
\spanishtranslation{escarabajo buey}
\clarification{insecto}

\entry{xkukuchyopom}
\partofspeech{s}
\spanishtranslation{chapulín fraile}
\clarification{insecto verde}

\entry{xkuchijel}
\partofspeech{s}
\spanishtranslation{cargador}
\clarification{sobre la espalda}

\entry{xkulañtye}
\partofspeech{s esp}
\spanishtranslation{cilantro}
\clarification{planta}

\entry{xkulix}
\partofspeech{s}
\spanishtranslation{mostaza}
\clarification{planta}

\entry{xkulukab}
\partofspeech{s}
\spanishtranslation{francolina}
\spanishtranslation{gallina de monte}
\spanishtranslation{gran tinamú}
\clarification{ave}

\entry{xkuway}
\partofspeech{s}
\spanishtranslation{pico blanco}
\spanishtranslation{piquiamarillo}
\clarification{ave}

\entry{xk'ajbasajk'}
\partofspeech{s}
\spanishtranslation{chapulín}
\clarification{insecto chico}

\entry{xk'ajk'äy aty}
\spanishtranslation{mútila mora}
\clarification{tipo de insecto como avispa; grande, pica fuerte y tiene alas coloradas}
\variation{xk'ajk'äy'aty}

\entry{xk'ajtyiya}
\partofspeech{s}
\onedefinition{1}
\spanishtranslation{el que pide ayuda}
\onedefinition{2}
\spanishtranslation{el que pide esposa}

\entry{xk'amäjel}
\partofspeech{s}
\spanishtranslation{enfermo}
\secondaryentry{k'amäjel}
\secondpartofspeech{s}
\secondtranslation{enfermedad}

\entry{xk'äjk'äs}
\partofspeech{s}
\spanishtranslation{luciérnaga}
\clarification{insecto}

\entry{xk'äjñibäyel}
\partofspeech{s}
\spanishtranslation{empleado}
\clarification{a la fuerza en una finca o en el gobierno}

\entry{xk'äñchäy}
\partofspeech{s}
\spanishtranslation{carpa}
\clarification{pez}

\entry{xk'äñty'añ}
\partofspeech{s}
\onedefinition{1}
\spanishtranslation{discípulo}
\onedefinition{2}
\spanishtranslation{el que acusa}
\dialectvariant{Sab.}
\dialectword{ajkäñty'añ}

\entry{xk'äñäñej}
\partofspeech{s}
\spanishtranslation{tipo de víbora}

\entry{xk'äskujel}
\partofspeech{s}
\spanishtranslation{persona con hueso quebrado}

\entry{xk'ok' chij}
\spanishtranslation{conjoyo}
\spanishtranslation{izote}
\clarification{árbol}

\entry{xk'o'k'obäläl}
\partofspeech{s}
\spanishtranslation{tipo de langosta}
\clarification{verde, comestible}

\entry{xk'o'loch}
\partofspeech{s}
\spanishtranslation{oreja de palo}
\clarification{hongo}

\entry{xk'umajtye'}
\partofspeech{s}
\spanishtranslation{zarza}
\clarification{arbusto que da fruta colorada}

\entry{xk'ux tsuk}
\spanishtranslation{coralillo}
\clarification{víbora}

\entry{xchañtye'}
\partofspeech{s}
\spanishtranslation{cuchunuc, madre de cacao}
\clarification{árbol}

\entry{xchañäl}
\partofspeech{s}
\spanishtranslation{mirón}
\cholexample{Jiñi xchañäl woli tyityo'ol xämbal ñumel.}
\exampletranslation{Ese mirón está parrandeando.}

\entry{xchawatye'}
\partofspeech{s}
\spanishtranslation{tipo de árbol}
\culturalinformation{Información cultural: La fruta tiene semilla chica, como arroz. La utilizan los brujos para remedios: raspan la cáscara, la tuestan y la dan al enfermo contra dolores de huesos. También lo usan como remedio para “el aire” y el mareo: queman las hojas con el pelo del puerco y echan el humo encima del enfermo.}

\entry{xchawa'ik'}
\partofspeech{s}
\spanishtranslation{guaco}
\clarification{planta}

\entry{xchäkäl aty}
\partofspeech{s}
\spanishtranslation{tipo de insecto}

\entry{xchäk chab ixim}
\spanishtranslation{maíz negro}

\entry{xchäkjabäñtye'}
\partofspeech{s}
\spanishtranslation{árbol grande}
\clarification{de madera colorada y dura, con fruta amarga}

\entry{xchäk jalä'tye'}
\spanishtranslation{tipo de árbol}
\clarification{sirve para postes y leña}

\entry{xchäläl}
\partofspeech{s}
\onedefinition{1}
\spanishtranslation{codorniz, perdiz chica, tinamú pequeño}
\clarification{pájaro con patas verdes}
\onedefinition{2}
\spanishtranslation{mancolón}
\clarification{pájaro que anda en el suelo; otro tipo de tinamú}

\entry{xchäkerech'}
\partofspeech{s}
\spanishtranslation{trepador}
\clarification{ave}

\entry{xchejkel}
\partofspeech{s}
\spanishtranslation{timbrillo, palo de pulque}
\clarification{arbusto}
\culturalinformation{Información cultural: Se deja fermentar la cáscara para curar pieles y preparar aguardiente.}

\entry{xchijty}
\partofspeech{s}
\onedefinition{1}
\spanishtranslation{espía}
\onedefinition{2}
\spanishtranslation{el que está emboscado}

\entry{xchilib}
\partofspeech{s}
\spanishtranslation{varisco}
\clarification{Planta de un metro de altura; la hoja es como la del chile; la fruta no es comestible.}
\culturalinformation{Información cultural: La planta entera se usa para escoba.}

\entry{xchoñoñel}
\partofspeech{s}
\spanishtranslation{vendedor}

\entry{xchumchumñiyel}
\partofspeech{s}
\spanishtranslation{nómada}
\clarification{La persona que está aquí un año, otro año en otra parte; un poco trastornado.}

\entry{xchumtyäl}
\partofspeech{s}
\spanishtranslation{habitantes}
\cholexample{Che' mi amajlel tyi yajalóñ yom mi añumel ya' tyi yojlil xchumtyäl cha'añ hidalgo.}
\exampletranslation{Cuando vayas a Yajalón debes pasar por en medio de los habitantes de Hidalgo.}
\dialectvariant{Sab.}
\dialectword{ajchumtyäl}

\entry{xchuwaña}
\relevantdialect{Tila}
\partofspeech{s}
\spanishtranslation{sacristán}

\entry{xchu' bu'ul}
\spanishtranslation{tipo de frijol grande}

\entry{xch'ajch'ib}
\partofspeech{s}
\spanishtranslation{arbusto}
\clarification{tiene hojas como las de la palma}

\entry{xch'aj päm}
\spanishtranslation{pitorreal}
\spanishtranslation{tucancillo collarejo}
\clarification{ave}

\entry{xch'ajtye'}
\partofspeech{s}
\spanishtranslation{árbol amargo}

\entry{xch'asip}
\partofspeech{s}
\spanishtranslation{mostacilla}
\clarification{una garrapata chica; insecto}

\entry{xch'ebak}
\partofspeech{s}
\spanishtranslation{tenguayaca}
\clarification{tipo de mojarra; pez de río}

\entry{xch'ejku'}
\partofspeech{s}
\onedefinition{1}
\spanishtranslation{cheje}
\clarification{carpintero listado}
\onedefinition{2}
\spanishtranslation{cheje}
\clarification{ave arrendajo}

\entry{xch'ekejk}
\partofspeech{s}
\spanishtranslation{pajuil, chachalaca negra}
\clarification{ave}

\entry{xch'e'}
\partofspeech{s}
\spanishtranslation{pájaro chico}
\clarification{de color oscuro; se encuentra en los acahuales}

\entry{xch'ich'bäty}
\partofspeech{s}
\spanishtranslation{sangre de dragón, árbol de sangre}

\entry{xch'ich'itye'}
\partofspeech{s}
\spanishtranslation{cualquier palo delgado y fuerte que se usa para castigar a los niños}

\entry{xch'ijch'ip}
\partofspeech{s}
\spanishtranslation{cualquier tipo de pajarito}

\entry{xch'ix pajch'}
\partofspeech{s}
\spanishtranslation{piñuela, piña silvestre}
\clarification{planta}
\dialectvariant{Tila}
\dialectword{ty'utspajch'}

\entry{xch'ok}
\partofspeech{s}
\spanishtranslation{niña}
\clarification{muchacha no casada}
\secondaryentry{xch'oktyobä x'ixik}
\secondtranslation{soltera}

\entry{xej}
\partofspeech{s}
\spanishtranslation{vómito}

\entry{-xejty}
\nontranslationdef{Sufijo numeral para contar objetos convexos.}

\entry{xejtyañ}
\partofspeech{vt}
\spanishtranslation{vomitar}

\entry{xejty'el}
\partofspeech{vt}
\spanishtranslation{quebrarse}
\clarification{barro, tortilla}

\entry{*xejty'il}
\partofspeech{s}
\spanishtranslation{pedazo}
\clarification{papel, tortilla}
\variation{*xujty'el}

\entry{xejwel}
\partofspeech{vi}
\spanishtranslation{volcarse}
\cholexample{Tsa' xejwi jukub ikolem ja'.}
\exampletranslation{Se volcó el cayuco en el río.}

\entry{xep}
\partofspeech{s}
\nontranslationdef{Una designación diminutiva para jóvenes y hombres.}

\entry{xep'}
\partofspeech{adv}
\spanishtranslation{poco}
\cholexample{Tsa' xep' ochi buty' ja' tyi cholel.}
\exampletranslation{Entró un poco la inundación en la milpa.}

\entry{xep'el}
\partofspeech{adj}
\spanishtranslation{muy reducido}
\cholexample{Xep'el tsa' käle jiñi lum.}
\exampletranslation{Muy reducido quedó el terreno.}

\entry{xekel}
\partofspeech{adj}
\spanishtranslation{bien crecidas}
\clarification{hojas}
\cholexample{Xekel iyopol tye'.}
\exampletranslation{Están bien crecidas las hojas.}

\entry{xetychokoñ}
\partofspeech{vt}
\spanishtranslation{colocar con objeto redondo}

\entry{xetyekña}
\partofspeech{adv}
\nontranslationdef{Se relaciona con un objeto redondo; p. ej.:}
\cholexample{Xetyekña tsa' majli pixol tyi ik'.}
\exampletranslation{El sombrero se fue girando con el viento.}

\entry{xetyel}
\partofspeech{adj}
\spanishtranslation{dejado, puesto}
\clarification{objeto redondo}
\cholexample{Xetyel añ jiñi tyza tyi mesa.}
\exampletranslation{La taza está puesta sobre la mesa.}

\entry{xewel}
\partofspeech{adj}
\spanishtranslation{inclinado}
\clarification{en el agua}
\cholexample{Xewel jiñi jukub tyi ja'.}
\exampletranslation{Ese cayuco está inclinado en el río.}

\entry{xewuña}
\partofspeech{adv}
\spanishtranslation{ladeándose}
\cholexample{Xewuña woli tyi majlel jiñi jukub, mach weñ melbilik.}
\exampletranslation{El cayuco va ladeándose; no es capaz de navegar.}

\entry{xex}
\partofspeech{s}
\onedefinition{1}
\spanishtranslation{camarón}
\onedefinition{2}
\spanishtranslation{piedras chicas revueltas con barro}

\entry{xexkokil}
\partofspeech{s}
\spanishtranslation{lugar de camarones}

\entry{xiba}
\partofspeech{s}
\onedefinition{1}
\spanishtranslation{diablo}
\onedefinition{2}
\spanishtranslation{brujo}
\secondaryentry{ts ixibäjlel}
\secondtranslation{espíritu malo de él, demonio de él}

\entry{xibañ}
\partofspeech{vt}
\spanishtranslation{peinar}

\entry{xiba yopom}
\spanishtranslation{judas}
\culturalinformation{Información cultural: figura que se cuelga durante la Semana Santa}

\entry{xikñi'iñ}
\partofspeech{vt}
\spanishtranslation{echar puntal}
\clarification{para reforzar}

\entry{xik'}
\defsuperscript{1}
\partofspeech{vt}
\spanishtranslation{atizar}
\cholexample{Yom mi axik' ak'ajk.}
\exampletranslation{Debes atizar tu fuego.}

\entry{xik'}
\defsuperscript{2}
\partofspeech{vt}
\spanishtranslation{obligar}
\cholexample{Woli ixik' tyi e'tyel iyalobil.}
\exampletranslation{Él está obligando a su hijo a trabajar.}

\entry{xik' sajp}
\spanishtranslation{jaguar, tigre americano}

\entry{xij}
\partofspeech{part}
\spanishtranslation{Palabra que se usa para arrear gallinas o pavos.}

\entry{xijk'otye'añ}
\partofspeech{vt}
\spanishtranslation{apuntalar}
\cholexample{Yomix mi axijk'otye'añ awotyoty.}
\exampletranslation{Ya debes apuntalar tu casa.}

\entry{xijiñ}
\partofspeech{adj}
\spanishtranslation{desagradable, asqueroso}
\clarification{olor}
\cholexample{Xijiñ jiñi ch'ich'.}
\exampletranslation{La sangre tiene un olor desagradable.}
\secondaryentry{ixijñal}
\secondpartofspeech{s}
\secondtranslation{olor desagradable}

\entry{xijiñtye'}
\partofspeech{s}
\spanishtranslation{chaperla}
\spanishtranslation{balché}
\clarification{árbol}
\dialectvariant{Tila}
\dialectword{yaxokiñtye'}

\entry{-xijp}
\nontranslationdef{Sufijo numeral para contar envoltorios chicos; p. ej.:}
\cholexample{Tsa' imäñä juñxijp azukal.}
\exampletranslation{Compró un envoltorio de azúcar.}

\entry{xijty'añ}
\partofspeech{vt}
\spanishtranslation{apoyarse con}
\clarification{palo, piedra}
\cholexample{Añ itye' cha'añ mi ixijty'añ tyi xämbal.}
\exampletranslation{Tiene el bastón que usa para apoyarse al caminar.}

\entry{xijty'el}
\partofspeech{vi}
\spanishtranslation{rendir}
\cholexample{Mi ixijty'el jiñi bu'ul che' mi lakch'äx.}
\exampletranslation{El frijol rinde cuando se hierve.}

\entry{ximal}
\partofspeech{adv}
\spanishtranslation{en medio}
\cholexample{Ximal mi kaj lakäk' tsaläl.}
\exampletranslation{Vamos a poner un cuarto en medio.}

\entry{xiñ}
\partofspeech{adv}
\spanishtranslation{en medio}

\entry{xiñk'äb}
\partofspeech{s}
\spanishtranslation{dedo mayor}

\entry{xiñich'}
\partofspeech{s}
\spanishtranslation{hormiga}
\clarification{insecto}

\entry{xiñil}
\relevantdialect{Sab.}
\partofspeech{adv}
\spanishtranslation{en medio}
\cholexample{Tyik melbe iyotylel kixim tyi xiñil kchol.}
\exampletranslation{Hice la troje en medio de mi milpa.}

\entry{xiñ pañchañ uw}
\nontranslationdef{Se dice cuando la luna está en cuarto menguante.}

\entry{xiñk'iñil}
\partofspeech{s}
\spanishtranslation{mediodía}

\entry{xiñol}
\partofspeech{adj}
\spanishtranslation{mitad}
\cholexample{Tyi' xiñol jach tsa' bujty'i tyi wiñikob jiñi otyoty.}
\exampletranslation{Nada más la mitad de la casa se llenó de gente.}

\entry{xiñolaj}
\partofspeech{s}
\spanishtranslation{mujer que no es indígena}

\entry{xip}
\partofspeech{adv}
\nontranslationdef{Se relaciona con un objeto envuelto en papeles o plumas.}

\entry{xipikña}
\partofspeech{adj}
\spanishtranslation{envuelta}
\cholexample{Xipikña mi ich'äm majlel iwuts'oñel.}
\exampletranslation{Sobre la cabeza lleva envuelta su ropa para lavar.}

\entry{xipil}
\partofspeech{adj}
\spanishtranslation{botado}
\clarification{ave}
\cholexample{Ya' xipil muty tyi jumpaty.}
\exampletranslation{Ahí está botada una gallina afuera.}

\entry{xipulañ}
\partofspeech{vt}
\spanishtranslation{juntar}
\clarification{ropa}
\cholexample{Jiñi x'ixik woli ixipulañ ipislel.}
\exampletranslation{La mujer está juntando su ropa.}

\entry{xity}
\partofspeech{adv}
\spanishtranslation{boca abajo}
\cholexample{Tsa' ixity käyä ibalde.}
\exampletranslation{Dejó su balde boca abajo.}

\entry{xitychokoñ}
\partofspeech{vt}
\spanishtranslation{poner boca abajo}
\cholexample{Yom mi axitychokoñ jiñi p'ejty ya' tyi lum.}
\exampletranslation{Debes poner la olla boca abajo en el suelo.}

\entry{xityikña}
\partofspeech{adv}
\spanishtranslation{de cabeza}
\cholexample{Xityikña tsa' yajli tyibältyäl jiñi ch'ityoñ.}
\exampletranslation{El niño se cayó de cabeza.}

\entry{xityil}
\partofspeech{adj}
\spanishtranslation{boca abajo}

\entry{xitytyek'}
\partofspeech{vt}
\spanishtranslation{empujar}
\cholexample{Tsa' ixitytyek'e jiñi xyäk'äjel.}
\exampletranslation{Empujó al borracho.}

\entry{xity'}
\partofspeech{adv}
\spanishtranslation{rápidamente}
\clarification{levantarse, pararse}
\cholexample{Tsa' xity' ch'ojiyoñ.}
\exampletranslation{Me levanté rápidamente.}

\entry{xix}
\partofspeech{s}
\spanishtranslation{asiento}
\clarification{de café}

\entry{xixbañ}
\partofspeech{vt}
\spanishtranslation{adormecer}
\cholexample{Woli ixixbañ iyalobil cha'añ mi iwäyel.}
\exampletranslation{Está adormeciendo a su hijo.}

\entry{xixjol}
\partofspeech{s}
\spanishtranslation{caspa}

\entry{xiyäb}
\partofspeech{s}
\spanishtranslation{peine}

\entry{xiye'}
\partofspeech{s}
\spanishtranslation{gavilán}
\clarification{ave grande}

\entry{xjabi bu'ul}
\spanishtranslation{frijol del año}

\entry{xjak'oñel}
\partofspeech{s}
\spanishtranslation{obediente}

\entry{xjol max}
\spanishtranslation{flor de corazón}
\spanishtranslation{yolosóchil}
\clarification{árbol}

\entry{xjuk'tye'}
\partofspeech{s}
\spanishtranslation{carpintero}
\clarification{ave}

\entry{xjuy'ul}
\partofspeech{s}
\spanishtranslation{tipo de insecto}

\entry{xlasobits'}
\partofspeech{s}
\spanishtranslation{cuajinicuil}
\spanishtranslation{jinicuil}
\clarification{árbol}

\entry{xlecheñtyo'}
\partofspeech{s}
\spanishtranslation{guano}
\clarification{palma}

\entry{xlemoñel}
\partofspeech{s}
\spanishtranslation{tomador}
\clarification{de aguardiente}

\entry{xles ak'}
\partofspeech{s}
\spanishtranslation{tartamudo}

\entry{xlilik}
\partofspeech{s}
\spanishtranslation{gavilán}
\clarification{ave chica}

\entry{xlok' joläl}
\partofspeech{s}
\spanishtranslation{peluquero}

\entry{xloty}
\partofspeech{s}
\spanishtranslation{mentiroso}
\secondaryentry{loty}
\secondpartofspeech{s}
\secondtranslation{mentira}

\entry{xlotyiya}
\partofspeech{s}
\spanishtranslation{testigo falso}

\entry{xlo' patye'}
\spanishtranslation{tipo de hongo}
\clarification{colorado, comestible}

\entry{xlumil bu'ul}
\partofspeech{s}
\spanishtranslation{frijol de tierra}

\entry{xlu'}
\defsuperscript{1}
\partofspeech{s}
\spanishtranslation{bagre barrigón}
\clarification{pez}
\dialectvariant{Tila}
\dialectword{ajlu'}

\entry{xlu'}
\defsuperscript{2}
\partofspeech{s}
\spanishtranslation{puerco sin pelo}

\entry{xmañk'uk'}
\partofspeech{s}
\spanishtranslation{quetzal}
\clarification{ave}

\entry{xmañchajk}
\partofspeech{s}
\spanishtranslation{lagartija trepadora}
\spanishtranslation{anolis}

\entry{xma'ity}
\partofspeech{s}
\spanishtranslation{gente que está adentro de la tierra}
\culturalinformation{Información cultural: Según la creencia, es como otro mundo. Nosotros no podemos ir allá, pero ellos vienen acá. Pasan por las casas donde son invitados por medio del <ajaw>. Los <xma'it> no toman alimento, pero olfatean el olor de la comida donde visitan. Dejan la presa que queda en los platos. Con eso les basta. Al terminar de comer desaparecen. Personas que tienen comunicación con el <ajaw> pueden oír a medianoche el canto de los gallos de los <xma'it> debajo de la tierra. Dicen que cuando aquí, en la tierra es de noche, allá abajo es de día.}

\entry{xmäjmäs}
\partofspeech{s}
\spanishtranslation{luciérnaga}
\clarification{insecto}

\entry{xmäläl}
\partofspeech{s}
\spanishtranslation{topota}
\clarification{pez}

\entry{xmäñäksi}
\partofspeech{s}
\spanishtranslation{larva del mosquito}

\entry{xmel waj}
\partofspeech{s}
\spanishtranslation{molendera}
\cholexample{Jiñi xmel waj woli imel waj cha'añ mi ik'uxob jiñi x'e'tyelob.}
\exampletranslation{La molendera está haciendo tortillas para que coman los trabajadores.}

\entry{Xmikimañesa}
\partofspeech{s}
\spanishtranslation{nombre de niña}

\entry{xmoja}
\partofspeech{s}
\spanishtranslation{prostituta}

\entry{xmukuy}
\partofspeech{s}
\spanishtranslation{paloma alas blancas, paloma real}

\entry{xmulil}
\partofspeech{s}
\spanishtranslation{pecador}
\dialectvariant{Sab.}
\dialectword{ajmulil}
\secondaryentry{mulil}
\secondpartofspeech{s}
\secondtranslation{maldad; delito; culpa}

\entry{xñichimtye'}
\partofspeech{s}
\spanishtranslation{flor de mayo}
\clarification{árbol}

\entry{xñakom}
\partofspeech{s}
\spanishtranslation{perdiz chica}
\clarification{ave de tierra caliente, patas de color verde, es comestible}

\entry{xñakow}
\partofspeech{s}
\spanishtranslation{tinamú canelo, perdiz canela}
\clarification{ave}

\entry{xña'ak'bal}
\partofspeech{s}
\spanishtranslation{tipo de grillo}

\entry{xña'a xu'}
\partofspeech{s}
\spanishtranslation{culebra arroyera}

\entry{xñätyechil}
\partofspeech{s}
\spanishtranslation{insecto que habita en la tierra}

\entry{xñejep'}
\partofspeech{s}
\spanishtranslation{mujer de edad}

\entry{xñojol}
\partofspeech{s}
\spanishtranslation{oreja de palo}
\clarification{hongo colorado, no comestible}

\entry{xñox}
\conjugationtense{variante}
\conjugationverb{ñox}
\spanishtranslation{viejo}

\entry{xñumi}
\partofspeech{vt irr}
\spanishtranslation{ya pasó}
\cholexample{Xñumi ili jabil.}
\exampletranslation{Ya pasó este año.}

\entry{xñusaty'añ}
\partofspeech{s}
\spanishtranslation{persona desobediente}
\dialectvariant{Sab.}
\dialectword{ñuñty'añ}

\entry{Xoktyik}
\relevantdialect{Tila}
\partofspeech{s}
\spanishtranslation{nombre de una colonia}

\entry{xoj}
\partofspeech{vt}
\spanishtranslation{vestir}
\cholexample{Mi kaj kxoj tsiji'bä kbujk.}
\exampletranslation{Voy a vestirme con una camisa nueva.}

\entry{*xojñil}
\partofspeech{s}
\spanishtranslation{cuña}
\secondaryentry{ixojñil}
\secondpartofspeech{s}
\secondtranslation{cuña}

\entry{xojñiñ}
\partofspeech{vt}
\spanishtranslation{acuñar}
\cholexample{Mi lakxojñiñ jiñi mesa cha'añ mach mi iñijkañ ibä.}
\exampletranslation{Acuñamos la mesa para que no se mueva.}

\entry{*xojob}
\partofspeech{s}
\spanishtranslation{claridad, rayos}
\secondaryentry{ixojob chajk}
\secondtranslation{relámpago}
\secondaryentry{ixojob k'iñ}
\secondtranslation{rayos del sol}
\secondaryentry{ixojob uw}
\secondtranslation{rayos de la luna}

\entry{*xojoblel}
\partofspeech{s}
\spanishtranslation{reflexión, resplandor de la luz}
\cholexample{Kabäl ixojoblel k'iñ.}
\exampletranslation{Hay mucho resplandor del sol.}

\entry{xojokña}
\partofspeech{adj}
\spanishtranslation{aromático}

\entry{-xojty}
\nontranslationdef{Sufijo numeral para contar aros; p. ej.:}
\cholexample{Mi kaj kch'äm jiñi juñxojty alambre.}
\exampletranslation{Voy a agarrar este aro hecho de metal.}

\entry{xopiñ}
\partofspeech{vt}
\spanishtranslation{enrollar}
\clarification{ropa}

\entry{xoty}
\partofspeech{adv}
\spanishtranslation{en forma redonda}
\cholexample{Mi lakxoty mel tye' cha'añ ch'äch'ak.}
\exampletranslation{Hacemos el palo en forma redonda para un yagual.}

\entry{xotyilañ}
\partofspeech{vt}
\spanishtranslation{enrollar}
\clarification{alambre, soga}

\entry{xotyokña}
\partofspeech{adv}
\spanishtranslation{moviéndose}
\clarification{en un círculo}
\cholexample{Xotyokña tsa' imäkä majlel iyotyoty yik'oty korral.}
\exampletranslation{Hizo la cerca en círculo alrededor de su casa.}

\entry{xotyol}
\partofspeech{adj}
\spanishtranslation{redondo}
\cholexample{Xotyol jiñi mätyäk'äbäl.}
\exampletranslation{Ese anillo es redondo.}
\secondaryentry{xotyolbä tye'}
\secondtranslation{aro}

\entry{xotyo'chel}
\partofspeech{s}
\spanishtranslation{coralillo}
\clarification{víbora}

\entry{*xotytyilel}
\partofspeech{s}
\onedefinition{1}
\spanishtranslation{circunferencia}
\onedefinition{2}
\spanishtranslation{medida de una casa adentro, medida de un cafetal, medida de un potrero}

\entry{xoty'}
\partofspeech{vt}
\onedefinition{1}
\spanishtranslation{partir}
\clarification{leña}
\onedefinition{2}
\spanishtranslation{terminar culto o fiesta}

\entry{-xoxañ}
\nontranslationdef{Sufijo que se presenta con raíces adjetivas que indican color y se refiere a la reflexión del color.}

\entry{xoy}
\defsuperscript{1}
\partofspeech{s}
\spanishtranslation{avispa de cabeza amarilla}

\entry{xoy}
\defsuperscript{2}
\partofspeech{vt}
\spanishtranslation{dar vuelta}
\cholexample{Mi ixoy majlel bij cha'añ mach yomik ñumel tyi' yojlil xchumtyäl.}
\exampletranslation{Da vuelta para no pasar en medio de la ranchería.}

\entry{xoy bij}
\partofspeech{s}
\spanishtranslation{vuelta}

\entry{xoyometyel}
\partofspeech{adj}
\spanishtranslation{sinuoso}
\clarification{camino, vereda}
\cholexample{Xoyometyel ibijlel kchol.}
\exampletranslation{El camino de mi milpa es sinuoso.}
\alsosee{k'ocholmetyel}

\entry{xo'tyäl}
\partofspeech{s}
\spanishtranslation{rincón}

\entry{xpak'}
\partofspeech{s}
\spanishtranslation{sembrador}
\dialectvariant{Sab.}
\dialectword{ajp'ujp'uya}

\entry{xpajlek'}
\partofspeech{s}
\onedefinition{1}
\spanishtranslation{pavo}
\cholexample{Mach yomix kolel xpajlek'.}
\exampletranslation{No quiere crecer el pavo.}
\onedefinition{2}
\spanishtranslation{término de regaño}

\entry{xpajtyo' tsuk}
\spanishtranslation{tipo de uva silvestre con fruta colorada y agria}
\culturalinformation{Información cultural: Se come el racimo de las frutitas; se mascan para tomar el jugo.}

\entry{xpampaja'}
\partofspeech{s}
\spanishtranslation{insecto del agua}

\entry{xpapañichim}
\partofspeech{s}
\spanishtranslation{campana}
\clarification{arbusto}

\entry{xpapastye'}
\partofspeech{s}
\spanishtranslation{cuaulote}
\clarification{árbol}

\entry{xpapätyal}
\partofspeech{s}
\spanishtranslation{guayabillo}
\clarification{árbol}

\entry{xpasa'}
\partofspeech{s}
\spanishtranslation{tipo de pájaro amarillo}

\entry{xpay}
\partofspeech{s}
\nontranslationdef{una comisión de personas que llevan un mensaje o una carga}

\entry{xpäklojm}
\partofspeech{s}
\spanishtranslation{culeca (reg.), clueca}

\entry{xpäk'oñel}
\partofspeech{s}
\onedefinition{1}
\spanishtranslation{sembrador}
\onedefinition{2}
\spanishtranslation{el que mancha}

\entry{xpäsbij}
\partofspeech{s}
\spanishtranslation{guía}

\entry{xpäyoñel}
\partofspeech{s}
\spanishtranslation{llamador}

\entry{xpek' ja'as}
\spanishtranslation{plátano enano}

\entry{xpech bu'ul}
\spanishtranslation{frijol pato}

\entry{xpejka xotyäl}
\spanishtranslation{hechicero}

\entry{xpeloña}
\partofspeech{s esp}
\spanishtranslation{gallina}
\clarification{sin plumaje en el cuello}

\entry{xpekejk}
\partofspeech{s}
\spanishtranslation{sapo}

\entry{xpermax}
\partofspeech{s}
\spanishtranslation{frijol de rabia}

\entry{xpetstyok'}
\partofspeech{s}
\spanishtranslation{tarántula}

\entry{xpiñtsik}
\partofspeech{s}
\spanishtranslation{tucancillo collarejo}
\clarification{ave}

\entry{xpokok}
\partofspeech{s}
\spanishtranslation{sapo grande}

\entry{xpok'jol}
\partofspeech{s}
\spanishtranslation{calvo}

\entry{xpojkäm}
\partofspeech{s}
\spanishtranslation{botil}
\clarification{reg.; frijol grande y colorado}

\entry{xpomaros pätyal}
\spanishtranslation{pomarrosa}
\clarification{árbol}

\entry{xpomtye'}
\partofspeech{s}
\spanishtranslation{aguacatillo}
\clarification{árbol}
\dialectvariant{Tila}
\dialectword{tyremeñtyiñatye'}

\entry{xpots'}
\partofspeech{s}
\spanishtranslation{ciego}
\cholexample{Yom juñtyikil mu'bä ityoj'esañ majlel jiñi xpots'.}
\exampletranslation{El ciego necesita que uno lo guíe.}

\entry{xpujm}
\partofspeech{s}
\spanishtranslation{tipo de árbol}
\culturalinformation{Información cultural: La cáscara se usa para bañarse.}

\entry{xpujyu'}
\partofspeech{s}
\spanishtranslation{tapacamino}
\clarification{ave chica}

\entry{xpumuk}
\partofspeech{s}
\spanishtranslation{tipo de paloma que vive en el suelo}

\entry{xpuruwok}
\partofspeech{s}
\spanishtranslation{tortolita}
\clarification{ave chica de color blanco, moteado de rojo}

\entry{xpuy ch'ajañ}
\partofspeech{s}
\spanishtranslation{talismecate}
\spanishtranslation{cuero de toro}
\clarification{árbol}

\entry{xp'eltye'}
\partofspeech{s}
\spanishtranslation{aserrador}
\dialectvariant{Sab.}
\dialectword{ajp'eltye'}

\entry{xkejkex}
\partofspeech{s}
\spanishtranslation{queisque, grajo verde}
\clarification{ave verde y grande; come plátanos}

\entry{xkel}
\partofspeech{s}
\spanishtranslation{chachalaca olivácea, chachalaca común}
\clarification{ave}

\entry{xkeñzal}
\partofspeech{s}
\spanishtranslation{quetzal}
\clarification{ave}

\entry{xkekex}
\partofspeech{s}
\spanishtranslation{queisque, grajo verde}
\clarification{ave}
\dialectvariant{Tila}
\dialectword{peazul}

\entry{xkeräch' patye'}
\partofspeech{s}
\spanishtranslation{tipo de hongo negro}
\clarification{comestible}

\entry{xk'elojel}
\partofspeech{s}
\onedefinition{1}
\spanishtranslation{guardia}
\onedefinition{2}
\spanishtranslation{espía}

\entry{xk'iñijel}
\partofspeech{s}
\spanishtranslation{participante en una fiesta}
\secondaryentry{k'iñijel}
\secondpartofspeech{s}
\secondtranslation{fiesta}

\entry{xrokiñ ok}
\partofspeech{adj}
\spanishtranslation{cojo}

\entry{xsäklaya}
\partofspeech{s}
\spanishtranslation{buscador}
\cholexample{Tsa'ix chojki majlel juñtyikil xsäklaya.}
\exampletranslation{Ya fue enviado un buscador.}

\entry{xtyajbachim}
\partofspeech{s}
\spanishtranslation{gusano de agua}

\entry{xtyamijol}
\partofspeech{s}
\spanishtranslation{mujer de pelo largo}

\entry{xtya'jol}
\partofspeech{s}
\spanishtranslation{zopilote negro}

\entry{xtya'lety}
\partofspeech{s}
\spanishtranslation{tipo de frijol de tierra}
\clarification{no se enreda y no tiene punta}

\entry{xtyexelex}
\partofspeech{s}
\spanishtranslation{tijera, tijerilla}
\clarification{tipo de insecto}

\entry{xtye' bu'ul}
\spanishtranslation{chícharo}
\clarification{tipo de hierba}

\entry{xtyiñjol}
\partofspeech{s}
\spanishtranslation{flor de muerto}
\culturalinformation{Información cultural: La riegan alrededor del sepulcro. Se dice que el olor asciende al cielo.}

\entry{xtyi'jamuty}
\partofspeech{s}
\spanishtranslation{pájaro chico y colorado}

\entry{xtyi'ja'}
\partofspeech{s}
\spanishtranslation{piscoy, vaquero}
\clarification{ave}

\entry{xtyojkay}
\partofspeech{s}
\spanishtranslation{tipo de pájaro grande}
\clarification{rayado, negro; anda en el suelo}

\entry{xtyow}
\partofspeech{s}
\spanishtranslation{gavilán negro}
\clarification{de tierra caliente}

\entry{xtyukuk}
\partofspeech{s}
\spanishtranslation{tiuca}
\clarification{ave; tiene la cabeza amarilla arriba y blanca abajo; la cola es amarilla y las patas blancas; es de tierra fría y anda solita}

\entry{xtyuk' kajpe'}
\partofspeech{s}
\spanishtranslation{cortador de café}

\entry{xtyuk'oñel}
\partofspeech{s}
\spanishtranslation{persona que corta}
\clarification{fruta}

\entry{xtyuch' k'iñ}
\partofspeech{s}
\spanishtranslation{insecto palo, zacatón}

\entry{xtyujk'atye' wakax}
\spanishtranslation{toro}

\entry{xtyujtye'}
\partofspeech{s}
\spanishtranslation{rana}

\entry{xtyuñich}
\partofspeech{s}
\spanishtranslation{temanchile}
\clarification{hierba}

\entry{xtyutyuy}
\partofspeech{s}
\spanishtranslation{lechuza chica}
\clarification{ave}

\entry{xtyuts}
\partofspeech{s}
\spanishtranslation{paloma perdiz, paloma codorniz}

\entry{xty'ojty'ojbak}
\partofspeech{s}
\onedefinition{1}
\spanishtranslation{hormiga}
\clarification{negra, grande; pica}
\onedefinition{2}
\spanishtranslation{insecto parecido a la lechuza}

\entry{xuk'chokoñ}
\partofspeech{vt}
\onedefinition{1}
\spanishtranslation{afirmar}
\cholexample{Yom mi axuk'chokoñ jiñi oy.}
\exampletranslation{Debes afirmar el horcón.}
\onedefinition{2}
\spanishtranslation{confirmar}
\cholexample{Yom mi axuk'chokoñ aty'añ.}
\exampletranslation{Debes confirmar tu palabra.}

\entry{xuk'tyäl}
\partofspeech{vi}
\spanishtranslation{quedarse firme}
\cholexample{Mi ixuk'tyäl postye che' tsäts ts'äpbil.}
\exampletranslation{El poste se queda firme cuando está bien sembrado.}

\entry{xuk'ukña}
\partofspeech{adv}
\spanishtranslation{despacio}
\cholexample{Xuk'ukña mi icha'leñ xämbal.}
\exampletranslation{Camina despacio.}

\entry{xuk'ul}
\onedefinition{1}
\partofspeech{adj}
\spanishtranslation{firme}
\cholexample{Xuk'ul jiñi otyoty.}
\exampletranslation{Esa casa está firma.}
\onedefinition{2}
\partofspeech{adj}
\spanishtranslation{fiel}
\cholexample{Xuk'ul jiñi wiñik.}
\exampletranslation{Ese hombre es fiel.}
\onedefinition{3}
\partofspeech{adv}
\relevantdialect{Sab.}
\spanishtranslation{despacio, seguramente}
\cholexample{Xuk'ul yom mi amajlel tyi bij.}
\exampletranslation{Debes ir despacio en el camino.}

\entry{xuchijp}
\partofspeech{s}
\spanishtranslation{guaqueque negro, agutí}
\clarification{mamífero}

\entry{*xujk}
\partofspeech{s}
\spanishtranslation{esquina}

\entry{xujk'äb}
\partofspeech{s}
\spanishtranslation{codo}

\entry{xujk'uñtye'}
\partofspeech{s}
\spanishtranslation{tronco}

\entry{*xujk'uñtye'lel}
\partofspeech{s}
\spanishtranslation{tronco}
\clarification{de árbol, maíz o plátano}

\entry{xujch'}
\partofspeech{s}
\spanishtranslation{ladrón}
\dialectvariant{Sab.}
\dialectword{ajxujch'}

\entry{xujch'iñ}
\partofspeech{vt}
\spanishtranslation{robar}

\entry{xujlel}
\partofspeech{vi}
\spanishtranslation{quebrarse}
\clarification{hueso, palo}
\cholexample{Yomix xujlel ikukujlel kotyoty.}
\exampletranslation{La viga de mi casa ya se quiere quebrar.}

\entry{xujlem}
\partofspeech{adj}
\spanishtranslation{quebrado}
\clarification{hueso, madera}
\cholexample{Xujlem ilápiz jiñi alob.}
\exampletranslation{El lápiz del niño se quebró (lit: está quebrado).}

\entry{-xujty'}
\nontranslationdef{Sufijo numeral para contar pedazos; p. ej.:}
\cholexample{Mi laj k'ux juñxujty' pañ.}
\exampletranslation{Comemos un pedazo de pan.}

\entry{xujty'el}
\partofspeech{vi}
\onedefinition{1}
\spanishtranslation{despedazarse}
\cholexample{Yom mi ixujty'el jiñi waj che' mi laj k'uxe'.}
\exampletranslation{La tortilla debe despedazarse cuando la comemos.}
\onedefinition{2}
\spanishtranslation{quebrarse}
\cholexample{Tsa' ujtyi tyi xujty'el klápiz.}
\exampletranslation{Se acaba de quebrar mi lápiz.}

\entry{*xujty'el}
\conjugationtense{variante}
\conjugationverb{*xejty'il}
\spanishtranslation{pedazo}
\clarification{de tela, o de madera}

\entry{xujty'om}
\partofspeech{s}
\spanishtranslation{mitad}
\clarification{vela}
\cholexample{Xujty'om jach tsa' käle.}
\exampletranslation{Solamente quedó la mitad de la vela.}

\entry{xul}
\partofspeech{vt}
\spanishtranslation{quebrar}
\clarification{hueso, palo}
\cholexample{Tsa' ixulu jiñi tye'.}
\exampletranslation{Quebró el palo.}

\entry{xulub}
\partofspeech{s}
\spanishtranslation{cuerno}

\entry{xuñ}
\partofspeech{s}
\spanishtranslation{camarón}
\clarification{crustáceo}

\entry{xuñxulu'}
\partofspeech{s}
\spanishtranslation{gavilán}
\clarification{chico}

\entry{xuty}
\partofspeech{s}
\spanishtranslation{hermano menor}
\secondaryentry{*xuty laj k'äb}
\secondtranslation{dedo meñique}
\secondaryentry{*xuty wich'}
\secondtranslation{punta de ala}

\entry{xutyäl}
\partofspeech{s}
\spanishtranslation{último hijo}

\entry{xutyuñtye'}
\partofspeech{s}
\spanishtranslation{tizón}
\clarification{palo quemado}

\entry{xuty'}
\partofspeech{vt}
\spanishtranslation{partir}
\cholexample{Mi lakxuty' jiñi waj che' mi laj k'ux.}
\exampletranslation{Partimos la tortilla al comerla.}
\secondaryentry{xuty' ochel}
\secondtranslation{meter por pedazos}

\entry{xux}
\partofspeech{s}
\spanishtranslation{avispa}
\clarification{insecto}

\entry{xuxjoño}
\relevantdialect{Sab.}
\partofspeech{adj}
\spanishtranslation{envidioso}

\entry{xuxuk' pimel}
\spanishtranslation{repollo, col}

\entry{xuyiña}
\partofspeech{adv}
\spanishtranslation{a punto de caer}
\clarification{carga de animal}
\cholexample{Xuyiña ikuch jiñi mula.}
\exampletranslation{La carga de la mula está a punto de caer.}

\entry{xu'}
\partofspeech{s}
\spanishtranslation{arriera}
\clarification{tipo de hormiga}

\entry{*xu'chajk}
\partofspeech{s}
\spanishtranslation{relámpago}

\entry{*xu'ch'il}
\partofspeech{s}
\spanishtranslation{savia del pino, resina}
\alsosee{yetsel tye'}

\entry{*xu'il}
\partofspeech{s}
\spanishtranslation{pedazo, astilla}
\secondaryentry{ixu'il tye'}
\secondpartofspeech{s}
\secondtranslation{nudo de la madera}
\secondaryentry{ixu'il chajk}
\secondpartofspeech{s}
\secondtranslation{relámpago}

\entry{Xu'wits}
\partofspeech{s}
\spanishtranslation{Cerro Firme, nombre de Tumbalá}

\entry{xwak che'e'}
\spanishtranslation{codorniz común}
\clarification{ave}

\entry{xwak ch'e'}
\spanishtranslation{cojolita}
\clarification{ave}

\entry{xwakway}
\relevantdialect{Tila}
\partofspeech{s}
\spanishtranslation{pico blanco, piquiamarillo}
\clarification{ave}
\alsosee{xkuway}

\entry{xwachiñ}
\partofspeech{s}
\spanishtranslation{tipo de pájaro de color negro}

\entry{xwach' bu'ul}
\spanishtranslation{frijol de castilla}

\entry{xwajch' me'}
\spanishtranslation{venado colorado, temazate}

\entry{xwaj'uñ}
\partofspeech{s}
\spanishtranslation{hoja de tamales}

\entry{xwarach'}
\partofspeech{s}
\spanishtranslation{gallina crespa}

\entry{xwax}
\partofspeech{s}
\spanishtranslation{jonote}
\clarification{árbol}
\culturalinformation{Información cultural: La cáscara se utiliza para amarrar las cercas, también para postes y leña.}

\entry{xwayeñch'ix}
\partofspeech{s}
\spanishtranslation{tipo de planta}
\clarification{con espinas}

\entry{xwech}
\conjugationtense{variante}
\conjugationverb{\textsuperscript{2}wech}
\spanishtranslation{armadillo}
\clarification{mamífero}

\entry{xwets' mula}
\spanishtranslation{arriero}

\entry{xwich'}
\partofspeech{s}
\spanishtranslation{avión}

\entry{xwilis}
\partofspeech{s}
\spanishtranslation{vencejo listado}
\spanishtranslation{golondrina de cueva}
\clarification{ave}

\entry{xwirischañ}
\partofspeech{s}
\spanishtranslation{golondrina}

\entry{xwis}
\partofspeech{s}
\spanishtranslation{tipo de pájaro negro}

\entry{xworch'ich'}
\partofspeech{s}
\spanishtranslation{tipo de pájaro negro}

\entry{xwukpik}
\partofspeech{s}
\spanishtranslation{guardabarranco}
\spanishtranslation{jilguero común}
\clarification{ave}

\entry{xwujty}
\partofspeech{s}
\spanishtranslation{brujo, curandero}
\culturalinformation{Información cultural: Se cree que tiene un compañero que es un espíritu malo, el <tentsun> o el <ajtso'>.}
\dialectvariant{Sab.}
\dialectword{awujtyaya}
\secondaryentry{wujtyiñtyel}
\secondpartofspeech{vi}
\secondtranslation{curarse}

\entry{xwukip}
\partofspeech{s}
\spanishtranslation{péndulo de corona}
\spanishtranslation{turco real}
\clarification{ave}

\entry{xyäx chup}
\partofspeech{s}
\spanishtranslation{larva de mariposa}

\entry{xyäxñich}
\partofspeech{s}
\spanishtranslation{árbol}
\clarification{de madera blanca y flor verde}

\entry{xyäx p'ok}
\partofspeech{s}
\spanishtranslation{lagartija de color verde}

\entry{xyk'ak'}
\partofspeech{s}
\spanishtranslation{tipo de bejuco}

\entry{xyok muty patye'}
\partofspeech{s}
\spanishtranslation{hongo}

\entry{xyojch'oya}
\partofspeech{s}
\spanishtranslation{espía}
\secondaryentry{yojch'oñ}
\secondpartofspeech{vt}
\secondtranslation{espiar}

\alphaletter{X'}

\entry{x'ak'bety}
\partofspeech{s}
\spanishtranslation{el que lleva dinero}
\clarification{para pagar la deuda de otra persona}

\entry{x'ak'juñ}
\partofspeech{s}
\spanishtranslation{mensajero}
\alsosee{ak'}

\entry{x'ak'xi'}
\partofspeech{s}
\spanishtranslation{pájaro negro}

\entry{x'ajlum}
\partofspeech{s}
\spanishtranslation{tijera gris}
\spanishtranslation{mosquera tijereta}
\clarification{ave}

\entry{x'ajñel}
\partofspeech{s}
\spanishtranslation{persona que está corriendo}

\entry{x'al mis}
\partofspeech{s}
\spanishtranslation{viborina}
\culturalinformation{Información cultural: La semilla de esta hierba se usa para el piquete de culebra o de alacrán.}

\entry{x'almis kajpe'}
\partofspeech{s}
\spanishtranslation{hierba grande}
\culturalinformation{Información cultural: La fruta se usa para una bebida.}

\entry{x'alty'añ}
\partofspeech{s}
\spanishtranslation{orador, el que habla}
\dialectvariant{Sab.}
\dialectword{ajsubty'añ}

\entry{x'alum}
\partofspeech{s}
\spanishtranslation{golondrina}

\entry{x'amiku}
\partofspeech{s}
\spanishtranslation{tzeltal}

\entry{x'araweño}
\partofspeech{s}
\spanishtranslation{hierbabuena}
\alsosee{araweño}

\entry{X'axija'}
\partofspeech{s}
\spanishtranslation{nombre de ranchería}

\entry{x'axux}
\partofspeech{s}
\spanishtranslation{ajo}

\entry{x'ek'}
\partofspeech{s}
\spanishtranslation{chaya, malamujer}
\clarification{árbol}

\entry{x'eñtyoj}
\partofspeech{s}
\spanishtranslation{eneldo}
\clarification{hierba}

\entry{x'e'tyel}
\partofspeech{s}
\spanishtranslation{trabajador}

\entry{x'ik' bäk'tyal}
\partofspeech{s}
\spanishtranslation{animal o ave negro}

\entry{x'ik'chäy}
\partofspeech{s}
\spanishtranslation{mojarra jaspeada}
\clarification{pez}

\entry{x'ik'ujts'}
\partofspeech{s}
\spanishtranslation{espatifilo}
\clarification{planta}

\entry{x'ik'uts}
\partofspeech{s}
\nontranslationdef{fruta comestible de una planta que madura en junio}

\entry{x'ichtyäk'}
\partofspeech{s}
\nontranslationdef{planta de aproximadamente ochenta centímetros de altura}
\clarification{da fruta verde, no comestible; es comida para la paloma}

\entry{x'ilo}
\partofspeech{s}
\spanishtranslation{curander}

\entry{x'ixik}
\partofspeech{s}
\spanishtranslation{mujer}

\entry{x'joch'}
\partofspeech{s}
\spanishtranslation{lechuza chica}
\clarification{de color amarillo}

\entry{x'ob}
\partofspeech{s}
\nontranslationdef{planta que alcanza un metro y medio de altura}
\culturalinformation{Información cultural: Tiene gajos colorados. Las hojas se utilizan para lavar. Da frutitas redondas, moradas, con mucha semilla; se ponen negras cuando ya están maduras. Se dice que una mujer de edad puede usar las hojas para avarse; pero niñas no, porque les puede dar viruela.}

\entry{x'obes}
\partofspeech{s}
\spanishtranslation{sitit}
\clarification{árbol chico de madera suave que da flor en abril, y que es aromático}

\entry{x'oñkoñak}
\partofspeech{s}
\spanishtranslation{sapo grande}

\entry{x'ujkuts}
\partofspeech{s}
\spanishtranslation{paloma ocotera, paloma de collar}

\entry{x'uma'}
\partofspeech{s}
\spanishtranslation{mudo}

\entry{x'uxkuruñwech patye'}
\partofspeech{s}
\spanishtranslation{hongo gris, comestible}

\entry{x'yok muty patye'}
\partofspeech{s}
\spanishtranslation{tipo de hongo blanco, comestible}

\alphaletter{Y}

\entry{-ya}
\nontranslationdef{Sufijo que se presenta con raíces transitivas para formar una raíz sustantiva; p. ej.:}
\cholexample{k'ajtyiya}
\exampletranslation{pregunta}

\entry{*yabejlel}
\partofspeech{s esp}
\spanishtranslation{llave}

\entry{yak}
\partofspeech{s}
\spanishtranslation{trampa}

\entry{yaj}
\defsuperscript{1}
\partofspeech{s}
\spanishtranslation{tumor, papera}

\entry{*yaj}
\defsuperscript{2}
\partofspeech{s}
\spanishtranslation{rendija}
\clarification{de una tabla}

\entry{*yaj}
\defsuperscript{3}
\partofspeech{s}
\spanishtranslation{hermano}

\entry{yajbiñ}
\partofspeech{vt}
\spanishtranslation{remoler}
\cholexample{Tsa'ix ujtyi iyajbiñ iwaj jiñi x'ixik.}
\exampletranslation{La mujer ya terminó de remoler su nixtamal.}

\entry{yajkañ}
\partofspeech{vt}
\spanishtranslation{escoger}
\cholexample{Yom mi lakyajkañ jiñi weñbä lakbujk.}
\exampletranslation{Debemos escoger ropa buena.}

\entry{yajkäbil}
\partofspeech{adj}
\spanishtranslation{escogido, elegido}
\cholexample{Maxtyo añix yajkäbil jiñi mu'bä iyochel tyi' ye'tyel ili jabil.}
\exampletranslation{Todavía no está elegida la persona que será la autoridad de este año.}

\entry{-yajl}
\nontranslationdef{Sufijo numeral para contar veces; p. ej.:}
\cholexample{Tsajñiyoñ cha'yajl tyi tyuxtylacha'añ mik säklañ ts'ak.}
\exampletranslation{Fui dos veces a Tuxtla en busca de medicina.}

\entry{yajlel}
\partofspeech{vi}
\spanishtranslation{caer}
\cholexample{Tsa' yajliyoñ tyi ja'.}
\exampletranslation{Me caí en el agua.}

\entry{yajlem}
\partofspeech{adj}
\spanishtranslation{caído}
\cholexample{Yajlem juñts'ijty tye' ya' tyi' yojlil bij.}
\exampletranslation{Un árbol está caído en medio del camino.}

\entry{*yajlib xwich'}
\partofspeech{s}
\spanishtranslation{campo de aterrizaje}

\entry{*yajñib k'ajk}
\partofspeech{s}
\spanishtranslation{lámpara}

\entry{*yajñib chab}
\partofspeech{s}
\spanishtranslation{abejera, colmena}

\entry{*yajñib ñich k'ajk}
\partofspeech{s}
\spanishtranslation{brasero}

\entry{yajñel}
\partofspeech{vi}
\onedefinition{1}
\spanishtranslation{quedarse, morar}
\cholexample{Wäx abi mi kajel iyajñel lakik'oty.}
\exampletranslation{Aquí va a quedarse con nosotros.}
\onedefinition{2}
\spanishtranslation{continuar}
\cholexample{Che'äch tyi pejtyelel ora mi kaj iyajñel jiñi komol ty'añ.}
\exampletranslation{Así va a continuar el acuerdo.}
\onedefinition{3}
\spanishtranslation{llegar (siempre)}
\cholexample{Mi iyajñel tyi yajalóñ.}
\exampletranslation{Siempre llega a Yajalón.}

\entry{*yajñelañ}
\relevantdialect{Sab.}
\partofspeech{s}
\spanishtranslation{principio}
\alsosee{*tyejchilañ}

\entry{yajñesañ}
\partofspeech{vt}
\spanishtranslation{perseguir}
\cholexample{Woli iyajñesañ majlel wax jiñik ts'i'.}
\exampletranslation{Mi perro está persiguiendo un gato silvestre.}

\entry{yajpel}
\partofspeech{vi}
\spanishtranslation{apagarse}
\cholexample{Mux ikajel iyajpel jiñi kas.}
\exampletranslation{Ya se va a apagar el candil.}

\entry{yajpem}
\partofspeech{adj}
\spanishtranslation{apagado}
\cholexample{Yajpem ik'ajk tyi' yotyoty che' tsa' k'otyiyoñ.}
\exampletranslation{La luz estaba apagada cuando llegué a su casa.}

\entry{*yajtyi'al}
\partofspeech{s}
\spanishtranslation{primero, crianza primera}

\entry{*yajwälel}
\relevantdialect{Sab.}
\partofspeech{s}
\spanishtranslation{paz}
\cholexample{Mi ichumtyälob jiñi kixtyañob tyi yajwälel.}
\exampletranslation{Esa gente vive en paz.}

\entry{yajyaj}
\partofspeech{adj}
\spanishtranslation{delgado, flaco}
\cholexample{Yajyaj jiñi aläl cha'añ pejtyelel ora mi ik'am'añ.}
\exampletranslation{El niño está flaco porque siempre se enferma.}

\entry{yaj'añ}
\partofspeech{vi}
\spanishtranslation{enflaquecer}
\cholexample{Jiñi xk'amäjel wolix iyaj'añ.}
\exampletranslation{Ese enfermo está enflaqueciendo.}

\entry{yaj'ojl}
\partofspeech{part}
\nontranslationdef{expresión de desprecio}
\clarification{dirigida a un joven por uno de mayor edad.}

\entry{*yal i k'äb}
\partofspeech{s}
\spanishtranslation{meñique}

\entry{*yal la kok}
\partofspeech{s}
\spanishtranslation{dedo de los pies}

\entry{yalobtyesañ}
\partofspeech{vt}
\spanishtranslation{adoptar}
\cholexample{Mi kajel iyalobtyesañ iyalobil ichich.}
\exampletranslation{Va a adoptar al hijo de su hermana.}

\entry{yambä}
\partofspeech{adj}
\spanishtranslation{otro}
\cholexample{Añ tyak yambä wits tyi' joytyälel.}
\exampletranslation{Hay otros cerros alrededor.}

\entry{yame}
\partofspeech{s esp}
\spanishtranslation{raíz de chayote, chayotestle}

\entry{*yara}
\relevantdialect{Sab.}
\partofspeech{s}
\spanishtranslation{criada}

\entry{yaräjäl}
\relevantdialect{Sab.}
\partofspeech{s}
\spanishtranslation{criada}

\entry{*yaty}
\partofspeech{s}
\spanishtranslation{testículos}

\entry{*yaty k'ajk}
\partofspeech{s}
\spanishtranslation{llama de fuego}

\entry{yats'}
\partofspeech{s}
\spanishtranslation{mal de ojo}
\culturalinformation{Información cultural: La enfermedad que sufre una criatura cuando una persona ve la ropa de la criatura, según dicen. La mujer que la ve tiene que lavar el pañal de la criatura. Si no quiere venir a lavar el pañal, tiene que mandarle su vestido a la mamá de la criatura. Así se alivia.}

\entry{yax}
\partofspeech{adj}
\spanishtranslation{verde}
\clarification{no maduro}
\cholexample{Yaxtyo ik'u' chobälel.}
\exampletranslation{La broza de la milpa todavía está verde.}

\entry{yaxajachañ}
\partofspeech{s}
\onedefinition{1}
\spanishtranslation{bejuquillo verde}
\clarification{reptil}
\onedefinition{2}
\spanishtranslation{ranera verde}
\clarification{reptil}

\entry{yaxä'tye'}
\partofspeech{s}
\spanishtranslation{paloverde}

\entry{yaxokiñtye'}
\relevantdialect{Tila}
\partofspeech{s}
\spanishtranslation{chaperla}
\spanishtranslation{balché}
\clarification{árbol}
\alsosee{xijiñtye'}

\entry{yaxum}
\relevantdialect{Sab.}
\partofspeech{s}
\spanishtranslation{maíz negro}

\entry{ya'}
\defsuperscript{1}
\partofspeech{adv}
\spanishtranslation{allá}
\cholexample{Tsa' majli ya'ba'añ ityaty.}
\exampletranslation{Se fue allá, a donde está su padre.}

\entry{*ya'}
\defsuperscript{2}
\partofspeech{s}
\spanishtranslation{pierna, muslo}

\entry{ya'äch}
\partofspeech{adv}
\spanishtranslation{ahí}
\cholexample{Ya'äch añ tyi' yotyoty.}
\exampletranslation{Ahí está en su casa.}

\entry{ya'i}
\partofspeech{adv}
\spanishtranslation{allí, ahí}
\cholexample{Ya'i yom mi ak'otyel.}
\exampletranslation{Debes llegar allí.}

\entry{*ya'lel}
\defsuperscript{1}
\partofspeech{s}
\onedefinition{1}
\spanishtranslation{líquido}
\onedefinition{2}
\spanishtranslation{jugo}
\secondaryentry{iya'lel lakwuty}
\secondtranslation{lágrima}
\secondaryentry{iya'lel lakej}
\secondtranslation{saliva}
\secondaryentry{iya'lel we'eläl}
\secondtranslation{caldo}

\entry{*ya'lel}
\defsuperscript{2}
\relevantdialect{Sab.}
\partofspeech{s}
\onedefinition{1}
\spanishtranslation{jugo}
\cholexample{Maxtyo añik iya'lel jiñi alaxax.}
\exampletranslation{Las naranjas todavía no tienen jugo.}
\onedefinition{2}
\spanishtranslation{caldo}
\cholexample{Tsäwañix iya'lel we'eläl.}
\exampletranslation{El caldo de la carne ya está frío.}
\onedefinition{3}
\spanishtranslation{aguardiente}
\cholexample{Mach yomik ikäy ijap iya'lel.}
\exampletranslation{No quiere dejar de tomar aguardiente.}

\entry{*yäbäklel}
\partofspeech{s}
\spanishtranslation{carbón}

\entry{yäk}
\partofspeech{adj}
\spanishtranslation{borracho}
\dialectvariant{Sab.}
\dialectword{k'ixiñ}

\entry{*yäklel}
\partofspeech{s}
\spanishtranslation{su tiempo}
\cholexample{Weñ iyäkleltyo ak'iñ.}
\exampletranslation{Todavía ahora es el tiempo de limpia.}

\entry{yäk'añ}
\partofspeech{vi}
\spanishtranslation{emborracharse}

\entry{yäk'bal}
\partofspeech{s}
\onedefinition{1}
\spanishtranslation{fruta}
\onedefinition{2}
\spanishtranslation{obra}

\entry{*yäk'bal}
\partofspeech{s}
\nontranslationdef{período de un día y una noche después de la luna nueva}

\entry{yäk'ñañ}
\partofspeech{vt}
\spanishtranslation{limpiar}
\clarification{milpa, cafetal}
\cholexample{Tsa' majli iyäk'ñañ ichol.}
\exampletranslation{Se fue a limpiar su milpa.}

\entry{*yäch'lel}
\partofspeech{s}
\spanishtranslation{humedad}
\alsosee{ach'}

\entry{yäjil}
\relevantdialect{Sab.}
\partofspeech{s}
\spanishtranslation{vergüenza}
\cholexample{Añ iyäjil wiñik.}
\exampletranslation{El hombre tiene vergüenza.}

\entry{yäjmañ}
\partofspeech{vt}
\spanishtranslation{mecer}
\cholexample{Yom mi ayäjmañ tyi ab jiñi aläl.}
\exampletranslation{Debes mecer al niño en la hamaca.}

\entry{yäjñel}
\partofspeech{vi}
\spanishtranslation{cambiar}
\cholexample{Wolix iyäjñel iwuty jiñi alob.}
\exampletranslation{Ya está cambiando la cara de ese niño.}

\entry{yäjyäx}
\partofspeech{adj}
\onedefinition{1}
\spanishtranslation{verde}
\cholexample{Laj yäjyäx iyopol tye'.}
\exampletranslation{Todas las hojas de los árboles son verdes.}
\onedefinition{2}
\spanishtranslation{azul}
\cholexample{Yäjyäx jiñi pañchañ.}
\exampletranslation{El cielo es azul.}

\entry{yäjyäx xajlel}
\partofspeech{s}
\spanishtranslation{obsidiana}

\entry{yäñ}
\partofspeech{vt}
\onedefinition{1}
\spanishtranslation{cambiar}
\clarification{color o dibujo}
\cholexample{Tsa' iyämbe ibojñil otyoty.}
\exampletranslation{Cambió el color de la casa.}
\onedefinition{2}
\spanishtranslation{cambiar}
\clarification{lugar}
\cholexample{Tsa' iyäñä ichumlib.}
\exampletranslation{Cambió su habitación.}

\entry{yäñäl}
\partofspeech{adj}
\spanishtranslation{cambiado}
\cholexample{Yäñäl iwuty jiñi wiñik.}
\exampletranslation{La cara de ese hombre está cambiada.}

\entry{yäp}
\partofspeech{vt}
\spanishtranslation{apagar}
\cholexample{Mi lakwujtyañ jiñi kas che' mi lakyäp.}
\exampletranslation{Soplamos el candil al apagarlo.}

\entry{yäpyäpña}
\partofspeech{adj}
\spanishtranslation{apagándose}
\clarification{un candil cuando le falta petróleo}

\entry{yäkel}
\relevantdialect{Sab.}
\partofspeech{part}
\nontranslationdef{Palabra que indica el aspecto continuativo; p. ej.:}
\cholexample{Yäkel tyi tyroñel jiñi wiñik.}
\exampletranslation{El hombre está trabajando.}
\alsosee{woli}

\entry{yäsañ}
\partofspeech{vt}
\spanishtranslation{dejar caer}
\cholexample{Jiñi xch'ok tsa' ujtyi iyäsañ iyijts'iñ.}
\exampletranslation{La niña acaba de dejar caer a su hermanito.}

\entry{yäsiyel}
\partofspeech{vi}
\spanishtranslation{descomponerse}
\cholexample{Wolix iyäsiyel jiñi laklum.}
\exampletranslation{Ya se está descomponiendo nuestra tierra.}

\entry{yäty'}
\partofspeech{vt}
\spanishtranslation{apretar}
\cholexample{Tsäts woli iyäty' isi'.}
\exampletranslation{Está apretando duro su tercio de leña.}

\entry{yäts'}
\partofspeech{vt}
\spanishtranslation{exprimir}
\cholexample{Mi iyäts' jiñi pisil che' tsa'ix ujtyi iwuts.}
\exampletranslation{Exprime la ropa cuando termina de lavarla.}

\entry{*yäts'mil}
\partofspeech{s}
\spanishtranslation{su sabor salado}

\entry{yäts'oñel}
\partofspeech{s}
\spanishtranslation{acción de exprimir}
\cholexample{Jiñi kña' woli tyi yäts'oñel.}
\exampletranslation{Mi mamá está exprimiendo la ropa (lit: está haciendo la acción de exprimir la ropa).}

\entry{yäx}
\partofspeech{adj}
\onedefinition{1}
\spanishtranslation{claro}
\spanishtranslation{limpio}
\clarification{agua}
\cholexample{Yäx yom jiñi ja' mu'bä ajap.}
\exampletranslation{Quiere que tomes agua limpia.}
\onedefinition{2}
\spanishtranslation{aguado}
\cholexample{Yäx ja' jiñi sa' mik jap.}
\exampletranslation{Yo tomo el pozol aguado.}
\secondaryentry{yäxmulañ}
\secondpartofspeech{adj}
\secondtranslation{azul o verde oscuro}
\secondaryentry{yäxmojañ}
\secondpartofspeech{adj}
\secondtranslation{azul, verde}
\clarification{reflejado}

\entry{yäxak'ach}
\partofspeech{s}
\spanishtranslation{pavo silvestre}

\entry{yäxajachañ}
\partofspeech{s}
\spanishtranslation{coralillo}
\clarification{víbora}

\entry{yäxäl}
\partofspeech{adj}
\spanishtranslation{prevenido}
\cholexample{Yom yäxäl lako kome añ laj koñtyra.}
\exampletranslation{Debemos estar prevenidos porque tenemos enemigos.}

\entry{yäxkäñañ}
\partofspeech{adj}
\spanishtranslation{aguado}
\cholexample{Yäxkäñañ jiñi sa'.}
\exampletranslation{El pozol está aguado.}

\entry{yäxk'äñcho}
\partofspeech{s}
\spanishtranslation{nauyaca verde}
\clarification{reptil venenoso arbóreo}

\entry{*yäxel}
\partofspeech{s}
\spanishtranslation{de color verde o azul}

\entry{yäx i yo}
\partofspeech{vt}
\spanishtranslation{vigilar}
\cholexample{Mi iyäx iyo tyi pejtyelel ora jiñi wiñik.}
\exampletranslation{Ese hombre vigila todo el tiempo.}

\entry{yäx jaj}
\partofspeech{s}
\spanishtranslation{mosca verde}

\entry{yäxlemañ}
\partofspeech{adj}
\spanishtranslation{azul}
\cholexample{Yäxlemañ jiñi ñoja' che' ñajty mi laj k'el.}
\exampletranslation{El río se ve azul cuando estamos lejos.}

\entry{yäxluluy}
\partofspeech{s}
\spanishtranslation{jocotillo}
\clarification{árbol}
\alsosee{ichitye'}

\entry{Yäxlumil}
\partofspeech{s}
\spanishtranslation{Tierra Verde}
\clarification{colonia}

\entry{yäxme'}
\partofspeech{s}
\spanishtranslation{venado azul}
\clarification{mamífero grande}

\entry{yäxmojañ}
\partofspeech{adj}
\spanishtranslation{morado}

\entry{yäxmulañ}
\partofspeech{adj}
\spanishtranslation{azul oscuro}

\entry{yäxñich}
\relevantdialect{Sab.}
\partofspeech{s}
\spanishtranslation{tabaquillo}
\clarification{árbol}

\entry{yäxpiyañ}
\partofspeech{adj}
\spanishtranslation{despejado}
\cholexample{Yäxpiyañ jiñi pañchañ.}
\exampletranslation{El cielo está despejado.}

\entry{yäxkich'añ}
\partofspeech{adj}
\spanishtranslation{verdoso}
\cholexample{Yäxkich'añix jiñi ja'.}
\exampletranslation{Toda el agua está verdosa.}

\entry{yäxtye'}
\partofspeech{s}
\spanishtranslation{ceiba}
\clarification{árbol grande}

\entry{*yäx'al}
\partofspeech{s}
\onedefinition{1}
\spanishtranslation{primer hijo}
\onedefinition{2}
\spanishtranslation{crianza primera}

\entry{yäx'aläl}
\partofspeech{s}
\spanishtranslation{primer hijo o hija}

\entry{yäx'añ}
\partofspeech{vi}
\onedefinition{1}
\spanishtranslation{palidecer}
\cholexample{Woli iyäx'añ iwuty cha'añ tsa' ikuju ibä.}
\exampletranslation{Está palideciendo porque se golpeó.}
\onedefinition{2}
\spanishtranslation{ponerse morado}

\entry{*yä'k'il}
\partofspeech{s}
\spanishtranslation{guía de una planta}

\entry{yebal}
\partofspeech{adv}
\spanishtranslation{debajo}
\cholexample{Päkäl xña'muty tyi' yebal ch'ak.}
\exampletranslation{La gallina está echada debajo de la cama.}

\entry{Yebalch'eñ}
\partofspeech{s}
\spanishtranslation{Abajo de la Cueva}
\clarification{colonia}

\entry{yebeñ}
\partofspeech{vt}
\spanishtranslation{dar}
\cholexample{Mi lakyebeñ isa' jiñi ak'ach.}
\exampletranslation{Le damos de comer con la mano a los pavitos.}

\entry{yebil}
\partofspeech{adj}
\spanishtranslation{agarrado}
\clarification{en la mano}
\cholexample{Yebil jiñibaltye tyi' k'äb.}
\exampletranslation{Tiene agarrado el balde en la mano.}

\entry{*yeklib i bäl ñäk'äl}
\spanishtranslation{mesa para comer}

\entry{*yej}
\partofspeech{s}
\onedefinition{1}
\spanishtranslation{filo}
\onedefinition{2}
\spanishtranslation{boca}

\entry{yejmel}
\partofspeech{vi}
\spanishtranslation{derrumbarse}
\cholexample{Woli yejmel jubel jiñi lum.}
\exampletranslation{La tierra está derrumbándose.}

\entry{*yejtyal}
\partofspeech{s}
\onedefinition{1}
\spanishtranslation{marca}
\cholexample{Añ iyejtyal jiñi mula.}
\exampletranslation{Esa mula tiene su marca.}
\onedefinition{2}
\spanishtranslation{foto}
\cholexample{Tsa'ix klok'sa iyejtyal kwuty.}
\exampletranslation{Ya me fui a sacar mi foto.}

\entry{-yel}
\nontranslationdef{Sufijo que se presenta con raíces atributivas para formar otra raíz sustantiva que indica calidad; p. ej.:}
\cholexample{ityijikñäyel}
\exampletranslation{su felicidad.}

\entry{*yesomal}
\partofspeech{s}
\spanishtranslation{costumbres vanas}

\entry{yesumil}
\relevantdialect{Tila}
\partofspeech{s}
\spanishtranslation{fantasma}
\culturalinformation{Información cultural: Se cree que muere en la cruz.}

\entry{*yetsel}
\partofspeech{s}
\spanishtranslation{savia}
\secondaryentry{*yetsel tye'}
\secondtranslation{trementina}

\entry{ye'}
\partofspeech{vt}
\spanishtranslation{llevar, tomar}
\clarification{en la mano}
\cholexample{Mi lakye' majlel ch'ejew.}
\exampletranslation{Llevamos en la mano un cajete.}

\entry{ye'eb}
\partofspeech{s}
\spanishtranslation{sereno}

\entry{ye'el}
\partofspeech{adj}
\spanishtranslation{agarrado}
\clarification{en la mano}
\cholexample{Ye'el icha'añ ilatyuba' woli ik'ux iwaj.}
\exampletranslation{Tiene agarrado el plato donde está comiendo.}

\entry{Ye'wits}
\partofspeech{s}
\spanishtranslation{Abajo del Cerro}
\clarification{colonia}

\entry{*yik'al i yej}
\partofspeech{s}
\spanishtranslation{caries}

\entry{yik'oty}
\partofspeech{part}
\onedefinition{1}
\spanishtranslation{con}
\cholexample{Tsa' majli yik'oty ipi'äl.}
\exampletranslation{Se fue con su esposo.}
\onedefinition{2}
\spanishtranslation{y}
\cholexample{Ysi' yäk'e kijñam tyumuty yik'oty welux.}
\exampletranslation{Le dio huevos y puerros a mi esposa.}
\variation{yity'ok}
\secondaryentry{kik'oty}
\secondpartofspeech{part}
\secondtranslation{yo con él}

\entry{*yilal}
\partofspeech{s}
\spanishtranslation{bastante}
\cholexample{Iyilaltyo tsa' käle iye'tyel.}
\exampletranslation{Quedó bastante de su trabajo.}

\entry{yilbejtyañ}
\relevantdialect{Sab.}
\partofspeech{vt}
\spanishtranslation{probar}
\cholexample{Jiñi wokol mi iyilbejtyañ apusik'al.}
\exampletranslation{La dificultad está probándote.}

\entry{yilol jach}
\spanishtranslation{espontáneamente, de por sí}
\cholexample{Yilol jach tsa' tyejchi ik'amäjel.}
\exampletranslation{De por sí comenzó su enfermedad.}

\entry{*yik'el lum}
\onedefinition{1}
\spanishtranslation{tierra negra}
\onedefinition{2}
\spanishtranslation{abono}

\entry{*yik'il}
\partofspeech{s}
\spanishtranslation{brisa de un río}

\entry{yity'ok}
\conjugationtense{variante}
\conjugationverb{yik'oty}
\spanishtranslation{con}

\entry{yokä}
\defsuperscript{1}
\partofspeech{adj}
\spanishtranslation{bastante}
\cholexample{Yokä añäch e'tyel.}
\exampletranslation{Hay bastante trabajo.}

\entry{yokä}
\defsuperscript{2}
\relevantdialect{Sab.}
\partofspeech{adv}
\spanishtranslation{ni siquiera}

\entry{yokä ja'as}
\relevantdialect{Sab.}
\spanishtranslation{plátano enano}

\entry{*yok ja'lel}
\partofspeech{s}
\spanishtranslation{zanja}

\entry{*yok'äjib}
\partofspeech{s}
\spanishtranslation{añadidura}
\cholexample{Mach lajalik its'ijbal iyok'äjib cha'añ ibujk.}
\exampletranslation{La añadidura del vestido no es del mismo color.}

\entry{*yok'beñal}
\partofspeech{s}
\spanishtranslation{podredumbre}

\entry{*yok'liyel}
\partofspeech{s}
\spanishtranslation{lodo}
\cholexample{Kabäl iyok'liyel iyok.}
\exampletranslation{Tiene mucho lodo en su pie.}

\entry{*yochib}
\partofspeech{s}
\spanishtranslation{entrada}

\entry{yojch'oñ}
\partofspeech{vt}
\spanishtranslation{espiar}
\alsosee{xyojch'oya}

\entry{*yojlil tye'}
\partofspeech{s}
\spanishtranslation{corazón del árbol}

\entry{yojlom}
\partofspeech{s}
\spanishtranslation{tuétano}

\entry{yojyoñ}
\partofspeech{vt}
\spanishtranslation{embrocar}
\cholexample{Wolik yojyoñ lok'el ixim tyi koxtyal.}
\exampletranslation{Estoy embrocando maíz del costal.}

\entry{yol}
\defsuperscript{1}
\partofspeech{adv}
\spanishtranslation{rápidamente}
\clarification{la forma de tomar algo líquido}
\cholexample{Tsa' iyol japä jump'is sa'.}
\exampletranslation{Tomó rápidamente una taza de pozol.}

\entry{yol}
\defsuperscript{2}
\relevantdialect{Sab.}
\partofspeech{s}
\spanishtranslation{perrito de agua}
\clarification{insecto}

\entry{yolokña}
\defsuperscript{2}
\nontranslationdef{Se relaciona con la manera como se arrastra una culebra; p. ej.}
\cholexample{Yolokña mi imajlel jiñi lukum.}
\exampletranslation{Esa culebra se va arrastrando.}

\entry{yolokña}
\defsuperscript{1}
\partofspeech{adv}
\spanishtranslation{en raudal}

\entry{yolk'iñ}
\partofspeech{vt}
\spanishtranslation{batir}
\cholexample{Woli iyolk'iñ ok'ol yik'oty iyok.}
\exampletranslation{Está batiendo lodo con su pie.}

\entry{yolyolña}
\partofspeech{adv}
\spanishtranslation{fuertemente}
\clarification{forma en que corre el agua}
\cholexample{Yolyolña woli tyi xämbal ja'.}
\exampletranslation{Es río baja fuertemente.}

\entry{yom}
\partofspeech{part}
\onedefinition{1}
\spanishtranslation{está bien}
\clarification{una respuesta}
\cholexample{Yom, kuku che' jiñi.}
\exampletranslation{Está bien, vete pues.}
\onedefinition{2}
\spanishtranslation{mejor}
\clarification{de salud}
\cholexample{Yomix pañimil woli iyubiñ kña'.}
\exampletranslation{Mi mamá ya se siente mejor.}
\onedefinition{3}
\spanishtranslation{deber}
\cholexample{Yom mi amajlel ak'el jiñi xk'amäjel.}
\exampletranslation{Debes ir a ver al enfermo.}

\entry{yomal}
\partofspeech{adv}
\spanishtranslation{casi}
\cholexample{Yomal kolelix ichämel.}
\exampletranslation{Casi se muere.}

\entry{yomäch}
\spanishtranslation{sí, es conveniente}

\entry{yom i tyaj xiñk'iñil}
\spanishtranslation{ya falta poco para el mediodía}

\entry{yomix bäjlel k'iñ}
\spanishtranslation{ya se está poniendo el sol}

\entry{yomojax}
\spanishtranslation{ser urgente}
\cholexample{Yomojax mi'bäk' ujtyel iye'tyel.}
\exampletranslation{Es urgente que termine pronto su trabajo.}

\entry{yomol}
\partofspeech{adj}
\spanishtranslation{amontonado}
\cholexample{Ya' yomol jiñi si' tyi lum.}
\exampletranslation{Ahí está amontonada la leña en el suelo.}

\entry{yomox}
\spanishtranslation{ya quiere}
\cholexample{Yomox sujtyel majlel.}
\exampletranslation{Ya quiere regresarse.}

\entry{yopmal}
\relevantdialect{Sab.}
\partofspeech{s}
\spanishtranslation{hoja}
\cholexample{Ma'añix iyopmal jiñi tye'.}
\exampletranslation{Ese árbol ya no tiene hojas.}

\entry{yopol}
\partofspeech{s}
\spanishtranslation{hoja}
\secondaryentry{*yopol otyoty}
\secondtranslation{techo}

\entry{yopom}
\partofspeech{s}
\spanishtranslation{una sola hoja}

\entry{*yopotye'}
\partofspeech{s}
\spanishtranslation{follaje}

\entry{*yokejty}
\partofspeech{s}
\spanishtranslation{tenamaste}
\clarification{las piedras que componen al fogón}

\entry{yokeñ}
\relevantdialect{Sab.}
\partofspeech{adv}
\spanishtranslation{verdaderamente}
\cholexample{Mik yokeñ subeñety.}
\exampletranslation{Te digo verdaderamente.}

\entry{*yorajlel}
\partofspeech{s}
\spanishtranslation{tiempo}

\entry{yosiñ}
\partofspeech{vt}
\spanishtranslation{adorar}
\cholexample{Ya' tyi ch'eñ mi iyosiñ idios tye'.}
\exampletranslation{Adora a su ídolo en la cueva.}

\entry{*yotylel}
\partofspeech{s}
\onedefinition{1}
\spanishtranslation{almacén}
\clarification{maíz}
\onedefinition{2}
\spanishtranslation{casa de animales}

\entry{*yotylel chab}
\partofspeech{s}
\spanishtranslation{colmena}

\entry{*yotyoty xux}
\partofspeech{s}
\spanishtranslation{panal de avispa}

\entry{yoty'}
\partofspeech{vt}
\spanishtranslation{ejercer presión}
\clarification{sobre el estómago}

\entry{yots'}
\partofspeech{vt}
\spanishtranslation{apretar}
\clarification{con la mano}
\cholexample{Tsa' iyots'o juñkojty tsuk tyi' k'äb.}
\exampletranslation{Apretó un ratón con la mano.}

\entry{yowix}
\partofspeech{s}
\spanishtranslation{vapor}

\entry{yowyowña}
\partofspeech{adj}
\spanishtranslation{caluroso}
\clarification{adentro del cuerpo}
\cholexample{Yowyowña ipusik'al woli iyubiñ.}
\exampletranslation{Tiene caluroso el estómago.}

\entry{yo'okñajax}
\partofspeech{adj}
\spanishtranslation{caluroso, con fiebre}
\cholexample{Yo'okñajax kpusik'al mi kubiñ.}
\exampletranslation{Mi corazón está caluroso.}

\entry{*yuk'}
\partofspeech{s}
\spanishtranslation{verdura que comen los guajolotes}

\entry{yuch' muty}
\spanishtranslation{piojo de gallina}

\entry{yujkuñ}
\partofspeech{vt}
\spanishtranslation{sacudir}
\clarification{planta, árbol}
\cholexample{Woli iyujkuñ itye'el alaxax.}
\exampletranslation{Está sacudiendo una mata de naranjas.}

\entry{yujil}
\partofspeech{vi}
\spanishtranslation{saber}
\cholexample{Jiñi ch'ityoñ yujil juñ.}
\exampletranslation{El muchacho sabe leer.}

\entry{yujlel}
\partofspeech{vi}
\spanishtranslation{bajar por un palo}

\entry{yujmañ}
\partofspeech{vt}
\spanishtranslation{agitar}
\cholexample{Yom mi ayujmañ ts'ak ambä tyi limetye.}
\exampletranslation{Hay que agitar la medicina que está en la botella.}

\entry{yujkel}
\partofspeech{s}
\spanishtranslation{temblor}

\entry{*yujtyibal}
\partofspeech{s}
\spanishtranslation{fin}

\entry{*yujts'il}
\partofspeech{s}
\spanishtranslation{aroma, olor}

\entry{yujyum}
\partofspeech{s}
\spanishtranslation{bolsero espalda amarilla}
\spanishtranslation{calandria real}
\clarification{ave}

\entry{yulyulña}
\partofspeech{adv}
\spanishtranslation{ondeándose}
\cholexample{Yulyulña mi imajlel ja'.}
\exampletranslation{El río fluye ondeándose.}

\entry{yumañ}
\partofspeech{vt}
\spanishtranslation{tener por jefe}
\cholexample{¿majki yom mi ayumañ?}
\exampletranslation{¿A quién quieres tener por jefe?}

\entry{-yumañ}
\nontranslationdef{Sufijo que se presenta con raíces adjetivas que indican color y se refiere a un líquido.}

\entry{yumäl}
\relevantdialect{Sab.}
\partofspeech{s}
\onedefinition{1}
\spanishtranslation{dueño de una finca}
\onedefinition{2}
\spanishtranslation{una autoridad que no es indígena}
\onedefinition{3}
\spanishtranslation{dueño del cerro}
\culturalinformation{Información cultural: Se dice que es un espíritu que molesta al hombre; los curanderos oran al dueño del cerro.}
\secondaryentry{iyum}
\secondtranslation{su dueño, su patrón}

\entry{yumäñ}
\partofspeech{vt}
\spanishtranslation{servir, ser súbdito}
\cholexample{Joñoñ mik yumäñ jiñi wiñik.}
\exampletranslation{Yo sirvo a ese hombre.}

\entry{*yumäñtyel}
\partofspeech{s}
\spanishtranslation{jurisdicción}
\cholexample{Jiñi ambä iye'tyel tyi yajalóñ añ iyumäñtyel tyi pejtyelel yajalóñ yik'oty tyila.}
\exampletranslation{La autoridad en Yajalón tiene su jurisdicción en todo Yajalón, Tumbalá y Tila.}

\entry{*yumijel}
\partofspeech{s}
\spanishtranslation{hermano de mi padre}

\entry{yumilañ}
\partofspeech{vt}
\spanishtranslation{agitar}
\cholexample{Mi lakyumilañ kajpe' tyi vaso.}
\exampletranslation{Agitamos el café en la taza.}

\entry{yukiña}
\defsuperscript{1}
\partofspeech{adj}
\spanishtranslation{ladeándose}
\cholexample{Yukiña jiñi otyoty cha'añ mi iñijkañ yujkel.}
\exampletranslation{La casa se está ladeando por el temblor.}

\entry{yukiña}
\defsuperscript{2}
\partofspeech{adj}
\spanishtranslation{meneando}
\cholexample{Yukiña ik'äb tye' tyi ik'.}
\exampletranslation{Los gajos del árbol se están meneando por el viento.}

\entry{*yutybal}
\partofspeech{s}
\spanishtranslation{cáliz}

\entry{*yutslel}
\partofspeech{s}
\spanishtranslation{bondad}
\cholexample{Kabäl iyutslel jiñi wiñik.}
\exampletranslation{Ese hombre es muy bondadoso (lit: tiene mucha bondad).}

\entry{yu'}
\partofspeech{vt}
\spanishtranslation{jalar}
\cholexample{Mi lakyu' jubel ik'äb kajpe' che' mi laktyuk'.}
\exampletranslation{Jalamos para abajo el gajo de café para cortar la fruta.}
\end{multicols*}\part{ESPAÑOL – CH'OL}\begin{multicols*}{2}

\entry{abajo}
\partofspeech{adv}
\spanishtranslation{eñtyäl}

\entry{Abajo de la Cueva}
\spanishtranslation{Yebalch'eñ}
\clarification{colonia}

\entry{Abajo del Cerro}
\spanishtranslation{Ye'wits}
\clarification{colonia}

\entry{abandonar}
\partofspeech{vt}
\secondaryentry{abandonado}
\secondtranslation{käyäl}

\entry{abdomen}
\partofspeech{m}
\onedefinition{1}
\spanishtranslation{jo'ñal}
\onedefinition{2}
\spanishtranslation{pusik'al}
\clarification{región central del cuerpo}

\entry{abeja}
\partofspeech{f}
\onedefinition{1}
\spanishtranslation{*chäñil chab}
\onedefinition{2}
\spanishtranslation{chäkchab}
\clarification{colorada}
\onedefinition{3}
\spanishtranslation{chäk ox}
\clarification{grande y colorada}

\entry{abejera}
\partofspeech{f}
\spanishtranslation{*yajñib chab}

\entry{abertura}
\partofspeech{f}
\spanishtranslation{*kawtyilel}

\entry{abierto}
\partofspeech{adj}
\onedefinition{1}
\spanishtranslation{kawakña, kawal}
\clarification{boca}
\onedefinition{2}
\spanishtranslation{kalal}
\clarification{pared o tabla}
\onedefinition{3}
\spanishtranslation{jajmeñ,tyokol}
\clarification{puerta de una casa}
\onedefinition{4}
\spanishtranslation{japal}
\clarification{tablas, setos}
\onedefinition{5}
\spanishtranslation{utsil}
\clarification{Tila; tiempo}
\onedefinition{6}
\spanishtranslation{xalal}
\clarification{corral}
\secondaryentry{abierto abajo}
\secondtranslation{choxol}

\entry{ablandar}
\partofspeech{vt}
\secondaryentry{ablandarse}
\secondtranslation{k'uñ'añ}

\entry{abofetear}
\partofspeech{vt}
\spanishtranslation{poch'iñ}

\entry{abono}
\partofspeech{m}
\spanishtranslation{*yik'el lum}
\clarification{fertilizante}

\entry{aborrecer}
\partofspeech{vt}
\spanishtranslation{ts'a'leñ}
\secondaryentry{aborrecido}
\secondtranslation{ts'a'lebil}

\entry{abortar}
\partofspeech{vt}
\spanishtranslation{ñusañ}

\entry{abrazar}
\partofspeech{vt}
\onedefinition{1}
\spanishtranslation{mek', tyul mek'}
\onedefinition{2}
\spanishtranslation{luts'}
\clarification{con alas}
\secondaryentry{abrazado}
\secondtranslation{mek'el; *mek'bal}
\clarification{cosa o criatura}
\secondaryentry{cosa abrazada}
\secondtranslation{mek'bal}

\entry{abrir}
\partofspeech{vt}
\onedefinition{1}
\spanishtranslation{kal}
\clarification{pared}
\onedefinition{2}
\spanishtranslation{kañ}
\clarification{los ojos}
\onedefinition{3}
\spanishtranslation{kaw}
\clarification{boca}
\onedefinition{4}
\spanishtranslation{\textsuperscript{2}ch'ep, ch'ewe}
\clarification{bolsa, costal, morral, ojos}
\onedefinition{5}
\spanishtranslation{japuñ}
\clarification{brecha, zanja}
\onedefinition{6}
\spanishtranslation{\textsuperscript{1}jam}
\clarification{casa, libro, caja}
\secondaryentry{abrir un poco}
\secondtranslation{ch'ip}
\clarification{costal, cartón, ojos}
\secondaryentry{abrirse}
\secondtranslation{jam ibä, jajmel}

\entry{absorber}
\partofspeech{vt}
\spanishtranslation{ch'äch'}
\clarification{algodón, trapo, tierra}

\entry{abuela}
\partofspeech{f}
\onedefinition{1}
\spanishtranslation{ko'äl, chuchu'}
\onedefinition{2}
\spanishtranslation{\textsuperscript{2}kolaj}
\clarification{Tila}

\entry{abuelo}
\partofspeech{m}
\spanishtranslation{*tatuch}
\clarification{paterno}

\entry{abultado}
\partofspeech{adj}
\spanishtranslation{bujul}
\clarification{cerrito}
\spanishtranslation{\textsuperscript{2}bulul}
\secondaryentry{abultadamente}
\secondtranslation{bultyäl}
\clarification{señalando el tamaño de una hinchazón}

\entry{abundancia}
\partofspeech{f}
\spanishtranslation{*k'amel, *k'amlel, *p'ejwlel}

\entry{abundante}
\partofspeech{adj}
\spanishtranslation{sejel}

\entry{acabar}
\partofspeech{vi}
\spanishtranslation{jojmel}

\entry{acahual}
\partofspeech{m}
\onedefinition{1}
\spanishtranslation{tye'e pimel}
\clarification{hierba}
\onedefinition{2}
\spanishtranslation{wumälel}
\clarification{milpa después de una cosecha}

\entry{acariciar}
\partofspeech{vt}
\spanishtranslation{jajpiñ}

\entry{acaso}
\partofspeech{adv}
\spanishtranslation{k'o'awom}
\clarification{Sab.}

\entry{acercar}
\partofspeech{vt}
\spanishtranslation{läk'tyesañ}
\secondaryentry{acercarse}
\secondtranslation{ñoch; ñochtyañ}
\clarification{a una persona o animal}

\entry{acezar}
\partofspeech{vi}
\secondaryentry{acezando}
\secondtranslation{lejlejña}
\clarification{perro}

\entry{achicar}
\partofspeech{vt}
\spanishtranslation{ch'och'oktyesañ}

\entry{achiote}
\partofspeech{m}
\spanishtranslation{jo'ox}
\clarification{árbol}

\entry{aclarar}
\partofspeech{vt}
\spanishtranslation{sujmityesañ}
\secondaryentry{aclararse}
\secondtranslation{jajmel}
\clarification{el tiempo}
\secondaryentry{ya está aclarando}
\secondtranslation{säkkolañ}
\clarification{de mañana}

\entry{acompañar}
\partofspeech{vt}
\spanishtranslation{pi'leñ}
\secondaryentry{lo que acompaña}
\secondtranslation{ñujp}

\entry{aconsejar}
\partofspeech{vt}
\secondtranslation{tyumbiñ}
\clarification{Sab.}

\entry{acordar}
\partofspeech{vt}
\spanishtranslation{k'ajtyisañ}
\clarification{Sab.}
\secondaryentry{acordado}
\secondtranslation{k'ajal}
\secondaryentry{acordarse}
\secondtranslation{k'ajatyañ}
\clarification{Sab.}
\secondaryentry{acordarse de}
\secondtranslation{k'ajtyesañ}

\entry{acortar}
\partofspeech{vt}
\secondtranslation{kom'esañ}

\entry{acostar}
\partofspeech{vt}
\onedefinition{1}
\spanishtranslation{ñolchokoñ,tyots'chokoñ}
\onedefinition{2}
\spanishtranslation{jämchokoñ}
\clarification{en una hamaca}
\secondaryentry{acostado}
\secondtranslation{tyots'ol}
\clarification{boca arriba}
\secondtranslation{le'ekña}
\clarification{las piernas abiertas}
\secondtranslation{limil}
\clarification{toda la familia por enfermedad}
\secondtranslation{mochol}
\clarification{con los pies encogidos}
\secondaryentry{acostarse}
\secondtranslation{ñoltyäl,tyots'tyäl}
\secondtranslation{jämtyäl}
\clarification{en hamaca}
\secondtranslation{metyañ}
\clarification{sobre}
\secondaryentry{acostado boca arriba}
\secondtranslation{ch'ebñolol}
\secondaryentry{acostado con las piernas estiradas}
\secondtranslation{ch'ebtyots'ol}
\secondaryentry{acostado de lado}
\secondtranslation{ts'ej ñolol}
\secondaryentry{acostarse de lado}
\secondtranslation{ts'ej ñoltyäl}
\secondaryentry{acostarse boca arriba}
\secondtranslation{ch'a'tyäl}
\secondaryentry{acostarse boca abajo}
\secondtranslation{päktyäl}

\entry{acostumbrar}
\partofspeech{vt}
\spanishtranslation{ñämtyesañ}
\secondaryentry{acostumbrarse}
\secondtranslation{ñäm'añ}

\entry{activo}
\partofspeech{adj}
\spanishtranslation{bäx}

\entry{acuerdo}
\partofspeech{m}
\spanishtranslation{tyratyo}

\entry{acuñar}
\partofspeech{vt}
\spanishtranslation{xojñiñ}

\entry{acurrucarse}
\partofspeech{prnl}
\secondaryentry{acurrucado}
\secondtranslation{petspetsña}
\secondtranslation{lutsul}
\clarification{persona o animal}

\entry{acusar}
\onedefinition{1}
\partofspeech{vt}
\spanishtranslation{\textsuperscript{2}jop'}
\onedefinition{2}
\partofspeech{vi}
\spanishtranslation{jojp'el}
\clarification{falsamente}
\secondaryentry{el que acusa}
\secondtranslation{xk'äñty'añ}

\entry{adelante}
\partofspeech{adv}
\onedefinition{1}
\spanishtranslation{ñaxañ}
\onedefinition{2}
\spanishtranslation{ajapam, ñajañ}
\clarification{Sab.}

\entry{adentro}
\partofspeech{adv}
\spanishtranslation{mal}
\secondaryentry{adentro de la casa}
\spanishtranslation{imal otyoty}

\entry{adiós}
\partofspeech{interj}
\spanishtranslation{¡kotyo!}

\entry{adolorido}
\partofspeech{adj}
\spanishtranslation{k'ux}

\entry{adoptar}
\partofspeech{vt}
\spanishtranslation{yalobtyesañ}
\clarification{como a un hijo}

\entry{adorar}
\partofspeech{vt}
\onedefinition{1}
\spanishtranslation{ch'ujutyesañ}
\onedefinition{2}
\spanishtranslation{diosiñ, yosiñ}
\clarification{esp.}
\onedefinition{3}
\spanishtranslation{ch'ujuña'iñ}
\clarification{como a la madre santa}

\entry{adormecer}
\partofspeech{vt}
\onedefinition{1}
\spanishtranslation{wäysañ, wäytyesañ}
\onedefinition{2}
\spanishtranslation{xixbañ}
\clarification{criatura}

\entry{adornar}
\partofspeech{vt}
\spanishtranslation{ch'äl}

\entry{adorno}
\partofspeech{m}
\spanishtranslation{ch'äjlil, ch'äloñib}

\entry{adueñarse}
\partofspeech{prnl}
\spanishtranslation{cha'añiñ}

\entry{adulterio}
\partofspeech{m}
\spanishtranslation{*ts'i'lel}

\entry{afeitar}
\partofspeech{vt}
\spanishtranslation{jujchiñ}
\clarification{barbas}

\entry{afilar}
\partofspeech{vt}
\spanishtranslation{\textsuperscript{1}juk', tyajpuñ}
\clarification{machete, cuchillo}
\secondaryentry{afilado}
\secondtranslation{jay}
\secondtranslation{\textsuperscript{1}säkpochañ}
\clarification{bien}

\entry{afirmar}
\partofspeech{vt}
\onedefinition{1}
\spanishtranslation{xuk'chokoñ}
\clarification{horcón de casa}
\onedefinition{2}
\spanishtranslation{tyejñañ}
\clarification{la tierra}

\entry{aflojar}
\partofspeech{vt}
\spanishtranslation{\textsuperscript{1}loj}

\entry{afuera}
\partofspeech{adv}
\spanishtranslation{-jumpaty}

\entry{agachar}
\partofspeech{vt}
\secondaryentry{agacharse}
\secondtranslation{ñuktyäl}
\secondtranslation{tyiñtyäl}
\clarification{Sab.}
\secondtranslation{wutstyäl}
\clarification{en posición acurrucada}
\secondaryentry{agachadamente}
\secondtranslation{päkäkña}
\clarification{gallina}
\secondtranslation{wutsukña}
\secondaryentry{agachado}
\secondtranslation{k'uch buchul}
\secondtranslation{kotyol}
\clarification{personas}
\secondtranslation{luts}
\clarification{al estar sentado}
\secondtranslation{\textsuperscript{2}ñuk}
\clarification{forma de persona caída en el camino}
\secondtranslation{ñukñukña}
\clarification{al caminar}
\secondtranslation{wutsul}
\clarification{en posición acurrucada}
\secondaryentry{boca agachado}
\secondtranslation{ñukye'el}
\clarification{persona, cubeta}
\secondaryentry{en posición agachada}
\secondtranslation{wutswutsña}
\secondaryentry{poner agachado}
\secondtranslation{wutschokoñ}

\entry{agarrar}
\onedefinition{1}
\partofspeech{vt}
\spanishtranslation{chuk}
\onedefinition{2}
\partofspeech{vt}
\spanishtranslation{ye'}
\clarification{en la mano}
\onedefinition{3}
\partofspeech{vt}
\spanishtranslation{ch'äm}
\clarification{llevar}
\onedefinition{4}
\partofspeech{vt}
\spanishtranslation{käl}
\clarification{con la boca}
\onedefinition{5}
\partofspeech{vt}
\spanishtranslation{jämch'äm}
\clarification{rápidamente}
\onedefinition{6}
\partofspeech{vi}
\spanishtranslation{chujkel}
\secondaryentry{agarrado}
\secondtranslation{chukbil, yebil, ye'el}
\clarification{en la mano}
\secondtranslation{\textsuperscript{2}lak}
\clarification{objeto largo}

\entry{agitar}
\partofspeech{vt}
\onedefinition{1}
\spanishtranslation{yujmañ, yumilañ}
\onedefinition{2}
\spanishtranslation{chejpañ}
\clarification{costal}

\entry{agobiar}
\partofspeech{vt}
\secondaryentry{agobiarse}
\secondtranslation{lijk'el}

\entry{agotar}
\partofspeech{vt}
\onedefinition{1}
\spanishtranslation{säjlel}
\onedefinition{2}
\spanishtranslation{tyojtyojil}
\clarification{dinero o alimentos}

\entry{agriar}
\partofspeech{vt}
\spanishtranslation{paj'añ}

\entry{agrio}
\partofspeech{adj}
\spanishtranslation{paj}

\entry{agrupar}
\partofspeech{vt}
\secondaryentry{agrupadamente}
\secondtranslation{momotyña}

\entry{agua}
\partofspeech{f}
\spanishtranslation{ja'}
\secondaryentry{agua que sale de una cueva}
\secondtranslation{ch'eñbä ja'}
\secondaryentry{forma en que corre el agua}
\secondtranslation{yolyolña}

\entry{Agua Brotante}
\spanishtranslation{Pasija'}
\clarification{Sab.; colonia}

\entry{Agua de Cal}
\spanishtranslation{Tyañija'}
\clarification{colonia}

\entry{aguacatillo}
\partofspeech{m}
\onedefinition{1}
\spanishtranslation{xpomtye'}
\clarification{árbol}
\onedefinition{2}
\spanishtranslation{tyremeñtyiñatye'}
\clarification{Tila; árbol}

\entry{aguacero}
\partofspeech{m}
\spanishtranslation{ja'lel}

\entry{aguado}
\partofspeech{adj}
\spanishtranslation{yäx, yäxkäñañ}

\entry{aguantar}
\partofspeech{vt}
\onedefinition{1}
\spanishtranslation{läty'}
\clarification{carga o dolor}
\onedefinition{2}
\spanishtranslation{mäl}
\clarification{día de trabajo}

\entry{aguardiente}
\partofspeech{m}
\onedefinition{1}
\spanishtranslation{lembal}
\onedefinition{2}
\spanishtranslation{chicha}
\clarification{la esencia del jugo de la caña ya fermentado}
\onedefinition{3}
\spanishtranslation{ts'a'añ, \textsuperscript{2}*ya'lel}
\clarification{Sab.}

\entry{aguazal}
\partofspeech{m}
\spanishtranslation{ja'lumil}

\entry{agüero}
\partofspeech{m}
\spanishtranslation{iyejtyal wokol}

\entry{águila real}
\partofspeech{f}
\spanishtranslation{kolembä xiye'}
\clarification{ave}

\entry{aguja}
\partofspeech{f}
\spanishtranslation{akuxañ}
\secondaryentry{aguja para quitar espinas}
\secondtranslation{p'iko'ch'ix}

\entry{agujerar}
\partofspeech{vt}
\spanishtranslation{ch'ub}
\clarification{lámina, tabla}
\secondaryentry{agujerado}
\secondtranslation{ch'äch'äñtyik}
\clarification{tela, lámina}
\secondtranslation{ch'ubul}
\clarification{madera, balde, cuero, lámina}

\entry{agutí}
\partofspeech{m}
\spanishtranslation{ujchib, xuchijp}
\clarification{mamífero}

\entry{ahí}
\partofspeech{adv}
\spanishtranslation{ya'äch, ya'i}

\entry{ahijado}
\partofspeech{m}
\spanishtranslation{*jala'al}

\entry{ahogar}
\partofspeech{vt}
\spanishtranslation{jik'tyañ}
\clarification{con comida}

\entry{ahora}
\partofspeech{adv}
\spanishtranslation{wäle}

\entry{ahorquillar}
\partofspeech{vt}
\secondaryentry{ahorquillado}
\spanishtranslation{xakal, xäk'äl}
\clarification{ramas de árbol}

\entry{ahumar}
\partofspeech{vt}
\secondaryentry{ahumado}
\secondtranslation{i'ik'ix tyi buts'}

\entry{aire}
\partofspeech{m}
\onedefinition{1}
\spanishtranslation{ik'}
\onedefinition{2}
\spanishtranslation{ik'uyel}
\clarification{del estómago}

\entry{ajo}
\partofspeech{m}
\spanishtranslation{axux, x'axux}

\entry{ajustar}
\partofspeech{vt}
\secondaryentry{ajustado}
\secondtranslation{jaxakña}

\entry{ala}
\partofspeech{f}
\spanishtranslation{wich'}

\entry{alacrán}
\partofspeech{m}
\spanishtranslation{siñañ}

\entry{alambrado}
\partofspeech{m}
\spanishtranslation{ch'ix tyak'iñ}

\entry{alargar}
\partofspeech{vt}
\spanishtranslation{päl'esañ}
\clarification{mecate, soga}

\entry{alcalde}
\partofspeech{m}
\spanishtranslation{ajkal}

\entry{alcanzar}
\partofspeech{vt}
\onedefinition{1}
\spanishtranslation{\textsuperscript{1}tyaj}
\onedefinition{2}
\spanishtranslation{jastyiyel}
\clarification{alimento, paga, ropa}
\secondaryentry{apenas alcanza}
\secondtranslation{k'ebel}

\entry{alentar}
\partofspeech{vt}
\spanishtranslation{moñ}
\clarification{niño}

\entry{algodón}
\partofspeech{m}
\spanishtranslation{tyiñäm, tyäñäm}

\entry{alguacil}
\partofspeech{m}
\spanishtranslation{wasil}

\entry{alimentar}
\partofspeech{vt}
\spanishtranslation{buk'sañ}
\clarification{animales}

\entry{alimento}
\partofspeech{m}
\onedefinition{1}
\spanishtranslation{\textsuperscript{2}bäl}
\clarification{en el plato}
\onedefinition{2}
\spanishtranslation{*we'el, we'eläl,bälñäk'äl}

\entry{alinear}
\partofspeech{vt}
\spanishtranslation{tsol}
\clarification{poner en fila}

\entry{allá}
\partofspeech{adv}
\onedefinition{1}
\spanishtranslation{ya'}
\onedefinition{2}
\spanishtranslation{ixixi, ixtyi}
\clarification{hasta allá}
\onedefinition{3}
\spanishtranslation{ixäch}
\clarification{señalando}

\entry{allí}
\partofspeech{adv}
\onedefinition{1}
\spanishtranslation{ya'i}
\onedefinition{2}
\spanishtranslation{ixba'añ}
\clarification{donde está eso}

\entry{almacén}
\partofspeech{m}
\spanishtranslation{*yotylel}
\clarification{de maíz}

\entry{almácigo}
\partofspeech{m}
\spanishtranslation{almasio}

\entry{almohada}
\partofspeech{f}
\spanishtranslation{k'äñjoläl}

\entry{alomar}
\partofspeech{vt}
\secondaryentry{alomado}
\secondtranslation{p'uchul}

\entry{alrededor}
\onedefinition{1}
\partofspeech{adv}
\spanishtranslation{selol}
\onedefinition{2}
\partofspeech{m}
\spanishtranslation{*sutyolel}
\clarification{sing.}
\spanishtranslation{*joytyilel}
\clarification{pl.}

\entry{alto}
\partofspeech{adj}
\spanishtranslation{chañ}
\secondaryentry{lugar alto}
\secondtranslation{k'eloñib}

\entry{altura}
\partofspeech{f}
\onedefinition{1}
\spanishtranslation{*chañelal, *chañlel}
\onedefinition{2}
\spanishtranslation{*kotyilel, *pitytyälel}
\clarification{de animales}
\onedefinition{3}
\spanishtranslation{*chaxtyiklel}
\clarification{tapesco para secar café, frijol}
\onedefinition{4}
\spanishtranslation{*ñojal}
\clarification{de niños o animales}
\onedefinition{5}
\spanishtranslation{*wa'tyilel}
\clarification{de persona}

\entry{amable}
\partofspeech{adj}
\secondaryentry{es muy amable}
\secondtranslation{uts ipusik'al}

\entry{Amado Nervo}
\spanishtranslation{Amarko}
\clarification{colonia}

\entry{amanecer}
\partofspeech{m}
\secondaryentry{al amanecer}
\secondtranslation{säk'ajel}

\entry{amansar}
\partofspeech{vt}
\onedefinition{1}
\spanishtranslation{utsbiñ}
\onedefinition{2}
\spanishtranslation{mañsojiyel}
\clarification{caballo o mula}

\entry{amar}
\partofspeech{vt}
\spanishtranslation{k'uxbiñ}

\entry{amargo}
\partofspeech{adj}
\spanishtranslation{ch'aj}

\entry{amarillo}
\partofspeech{adj}
\onedefinition{1}
\spanishtranslation{k'äñk'äñ, k'äñtyijañ}
\onedefinition{2}
\spanishtranslation{k'äñlemañ}
\clarification{milpa, pero sin cosecha}
\secondaryentry{ponerse amarillo}
\secondtranslation{*k'äñajel}
\secondaryentry{amarillo y plano}
\secondtranslation{k'äñwechañ}
\clarification{objeto}

\entry{amarrador}
\partofspeech{m}
\spanishtranslation{*käjchil}
\clarification{de tela}

\entry{amarrar}
\onedefinition{1}
\partofspeech{vt}
\spanishtranslation{käch}
\onedefinition{2}
\partofspeech{vt}
\spanishtranslation{\textsuperscript{1}jop'}
\clarification{enaguas}
\onedefinition{3}
\partofspeech{vt}
\spanishtranslation{mochilañ, mochiñ}
\clarification{animal}
\onedefinition{4}
\partofspeech{vi}
\spanishtranslation{käjchel}
\secondaryentry{amarrar carga}
\secondtranslation{p'äty}
\secondaryentry{amarrar repetidas veces}
\secondtranslation{mochiñ}
\clarification{costal}
\secondaryentry{amarrado}
\secondtranslation{kächäl}
\secondtranslation{jity'bil}
\clarification{cerco, puente}
\secondtranslation{suty' kächbil}
\clarification{un rollo}

\entry{amate}
\partofspeech{m}
\spanishtranslation{\textsuperscript{2}juñ}
\clarification{árbol}

\entry{Amate Hollado por Dentro}
\spanishtranslation{Ty'obojuñ}
\clarification{colonia}

\entry{amigo}
\partofspeech{m}
\spanishtranslation{pi'äl}

\entry{amonestación}
\partofspeech{f}
\spanishtranslation{*tyik'ol}

\entry{amontonar}
\onedefinition{1}
\partofspeech{vt}
\spanishtranslation{much'chokoñ}
\clarification{granos de café, piedra}
\onedefinition{2}
\partofspeech{vt}
\spanishtranslation{p'ulchokoñ}
\clarification{maíz, frijol, piedra}
\secondaryentry{amontonarse}
\secondtranslation{mojtyañ}
\clarification{moscas en comida}
\secondaryentry{amontonado}
\secondtranslation{p'ul, p'ulul, p'ulukña}
\secondtranslation{jomol}
\clarification{gente}
\secondtranslation{much'ul}
\clarification{como granos de café, piedra}
\secondtranslation{p'uchul}
\clarification{cosas}
\secondtranslation{wamal}
\clarification{muchas cosas}
\secondtranslation{*witsikña}
\clarification{maíz, café, frijol}
\secondtranslation{wotsol}
\clarification{ropa, arriera o avispa ya muerta}
\secondtranslation{yomol}
\clarification{rollo de palos}
\secondaryentry{amontonados así}
\secondtranslation{omtyäl}
\clarification{señalando personas, animales}

\entry{ampliar}
\onedefinition{1}
\partofspeech{vt}
\spanishtranslation{ñuk'esañ}
\onedefinition{2}
\partofspeech{vi}
\spanishtranslation{ñuk'esäñtyel}

\entry{ampolla}
\partofspeech{f}
\spanishtranslation{*wosmäjel}

\entry{ampollar}
\partofspeech{vt}
\spanishtranslation{wosmäl}
\clarification{piel}
\secondaryentry{ampollado}
\secondtranslation{wosol}

\entry{analizar}
\partofspeech{vt}
\spanishtranslation{ñejñañ}
\clarification{a un paciente}

\entry{ancho}
\partofspeech{adj}
\onedefinition{1}
\spanishtranslation{pechekña}
\clarification{piedra}
\onedefinition{2}
\spanishtranslation{wech wech}
\clarification{lámina}
\onedefinition{3}
\spanishtranslation{pats'al}
\clarification{el pie}
\secondaryentry{ancho y largo}
\secondtranslation{pach'al}
\clarification{pie, mano}

\entry{anchura}
\partofspeech{f}
\spanishtranslation{*leñtyilel, *weltyiklel}
\secondaryentry{anchura de la mano}
\secondtranslation{*pats'tyilel laj k'äb}

\entry{anciano}
\partofspeech{m}
\spanishtranslation{¡tyatye!}
\clarification{Sab.; saludo a un anciano de un joven}

\entry{andamiada}
\partofspeech{f}
\spanishtranslation{*wa'lib}
\clarification{de un edificio, donde el cazador espera la caza}
\alsosee{miradero}

\entry{andar}
\partofspeech{vi}
\spanishtranslation{xäñ, mi icha'leñ xämbal}

\entry{anginas}
\partofspeech{f}
\spanishtranslation{*ñujp'ib lakuk'ib}

\entry{angosto}
\partofspeech{adj}
\onedefinition{1}
\spanishtranslation{läts'äl}
\clarification{camino}
\onedefinition{2}
\spanishtranslation{\textsuperscript{1}silikña}
\clarification{tabla}
\onedefinition{3}
\spanishtranslation{tsäyäl}
\clarification{cintura}

\entry{anguila}
\partofspeech{f}
\spanishtranslation{mijts'ity}
\clarification{pez}
\secondaryentry{falsa anguila}
\secondtranslation{mits'ijty}
\clarification{pez}

\entry{anillo}
\partofspeech{m}
\spanishtranslation{mätyk'äb}

\entry{animal}
\partofspeech{m}
\secondaryentry{animal doméstico}
\secondtranslation{aläk'äl}
\secondtranslation{aläk'il}
\clarification{Sab.}
\secondaryentry{animal silvestre}
\secondtranslation{bälmatye'el}
\secondaryentry{animal muerto que ya apesta}
\secondtranslation{cheñek}
\secondaryentry{animal negro}
\secondtranslation{x'ik'bäk'tyal}

\entry{ano}
\partofspeech{m}
\spanishtranslation{*ch'uyity}

\entry{anochecer}
\partofspeech{vi}
\spanishtranslation{ik'añ}

\entry{anona}
\partofspeech{f}
\onedefinition{1}
\spanishtranslation{k'ewex}
\clarification{árbol}
\onedefinition{2}
\spanishtranslation{k'ätsats}
\clarification{Tila; árbol}
\secondaryentry{anona morada}
\secondtranslation{pox}

\entry{ansiar}
\partofspeech{vt}
\spanishtranslation{k'upiñ}

\entry{anteayer}
\partofspeech{adv}
\spanishtranslation{chäbiji}

\entry{antemano}
\secondaryentry{de antemano}
\secondtranslation{wäñ}

\entry{antepasados}
\partofspeech{m}
\spanishtranslation{*päk'il}

\entry{anteriormente}
\partofspeech{adv}
\spanishtranslation{wajalix}

\entry{anudar}
\partofspeech{vt}
\secondaryentry{anudado}
\secondtranslation{k'ojlostyik}

\entry{anular}
\partofspeech{vt}
\spanishtranslation{jem}
\clarification{plan}

\entry{anunciar}
\partofspeech{vt}
\spanishtranslation{puk ty'añ}
\secondaryentry{anunciado}
\secondtranslation{tsiktyesäbil}

\entry{anzuelo}
\partofspeech{m}
\spanishtranslation{luko' chäy}

\entry{añadidura}
\partofspeech{f}
\spanishtranslation{*yok'äjib}

\entry{añadir}
\partofspeech{vt}
\spanishtranslation{tyäk'}

\entry{año}
\partofspeech{m}
\spanishtranslation{jab}
\secondaryentry{cada año}
\secondtranslation{ja'jabil}
\secondaryentry{fin del año}
\secondtranslation{ijilibal jabil}
\secondaryentry{todos los años}
\secondtranslation{pejtyel jabil}

\entry{añublo}
\partofspeech{m}
\spanishtranslation{*kuxemal, *kuxil}

\entry{apachurrar}
\onedefinition{1}
\partofspeech{vt}
\spanishtranslation{puch'}
\onedefinition{2}
\partofspeech{vi}
\spanishtranslation{tyup}

\entry{apagar}
\onedefinition{1}
\partofspeech{vt}
\spanishtranslation{yäp}
\onedefinition{2}
\partofspeech{vt}
\spanishtranslation{mäyxuñ}
\clarification{la vista}
\secondaryentry{apagarse}
\secondtranslation{yajpel}
\secondaryentry{apagándose}
\secondtranslation{yäpyäpña}
\clarification{candil cuando le falta petróleo}
\secondaryentry{apagado}
\secondtranslation{yajpem}

\entry{aparecer}
\partofspeech{vi}
\spanishtranslation{tsiktyiyel}

\entry{apellido}
\partofspeech{m}
\spanishtranslation{*jol ik'aba'}

\entry{apenas}
\partofspeech{adv}
\spanishtranslation{komkatsa', tsal}
\secondaryentry{apenas alcanza}
\secondtranslation{k'ebel}

\entry{apestoso}
\partofspeech{adj}
\spanishtranslation{tyuw, tyuweñ}

\entry{aplanar}
\partofspeech{vt}
\secondaryentry{aplanado y ancho}
\secondtranslation{petsekña}
\secondaryentry{en forma aplanada}
\secondtranslation{petsel}

\entry{aplastar}
\partofspeech{vt}
\onedefinition{1}
\spanishtranslation{ñety', tyeñ}
\clarification{con instrumento}
\onedefinition{2}
\spanishtranslation{pechtyeñ}
\clarification{en trampa}

\entry{apoyar}
\partofspeech{vt}
\secondaryentry{apoyarse}
\secondtranslation{xijty'añ}
\clarification{con palo, piedra}

\entry{aprender}
\partofspeech{vt}
\spanishtranslation{käñ, ñop, sojlel}

\entry{apresurar}
\partofspeech{vt}
\secondaryentry{apresúrate}
\secondtranslation{¡kukuñuñ!}

\entry{apretar}
\partofspeech{vt}
\onedefinition{1}
\spanishtranslation{pets'}
\clarification{con la mano}
\onedefinition{2}
\spanishtranslation{wity'}
\clarification{cincha de mula o rollo de leña}
\onedefinition{3}
\spanishtranslation{yäty'}
\clarification{leña al amarrarla}
\onedefinition{4}
\spanishtranslation{yots'}
\clarification{en la mano}
\secondaryentry{apretado}
\secondtranslation{luty'ul}
\clarification{piernas}

\entry{apuntalar}
\partofspeech{vt}
\spanishtranslation{xijk'otye'añ}

\entry{apuntar}
\partofspeech{vt}
\spanishtranslation{ñejñañ,tyojbiñ}
\clarification{con un arma}

\entry{apuñalar}
\partofspeech{vt}
\spanishtranslation{jek'}

\entry{apurar}
\partofspeech{vt}
\spanishtranslation{p'ulbeñ, sebuñ}
\secondaryentry{apúrate}
\secondtranslation{¡la'ñuñ!, ¡se'ñuñ!}
\secondaryentry{debes apurarte}
\secondtranslation{¡se'sebety!}

\entry{aquél}
\partofspeech{pron}
\spanishtranslation{ixiyi}
\secondaryentry{es aquél}
\secondtranslation{jiñjiñi}

\entry{aquí}
\partofspeech{adv}
\onedefinition{1}
\spanishtranslation{äj, ilayi, um, wä', wä'i}
\onedefinition{2}
\spanishtranslation{ilawä}
\clarification{Sab.}
\secondaryentry{aquí está}
\secondtranslation{umba'añ, wä'añ, uñ tsa'}

\entry{aquietar}
\onedefinition{1}
\partofspeech{vt}
\spanishtranslation{ñäch'tyesañ}
\clarification{a una persona}
\onedefinition{2}
\partofspeech{vi}
\spanishtranslation{ñäjch'el}

\entry{araña}
\partofspeech{f}
\spanishtranslation{am}

\entry{árbol}
\partofspeech{m}
\spanishtranslation{tye'}
\clarification{palo, madera}
\secondaryentry{árbol de sangre}
\secondtranslation{xch'ich'bäty}

\entry{Arboleda de Madera Colorada}
\spanishtranslation{Chäkluñtye'el}
\clarification{colonia}

\entry{arbusto}
\partofspeech{m}
\onedefinition{1}
\spanishtranslation{ch'upujk}
\onedefinition{2}
\spanishtranslation{xch'ajch'ib}
\clarification{tiene hojas como las de la palma}

\entry{arco}
\partofspeech{m}
\spanishtranslation{jaläjp}
\clarification{arma, puente}
\secondaryentry{arco iris}
\secondtranslation{ty'ox ja'}

\entry{ardilla}
\partofspeech{f}
\spanishtranslation{chuch, chäkjocho chuch}

\entry{arena}
\partofspeech{f}
\spanishtranslation{ji'}

\entry{arenal}
\partofspeech{m}
\spanishtranslation{\textsuperscript{1}ji'il}

\entry{arenoso}
\partofspeech{adj}
\secondaryentry{lugar arenoso}
\secondtranslation{ji'lumil}

\entry{arete}
\partofspeech{m}
\spanishtranslation{uya'}

\entry{arisco}
\partofspeech{adj}
\spanishtranslation{p'ip'}

\entry{arma}
\partofspeech{f}
\spanishtranslation{juloñib}

\entry{armadillo}
\partofspeech{m}
\spanishtranslation{ib}
\clarification{Tila}
\spanishtranslation{\textsuperscript{2}wech, xwech}

\entry{armónica}
\partofspeech{f}
\spanishtranslation{labityo}

\entry{aro}
\partofspeech{m}
\spanishtranslation{xotyolbä tye'}

\entry{aroma}
\partofspeech{m}
\spanishtranslation{*yujts'il}

\entry{aromático}
\partofspeech{adj}
\spanishtranslation{xojokña}

\entry{arquear}
\partofspeech{vt}
\spanishtranslation{k'uch}

\entry{arracimarse}
\partofspeech{prnl}
\secondaryentry{arracimado}
\spanishtranslation{palal}

\entry{arrancar}
\partofspeech{vt}
\onedefinition{1}
\spanishtranslation{bok}
\clarification{mata}
\onedefinition{2}
\spanishtranslation{bots'}
\clarification{poste}
\onedefinition{3}
\spanishtranslation{k'ol}
\clarification{hijuelo de plátano}
\onedefinition{4}
\spanishtranslation{jots'}
\clarification{poste, diente}
\secondaryentry{arrancarse}
\spanishtranslation{bojkel}
\clarification{camote, cabello}
\spanishtranslation{k'ojlel}
\clarification{hijuelo de plátano}
\spanishtranslation{k'ojkel}
\clarification{dientes}
\secondaryentry{arrancado}
\secondtranslation{bokbil}
\clarification{camote, papas, cebollas}
\secondtranslation{jots'bil}
\clarification{dientes, postes}

\entry{arrastrar}
\partofspeech{vt}
\spanishtranslation{jexbañ, jexk'uñ}
\clarification{viga, persona, animal; en el suelo}
\secondaryentry{arrastrando}
\secondtranslation{jäläkña, jolokña}
\clarification{culebra}
\secondtranslation{jetyjetyña, jexjexña}
\clarification{viga, persona, animal; en el suelo}
\secondaryentry{arrastrándose}
\secondtranslation{jexekña}

\entry{arrear}
\onedefinition{1}
\partofspeech{vt}
\spanishtranslation{ñijkañ}
\onedefinition{2}
\partofspeech{vt}
\spanishtranslation{wets'}
\onedefinition{3}
\partofspeech{vi}
\spanishtranslation{ñijkäñtyel}

\entry{arriba}
\partofspeech{adv}
\spanishtranslation{tyi chañ}

\entry{arriera}
\partofspeech{f}
\spanishtranslation{xu'}
\clarification{hormiga}

\entry{arriero}
\partofspeech{m}
\spanishtranslation{arieru, xwets' mula}

\entry{arrodillar}
\partofspeech{vt}
\spanishtranslation{ñokchokoñ}
\secondaryentry{arrodillarse}
\spanishtranslation{ñoktyäl}
\secondaryentry{arrodillado}
\secondtranslation{ñokol}

\entry{arroyo}
\partofspeech{m}
\onedefinition{1}
\spanishtranslation{arayojil, ja', pa'}
\onedefinition{2}
\spanishtranslation{jajpa'}
\clarification{pasa en medio de dos cerritos}

\entry{Arroyo Amarillo}
\spanishtranslation{K'äñk'ämpa'}
\clarification{finca}

\entry{Arroyo con Caracoles}
\spanishtranslation{Puypa'}
\clarification{lugar}

\entry{Arroyo de Bambú}
\spanishtranslation{Che'opa'}
\clarification{colonia}

\entry{Arroyo de Piedra Negra}
\spanishtranslation{Ik'tyumpa'}
\clarification{Tila; ranchería}

\entry{Arroyo de Tierra Blanca}
\spanishtranslation{Säklumpa'}
\clarification{colonia}

\entry{Arroyo del Palmar}
\spanishtranslation{Aktye'pa'}
\clarification{colonia}

\entry{Arroyo del Palo Colorado}
\spanishtranslation{Chäktye'pa'}
\clarification{colonia}

\entry{Arroyo Hondo}
\spanishtranslation{Tyambäpa'}
\clarification{colonia}

\entry{arroz}
\partofspeech{m}
\spanishtranslation{arus}

\entry{arrugar}
\partofspeech{vt}
\spanishtranslation{\textsuperscript{1}sowilañ}
\clarification{trapo}
\secondaryentry{arrugado}
\secondtranslation{lu'ichtyik}
\clarification{piel de una persona}
\secondtranslation{mujchik'}
\clarification{piel o fruta seca}
\secondtranslation{wojtsik'tyik}
\clarification{ropa, trapo}

\entry{asa}
\partofspeech{f}
\spanishtranslation{chukoñib}
\clarification{de trastos}

\entry{asar}
\partofspeech{vt}
\spanishtranslation{pojpoñ}
\secondaryentry{asado}
\secondtranslation{pojpobil}

\entry{ascender}
\partofspeech{vi}
\spanishtranslation{letsel}

\entry{asco}
\partofspeech{m}
\secondaryentry{tener asco por mal olor}
\secondtranslation{bi'leñ}

\entry{aserrador}
\partofspeech{m}
\onedefinition{1}
\spanishtranslation{xp'eltye'}
\onedefinition{2}
\spanishtranslation{ajp'eltye'}
\clarification{Sab.}

\entry{aserrar}
\partofspeech{vt}
\spanishtranslation{p'el}
\clarification{madera}

\entry{así}
\partofspeech{adv}
\onedefinition{1}
\spanishtranslation{\textsuperscript{1}abi, che', che'i, kiñtyäl}
\onedefinition{2}
\spanishtranslation{chebi}
\clarification{Sab.}
\secondaryentry{así como}
\secondtranslation{che'bajche'}
\secondaryentry{así debes decir}
\secondtranslation{che'ety}
\secondaryentry{así dice}
\secondtranslation{che'eñ}
\secondaryentry{así dicen}
\secondtranslation{che'ob}
\clarification{terminación}
\secondaryentry{así digo yo}
\secondtranslation{cho'oñ}
\secondaryentry{así es}
\secondtranslation{che'kuyi}
\clarification{respuesta}
\secondaryentry{así nada más}
\secondtranslation{chejachi, che' jach}
\secondaryentry{así suena}
\secondtranslation{p'äklaw}
\clarification{sonido de la lluvia}
\secondaryentry{así también}
\secondtranslation{che' ja'el}

\entry{asiento}
\partofspeech{m}
\onedefinition{1}
\spanishtranslation{*buchlib}
\onedefinition{2}
\spanishtranslation{xix}
\clarification{de café}

\entry{asolear}
\partofspeech{vt}
\secondaryentry{asolearse}
\secondtranslation{k'ix k'iñ}

\entry{áspero}
\partofspeech{adj}
\onedefinition{1}
\spanishtranslation{k'ojlostyik}
\clarification{madera}
\onedefinition{2}
\spanishtranslation{jeñk'extyik}
\clarification{árbol}
\onedefinition{3}
\spanishtranslation{jejñex}
\clarification{piedra o tabla}
\onedefinition{4}
\spanishtranslation{tsäts}
\clarification{palabra}

\entry{asqueroso}
\partofspeech{adj}
\onedefinition{1}
\spanishtranslation{bi'ijtyik}
\onedefinition{2}
\spanishtranslation{xijiñ}
\clarification{olor}

\entry{astilla}
\partofspeech{f}
\spanishtranslation{*xu'il}
\secondaryentry{astilla de palo}
\secondtranslation{*sajl tye'}

\entry{astillar}
\partofspeech{vt}
\secondaryentry{astillado}
\secondtranslation{xalal}

\entry{atajar}
\partofspeech{vt}
\spanishtranslation{mäktyañ}

\entry{atardecer}
\partofspeech{vi}
\secondaryentry{atardeciendo}
\secondtranslation{mächäkña}

\entry{atizar}
\partofspeech{vt}
\spanishtranslation{\textsuperscript{1}xik'}
\clarification{fuego}

\entry{atole}
\partofspeech{m}
\spanishtranslation{ul}
\clarification{de masa}

\entry{atorar}
\partofspeech{vt}
\secondaryentry{atorado}
\secondtranslation{kets'}

\entry{atrás}
\partofspeech{adv}
\onedefinition{1}
\spanishtranslation{wi'il}
\clarification{posición de persona al caminar}
\onedefinition{2}
\spanishtranslation{*paty}
\clarification{Sab.}

\entry{atrasado}
\partofspeech{adj}
\spanishtranslation{tyoñkots}

\entry{atravesar}
\partofspeech{vt}
\secondaryentry{atravesado}
\secondtranslation{k'äty, k'ätyäl}
\secondaryentry{atravesando}
\secondtranslation{ñäkäkña}

\entry{aumentar}
\partofspeech{vt}
\onedefinition{1}
\spanishtranslation{p'ew}
\clarification{bienes}
\onedefinition{2}
\spanishtranslation{p'ojlesañ}
\clarification{personas, animales}

\entry{aunque}
\partofspeech{conj}
\onedefinition{1}
\spanishtranslation{añkese}
\onedefinition{2}
\spanishtranslation{añkimi}
\clarification{Sab.}

\entry{autoridad}
\partofspeech{f}
\onedefinition{1}
\spanishtranslation{ambä iye'tyel}
\onedefinition{2}
\spanishtranslation{mayor}
\onedefinition{3}
\spanishtranslation{yumäl}
\clarification{Sab.; no indígena}

\entry{auxiliador}
\partofspeech{m}
\spanishtranslation{ajkoltyaya}
\clarification{Sab.}

\entry{auxiliar}
\partofspeech{m}
\spanishtranslation{wasil}
\clarification{de ayuntamiento}

\entry{auxilio}
\partofspeech{m}
\spanishtranslation{koltyäñtyel}

\entry{avanzar}
\partofspeech{vi}
\spanishtranslation{jojyel}

\entry{ave}
\partofspeech{f}
\secondaryentry{tipo de ave}
\spanishtranslation{tsuñkay}
\clarification{tipo de ave}
\secondaryentry{ave negra}
\secondtranslation{x'ik'bäk'tyal}

\entry{avergonzar}
\partofspeech{vt}
\spanishtranslation{kisñiñ}

\entry{avión}
\partofspeech{m}
\spanishtranslation{xwich'}

\entry{avisar}
\partofspeech{vt}
\spanishtranslation{sub}

\entry{avispa}
\partofspeech{f}
\spanishtranslation{xux}
\secondaryentry{avispa polista}
\spanishtranslation{jobeñtye' xux}
\secondaryentry{avispa de cabeza amarilla}
\secondtranslation{\textsuperscript{1}xoy}

\entry{ayer}
\partofspeech{adv}
\onedefinition{1}
\spanishtranslation{ak'bi}
\onedefinition{2}
\spanishtranslation{\textsuperscript{2}abi}
\clarification{Sab.}

\entry{ayuda}
\partofspeech{f}
\spanishtranslation{koltyaya}

\entry{ayudante}
\partofspeech{m}
\spanishtranslation{xkoltyaya}

\entry{ayudar}
\partofspeech{vt}
\spanishtranslation{koltyañ}
\secondaryentry{ayudado}
\secondtranslation{koltyäbil}

\entry{ayunar}
\partofspeech{vi}
\spanishtranslation{ch'ajbañ}

\entry{ayuno}
\partofspeech{m}
\spanishtranslation{ch'ajb}

\entry{azúcar}
\partofspeech{m}
\spanishtranslation{asukal}

\entry{azul}
\partofspeech{adj}
\onedefinition{1}
\spanishtranslation{yäjyäx}
\onedefinition{2}
\spanishtranslation{yäxlemañ}
\clarification{río, mar}
\onedefinition{3}
\spanishtranslation{yäxmulañ}
\clarification{oscuro}
\onedefinition{4}
\spanishtranslation{yäxmojañ}
\clarification{reflejado}
\secondaryentry{de color azul}
\secondtranslation{*yäxel}

\entry{babear}
\partofspeech{vi}
\secondaryentry{babeando}
\secondtranslation{ch'äyäkña}

\entry{bagre barrigón}
\onedefinition{1}
\spanishtranslation{\textsuperscript{1}xlu'}
\clarification{pez}
\onedefinition{2}
\spanishtranslation{*ajlu'}
\clarification{Tila; pez}

\entry{baile}
\partofspeech{m}
\spanishtranslation{soñ}

\entry{bajar}
\partofspeech{vi}
\onedefinition{1}
\spanishtranslation{jubel}
\onedefinition{2}
\spanishtranslation{jets}
\clarification{asiento de café, pozol}
\onedefinition{3}
\spanishtranslation{ju'sañ}
\clarification{alguna cosa}
\onedefinition{4}
\spanishtranslation{sajp'el}
\clarification{agua}
\secondaryentry{bajar y quedar}
\secondtranslation{sämtyäl}
\clarification{neblina}
\secondaryentry{bajar por un palo}
\secondtranslation{yujlel}

\entry{bajo}
\partofspeech{adj}
\onedefinition{1}
\spanishtranslation{jubeñ}
\clarification{precios}
\onedefinition{2}
\spanishtranslation{pek'}
\clarification{altura de casa}
\onedefinition{3}
\spanishtranslation{säp'äl}
\clarification{agua}

\entry{balanza}
\partofspeech{f}
\spanishtranslation{*p'isoñib}

\entry{balché}

\partofspeech{f}
\spanishtranslation{xijiñtye'}
\clarification{árbol}

\entry{balde}
\partofspeech{adv}
\secondaryentry{en balde}
\secondtranslation{pojob}
\clarification{Sab.}
\secondaryentry{de balde}
\secondtranslation{lolom jach, milik}
\secondtranslation{\textsuperscript{1}pojol}
\clarification{Tila}

\entry{balsa}
\partofspeech{f}
\spanishtranslation{poy}

\entry{bambú}
\partofspeech{m}
\onedefinition{1}
\spanishtranslation{chejp}
\clarification{planta}
\onedefinition{2}
\spanishtranslation{\textsuperscript{2}k'äñchejb}
\clarification{amarillo}
\onedefinition{3}
\spanishtranslation{jimba}
\clarification{Tila}
\secondaryentry{lugar donde hay mucho bambú}
\secondtranslation{chejboy, ch'ijbol}

\entry{banca}
\partofspeech{f}
\onedefinition{1}
\spanishtranslation{tyem}
\onedefinition{2}
\spanishtranslation{wañku}
\clarification{Sab.}
\secondaryentry{banqueta}
\secondtranslation{ts'ajk}

\entry{bañadero}
\partofspeech{m}
\secondaryentry{bañadero de puerco}
\secondtranslation{ts'ämi ichityam}

\entry{bañar}
\partofspeech{vt}
\spanishtranslation{ts'äñsañ}
\secondaryentry{bañado}
\secondtranslation{ts'äñsäbil}
\secondaryentry{bañarse}
\secondtranslation{ts'äñsäñtyel}

\entry{baño}
\partofspeech{m}
\spanishtranslation{ts'ämel}
\secondaryentry{baño de vapor}
\secondtranslation{pus}

\entry{barba}
\partofspeech{f}
\spanishtranslation{*tsuktyi', *tsutsel lakchoj}

\entry{barbilla}
\partofspeech{f}
\spanishtranslation{*xäk'tyi'}

\entry{barco}
\partofspeech{m}
\spanishtranslation{xäñwibäl}
\clarification{Sab.}

\entry{barrer}
\onedefinition{1}
\partofspeech{vt}
\spanishtranslation{misuñ}
\onedefinition{2}
\partofspeech{vi}
\spanishtranslation{misujel}
\onedefinition{3}
\partofspeech{vi}
\spanishtranslation{misuñtyel}
\clarification{parte de una ceremonia}

\entry{barro}
\partofspeech{m}
\spanishtranslation{wajpam}
\clarification{en la cara}

\entry{base}
\partofspeech{f}
\onedefinition{1}
\spanishtranslation{k'äklib}
\onedefinition{2}
\spanishtranslation{ñaklib}
\clarification{Sab.; de una casa}
\secondaryentry{base de cola de aves}
\secondtranslation{*chuñchuñ muty, *k'ajk}

\entry{basilisco}
\partofspeech{m}
\spanishtranslation{tseljol}

\entry{bastante}
\partofspeech{adj}
\onedefinition{1}
\spanishtranslation{oñ}
\onedefinition{2}
\spanishtranslation{ts'iwil}
\onedefinition{3}
\spanishtranslation{wets'ekña}
\clarification{hombres, animales, nubes}
\onedefinition{4}
\spanishtranslation{*yilal, \textsuperscript{1}yokä}
\clarification{trabajo}

\entry{basura}
\partofspeech{f}
\spanishtranslation{misujeläl}

\entry{batea}
\partofspeech{f}
\spanishtranslation{batyiya tye'}

\entry{batir}
\partofspeech{vt}
\onedefinition{1}
\spanishtranslation{puk'}
\clarification{pozol}
\onedefinition{2}
\spanishtranslation{yolk'iñ}
\clarification{lodo}

\entry{bautismo}
\partofspeech{m}
\spanishtranslation{ch'ämja'}

\entry{bautizar}
\partofspeech{vt}
\spanishtranslation{mi iyäk' ch'ämja'}

\entry{beber}
\partofspeech{vt}
\spanishtranslation{jap}
\secondaryentry{acto de bebe}
\secondtranslation{uch'el}

\entry{becerro}
\partofspeech{m}
\spanishtranslation{alä wakax}

\entry{bejuco}
\partofspeech{m}
\onedefinition{1}
\spanishtranslation{\textsuperscript{1}ak'}
\onedefinition{2}
\spanishtranslation{kojkom}
\clarification{para amarrar cercos}
\secondaryentry{tipo de bejuco silvestre}
\secondtranslation{bayil}
\secondtranslation{sibikchuch}
\clarification{de tierra caliente}
\secondtranslation{xyk'ak'}
\secondaryentry{bejuco de tecomate}
\secondtranslation{bux}
\secondaryentry{tipo de bejuco grande}
\secondtranslation{ch'äb}
\secondaryentry{bejuco que se tiende en el suelo}
\secondtranslation{pepech'ak'}
\secondaryentry{bejuco de alalcrán}
\secondtranslation{siñañ ak'}
\secondaryentry{bejuco de uva}
\secondtranslation{tsejluk ak'}
\secondaryentry{tipo de bejuco chico y espinoso}
\secondtranslation{ts'äbab}

\entry{bejuquillo verde}
\spanishtranslation{yaxajachañ}
\clarification{reptil}

\entry{belleza}
\partofspeech{f}
\spanishtranslation{*ty'ojläwib}

\entry{besar}
\partofspeech{vt}
\spanishtranslation{ts'ujts'uñ}

\entry{bien}
\partofspeech{adv}
\spanishtranslation{uts'aty, weñ}

\entry{bienestar}
\partofspeech{m}
\spanishtranslation{*weñlel}

\entry{bifurcación}
\partofspeech{f}
\spanishtranslation{*xäk' bij}
\clarification{del camino}

\entry{blanco}
\partofspeech{adj}
\onedefinition{1}
\spanishtranslation{säsäk, säkchaxañ, säkwa'añ}
\onedefinition{2}
\spanishtranslation{säkluts'añ}
\clarification{piel}
\onedefinition{3}
\spanishtranslation{säkmotyañ}
\clarification{conjunto de piedras}
\onedefinition{4}
\spanishtranslation{säktyijañ}
\clarification{cabello, trenzas}
\onedefinition{5}
\spanishtranslation{säkts'ijañ, säkñup'añ}
\clarification{piedra}
\onedefinition{6}
\spanishtranslation{säkwelañ}
\clarification{tela}
\onedefinition{7}
\spanishtranslation{säkwolañ}
\clarification{pelo}
\secondaryentry{blanco oscuro}
\secondtranslation{säktyojañ}
\clarification{como nubes}

\entry{blando}
\partofspeech{adj}
\spanishtranslation{k'uñ}
\clarification{alimentos, tierra, palo}
\secondaryentry{muy blando}
\secondtranslation{k'uñyumañ}
\clarification{palo}

\entry{boa}
\partofspeech{f}
\spanishtranslation{uchchañ}
\clarification{reptil}

\entry{boca}
\partofspeech{f}
\spanishtranslation{ejäl, *yej}
\secondaryentry{boca abajo}
\secondtranslation{ñukye'el, xity, xityil}
\secondaryentry{boca agachado}
\secondtranslation{ñukye'el}
\clarification{persona, cubeta}
\secondaryentry{boca arriba}
\secondtranslation{ch'a'al}
\clarification{acostado}
\secondaryentry{poner boca abajo}
\secondtranslation{ñukchokoñ, xitychokoñ}

\entry{bofa}
\partofspeech{adj}
\spanishtranslation{wojts}
\clarification{madera, tierra}

\entry{bola}
\partofspeech{f}
\spanishtranslation{k'oxol}
\clarification{de masa, alimento, tierra}
\secondaryentry{hacer bola}
\secondtranslation{k'olilañ}
\clarification{sing.}
\secondtranslation{k'oliñ}
\clarification{pl.}
\secondtranslation{woxilañ}
\clarification{masa, tierra}

\entry{bolsa}
\partofspeech{f}
\onedefinition{1}
\spanishtranslation{borxa, *borxajlel}
\onedefinition{2}
\spanishtranslation{mukujk}
\clarification{de tela}
\secondaryentry{bolsa donde se envasa el azúcar}
\secondtranslation{mukuk pisil}

\entry{bolsero espalda amarilla}
\spanishtranslation{yujyum}
\clarification{calandria real; ave}

\entry{bondad}
\partofspeech{f}
\spanishtranslation{*yutslel}

\entry{bonito}
\partofspeech{adj}
\onedefinition{1}
\spanishtranslation{*ty'ojol}
\onedefinition{2}
\spanishtranslation{k'otya}
\clarification{Sab., Tila}
\onedefinition{3}
\spanishtranslation{tsäñtsäña}
\clarification{sonido}

\entry{bordar}
\partofspeech{vt}
\spanishtranslation{\textsuperscript{2}joch'}
\clarification{tela}

\entry{bordón}
\partofspeech{m}
\spanishtranslation{ñä'tye'}
\clarification{Sab.}

\entry{borracho}
\partofspeech{adj}
\onedefinition{1}
\spanishtranslation{lemel, yäk}
\onedefinition{2}
\spanishtranslation{\textsuperscript{1}k'ixiñ}
\clarification{Sab.}

\entry{bosque}
\partofspeech{m}
\onedefinition{1}
\spanishtranslation{tye'el}
\clarification{pequeño}
\onedefinition{2}
\spanishtranslation{ñojtye'el}
\clarification{Sab.; grande}
\secondaryentry{bosque sin monte bajo}
\secondtranslation{kolokña}

\entry{Bosque de Los Huesos}
\spanishtranslation{Baktye'el}
\clarification{ranchería}

\entry{bostezo}
\partofspeech{m}
\spanishtranslation{jayäb}

\entry{botar}
\partofspeech{vt}
\secondaryentry{botado}
\secondtranslation{xipil}
\clarification{ave}
\secondtranslation{ty'usul}
\clarification{cosa gruesa}

\entry{botella}
\partofspeech{f}
\spanishtranslation{limetye}

\entry{botil}
\spanishtranslation{(reg.)}
\partofspeech{m}
\spanishtranslation{xpojkäm}
\clarification{frijol grande y colorado}

\entry{bramar}
\partofspeech{vi}
\secondaryentry{bramando}
\secondtranslation{jujujña}
\clarification{tigre}

\entry{brasa}
\partofspeech{f}
\spanishtranslation{*ñich k'ajk}

\entry{brasero}
\partofspeech{m}
\spanishtranslation{*yajñib ñich k'ajk}

\entry{bravo}
\partofspeech{adj}
\spanishtranslation{ch'ejl}

\entry{brazo}
\partofspeech{m}
\spanishtranslation{k'äbäl}
\secondaryentry{su brazo}
\secondtranslation{ik'äb}
\secondaryentry{en los brazos}
\secondtranslation{\textsuperscript{2}loch'}

\entry{breve}
\partofspeech{adj}
\secondaryentry{hacer breve}
\secondtranslation{kom'esañ}

\entry{brilloso}
\partofspeech{adj}
\spanishtranslation{k'äñlemañ, säklemañ}
\clarification{lámina}
\secondaryentry{brilloso y blanco}
\secondtranslation{säktyilañ}
\secondaryentry{brilloso y liso}
\secondtranslation{elekña}

\entry{brincar}
\partofspeech{vi}
\onedefinition{1}
\spanishtranslation{lujty'el}
\clarification{caer en el mismo lugar}
\onedefinition{2}
\spanishtranslation{tyijp'ejl}
\clarification{con paso}
\secondaryentry{brincar con un pie}
\secondtranslation{ty'ijchiñ}

\entry{brisa}
\partofspeech{f}
\spanishtranslation{*yik'il}
\clarification{de río}

\entry{broma}
\partofspeech{f}
\spanishtranslation{alas ty'añ}

\entry{brotar}
\partofspeech{vi}
\spanishtranslation{pasel}
\clarification{planta}
\secondaryentry{brotado}
\secondtranslation{bulux}
\secondaryentry{brotando}
\secondtranslation{wulwulña}
\clarification{lentamente}

\entry{brote}
\partofspeech{m}
\secondaryentry{brote de agua}
\secondtranslation{bulux ja'}

\entry{broza}
\partofspeech{f}
\spanishtranslation{k'u'}
\clarification{de la milpa}

\entry{brujo}
\partofspeech{m}
\spanishtranslation{sts'äkaya, xiba, xwujty}

\entry{brumoso}
\partofspeech{adj}
\spanishtranslation{mäkäl}

\entry{bueno}
\partofspeech{adj}
\onedefinition{1}
\spanishtranslation{uts}
\onedefinition{2}
\spanishtranslation{uts'aty}
\clarification{una respuesta}
\secondaryentry{buen tiempo}
\secondtranslation{jamäl}
\secondaryentry{no está bueno}
\secondtranslation{mach weñik}

\entry{burbujear}
\partofspeech{vi}
\secondaryentry{burbujeante}
\secondtranslation{sorokña}

\entry{burla}
\partofspeech{f}
\spanishtranslation{wajal}

\entry{burlar}
\partofspeech{vt}
\spanishtranslation{wajleñ}

\entry{buscador}
\partofspeech{m}
\spanishtranslation{xsäklaya}

\entry{buscar}
\partofspeech{vt}
\spanishtranslation{sajkañ, säklañ}
\secondaryentry{buscar pleito}
\secondtranslation{tyech ty'añ}
\clarification{Sab.}

\entry{caballero}
\partofspeech{m}
\spanishtranslation{pujyu'}
\clarification{chotacabra; ave}

\entry{caballete}
\partofspeech{m}
\spanishtranslation{*jol otyoty, tyek'jol}
\clarification{de casa}

\entry{caballito del diablo}
\partofspeech{m}
\spanishtranslation{tyujlux}
\clarification{libélula; insecto}

\entry{caballo}
\partofspeech{m}
\spanishtranslation{kawayu', k'ächlibäl}

\entry{cabello}
\partofspeech{m}
\spanishtranslation{*tsutsel lakol}

\entry{cabeza}
\partofspeech{f}
\spanishtranslation{*jol}
\secondaryentry{cabeza blanca}
\secondtranslation{säkix ijol}
\clarification{vieja del monte}
\secondaryentry{cabeza de negro}
\secondtranslation{kulak'}
\clarification{bejuco grande}
\secondaryentry{de cabeza}
\secondtranslation{xityikña}
\secondaryentry{sin cabeza}
\secondtranslation{borol ibik'}

\entry{Cabeza del Arroyo}
\onedefinition{1}
\spanishtranslation{Jolja'}
\clarification{colonia}
\onedefinition{2}
\spanishtranslation{Jolñopa'}
\clarification{Tila; colonia}

\entry{Cabeza del Cerro}
\spanishtranslation{Jolwits}
\clarification{colonia}

\entry{Cabeza del Zacatal}
\spanishtranslation{Joljamil}
\clarification{colonia}

\entry{cacahuate}
\partofspeech{m}
\spanishtranslation{mañax bu'ul}

\entry{cacao}
\partofspeech{m}
\spanishtranslation{käkäw}

\entry{cacaotal}
\partofspeech{m}
\spanishtranslation{käkäwol}

\entry{cacarear}
\partofspeech{vi}
\secondaryentry{cacareando}
\secondtranslation{ch'e'ekña}

\entry{cacaté}
\partofspeech{m}
\secondaryentry{lugar donde hay muchas matas de cacaté}
\secondtranslation{*käkätye'ol}

\entry{cacería}
\partofspeech{f}
\spanishtranslation{julbäl}

\entry{cada}
\partofspeech{adj}
\secondaryentry{a cada rato}
\spanishtranslation{wo'wo}
\secondaryentry{cada año}
\spanishtranslation{ja'jabil}
\secondaryentry{cada cinco}
\spanishtranslation{jo'jo'p'ejl}
\secondaryentry{cada clase}
\spanishtranslation{jujuñchajp}
\secondaryentry{cada día}
\spanishtranslation{jujump'ejl k'iñ}
\secondaryentry{cada uno}
\spanishtranslation{jujump'ejl}
\secondaryentry{cada vez}
\spanishtranslation{jajayajl, jujuñyajl}
\secondaryentry{cada vez más}
\secondtranslation{utsi}

\entry{cadáver}
\partofspeech{m}
\spanishtranslation{ch'ujleläl}

\entry{cadena}
\partofspeech{f}
\spanishtranslation{kareña}

\entry{cadera}
\partofspeech{f}
\spanishtranslation{*jol ya'}

\entry{caer}
\partofspeech{vi}
\onedefinition{1}
\spanishtranslation{yajlel, bojkel}
\onedefinition{2}
\spanishtranslation{jojchel}
\clarification{calzón, pantalón}
\onedefinition{3}
\spanishtranslation{p'ajtyel}
\clarification{fruta, carne, dinero, tortillas}
\onedefinition{4}
\spanishtranslation{sijpel}
\clarification{trampa}
\onedefinition{5}
\spanishtranslation{wejtyuyel}
\clarification{arroz, frijol, maíz}
\secondaryentry{a punto de caer}
\secondtranslation{xuyiña}
\clarification{carga de animal}
\secondaryentry{dejar caer}
\secondtranslation{yäsañ; p'ätsañ}
\clarification{alimento o dinero}
\secondaryentry{manera de caer}
\secondtranslation{wity}
\secondaryentry{caído}
\secondtranslation{yajlem, bujchem}
\clarification{árbol, casa}
\secondaryentry{cayendo}
\secondtranslation{ts'äplaw}
\clarification{cuantas gotas}

\entry{café}
\partofspeech{m}
\onedefinition{1}
\spanishtranslation{kajpe'}
\onedefinition{2}
\spanishtranslation{kajwe'}
\clarification{Tila}
\onedefinition{3}
\spanishtranslation{kape}
\clarification{Sab.}
\secondaryentry{mata de café}
\secondtranslation{itye'el kajpe'}
\secondaryentry{fruta verde del café}
\secondtranslation{*ch'okel}

\entry{cafetal}
\partofspeech{m}
\onedefinition{1}
\spanishtranslation{kajpe'lel}
\onedefinition{2}
\spanishtranslation{kapejol}
\clarification{Sab.}

\entry{caída}
\partofspeech{f}
\secondaryentry{caída de agua}
\secondtranslation{*wejlib ja'}

\entry{caimito}
\partofspeech{m}
\spanishtranslation{tyak'iñ ch'ijty}
\clarification{palo de canela; árbol}

\entry{cajete}
\partofspeech{m}
\spanishtranslation{ch'ejew}
\clarification{cazuela hecha de barro}

\entry{cajón}
\partofspeech{m}
\spanishtranslation{kajoñtye', kaxatye'}

\entry{cal}
\partofspeech{f}
\spanishtranslation{tyañ}

\entry{calabaza}
\partofspeech{f}
\spanishtranslation{ch'ujm}

\entry{calambre}
\partofspeech{m}
\spanishtranslation{lo'chij}

\entry{calandria}
\partofspeech{f}
\spanishtranslation{tyojty}
\clarification{ave}
\secondaryentry{calandria real}
\secondtranslation{yujyum}
\clarification{ave}

\entry{calavera}
\partofspeech{f}
\spanishtranslation{*bäkel joläl}

\entry{caldo}
\partofspeech{m}
\spanishtranslation{\textsuperscript{2}*ya'lel}
\clarification{Sab.}
\secondaryentry{sin caldo}
\secondtranslation{tyikiñjopol}
\clarification{frijol}

\entry{calentar}
\partofspeech{vt}
\onedefinition{1}
\spanishtranslation{k'ix}
\clarification{en el fuego}
\onedefinition{2}
\spanishtranslation{k'ixñesañ}
\clarification{poco}
\onedefinition{3}
\spanishtranslation{tyikwesañ}
\clarification{mucho}

\entry{calentura}
\partofspeech{f}
\spanishtranslation{k'ajk}

\entry{caliente}
\partofspeech{adj}
\onedefinition{1}
\spanishtranslation{\textsuperscript{2}k'ixiñ}
\clarification{poco}
\onedefinition{2}
\spanishtranslation{tyikäw}
\clarification{muy}

\entry{cáliz}
\partofspeech{m}
\spanishtranslation{*yutybal}

\entry{callado}
\partofspeech{adj}
\spanishtranslation{ämäl, ñäch'äl}

\entry{callar}
\partofspeech{vt}
\spanishtranslation{ch'ä'tyesañ}
\clarification{Sab.}
\secondaryentry{cállate}
\secondtranslation{¡ch'äbix!}

\entry{callejón}
\partofspeech{m}
\spanishtranslation{kayajoñ}

\entry{calmar}
\onedefinition{1}
\partofspeech{vt}
\spanishtranslation{läm, ñäjch'el}
\clarification{enfermedad}
\onedefinition{2}
\partofspeech{vi}
\spanishtranslation{lämtyäl}
\clarification{Sab.; enfermedad}

\entry{calor}
\partofspeech{m}
\spanishtranslation{tyikwal}
\secondaryentry{hacer calor}
\secondtranslation{k'ixiñ pañimil, tyikäw pañimil}

\entry{caluroso}
\partofspeech{adj}
\onedefinition{1}
\spanishtranslation{chäpäkña}
\clarification{adentro de la casa}
\onedefinition{2}
\spanishtranslation{yo'okñajax}
\clarification{con fiebre}
\onedefinition{3}
\spanishtranslation{yowyowña}
\clarification{adentro del cuerpo}

\entry{calvo}
\partofspeech{adj}
\spanishtranslation{chäkpiräñ jol, xpok'jol, ch'umjol}
\clarification{lit.: cabeza de calabaza}

\entry{calzoncillo}
\partofspeech{m}
\spanishtranslation{*wex, wexäl}

\entry{cama}
\partofspeech{f}
\onedefinition{1}
\spanishtranslation{ch'ak}
\clarification{fija}
\onedefinition{2}
\spanishtranslation{wäyib, wäyibäl}
\clarification{movible}

\entry{camaleón}
\partofspeech{m}
\spanishtranslation{o'chañ}
\clarification{reptil}

\entry{camarón}
\partofspeech{m}
\spanishtranslation{xex, xuñ}
\secondaryentry{lugar de camarones}
\secondtranslation{xexkokil}

\entry{cambiar}
\onedefinition{1}
\partofspeech{vi}
\spanishtranslation{k'axel, k'ex}
\onedefinition{2}
\partofspeech{vt}
\spanishtranslation{jaw}
\clarification{moneda}
\onedefinition{3}
\partofspeech{vt}
\spanishtranslation{yäñ}
\clarification{lugar, color o dibujo}
\secondaryentry{cambiado}
\secondtranslation{yäñäl}
\secondaryentry{cambiarse}
\secondtranslation{k'extyiyel}
\secondtranslation{yäjñel}
\clarification{de cara}
\secondaryentry{acción de cambiar}
\secondtranslation{k'exoñel}

\entry{cambio}
\partofspeech{m}
\spanishtranslation{sujtyib}
\clarification{dinero}

\entry{camilla}
\partofspeech{f}
\secondaryentry{camilla de un muerto}
\secondtranslation{*k'ejchil}

\entry{caminar}
\partofspeech{vt}
\spanishtranslation{xäñtyesañ}
\clarification{hacer}
\secondaryentry{caminando}
\secondtranslation{kotykotyña}
\clarification{persona, en cuatro patas}
\secondtranslation{ty'ity'ichñayok}
\clarification{de puntillas}
\secondtranslation{xäñwi}
\clarification{Sab.}
\secondaryentry{manera de caminar echando la cabeza hacia atrás}
\secondtranslation{ch'a'ch'a'ña}

\entry{camino}
\partofspeech{m}
\spanishtranslation{bij}
\secondaryentry{camino real}
\secondtranslation{kolem bij}
\secondaryentry{camino compuesto de palos atravesados}
\secondtranslation{ji'tye'ol}
\secondaryentry{camino principal}
\secondtranslation{ñoj bij}

\entry{camisa}
\partofspeech{f}
\spanishtranslation{bujkäl}

\entry{camote}
\partofspeech{m}
\spanishtranslation{ajkum}
\secondaryentry{primera raíz del camote}
\secondtranslation{*chumwi' ajkum}

\entry{campana}
\partofspeech{f}
\spanishtranslation{xpapañichim, pañpañichim}
\clarification{arbusto}

\entry{campo}
\partofspeech{m}
\secondaryentry{campo de aterrizaje}
\secondtranslation{*yajlib xwich'}

\entry{campomocha}
\partofspeech{f}
\spanishtranslation{tyuch'k'iñ}
\clarification{tipo de mantis}

\entry{canasta}
\partofspeech{f}
\spanishtranslation{chikib}

\entry{candado}
\partofspeech{m}
\secondaryentry{echar candado}
\secondtranslation{ñety}

\entry{candente}
\partofspeech{adj}
\spanishtranslation{chäktsäñañ}
\clarification{fierro}

\entry{candil}
\partofspeech{m}
\spanishtranslation{kañtyil}
\clarification{víbora}

\entry{cangrejo}
\partofspeech{m}
\spanishtranslation{mep'}

\entry{canícula}
\partofspeech{f}
\spanishtranslation{k'äñ jaläjp}
\clarification{temporada de sequía en agosto o septiembre}

\entry{canilla}
\partofspeech{f}
\spanishtranslation{*tseñek}
\clarification{de la pierna}

\entry{canoa}
\partofspeech{f}
\spanishtranslation{*tyilib}

\entry{cansado}
\partofspeech{adj}
\onedefinition{1}
\spanishtranslation{lujb, lujbeñ}
\clarification{persona, animal}
\onedefinition{2}
\spanishtranslation{sop'}
\clarification{pies, piernas}

\entry{cansancio}
\partofspeech{m}
\spanishtranslation{*lujbel, *lujbeñal}

\entry{cansar}
\partofspeech{vt}
\secondaryentry{cansarse}
\secondtranslation{lujb'añ}

\entry{cantar}
\partofspeech{vt}
\spanishtranslation{k'äyiñ, mi icha'leñ k'ay}
\secondaryentry{cantando}
\secondtranslation{jichikña}
\clarification{chicharra}

\entry{cántaro}
\partofspeech{m}
\onedefinition{1}
\spanishtranslation{uk'um}
\onedefinition{2}
\spanishtranslation{akaras}
\clarification{Tila}

\entry{cantear}
\partofspeech{vt}
\secondaryentry{canteado}
\secondtranslation{bexel}

\entry{cantidad}
\partofspeech{f}
\secondaryentry{cantidad de agua}
\secondtranslation{*pomlel}

\entry{canto}
\partofspeech{m}
\onedefinition{1}
\spanishtranslation{k'ay}
\onedefinition{2}
\spanishtranslation{uk'el}
\clarification{radio, tocadisco, gallo}
\onedefinition{3}
\spanishtranslation{kokorojo'}
\clarification{gallo}

\entry{caña}
\partofspeech{f}
\spanishtranslation{sik'äb}
\secondaryentry{caña agria}
\secondtranslation{pajtyo'}
\clarification{hierba}

\entry{cañal}
\partofspeech{m}
\spanishtranslation{sik'bal}

\entry{caoba}
\partofspeech{f}
\spanishtranslation{suts'ul}

\entry{capa}
\partofspeech{f}
\spanishtranslation{mojch'}

\entry{capacitado}
\partofspeech{adj}
\spanishtranslation{sojlem}

\entry{capar}
\partofspeech{vt}
\secondaryentry{capado}
\secondtranslation{kapom}
\clarification{animal}

\entry{capulín cimarrón}
\spanishtranslation{ch'ajañ, pomoy}
\clarification{majagua colorada; árbol}

\entry{capulinero}
\partofspeech{m}
\spanishtranslation{xboch jol}

\entry{cara}
\partofspeech{f}
\onedefinition{1}
\spanishtranslation{\textsuperscript{1}*wuty}
\onedefinition{2}
\spanishtranslation{säkpojañ}
\clarification{medio blanca}

\entry{caracol}
\partofspeech{m}
\onedefinition{1}
\spanishtranslation{puy}
\onedefinition{2}
\spanishtranslation{ty'oty'}
\clarification{del monte}

\entry{carbón}
\partofspeech{m}
\spanishtranslation{abäk, *yäbäklel}

\entry{carbúnculo}
\partofspeech{m}
\spanishtranslation{chäkajl}

\entry{carcajada}
\partofspeech{f}
\secondaryentry{a carcajadas}
\secondtranslation{jajakña}

\entry{cárcel}
\partofspeech{f}
\spanishtranslation{mäjkibäl, ñujp'ibäl}

\entry{carga}
\partofspeech{f}
\spanishtranslation{*kuch, kuchäl}

\entry{cargador}
\partofspeech{m}
\spanishtranslation{xkuchijel}

\entry{cargar}
\partofspeech{vt}
\onedefinition{1}
\spanishtranslation{*kuch, kuchijel}
\clarification{sobre espalda}
\onedefinition{2}
\spanishtranslation{k'ech}
\clarification{en hombro}
\secondaryentry{cargado}
\secondtranslation{kuchbil}
\clarification{sobre espalda}
\secondtranslation{k'echel}
\clarification{en hombro}
\secondaryentry{bien cargado}
\secondtranslation{pakal}
\clarification{de maíz, fruta}

\entry{carie}
\partofspeech{f}
\spanishtranslation{*yik'al iyej}

\entry{carne}
\partofspeech{f}
\spanishtranslation{we'eläl, *we'el}

\entry{caro}
\partofspeech{adj}
\spanishtranslation{letsem}

\entry{carpa}
\partofspeech{f}
\spanishtranslation{xk'äñchäy}
\clarification{pez}

\entry{carpintero}
\partofspeech{m}
\spanishtranslation{xjuk'tye'}
\clarification{ave}
\secondaryentry{carpintero real}
\secondtranslation{stselel, xbijmuty}
\clarification{ave}
\secondaryentry{carpintero listado}
\secondtranslation{xch'ejku}
\clarification{ave}

\entry{carrizo}
\partofspeech{m}
\spanishtranslation{amäy, \textsuperscript{1}pochob}

\entry{carta}
\partofspeech{f}
\spanishtranslation{\textsuperscript{1}juñ}

\entry{cartucho}
\partofspeech{m}
\spanishtranslation{ibäk' juloñib}
\clarification{de arma}

\entry{casa}
\partofspeech{f}
\spanishtranslation{otyoty}
\secondaryentry{casa de animales}
\spanishtranslation{*yotylel}
\secondaryentry{entrada de casa}
\secondtranslation{ityi' otyoty}

\entry{casarse}
\partofspeech{prnl}
\spanishtranslation{ñujpuñel, päy}
\secondaryentry{dar en casamiento}
\secondtranslation{sijiñ}

\entry{cascajo}
\partofspeech{m}
\spanishtranslation{k'äsäb}

\entry{cáscara}
\partofspeech{f}
\onedefinition{1}
\spanishtranslation{ipaty tye'}
\clarification{corteza}
\onedefinition{2}
\spanishtranslation{*sujl}
\clarification{de arroz, café, frijol, maíz}
\secondaryentry{cáscara de árbol que usan para paludismo}
\secondtranslation{makuliñ}
\secondaryentry{cáscara del maíz}
\secondtranslation{jomojch'}

\entry{casco}
\partofspeech{m}
\spanishtranslation{ejk'ach}
\clarification{de caballo}

\entry{casi}
\partofspeech{adv}
\onedefinition{1}
\spanishtranslation{kolel, yomal}
\onedefinition{2}
\spanishtranslation{ts'itya'}
\clarification{falta poco}

\entry{caspa}
\partofspeech{f}
\spanishtranslation{xixjol}

\entry{castellano}
\partofspeech{m}
\spanishtranslation{kastyiya}

\entry{castigo}
\partofspeech{m}
\spanishtranslation{tyojmulil}

\entry{catarro}
\partofspeech{m}
\spanishtranslation{sijmal}

\entry{catorce}
\partofspeech{adj}
\spanishtranslation{chäñlujump'ejl}

\entry{causa}
\partofspeech{f}
\secondaryentry{por causa de}
\secondtranslation{kaj}

\entry{cavar}
\partofspeech{vt}
\spanishtranslation{pik}

\entry{cayuco}
\partofspeech{m}
\spanishtranslation{jukub}

\entry{cazar}
\partofspeech{vt}
\spanishtranslation{kolmäjel}

\entry{cedro}
\partofspeech{m}
\spanishtranslation{chäktye', ch'ujtye'}
\clarification{árbol}

\entry{cegar}
\partofspeech{vt}
\spanishtranslation{\textsuperscript{1}mixuñ}
\clarification{por luz fuerte}

\entry{ceiba}
\partofspeech{f}
\spanishtranslation{yäxtye'}
\clarification{árbol}

\entry{ceja}
\partofspeech{f}
\spanishtranslation{mätsab}

\entry{ceniza}
\partofspeech{f}
\spanishtranslation{tyäñäl k'ajk, ityäñil k'ajk}

\entry{cepillar}
\partofspeech{vt}
\spanishtranslation{\textsuperscript{1}juk'}
\clarification{madera, dientes}

\entry{cepillo}
\partofspeech{m}
\spanishtranslation{poki'ejäl, juk'o'ejäl}
\clarification{dental}

\entry{cera}
\partofspeech{f}
\spanishtranslation{tya'chäb}

\entry{cerbatana}
\partofspeech{f}
\spanishtranslation{jots'amäy}
\clarification{hecha de carrizo}

\entry{cerca}
\partofspeech{adv}
\onedefinition{1}
\spanishtranslation{läk', läk'äl}
\onedefinition{2}
\spanishtranslation{ñochol}
\clarification{pegado}
\onedefinition{3}
\spanishtranslation{k'ajpa'añ}
\clarification{visible}

\entry{cerca}
\partofspeech{f}
\spanishtranslation{*bojtye'lel}
\clarification{de casa}

\entry{cercar}
\partofspeech{vt}
\spanishtranslation{korälijel}
\secondaryentry{cercado}
\secondtranslation{koraläjem}

\entry{cerco}
\partofspeech{m}
\secondaryentry{hacer cerco}
\secondtranslation{korälijel}

\entry{cerdito}
\partofspeech{m}
\onedefinition{1}
\spanishtranslation{k'oyem}
\clarification{puerquito, viuda; ave}
\onedefinition{2}
\spanishtranslation{ch'ämpäk'}
\clarification{Sab.}

\entry{cerdo}
\partofspeech{m}
\spanishtranslation{chityam}

\entry{cerradura}
\partofspeech{f}
\spanishtranslation{*ñejty'il}
\clarification{de una casa}

\entry{cerrar}
\partofspeech{vt}
\onedefinition{1}
\spanishtranslation{ñup'}
\clarification{casa}
\onedefinition{2}
\spanishtranslation{muts'}
\clarification{ojos}
\secondaryentry{cerrar con llave}
\secondtranslation{ts'oty}
\clarification{Sab.}
\secondaryentry{cerrado}
\secondtranslation{ñup'ul}
\clarification{casa}
\secondtranslation{k'o'mäkäl}
\clarification{de nubes}
\secondtranslation{mäkäkña}
\clarification{por nubes}

\entry{cerro}
\partofspeech{m}
\onedefinition{1}
\spanishtranslation{wits}
\onedefinition{2}
\spanishtranslation{boltyäl}
\clarification{Sab.}
\secondaryentry{cerro grande}
\secondtranslation{ñoj wits}

\entry{Cerro Alto}
\spanishtranslation{K'uk'wits}
\clarification{Tumbalá}

\entry{Cerro del Tambor}
\spanishtranslation{Lajtye'wits}
\clarification{cerro}

\entry{Cerro Firme}
\spanishtranslation{Xu'wits}

\entry{chachalaca olivácea}
\spanishtranslation{xk'el}
\clarification{chachalaca común; ave}

\entry{chamarra}
\partofspeech{f}
\spanishtranslation{\textsuperscript{2}tsuts}

\entry{champa}
\partofspeech{vt}
\spanishtranslation{lejchiñpaty}
\clarification{Sab.}

\entry{chanté}
\partofspeech{m}
\spanishtranslation{chañtye'}
\clarification{madre de cacao; árbol}

\entry{chaparro}
\partofspeech{adj}
\secondaryentry{chaparrito}
\secondtranslation{pek'}

\entry{chapaya}
\partofspeech{f}
\onedefinition{1}
\spanishtranslation{bajtyuñ}
\clarification{tierna}
\onedefinition{2}
\spanishtranslation{chapäy}
\clarification{fruta de palma}
\secondaryentry{chapay amargo}
\secondtranslation{tyuk}
\clarification{vegetal comestible}

\entry{chaperla}
\partofspeech{f}
\onedefinition{1}
\spanishtranslation{xijiñtye'}
\clarification{balché; árbol}
\onedefinition{2}
\spanishtranslation{yaxokiñtye'}
\clarification{Tila; árbol}

\entry{chaporrear}
\partofspeech{vt}
\onedefinition{1}
\spanishtranslation{wejlujel}
\clarification{chapodar}
\onedefinition{2}
\spanishtranslation{\textsuperscript{2}wejluñ}
\onedefinition{3}
\spanishtranslation{\textsuperscript{1}päty}
\clarification{camino}

\entry{chapotear}
\partofspeech{vi}
\secondaryentry{chapoteando}
\secondtranslation{ts'opts'opña}

\entry{chapulín}
\partofspeech{m}
\onedefinition{1}
\spanishtranslation{sajk'}
\clarification{insecto}
\onedefinition{2}
\spanishtranslation{xk'ajbasajk'}
\secondaryentry{chapulín fraile}
\secondtranslation{xkukuchyopom}
\clarification{langosta verde}

\entry{chaquiste}
\partofspeech{m}
\spanishtranslation{ch'ikijl}
\clarification{insectillo crepuscular muy voraz}

\entry{chaya}
\partofspeech{f}
\spanishtranslation{* \textsuperscript{2}ek', x'ek'}
\clarification{quelite}
\secondaryentry{chaya pica}
\secondtranslation{stsaja tyuñ}

\entry{chayote}
\partofspeech{m}
\spanishtranslation{ch'ijch'um, ñi'uk'}

\entry{chayotestle}
\spanishtranslation{(reg.)}
\partofspeech{m}
\spanishtranslation{yame}
\clarification{raíz de chayote}

\entry{chechén}
\partofspeech{m}
\spanishtranslation{ixtye'}
\clarification{malamujer; árbol}

\entry{cheje}
\partofspeech{m}
\spanishtranslation{xch'ejku'}
\clarification{ave}

\entry{chicantor}
\partofspeech{m}
\spanishtranslation{tsijkotso'}
\clarification{ave}

\entry{chícharo}
\partofspeech{m}
\spanishtranslation{xtye' bu'ul}
\clarification{tipo de hierba}

\entry{chicharra}
\partofspeech{vt}
\spanishtranslation{jijch, ts'ijkityiñ}

\entry{chicle}
\partofspeech{m}
\onedefinition{1}
\spanishtranslation{tyulum}
\clarification{árbol}
\onedefinition{2}
\spanishtranslation{tsak'iñ}
\clarification{goma de mascar}

\entry{chico}
\partofspeech{adj}
\onedefinition{1}
\spanishtranslation{bik'ity}
\onedefinition{2}
\spanishtranslation{chuty}
\clarification{Sab., Tila}
\secondaryentry{chiquito}
\secondtranslation{ch'o'ch'ok}
\secondtranslation{wijwistyäl}
\clarification{granos}
\secondtranslation{wisil}
\clarification{pedacito}

\entry{chicote}
\partofspeech{m}
\spanishtranslation{asiyal, pächi}
\clarification{látigo}

\entry{chicotear}
\partofspeech{vt}
\spanishtranslation{jäjluñ}
\clarification{persona, animal}

\entry{chicozapote}
\partofspeech{m}
\spanishtranslation{bik'tyi chä'tye', chä'tye'}
\clarification{árbol}

\entry{chiflar}
\partofspeech{vt}
\spanishtranslation{ch'uybañ}

\entry{chile}
\partofspeech{m}
\spanishtranslation{ich}

\entry{chinche}
\partofspeech{m}
\spanishtranslation{poch'}
\clarification{insecto}

\entry{chincuyu}
\partofspeech{f}
\spanishtranslation{pox}
\clarification{anona colorada; árbol}

\entry{chinino}
\partofspeech{m}
\spanishtranslation{koyoj}
\clarification{tipo de aguacate}

\entry{chiquero}
\partofspeech{m}
\spanishtranslation{*mäjkib chityam}

\entry{chiquiguao}
\partofspeech{m}
\spanishtranslation{ch'ix ajk}
\clarification{tipo de tortuga}

\entry{chiquitín}
\partofspeech{m}
\spanishtranslation{ts'ikijk}

\entry{chisme}
\partofspeech{m}
\onedefinition{1}
\spanishtranslation{jop'ty'añ, u'yaj}
\onedefinition{2}
\spanishtranslation{choñtyi'}
\clarification{Tila}

\entry{chispa}
\partofspeech{f}
\secondaryentry{echar chispas}
\secondtranslation{ts'iplaw}

\entry{chispeante}
\partofspeech{adj}
\spanishtranslation{ts'äylaw}

\entry{chiste}
\partofspeech{m}
\spanishtranslation{alas ty'añ}

\entry{chistoso}
\partofspeech{adj}
\secondaryentry{chistosamente}
\secondtranslation{tse'tyeñtyik}

\entry{chorrear}
\partofspeech{vi}
\secondaryentry{chorreando}
\secondtranslation{kelkelña}
\clarification{sangre, agua}

\entry{chorro}
\partofspeech{m}
\secondaryentry{chorrito de agua}
\secondtranslation{chulu' ja'}

\entry{choza}
\partofspeech{f}
\spanishtranslation{lejchempaty}

\entry{chupaflor}
\partofspeech{m}
\spanishtranslation{ts'uñuñ}
\clarification{ave}

\entry{chupamiel}
\partofspeech{m}
\spanishtranslation{ts'u' chab}
\clarification{mamífero}

\entry{chupar}
\partofspeech{vt}
\onedefinition{1}
\spanishtranslation{majts'añ}
\clarification{mano}
\onedefinition{2}
\spanishtranslation{ñul}
\clarification{dulce}
\onedefinition{3}
\spanishtranslation{ts'u'}
\clarification{naranja}
\onedefinition{4}
\spanishtranslation{ts'u'ts'uñ}
\clarification{jugo de fruta}

\entry{chuy}
\partofspeech{m}
\spanishtranslation{ts'iwi'}
\clarification{hierba}

\entry{cicatriz}
\partofspeech{f}
\spanishtranslation{*lojweñal}

\entry{ciego}
\onedefinition{1}
\partofspeech{adj}
\spanishtranslation{pots'}
\onedefinition{2}
\partofspeech{m}
\spanishtranslation{xpots'}
\secondaryentry{ciego con los ojos abiertos}
\spanishtranslation{tyokal wuty}

\entry{cielo}
\partofspeech{m}
\onedefinition{1}
\spanishtranslation{pañchañ}
\onedefinition{2}
\spanishtranslation{chañ, gloria}
\clarification{Tila}

\entry{ciénaga}
\partofspeech{f}
\spanishtranslation{ja'lumil, lumija'}

\entry{cierto}
\partofspeech{adj}
\spanishtranslation{isujm}
\secondaryentry{es cierto}
\secondtranslation{melel}

\entry{ciervo volante}
\partofspeech{m}
\spanishtranslation{jäx}
\clarification{escarbajo de los árboles; insecto}

\entry{cigarra}
\partofspeech{f}
\spanishtranslation{jichityiñ}
\clarification{cicada, chicarra; insecto}

\entry{cigarro}
\partofspeech{m}
\spanishtranslation{k'ujts}

\entry{cilantro}
\partofspeech{m}
\spanishtranslation{kulañtya, xkulañtye}

\entry{cima}
\partofspeech{f}
\spanishtranslation{pam}
\secondaryentry{cima de una loma}
\secondtranslation{bujulum}

\entry{cinco}
\partofspeech{adj}
\spanishtranslation{jo'p'ejl}
\secondaryentry{cada cinco}
\secondtranslation{jo'jo'p'ejl}

\entry{cinta}
\partofspeech{f}
\spanishtranslation{*ts'omtye'lel}
\clarification{de casa}

\entry{cinturón}
\partofspeech{m}
\spanishtranslation{kajchiñäk', siñcho}

\entry{Circundado por Agua}
\spanishtranslation{Joyo'ja'}
\clarification{colonia}

\entry{circundar}
\partofspeech{vt}
\secondaryentry{circundado}
\secondtranslation{joyol}

\entry{circunferencia}
\partofspeech{f}
\spanishtranslation{*xotytyilel}

\entry{clan}
\partofspeech{m}
\spanishtranslation{majchil}
\clarification{Tila}

\entry{clara}
\partofspeech{f}
\spanishtranslation{*säkel tyumuty}
\clarification{del huevo}

\entry{claridad}
\partofspeech{f}
\onedefinition{1}
\spanishtranslation{säkjamtyäl}
\clarification{adentro}
\onedefinition{2}
\spanishtranslation{*säklel}
\clarification{afuera}
\onedefinition{3}
\spanishtranslation{*xojob}
\clarification{del sol}

\entry{claro}
\partofspeech{adj}
\onedefinition{1}
\spanishtranslation{chäkkolañ}
\onedefinition{2}
\spanishtranslation{yäx}
\clarification{agua}
\onedefinition{3}
\spanishtranslation{chäklemañ}
\clarification{luz}
\onedefinition{4}
\spanishtranslation{jajmeñ, säkjamañ}
\clarification{el tiempo}
\onedefinition{5}
\spanishtranslation{k'iñlaw}
\clarification{la noche}
\secondaryentry{se ve claro}
\secondtranslation{säkxojañ}

\entry{clavar}
\partofspeech{vt}
\onedefinition{1}
\spanishtranslation{ch'ij}
\onedefinition{2}
\spanishtranslation{\textsuperscript{2}baj}
\clarification{Sab., Tila}

\entry{clavo}
\partofspeech{m}
\onedefinition{1}
\spanishtranslation{lawux}
\onedefinition{2}
\spanishtranslation{*jok'lib}
\clarification{para colgar cosas}

\entry{clueca}
\partofspeech{f}
\spanishtranslation{xpäklojm}

\entry{coa}
\partofspeech{f}
\spanishtranslation{pikoñib}

\entry{cobertor}
\partofspeech{m}
\spanishtranslation{*tyasil}

\entry{cobija}
\partofspeech{f}
\spanishtranslation{\textsuperscript{2}tsuts}

\entry{cocer}
\partofspeech{vt}
\onedefinition{1}
\spanishtranslation{tyik'añ}
\onedefinition{2}
\spanishtranslation{ch'äx}
\clarification{hervir}
\secondaryentry{cocido}
\secondtranslation{ch'äxbil}
\secondaryentry{falta de cocimiento}
\secondtranslation{k'äsix}
\clarification{frijol}

\entry{cocina}
\partofspeech{f}
\spanishtranslation{kusiñaj}

\entry{cocodrilo}
\partofspeech{m}
\spanishtranslation{ajiñ}
\clarification{de pantano, río}

\entry{cocohuite}
\partofspeech{m}
\spanishtranslation{chañtye'}
\clarification{árbol}

\entry{cocoyol}
\partofspeech{m}
\spanishtranslation{mäp}
\clarification{Sab.; corozo; árbol}

\entry{cocsán}
\partofspeech{m}
\spanishtranslation{bits'}
\clarification{cuajinicuil; árbol}

\entry{cocuyo}
\partofspeech{m}
\spanishtranslation{kukluñtya'}
\clarification{insecto}

\entry{codo}
\partofspeech{m}
\spanishtranslation{xujk'äb}

\entry{codorniz}
\partofspeech{f}
\onedefinition{1}
\spanishtranslation{*ñakob, ñakol, xwak che'e'}
\clarification{ave}
\onedefinition{2}
\spanishtranslation{xchäläl}
\clarification{perdiz chica; ave con patas verdes}
\secondaryentry{codorniz bolonchaco}
\secondtranslation{chañwox}
\secondaryentry{codorniz común}
\secondtranslation{tyojk'ay}

\entry{cohete}
\partofspeech{m}
\spanishtranslation{sibik}

\entry{cojear}
\partofspeech{vi}
\spanishtranslation{k'ijchiñ}
\secondaryentry{cojeando}
\secondtranslation{k'uyiña, k'ejcheñ k'ejcheñ, k'ichk'ichña}

\entry{cojo}
\partofspeech{adj}
\spanishtranslation{xrokiñ ok}

\entry{cojolita}
\partofspeech{f}
\spanishtranslation{xwak ch'e'}
\clarification{pava cojolita; ave}

\entry{cola}
\partofspeech{f}
\spanishtranslation{ñej}
\clarification{de aves y animales}
\secondaryentry{cola larga de gallina o gallo}
\secondtranslation{xbits}
\secondaryentry{cola que es puro hueso}
\secondtranslation{xbakñej}
\secondaryentry{cola de fuego}
\secondtranslation{ik' lukum}
\clarification{Tila; reptil}

\entry{colar}
\partofspeech{vt}
\spanishtranslation{chik, chijkañ}

\entry{colectivo}
\partofspeech{adj}
\spanishtranslation{*tyem cha'añ}

\entry{colgar}
\partofspeech{vt}
\onedefinition{1}
\spanishtranslation{ts'uychokoñ}
\onedefinition{2}
\spanishtranslation{\textsuperscript{2}jich', jich'chokoñ}
\clarification{libremente}
\onedefinition{3}
\spanishtranslation{\textsuperscript{2}jok', jok'chokoñ}
\clarification{contra una pared}
\secondaryentry{colgar en un palo}
\secondtranslation{likchokoñ}
\secondaryentry{colgarse}
\secondtranslation{jich'tyäl}
\secondtranslation{ts'ujyel}
\clarification{sube y baja}
\secondaryentry{colgado}
\secondtranslation{jich'bil}
\clarification{por alguien}
\secondtranslation{jich'chokobil}
\clarification{libremente}
\secondtranslation{jich'il}
\clarification{cosa chica}
\secondtranslation{jok'ol}
\clarification{cosa grande}
\secondaryentry{colgado de la mano}
\secondtranslation{ts'uye'el}

\entry{colibrí}
\partofspeech{m}
\spanishtranslation{ts'uñuñ}
\clarification{ave}

\entry{colindancia}
\partofspeech{f}
\spanishtranslation{ñup'}

\entry{colindar}
\partofspeech{vi}
\secondaryentry{colindado}
\secondtranslation{jaxal}
\clarification{Sab.; terreno}

\entry{colmena}
\partofspeech{f}
\spanishtranslation{*yajñib chab, *yotylel chab}

\entry{colocar}
\partofspeech{vt}
\onedefinition{1}
\spanishtranslation{k'äktyäl}
\clarification{atravesado}
\onedefinition{2}
\spanishtranslation{k'o'chokoñ}
\clarification{objeto esférico}
\onedefinition{3}
\spanishtranslation{ekchokoñ}
\clarification{boca arriba}
\onedefinition{4}
\spanishtranslation{suty'chokoñ}
\clarification{rollo de zacate, palitos}
\secondaryentry{colocar boca arriba}
\secondtranslation{ch'a'chokoñ}
\clarification{persona, animal}
\secondaryentry{colocar con objeto redondo}
\secondtranslation{xetychokoñ}
\secondaryentry{colocar de lado}
\secondtranslation{ts'ejchokoñ}
\secondaryentry{colocar encima de}
\secondtranslation{chumchokoñ}
\secondaryentry{colocado}
\secondtranslation{ekel}
\clarification{canasto, plato}

\entry{color}
\partofspeech{m}
\spanishtranslation{*bojñil, *ts'ijbal}

\entry{colorado}
\partofspeech{adj}
\onedefinition{1}
\spanishtranslation{chächäk}
\clarification{rojo}
\onedefinition{2}
\spanishtranslation{chäklak'añ}
\clarification{piel}
\onedefinition{3}
\spanishtranslation{chäkwatsañ}
\clarification{pelo}
\onedefinition{4}
\spanishtranslation{chäkyumañ}
\clarification{líquido}
\secondaryentry{colorado y maduro}
\secondtranslation{chäkix}
\secondaryentry{medio colorado}
\secondtranslation{chäkkojañ}
\clarification{trapo, perro}

\entry{columna}
\partofspeech{f}
\secondaryentry{columna vertebral}
\secondtranslation{ch'ixpaty}

\entry{comadre}
\partofspeech{f}
\spanishtranslation{kumale}

\entry{comadreja}
\partofspeech{f}
\spanishtranslation{sajbiñ}
\clarification{mamífero}

\entry{comal}
\partofspeech{m}
\spanishtranslation{semejty}

\entry{comején}
\partofspeech{m}
\spanishtranslation{ts'islum}
\clarification{hormiga blanca}

\entry{comenzar}
\partofspeech{vi}
\onedefinition{1}
\spanishtranslation{kajel}
\onedefinition{2}
\spanishtranslation{ñijlel}
\clarification{Sab.}
\secondaryentry{comenzar a ser}
\secondtranslation{ochel}

\entry{comer}
\partofspeech{vt}
\onedefinition{1}
\spanishtranslation{\textsuperscript{1}k'ux, uch'eñ}
\onedefinition{2}
\spanishtranslation{k'uxuk}
\clarification{negativo}
\onedefinition{3}
\spanishtranslation{cha'leñ we'el}
\clarification{carne}
\onedefinition{4}
\spanishtranslation{mäk'}
\clarification{alimento blando}
\secondaryentry{acto de comer}
\secondtranslation{uch'el}
\secondaryentry{comer demasiado}
\secondtranslation{jojmañ}
\secondaryentry{comer poco}
\secondtranslation{ñich'käm}
\secondaryentry{comer una bola}
\secondtranslation{kats'}
\secondaryentry{ganas de comer}
\secondtranslation{*sits'}
\clarification{carne}

\entry{comestible}
\partofspeech{adj}
\onedefinition{1}
\spanishtranslation{k'uxbil}
\onedefinition{2}
\spanishtranslation{*k'uxbälel}
\clarification{Sab.}
\onedefinition{3}
\spanishtranslation{mäk'bil}
\clarification{blando}

\entry{comida}
\partofspeech{f}
\spanishtranslation{*buk'bal}
\clarification{de animales}
\secondaryentry{comida que se ofrece a los santos}
\secondtranslation{kompirar}

\entry{comienzo}
\partofspeech{m}
\spanishtranslation{*tyejchibal}

\entry{comisión}
\partofspeech{f}
\spanishtranslation{xpay}
\clarification{personas que llevan mensaje o carga}

\entry{como}
\partofspeech{adv}
\spanishtranslation{bajche'}
\secondaryentry{comoquiera}
\secondtranslation{bajche' ikach}
\clarification{sin cuidado}
\secondtranslation{ñuki}

\entry{cómo}
\partofspeech{adv}
\spanishtranslation{¿bajche'?}
\secondaryentry{cómo está}
\secondtranslation{¿bajche' yilal?}

\entry{compadecerse}
\partofspeech{prnl}
\spanishtranslation{p'uñtyäñtyel}

\entry{compadre}
\partofspeech{m}
\spanishtranslation{kumpale}

\entry{compañero}
\partofspeech{m}
\onedefinition{1}
\spanishtranslation{pi'äl}
\onedefinition{2}
\spanishtranslation{ajaw}
\clarification{del diablo}
\onedefinition{3}
\spanishtranslation{wajmäl}
\clarification{término de desprecio}
\onedefinition{4}
\spanishtranslation{wäy}
\clarification{espíritu}

\entry{completar}
\partofspeech{vt}
\spanishtranslation{ts'äktyesañ}

\entry{completo}
\partofspeech{adj}
\onedefinition{1}
\spanishtranslation{ts'äkäl}
\onedefinition{2}
\spanishtranslation{sety'el}
\clarification{cantidad necesaria para ajustar}

\entry{componerse}
\partofspeech{prnl}
\onedefinition{1}
\spanishtranslation{lajmel}
\clarification{el tiempo o por ej.: una disputa}
\onedefinition{2}
\spanishtranslation{uts'atyiyel}
\clarification{Sab.; el tiempo}

\entry{comprar}
\partofspeech{vt}
\spanishtranslation{mäñ}

\entry{comprender}
\partofspeech{vt}
\spanishtranslation{ch'ämbeñ isujm, ña'tyañ}

\entry{comprensible}
\partofspeech{adj}
\spanishtranslation{tsikil}

\entry{comprensión}
\partofspeech{f}
\spanishtranslation{*ña'tyäbal}

\entry{comprometer}
\partofspeech{vt}
\secondaryentry{comprometida}
\secondtranslation{k'ajtyibil}
\clarification{muchacha}

\entry{con}
\partofspeech{prep}
\spanishtranslation{yik'oty, yity'ok}

\entry{concha}
\partofspeech{f}
\spanishtranslation{jujch}
\secondaryentry{concha vacía de caracol}
\secondtranslation{pupuy}

\entry{condición}
\partofspeech{f}
\secondaryentry{la condición en que está}
\secondtranslation{*ilal}

\entry{condimento}
\partofspeech{m}
\spanishtranslation{*ats'mil}

\entry{conejo}
\partofspeech{m}
\spanishtranslation{ty'ul}

\entry{conferencia}
\partofspeech{f}
\spanishtranslation{subty'añ}

\entry{confesar}
\partofspeech{vt}
\spanishtranslation{sub}

\entry{confirmar}
\partofspeech{vt}
\spanishtranslation{xuk'chokoñ}

\entry{conjoyo}
\partofspeech{m}
\spanishtranslation{xk'ok' chij}
\clarification{izote; tipo de yuca}

\entry{conocer}
\partofspeech{vt}
\onedefinition{1}
\spanishtranslation{käñ}
\onedefinition{2}
\spanishtranslation{pojleñ}
\clarification{Sab., Tila}
\secondaryentry{conocido}
\secondtranslation{käñäl}
\secondtranslation{\textsuperscript{2}pojol}
\clarification{Sab., Tila; camino}

\entry{conseguir}
\partofspeech{vt}
\spanishtranslation{mäñtyañ}
\clarification{agua}

\entry{consolar}
\partofspeech{vt}
\onedefinition{1}
\spanishtranslation{ch'äbtyesañ}
\clarification{un niño}
\onedefinition{2}
\spanishtranslation{ñuk'esäbeñ}
\onedefinition{3}
\spanishtranslation{ñukityesañ}
\clarification{Sab.}

\entry{constantemente}
\partofspeech{adv}
\spanishtranslation{belekña}

\entry{constelacion}
\partofspeech{vt}
\spanishtranslation{ek'xäñäb}

\entry{construir}
\partofspeech{vt}
\spanishtranslation{k'äl}
\clarification{casa}

\entry{consumir}
\onedefinition{1}
\partofspeech{vt}
\spanishtranslation{tsäk'mesañ}
\clarification{por hervir}
\onedefinition{2}
\partofspeech{vi}
\spanishtranslation{tsäk'mäl}
\clarification{en fuego}

\entry{consumo}
\partofspeech{m}
\secondaryentry{gran consumo}
\secondtranslation{*jojmäñtyel}
\clarification{de alimento}

\entry{contagiar}
\partofspeech{vt}
\spanishtranslation{tyoy}

\entry{contaminar}
\partofspeech{vt}
\spanishtranslation{xaxañ}

\entry{contar}
\partofspeech{vt}
\spanishtranslation{tsik}

\entry{contenido}
\partofspeech{m}
\spanishtranslation{\textsuperscript{2}bäl}

\entry{contento}
\partofspeech{adj}
\onedefinition{1}
\spanishtranslation{tyijikña}
\clarification{satisfecho}
\onedefinition{2}
\spanishtranslation{uts}
\clarification{bueno}
\onedefinition{3}
\spanishtranslation{utyesañ}
\clarification{Sab.}
\onedefinition{4}
\spanishtranslation{k'ajakña}
\clarification{Sab.}
\secondaryentry{hacer contento}
\secondtranslation{utsesañ}
\clarification{Sab.}

\entry{contestar}
\partofspeech{vt}
\spanishtranslation{jak'}

\entry{contiguo}
\partofspeech{adj}
\spanishtranslation{jaxäl}

\entry{continuamente}
\partofspeech{adv}
\spanishtranslation{chäñ, ñik'i}

\entry{continuar}
\partofspeech{vi}
\spanishtranslation{yajñel}

\entry{contrariar}
\partofspeech{vt}
\spanishtranslation{koñtyrajiñ}

\entry{contrato}
\partofspeech{m}
\spanishtranslation{tyratyo}

\entry{conveniente}
\partofspeech{adj}
\secondaryentry{es conveniente}
\secondtranslation{yomäch}

\entry{convertirse}
\partofspeech{prnl}
\spanishtranslation{päñtyiyel}

\entry{copia}
\partofspeech{f}
\spanishtranslation{*lok'omlel}
\clarification{Sab.}

\entry{copularse}
\partofspeech{prnl}
\spanishtranslation{\textsuperscript{2}pejkañ}
\clarification{con}

\entry{coraje}
\partofspeech{m}
\spanishtranslation{mich'ajel, *mich'lel}

\entry{coralillo}
\partofspeech{m}
\spanishtranslation{kiñtya laj ko', xk'ux tsuk, xotyo'chel, yäxajachañ}
\clarification{venenoso; reptil}

\entry{corazón}
\partofspeech{m}
\onedefinition{1}
\spanishtranslation{pusik'al}
\onedefinition{2}
\spanishtranslation{tyulum, *tyul mal tye', *yojlil tye'}
\clarification{de árbol}

\entry{corcho}
\partofspeech{m}
\spanishtranslation{majaw, ojol, poytye'}
\secondaryentry{corteza de corcho}
\secondtranslation{pätye'}

\entry{cordón}
\partofspeech{m}
\spanishtranslation{ch'ajñal}
\clarification{de piel, de mecate}

\entry{cornear}
\partofspeech{vt}
\spanishtranslation{lujchiñ}
\secondaryentry{corneado}
\secondtranslation{lujchiya}

\entry{correcamino}
\partofspeech{m}
\spanishtranslation{ajkuñts'u'}
\clarification{ave}

\entry{corrección}
\partofspeech{f}
\spanishtranslation{tyoj'esaty'añ}
\clarification{de acta o carta}

\entry{corregir}
\partofspeech{vt}
\spanishtranslation{jultyesañ ipusik'al}
\secondaryentry{el que corrige}
\secondtranslation{xtyojesaty'añ}

\entry{correr}
\partofspeech{vi}
\spanishtranslation{ajñel}
\secondaryentry{el que está corriendo}
\secondtranslation{x'ajñel}

\entry{corretear}
\partofspeech{vt}
\spanishtranslation{ajñesañ}

\entry{corriente}
\partofspeech{f}
\secondaryentry{corriente rápida}
\secondtranslation{wek'uña}

\entry{cortador}
\partofspeech{m}
\secondaryentry{cortador de café}
\secondtranslation{xtyuk' kajpe'}

\entry{cortar}
\partofspeech{vt}
\onedefinition{1}
\spanishtranslation{k'ok, tyuk'}
\clarification{cosa redonda}
\onedefinition{2}
\spanishtranslation{chij}
\clarification{fruta con palo}
\onedefinition{3}
\spanishtranslation{joty}
\clarification{con machete}
\onedefinition{4}
\spanishtranslation{p'asuñ}
\clarification{en trozos}
\onedefinition{5}
\spanishtranslation{kejkañ, tsep}
\clarification{gajos de árbol, zacate}
\onedefinition{6}
\spanishtranslation{setyuñ}
\clarification{palo, tabla}
\onedefinition{7}
\spanishtranslation{sety'}
\clarification{papel, tela, pelo}
\onedefinition{8}
\spanishtranslation{ty'oj}
\clarification{carne, piedra, madera}
\secondaryentry{cortar el pelo}
\secondtranslation{lok' jol}
\secondaryentry{corte alto}
\secondtranslation{ch'ixluktyik}
\clarification{vegetación}
\secondaryentry{el que corta}
\secondtranslation{xtyuk'oñel}
\clarification{fruta}
\secondaryentry{acto de cortar leña}
\secondtranslation{sibal}

\entry{corteza}
\partofspeech{f}
\spanishtranslation{ipaty tye'}
\secondaryentry{corteza de árbol que usan para paludismo}
\secondtranslation{makuliñ}

\entry{corto}
\partofspeech{adj}
\spanishtranslation{\textsuperscript{2}kom, p'ots, tyutsul}
\secondaryentry{muy corto}
\secondtranslation{komatyax}

\entry{corva}
\partofspeech{f}
\spanishtranslation{muk'}

\entry{cosa}
\partofspeech{f}
\secondaryentry{cosa entera}
\secondtranslation{-jump'ejlel}
\secondaryentry{muchas cosas}
\secondtranslation{bäl}
\secondaryentry{cosa abrazada}
\secondtranslation{mek'bal}

\entry{cosecha}
\partofspeech{f}
\spanishtranslation{tyroñeläl}
\clarification{Sab.}

\entry{cosmético}
\partofspeech{m}
\spanishtranslation{*bojñib wutyäl}

\entry{cosquilla}
\partofspeech{f}
\spanishtranslation{tyäkäch}
\secondaryentry{hacer cosquillas}
\secondtranslation{tyäkchuñ}

\entry{costal}
\partofspeech{m}
\spanishtranslation{koxtyal}

\entry{costilla}
\partofspeech{f}
\spanishtranslation{*ch'il'aty}
\clarification{persona, animal}

\entry{costumbre}
\partofspeech{f}
\secondaryentry{costumbres vanas}
\secondtranslation{*yesomal}

\entry{costurar}
\partofspeech{vt}
\spanishtranslation{ts'is}

\entry{coyol}
\partofspeech{m}
\spanishtranslation{chiñiño, kokoyol}
\clarification{árbol}

\entry{coyote}
\partofspeech{m}
\spanishtranslation{matye'el ts'i'}
\clarification{mamífero}

\entry{coyuntura}
\partofspeech{f}
\onedefinition{1}
\spanishtranslation{p'akel}
\onedefinition{2}
\spanishtranslation{*p'akel laj k'äb, ibik' laj k'äb}
\clarification{de la mano}

\entry{crecer}
\partofspeech{vi}
\spanishtranslation{kolel, ñuk'añ}
\secondaryentry{bien crecidas}
\secondtranslation{xekel}
\clarification{hojas}

\entry{creciente}
\partofspeech{adj}
\spanishtranslation{buty'ja'}

\entry{creer}
\partofspeech{vt}
\onedefinition{1}
\spanishtranslation{ñop}
\onedefinition{2}
\spanishtranslation{ch'ujbiñ}
\clarification{Tila}

\entry{crepúsculo}
\partofspeech{m}
\spanishtranslation{ik'ajel}

\entry{crespo}
\partofspeech{adj}
\spanishtranslation{ñoroch'}

\entry{cresta}
\partofspeech{f}
\spanishtranslation{*tsel muty}
\clarification{de pájaro, gallo}

\entry{creta}
\partofspeech{f}
\spanishtranslation{jis}
\clarification{gis}

\entry{cría}
\partofspeech{f}
\spanishtranslation{*al}
\secondaryentry{cría de puerco}
\secondtranslation{sts'ots'}
\secondaryentry{con cría}
\secondtranslation{chujkem}
\clarification{vaca, perra, puerca, yegua}

\entry{criada}
\partofspeech{f}
\spanishtranslation{*yara, yaräjäl}
\clarification{Sab.}

\entry{criado}
\partofspeech{m}
\spanishtranslation{moso}

\entry{crianza}
\partofspeech{f}
\secondaryentry{crianza primera}
\secondtranslation{*yajtyi'al, *yäx'al}

\entry{criar}
\partofspeech{vt}
\spanishtranslation{kosañ}
\secondaryentry{criarse}
\secondtranslation{kolemäyel}
\secondaryentry{criando}
\secondtranslation{bots}

\entry{criatura}
\partofspeech{f}
\spanishtranslation{aläl, ch'ok aläl, meme', xba'}
\secondaryentry{criatura chiquita que llora}
\secondtranslation{uñe'}
\secondaryentry{criatura del agua con muchos pies}
\secondtranslation{tyajbachim}

\entry{criticar}
\partofspeech{vt}
\spanishtranslation{wälilañ}
\secondaryentry{criticarse}
\secondtranslation{ajlel}

\entry{crucero}
\partofspeech{m}
\spanishtranslation{rusibal bij}

\entry{crudo}
\partofspeech{adj}
\spanishtranslation{tsij}

\entry{crujiente}
\partofspeech{adj}
\spanishtranslation{woch'okña}

\entry{cruz}
\partofspeech{f}
\spanishtranslation{rus}

\entry{Cruz Colorada}
\spanishtranslation{Chächäkruz}
\clarification{colonia}

\entry{cruzar}
\onedefinition{1}
\partofspeech{vi}
\spanishtranslation{k'axel}
\onedefinition{2}
\partofspeech{vt}
\spanishtranslation{k'axtyañ}

\entry{cuadrado}
\partofspeech{adj}
\spanishtranslation{joyxujk}

\entry{cuadro}
\partofspeech{m}
\spanishtranslation{lok'oñbaj}
\clarification{Tila}

\entry{cuajinicuil}
\partofspeech{m}
\onedefinition{1}
\spanishtranslation{bits', xlasobits'}
\clarification{árbol}
\onedefinition{2}
\spanishtranslation{bits'chajk}
\clarification{Tila; árbol}

\entry{cuál}
\partofspeech{pron}
\secondaryentry{cuál es}
\secondtranslation{¿bakibä?}

\entry{cualquiera}
\partofspeech{pron}
\spanishtranslation{ba'ikachbä,baki jachbä}

\entry{cuando}
\partofspeech{adv}
\onedefinition{1}
\spanishtranslation{che'}
\onedefinition{2}
\spanishtranslation{che'bä, che' ñak, \textsuperscript{3}ñak}
\clarification{Sab., Tila}
\secondaryentry{de vez en cuando}
\secondtranslation{bixeltyik, wojlel jach, xä'}

\entry{cuándo}
\partofspeech{adv}
\spanishtranslation{¿baki ora?, ¿jalaj?, ¿jaläxki?}

\entry{cuánto}
\partofspeech{adv}
\spanishtranslation{¿bajche'?, ¿jay-?}
\secondaryentry{en cuánto a}
\secondtranslation{ixku}

\entry{cuarta}
\partofspeech{f}
\spanishtranslation{juñajb}
\clarification{medida}
\secondaryentry{medir con la cuarta de la mano}
\secondtranslation{ñajbañ}

\entry{cuarto}
\partofspeech{m}
\spanishtranslation{tsaläl}

\entry{cuatro}
\partofspeech{adj}
\spanishtranslation{chämp'ejl}
\secondaryentry{cuatro días}
\secondtranslation{chäñi}

\entry{cuaulote}
\partofspeech{m}
\spanishtranslation{xpapastye'}
\clarification{guácimo; árbol}

\entry{cuautecomate}
\partofspeech{m}
\spanishtranslation{stsimajtye'}
\clarification{árbol}

\entry{cubeta}
\partofspeech{f}
\spanishtranslation{lucho'ja'}

\entry{cubrir}
\partofspeech{vt}
\spanishtranslation{mujlañ}
\clarification{con arena, hojas, tierra, zacate}

\entry{cucaracha}
\partofspeech{f}
\spanishtranslation{mako', pewal}
\clarification{insecto}

\entry{cuchara}
\partofspeech{f}
\spanishtranslation{luchoñib}

\entry{cuchichear}
\partofspeech{vi}
\spanishtranslation{mukuty'añ}

\entry{cuchillo}
\partofspeech{m}
\spanishtranslation{kuchilu}

\entry{cuchunuc}
\partofspeech{m}
\spanishtranslation{xchañtye'}
\clarification{madre de cacao; árbol}

\entry{cucu bobo}
\partofspeech{m}
\spanishtranslation{k'iñ tyuñimuty}
\clarification{cuclillo chiflador; ave}

\entry{cuello}
\partofspeech{m}
\spanishtranslation{bik'}

\entry{cuenca}
\partofspeech{f}
\spanishtranslation{japal}
\clarification{de río}

\entry{cuerda}
\partofspeech{f}
\secondaryentry{dar cuerda}
\secondtranslation{wilts'uñ}

\entry{cuerno}
\partofspeech{m}
\spanishtranslation{xulub}

\entry{cuero}
\partofspeech{m}
\spanishtranslation{pächi}
\secondaryentry{cuero de toro}
\secondtranslation{xpuy ch'ajañ}
\clarification{talismecate; árbol}

\entry{cuerpo}
\partofspeech{m}
\onedefinition{1}
\spanishtranslation{bäk'tyaläl}
\onedefinition{2}
\spanishtranslation{*pächälel}
\clarification{Sab.}

\entry{cuervo}
\partofspeech{m}
\spanishtranslation{i'ik' muty}
\clarification{ave}

\entry{cueva}
\partofspeech{f}
\spanishtranslation{ch'eñ}
\secondaryentry{lugar encima de una cueva}
\secondtranslation{pañch'eñ}

\entry{cueza}
\partofspeech{f}
\spanishtranslation{xko'siñ}
\clarification{raíz comestible de chayote}

\entry{cuiche}
\partofspeech{m}
\spanishtranslation{tyojk'ay}
\clarification{codorniz común; ave}

\entry{cuidado}
\partofspeech{m}
\spanishtranslation{tsajilety}

\entry{cuidador}
\partofspeech{m}
\onedefinition{1}
\spanishtranslation{xkäñañ, xkäñtyaya}
\secondaryentry{cuidador de la iglesia}
\secondtranslation{motyomaj}
\secondaryentry{cuidador de una criatura}
\secondtranslation{xkäñä aläl}

\entry{cuidar}
\partofspeech{vt}
\spanishtranslation{käñtyañ, chäkä k'el}
\secondaryentry{cuidarse}
\secondtranslation{tsajiñ}

\entry{cuipu}
\partofspeech{m}
\spanishtranslation{bäkch'umtye'}
\clarification{piñoncillo; árbol}

\entry{culebra}
\partofspeech{f}
\spanishtranslation{lukum}
\secondaryentry{culebra arroyera}
\secondtranslation{xña'a xu'}
\secondaryentry{culebra de agua}
\spanishtranslation{mijts'ity}
\clarification{tilcoate}
\secondaryentry{culebra voladora}
\secondtranslation{k'äñjixil}
\secondaryentry{tipo de culebra}
\secondtranslation{xk'äñäñej}
\secondaryentry{tipo de culebra venenosa}
\secondtranslation{chikixchañ}

\entry{culeca (reg.)}
\partofspeech{f}
\spanishtranslation{xpäklojm}
\clarification{clueca}

\entry{culpa}
\partofspeech{f}
\onedefinition{1}
\spanishtranslation{\textsuperscript{2}mul, mulil}
\onedefinition{2}
\spanishtranslation{ajmulil}
\clarification{Sab.}

\entry{culpable}
\partofspeech{adj}
\spanishtranslation{bäjäch}

\entry{culpar}
\partofspeech{vt}
\spanishtranslation{\textsuperscript{3}päk'}

\entry{cumbre}
\partofspeech{f}
\spanishtranslation{*ñi'wits}

\entry{cumplir}
\partofspeech{vi}
\spanishtranslation{ts'äktyesäñtyel}
\secondaryentry{no cumplir}
\secondtranslation{juchtyek'}

\entry{cuna}
\partofspeech{f}
\spanishtranslation{wäyib aläl}

\entry{cundido}
\partofspeech{adj}
\spanishtranslation{lañal}
\clarification{granos de sarampión}

\entry{cuña}
\partofspeech{f}
\spanishtranslation{*ñejty'il, *xojñil}

\entry{cuñada}
\partofspeech{f}
\onedefinition{1}
\spanishtranslation{*jawäñ}
\clarification{de mujer}
\onedefinition{2}
\spanishtranslation{*mu'}
\clarification{de hombre}

\entry{cuñado}
\partofspeech{m}
\spanishtranslation{*ja'añ}
\clarification{de hombre}

\entry{cura}
\partofspeech{m}
\onedefinition{1}
\spanishtranslation{pale}
\onedefinition{2}
\spanishtranslation{tyal'a}
\clarification{Tila}

\entry{curación}
\partofspeech{f}
\spanishtranslation{ilol}
\clarification{que hace un curandero}

\entry{curandero}
\partofspeech{m}
\onedefinition{1}
\spanishtranslation{x'ilo, sts'äkaya, xwujty}
\onedefinition{2}
\spanishtranslation{awujtyaya}
\clarification{Sab.}

\entry{curar}
\partofspeech{vt}
\onedefinition{1}
\spanishtranslation{lajmesañ, ts'äkañ}
\onedefinition{2}
\spanishtranslation{\textsuperscript{2}ilañ}
\clarification{por hechicería}
\secondaryentry{curarse}
\secondtranslation{ts'äkäñtyel, lajmel}
\secondtranslation{wujtyiñtyel}
\clarification{por hechicería}

\entry{curva}
\partofspeech{f}
\spanishtranslation{k'ojchel bij}
\clarification{del camino}

\entry{curvado}
\partofspeech{adj}
\onedefinition{1}
\spanishtranslation{lokokña}
\clarification{árbol, palo}
\onedefinition{2}
\spanishtranslation{lokol}
\clarification{palo, alambre}

\entry{daño}
\partofspeech{m}
\spanishtranslation{*jisälel}

\entry{dar}
\partofspeech{vt}
\onedefinition{1}
\spanishtranslation{\textsuperscript{2}ak'}
\onedefinition{2}
\spanishtranslation{yebeñ}
\clarification{comida a pollitos, pavos}
\secondaryentry{dar alimento}
\secondtranslation{we'sañ}
\secondaryentry{dar cuerda}
\secondtranslation{wilts'uñ}
\secondaryentry{dar patada}
\secondtranslation{tyoñtyek'}
\secondaryentry{dar vuelta}
\secondtranslation{\textsuperscript{2}xoy, wilts'uñ}
\secondaryentry{dar vueltas}
\secondtranslation{wijlel}
\secondaryentry{dáselo}
\secondtranslation{¡ak'eñ!}

\entry{de}
\partofspeech{prep}
\spanishtranslation{cha'añ}
\secondaryentry{de antemano}
\secondtranslation{wäñ}
\secondaryentry{de aquí a tres días}
\secondtranslation{uxi}
\secondaryentry{de balde}
\secondtranslation{lolom jach, milik}
\secondtranslation{\textsuperscript{1}pojol}
\clarification{Tila}

\entry{debajo}
\partofspeech{adv}
\spanishtranslation{yebal}

\entry{deber}
\partofspeech{m}
\spanishtranslation{*k'äjñibal}

\entry{deber}
\partofspeech{vi}
\spanishtranslation{yom}

\entry{débil}
\partofspeech{adj}
\onedefinition{1}
\spanishtranslation{k'uñ, k'uñlitsañ}
\clarification{cuerpo, palo largo}
\onedefinition{2}
\spanishtranslation{luk'law}
\clarification{palo o tabla}
\secondaryentry{débil todavía}
\secondtranslation{k'uñtyo}
\secondaryentry{débilmente}
\secondtranslation{mächäkña}

\entry{debilidad}
\partofspeech{f}
\spanishtranslation{k'uñukñiyel}

\entry{debilitar}
\partofspeech{vt}
\spanishtranslation{k'uñ'añ}

\entry{decir}
\partofspeech{vt}
\spanishtranslation{\textsuperscript{2}al, ajlel}
\secondaryentry{así debes decir}
\secondtranslation{che'ety}
\secondaryentry{así dice}
\secondtranslation{che'eñ}
\secondaryentry{así dicen}
\secondtranslation{che'ob}
\clarification{terminación}
\secondaryentry{así digo yo}
\secondtranslation{cho'oñ}
\secondaryentry{dicho}
\secondtranslation{subil, sujbem}

\entry{declarar}
\partofspeech{vt}
\spanishtranslation{tsiktyesañ}

\entry{dedicarse}
\partofspeech{prnl}
\spanishtranslation{esmañ}

\entry{dedo}
\partofspeech{m}
\secondaryentry{pulgar}
\spanishtranslation{*ña'al laj k'äb}
\secondaryentry{dedo índice}
\spanishtranslation{*tyuch'oñib}
\secondaryentry{dedo mayor}
\spanishtranslation{xiñk'äb}
\secondaryentry{dedo meñique}
\spanishtranslation{*xuty laj k'äb}
\secondaryentry{dedo de los pies}
\spanishtranslation{*yal lakok}

\entry{deficiente}
\partofspeech{adj}
\spanishtranslation{leko}

\entry{dejar}
\partofspeech{vt}
\spanishtranslation{setychokoñ}
\clarification{rollo de palos}
\secondaryentry{deja}
\secondtranslation{la'tyo}
\secondaryentry{dejado}
\secondtranslation{kälem, käyäl}
\secondtranslation{käyleñ}
\clarification{Sab.}
\secondtranslation{xetyel}
\clarification{objeto redondo}

\entry{delantal}
\partofspeech{m}
\spanishtranslation{welñäk'}

\entry{delgado}
\partofspeech{adj}
\onedefinition{1}
\spanishtranslation{jay}
\onedefinition{2}
\spanishtranslation{yajyaj}
\clarification{persona}
\secondaryentry{delgado y alto}
\secondtranslation{bilil}

\entry{delito}
\partofspeech{m}
\spanishtranslation{mulil}

\entry{demasiado}
\partofspeech{adj}
\spanishtranslation{käläx}

\entry{demolición}
\partofspeech{f}
\spanishtranslation{*jembal}

\entry{derecho}
\partofspeech{adj}
\spanishtranslation{\textsuperscript{2}ñoj}
\secondaryentry{derechamente}
\secondtranslation{ñäxäkña}

\entry{derramar}
\partofspeech{vt}
\spanishtranslation{bek'}
\secondaryentry{derramar un poco}
\secondtranslation{ts'uj}

\entry{derretir}
\partofspeech{vt}
\spanishtranslation{ulmesañ}
\secondaryentry{derretirse}
\secondtranslation{ulmäl}

\entry{derrumbar}
\partofspeech{vt}
\spanishtranslation{buch}
\clarification{árbol, casa}
\secondaryentry{derrumbarse}
\spanishtranslation{yejmel}
\clarification{tierra}

\entry{derrumbe}
\partofspeech{m}
\spanishtranslation{ejmel}

\entry{desagradable}
\partofspeech{adj}
\spanishtranslation{leko, xijiñ}

\entry{desajustado}
\partofspeech{adj}
\spanishtranslation{jelel}

\entry{desaparecer}
\partofspeech{vt}
\spanishtranslation{säk jilel}

\entry{desarmar}
\partofspeech{vt}
\spanishtranslation{\textsuperscript{2}poj}

\entry{desatar}
\partofspeech{vt}
\onedefinition{1}
\spanishtranslation{jity, tyik}
\onedefinition{2}
\spanishtranslation{tyil}
\clarification{la manera de}
\secondaryentry{desatado}
\secondtranslation{jitybil}

\entry{desbaratar}
\partofspeech{vt}
\spanishtranslation{jem}

\entry{descansadero}
\partofspeech{m}
\spanishtranslation{jijlibäl}
\clarification{en camino}

\entry{descansar}
\partofspeech{vi}
\onedefinition{1}
\spanishtranslation{k'aj lako}
\onedefinition{2}
\spanishtranslation{jijlel}
\clarification{por la noche}
\secondaryentry{descansa}
\secondtranslation{¡jijleñ!}

\entry{descanso}
\partofspeech{m}
\spanishtranslation{k'aj o}
\secondaryentry{lugar de descanso}
\secondtranslation{k'ajo'o'}
\clarification{para la noche}

\entry{descascarar}
\partofspeech{vt}
\spanishtranslation{cho'}
\clarification{plátano, frijol, maíz}

\entry{descendiente}
\partofspeech{m}
\onedefinition{1}
\spanishtranslation{*p'olbal}
\onedefinition{2}
\spanishtranslation{mam}
\clarification{Tila}

\entry{descomponer}
\partofspeech{vt}
\secondaryentry{descomponerse}
\secondtranslation{jejmel}
\clarification{carga, casa, plan}
\secondtranslation{yäsiyel}
\clarification{trabajo}

\entry{descoyuntar}
\partofspeech{vt}
\spanishtranslation{jits'kuyel}

\entry{descuartizar}
\partofspeech{vt}
\spanishtranslation{jajkuñ}

\entry{descubrir}
\partofspeech{vt}
\spanishtranslation{tsiktyiyel}

\entry{desenterrar}
\partofspeech{vt}
\spanishtranslation{k'ol}
\clarification{piedra}

\entry{desenvainar}
\partofspeech{vt}
\spanishtranslation{jots'}
\clarification{machete de su funda}

\entry{desenvolver}
\partofspeech{vt}
\spanishtranslation{chaw}

\entry{desgajar}
\partofspeech{vt}
\spanishtranslation{jak}

\entry{desgranar}
\partofspeech{vt}
\onedefinition{1}
\spanishtranslation{ixmañ}
\clarification{maíz}
\onedefinition{2}
\spanishtranslation{ty'ojchiñ}
\clarification{con el pulgar}
\secondaryentry{acción de desgranar}
\secondtranslation{ixom}

\entry{deshilachar}
\partofspeech{vt}
\secondaryentry{deshilacharse}
\secondtranslation{tyijlel}

\entry{desmayarse}
\partofspeech{prnl}
\onedefinition{1}
\spanishtranslation{puk' chämel}
\onedefinition{2}
\spanishtranslation{jits'kwäyel}
\clarification{Sab.}

\entry{desnivelar}
\partofspeech{vt}
\secondaryentry{desnivelado}
\secondtranslation{bexel}

\entry{desnudar}
\partofspeech{vt}
\spanishtranslation{pits'chokoñ}

\entry{desnudo}
\partofspeech{adj}
\onedefinition{1}
\spanishtranslation{bulux}
\clarification{pollo}
\onedefinition{2}
\spanishtranslation{chakal}
\clarification{Tila; general}
\onedefinition{3}
\spanishtranslation{pits'il}
\clarification{persona, plátano, maíz}
\onedefinition{4}
\spanishtranslation{tyañal}
\clarification{Sab.; niño}
\onedefinition{5}
\spanishtranslation{ts'ul}
\clarification{de pelo}
\secondaryentry{desnudo y parado}
\spanishtranslation{chak wa'al}

\entry{desobedecer}
\partofspeech{vt}
\spanishtranslation{ñusañ}

\entry{desobediente}
\partofspeech{adj}
\onedefinition{1}
\spanishtranslation{xñusaty'añ}
\clarification{persona}
\onedefinition{2}
\spanishtranslation{ñuñty'añ}
\clarification{Sab.; persona}

\entry{desocupar}
\partofspeech{vt}
\spanishtranslation{jochtyesañ, \textsuperscript{2}poj}
\secondaryentry{desocuparse}
\spanishtranslation{jochtyiyel}
\secondaryentry{desocupado}
\secondtranslation{jochokña, \textsuperscript{2}pojol}
\clarification{casa}
\secondtranslation{jochol}
\clarification{casa, tierra}

\entry{desordenar}
\partofspeech{vt}
\secondaryentry{desordenadamente}
\secondtranslation{bäñlaw}

\entry{desorientar}
\partofspeech{vt}
\secondaryentry{desorientado}
\secondtranslation{ch'älch'älña, sojkem}

\entry{despacio}
\partofspeech{adv}
\onedefinition{1}
\spanishtranslation{k'uñtye'}
\onedefinition{2}
\spanishtranslation{xuk'ukña}
\clarification{muy}
\onedefinition{3}
\spanishtranslation{xuk'ul}
\clarification{Sab.}

\entry{despedazar}
\partofspeech{vt}
\spanishtranslation{xujty'el}

\entry{despegar}
\partofspeech{vt}
\onedefinition{1}
\spanishtranslation{k'ajchel}
\clarification{una parte}
\onedefinition{2}
\spanishtranslation{k'ajlel}
\clarification{pintura, repello}
\secondaryentry{despegarse}
\spanishtranslation{ty'ajchel}
\clarification{una parte}
\secondaryentry{despegado}
\secondtranslation{k'ajchem}
\clarification{uña}
\secondtranslation{k'ajlem}
\clarification{pintura, repello}
\secondaryentry{despegando}
\secondtranslation{ts'ul}

\entry{despejado}
\partofspeech{adj}
\onedefinition{1}
\spanishtranslation{jamakña, k'elekña, yäxpiyañ}
\onedefinition{2}
\spanishtranslation{utsil}
\clarification{Tila}

\entry{despertar}
\partofspeech{vt}
\spanishtranslation{p'ixel, tyejchel}

\entry{despierto}
\partofspeech{adj}
\onedefinition{1}
\spanishtranslation{kañal lakwuty}
\onedefinition{2}
\spanishtranslation{p'ixil}
\clarification{Sab.}

\entry{desplumar}
\partofspeech{vt}
\spanishtranslation{tyujtyuñ}

\entry{despreciar}
\partofspeech{vt}
\secondaryentry{despreciado}
\secondtranslation{ts'a'lebil}

\entry{desprecio}
\partofspeech{m}
\secondaryentry{expresión de desprecio}
\secondtranslation{yaj'ojl}

\entry{desprender}
\partofspeech{vt}
\spanishtranslation{k'ojkel}
\secondaryentry{desprenderse}
\secondtranslation{sijpel}
\clarification{viga}

\entry{después}
\partofspeech{adv}
\onedefinition{1}
\spanishtranslation{wi'il}
\onedefinition{2}
\spanishtranslation{tsäktsäkña}
\clarification{por detrás}

\entry{destapador}
\partofspeech{m}
\spanishtranslation{jamoñib}
\clarification{de botellas}

\entry{destino}
\partofspeech{m}
\spanishtranslation{k'otyib}

\entry{destruir}
\partofspeech{vt}
\spanishtranslation{jem, jisañ}

\entry{desviación}
\partofspeech{f}
\spanishtranslation{*xäk' bij}
\clarification{del camino}

\entry{deuda}
\partofspeech{f}
\spanishtranslation{bety}

\entry{devolver}
\partofspeech{vt}
\spanishtranslation{sutyk'iñ}

\entry{día}
\partofspeech{m}
\spanishtranslation{k'iñ}
\secondaryentry{cada día}
\secondtranslation{jujump'ejl k'iñ}
\secondaryentry{de día}
\secondtranslation{k'iñil}
\secondaryentry{día especial}
\secondtranslation{*k'iñilel}
\secondaryentry{el décimo día}
\secondtranslation{lujuñij}
\secondaryentry{noveno día}
\secondtranslation{boloñij}
\secondaryentry{séptimo día}
\secondtranslation{wukñij}
\secondaryentry{todos los días}
\secondtranslation{pejtyel k'iñ}

\entry{diablo}
\partofspeech{m}
\onedefinition{1}
\spanishtranslation{tyeñtsuñ, xbakñej, xiba}
\onedefinition{2}
\spanishtranslation{bäwits, judío, judas}
\clarification{Sab.}

\entry{diarrea}
\partofspeech{f}
\spanishtranslation{tyukñäk'}

\entry{dibujo}
\partofspeech{m}
\spanishtranslation{*ts'ijbal}

\entry{diecinueve}
\partofspeech{adj}
\spanishtranslation{boloñlujump'ejl}

\entry{dieciocho}
\partofspeech{adj}
\spanishtranslation{waxäklujump'ejl}

\entry{dieciséis}
\partofspeech{adj}
\spanishtranslation{wäklujump'ejl}

\entry{diecisiete}
\partofspeech{adj}
\spanishtranslation{wuklujump'ejl}

\entry{diente}
\partofspeech{m}
\onedefinition{1}
\spanishtranslation{*bäkel ejäl}
\onedefinition{2}
\spanishtranslation{*pamäk ej}
\clarification{de enfrente}

\entry{diez}
\partofspeech{adj}
\onedefinition{1}
\spanishtranslation{lujump'ejl}
\onedefinition{2}
\spanishtranslation{lämp'ejl}
\clarification{Tila}

\entry{diferente}
\partofspeech{adv}
\spanishtranslation{jelchojk}

\entry{difícil}
\partofspeech{adj}
\spanishtranslation{\textsuperscript{1}wokol}
\secondaryentry{difícilmente}
\secondtranslation{komkatsa'}

\entry{dificultad}
\partofspeech{f}
\spanishtranslation{\textsuperscript{1}wokol}

\entry{diminutivo}
\partofspeech{m}
\spanishtranslation{xep}
\clarification{jóvenes, hombres}

\entry{dinero}
\partofspeech{m}
\spanishtranslation{tyak'iñ}
\secondaryentry{el que lleva dinero}
\secondtranslation{x'ak'bety}

\entry{dintel}
\partofspeech{m}
\spanishtranslation{*k'ätyloñtye'el}

\entry{Dios}
\partofspeech{m}
\spanishtranslation{*ch'ujutyaty}
\clarification{lit.: Padre Santo}
\secondaryentry{dios de la abundancia de plantas y animales}
\spanishtranslation{\textsuperscript{3}*ña'al}
\secondaryentry{diosa del agua de la creación}
\spanishtranslation{\textsuperscript{2}kolaj}
\clarification{Tila}

\entry{dirigir}
\partofspeech{vt}
\spanishtranslation{chäkä k'el}
\clarification{trabajadores}

\entry{discípulo}
\partofspeech{m}
\onedefinition{1}
\spanishtranslation{xk'äñty'añ}
\onedefinition{2}
\spanishtranslation{ajkäñty'añ}
\clarification{Sab.}

\entry{disentería}
\partofspeech{f}
\spanishtranslation{ch'ich'tya'}
\clarification{lit.: excremento con sangre}

\entry{disminuir}
\partofspeech{vt}
\spanishtranslation{tsiñ}
\clarification{como meteor}
\secondaryentry{disminuyendo}
\secondtranslation{tsiñtyäl}

\entry{disparar}
\partofspeech{vt}
\spanishtranslation{tyojmesañ}
\clarification{arma}

\entry{disparejo}
\partofspeech{adj}
\onedefinition{1}
\spanishtranslation{bolo'ty'axal}
\clarification{poco}
\onedefinition{2}
\spanishtranslation{bulubexel}
\clarification{piso de la casa}
\onedefinition{3}
\spanishtranslation{bujbujtyäl}
\clarification{terreno}
\onedefinition{4}
\spanishtranslation{ty'olol, ty'ojlox}
\clarification{pared, suelo}

\entry{disputa}
\partofspeech{f}
\spanishtranslation{letyo}

\entry{distinto}
\partofspeech{adj}
\spanishtranslation{jelchojk}

\entry{distribuir}
\partofspeech{vt}
\spanishtranslation{pujkel}

\entry{dividir}
\partofspeech{vt}
\spanishtranslation{ty'ox}
\secondaryentry{dividido}
\secondtranslation{ty'oxlem}
\clarification{terreno, dinero}
\secondtranslation{japal}
\clarification{cerro, montaña}

\entry{divulgar}
\partofspeech{vt}
\spanishtranslation{pujkel}

\entry{doblar}
\partofspeech{vt}
\onedefinition{1}
\spanishtranslation{k'ochilañ}
\clarification{mano}
\onedefinition{2}
\spanishtranslation{lok, lokilañ}
\clarification{palo, alambre}
\onedefinition{3}
\spanishtranslation{lux}
\clarification{rodilla, brazo, callejón}
\onedefinition{4}
\spanishtranslation{päk}
\clarification{milpa}
\secondaryentry{doblarse el tobillo}
\spanishtranslation{k'ujyel}
\secondaryentry{doblado}
\secondtranslation{litslitsña}
\clarification{manera}
\secondtranslation{luxul}
\clarification{rodilla, camino}
\secondtranslation{päjkem}
\clarification{papel, tela}
\secondaryentry{doblando}
\secondtranslation{luk'luk'ña}

\entry{doce}
\partofspeech{adj}
\spanishtranslation{lajchämp'ejl}

\entry{doler}
\partofspeech{vi}
\spanishtranslation{\textsuperscript{2}k'ux}
\secondaryentry{me duele el corazón}
\secondtranslation{k'ux kpusik'al}

\entry{doloroso}
\partofspeech{adj}
\spanishtranslation{tyäk'äkña}

\entry{dónde}
\partofspeech{adv}
\spanishtranslation{baki,ba'}
\secondaryentry{dónde está}
\secondtranslation{¿baki añ?, ¿bak añ?}

\entry{dondequiera}
\partofspeech{adv}
\onedefinition{1}
\spanishtranslation{baki jach}
\onedefinition{2}
\spanishtranslation{ba'ikal}
\clarification{Sab.}

\entry{dormir}
\partofspeech{vi}
\spanishtranslation{wäyel, mi icha'leñ wäyel}
\secondaryentry{dormir sin cobija}
\spanishtranslation{tsäñä wäyel}
\clarification{sufriendo el frío}
\secondaryentry{dormido}
\secondtranslation{wäyäl}

\entry{dormitar}
\partofspeech{vi}
\secondaryentry{acción de dormitar}
\secondtranslation{ñäkäb}

\entry{dorso}
\partofspeech{m}
\spanishtranslation{ipaty laj k'äb}
\clarification{de la mano}

\entry{dos}
\partofspeech{adj}
\spanishtranslation{cha'p'ejl}
\secondaryentry{dos dobladas}
\secondtranslation{cha'päjk}

\entry{dueño}
\partofspeech{m}
\spanishtranslation{yumäl}
\clarification{Sab.; espíritu malo}
\secondaryentry{dueño del cerro}
\secondtranslation{bäwits}
\clarification{Sab.}

\entry{dulce}
\partofspeech{adj}
\spanishtranslation{chäb, tsaj}

\entry{durable}
\partofspeech{adj}
\spanishtranslation{ch'äjy}
\clarification{camisa}

\entry{dureza}
\partofspeech{f}
\spanishtranslation{*tsätslel}

\entry{duro}
\partofspeech{adj}
\onedefinition{1}
\spanishtranslation{tsäts, \textsuperscript{1}wokol}
\onedefinition{2}
\spanishtranslation{ts'ujy}
\clarification{cáscara de fruta}
\secondaryentry{es duro su carácter}
\secondtranslation{tsäts ipusik'al}

\entry{echar}
\partofspeech{vt}
\onedefinition{1}
\spanishtranslation{päkchokoñ}
\clarification{gallina}
\onedefinition{2}
\spanishtranslation{päktyañ}
\clarification{sobre huevos}
\secondaryentry{echado}
\secondtranslation{päkäl}
\clarification{animal}
\secondaryentry{echar puntal}
\secondtranslation{xikñi'iñ}
\clarification{para reforzar}
\secondaryentry{echarse a perder}
\secondtranslation{pujch'el}
\secondaryentry{echarse a un lado por susto}
\secondtranslation{bijty'el}

\entry{eclipse}
\partofspeech{m}
\secondaryentry{haber eclipse}
\secondtranslation{pulel}
\clarification{Sab.}

\entry{edad}
\partofspeech{f}
\spanishtranslation{*jabilel}

\entry{edificar}
\partofspeech{vt}
\spanishtranslation{wa'chokoñ}

\entry{él}
\partofspeech{pron}
\spanishtranslation{jiñi}

\entry{elegido}
\partofspeech{adj}
\spanishtranslation{yajkäbil}

\entry{ella}
\partofspeech{pron}
\spanishtranslation{jiñi}

\entry{elote}
\partofspeech{m}
\spanishtranslation{wajtyañ}

\entry{embarazada}
\partofspeech{adj}
\spanishtranslation{käñtyäbil}

\entry{embarrar}
\partofspeech{vt}
\spanishtranslation{jujk'uñ}
\secondaryentry{embarrado}
\secondtranslation{pajk'ibil}
\clarification{pared}

\entry{emborracharse}
\partofspeech{prnl}
\spanishtranslation{yäk'añ}

\entry{emboscar}
\partofspeech{vt}
\secondaryentry{el que está emboscado}
\secondtranslation{xchijty}

\entry{embrocar}
\partofspeech{vt}
\onedefinition{1}
\spanishtranslation{kol}
\clarification{agua}
\onedefinition{2}
\spanishtranslation{yojyoñ}
\clarification{maíz, frijol, café}
\secondaryentry{embrocado}
\secondtranslation{ñukul}
\clarification{olla, canasto}

\entry{empapar}
\partofspeech{vt}
\spanishtranslation{\textsuperscript{1}chäy}
\clarification{algodón en agua}
\secondaryentry{empapado}
\secondtranslation{chäybil}
\clarification{algodón}
\secondtranslation{lutsañ}
\clarification{personas, animales}

\entry{empelotar}
\partofspeech{vt}
\secondaryentry{empelotado}
\secondtranslation{woxbil}

\entry{empezar}
\partofspeech{vt}
\spanishtranslation{tyech}

\entry{empinado}
\partofspeech{adj}
\spanishtranslation{bäläl}

\entry{empleado}
\partofspeech{m}
\onedefinition{1}
\spanishtranslation{wiñik, *wiñiklel}
\onedefinition{2}
\spanishtranslation{*wiñkilel}
\clarification{Sab.}
\secondaryentry{empleado a la fuerza en una finca o en el gobierno}
\secondtranslation{xk'äjñibäyel}

\entry{emplear}
\partofspeech{vt}
\spanishtranslation{wiñikañ}
\clarification{para trabajo}

\entry{empujar}
\partofspeech{vt}
\spanishtranslation{xitytyek'}

\entry{empuñar}
\partofspeech{vt}
\onedefinition{1}
\spanishtranslation{\textsuperscript{1}moch', mop'}
\clarification{mano}
\onedefinition{2}
\spanishtranslation{moch'ye', mop'ye'}
\clarification{con algo en la mano}
\secondaryentry{empuñado}
\secondtranslation{moch'ol}
\clarification{mano}

\entry{enagua}
\partofspeech{f}
\spanishtranslation{majtsäl}

\entry{enamorar}
\partofspeech{vt}
\spanishtranslation{\textsuperscript{2}pejkañ}

\entry{encaramar}
\partofspeech{vt}
\onedefinition{1}
\spanishtranslation{k'äk}
\onedefinition{2}
\spanishtranslation{k'äkchokoñ}
\clarification{sobre}

\entry{encarcelar}
\partofspeech{vt}
\spanishtranslation{käjchel}
\secondaryentry{encarcelado}
\secondtranslation{kächäl}

\entry{encargo}
\partofspeech{m}
\onedefinition{1}
\spanishtranslation{*e'tyel}
\onedefinition{2}
\spanishtranslation{tyroñel}
\clarification{Tila}

\entry{encendedor}
\partofspeech{m}
\spanishtranslation{*ts'äboñib}
\secondaryentry{encendedor de velas en la iglesia}
\secondtranslation{stsuk'ñichim}

\entry{encender}
\partofspeech{vt}
\onedefinition{1}
\spanishtranslation{\textsuperscript{2}tsuk'}
\clarification{fuego}
\onedefinition{2}
\spanishtranslation{ts'äb}
\clarification{lámpara}
\secondaryentry{encendido}
\secondtranslation{tsujk'em, ts'äyäl}

\entry{encerrar}
\partofspeech{vt}
\secondaryentry{encerrado}
\secondtranslation{ñup'ul}

\entry{encharcar}
\partofspeech{vt}
\secondaryentry{encharcado}
\secondtranslation{lämäl}
\secondaryentry{encharcarse}
\secondtranslation{lämtyäl}
\clarification{Sab.; agua}

\entry{encima}
\partofspeech{adv}
\spanishtranslation{ipamlel}
\secondaryentry{encima del pie}
\secondtranslation{ipaty lakok}

\entry{encoger}
\partofspeech{vt}
\spanishtranslation{\textsuperscript{2}tsäy}
\clarification{cintura}
\secondaryentry{encogerse}
\spanishtranslation{sombäl}
\clarification{ropa}

\entry{encontrar}
\partofspeech{vt}
\spanishtranslation{\textsuperscript{1}tyaj}

\entry{encrespar}
\partofspeech{vt}
\spanishtranslation{ñoroch'iyel}
\clarification{cabello}
\secondaryentry{encrespado}
\secondtranslation{murux}

\entry{enderezar}
\partofspeech{vt}
\spanishtranslation{tyoj'esañ}

\entry{endurecer}
\partofspeech{vt}
\secondaryentry{endurecerse}
\secondtranslation{ch'äjy'añ}
\clarification{tortilla, pan}

\entry{eneldo}
\partofspeech{m}
\spanishtranslation{x'eñtyoj}
\clarification{hierba}

\entry{enemigo}
\partofspeech{m}
\spanishtranslation{koñtyra}

\entry{energía}
\partofspeech{f}
\spanishtranslation{*bäxlel}

\entry{enfadar}
\partofspeech{vt}
\spanishtranslation{mich'esañ}

\entry{enfermar}
\partofspeech{vt}
\spanishtranslation{k'am'añ}

\entry{enfermedad}
\partofspeech{f}
\spanishtranslation{k'amäjel, chämel, \textsuperscript{2}wokol}
\secondaryentry{enfermedad mortal}
\spanishtranslation{ch'ämoñel}

\entry{enfermo}
\onedefinition{1}
\partofspeech{adj}
\spanishtranslation{\textsuperscript{1}k'am}
\onedefinition{2}
\partofspeech{m}
\spanishtranslation{xk'amäjel}

\entry{enflaquecer}
\partofspeech{vt}
\spanishtranslation{yaj'añ}

\entry{enfriar}
\partofspeech{vt}
\spanishtranslation{tsäñesañ}

\entry{engañar}
\partofspeech{vt}
\spanishtranslation{lotyiñ, lo'loñ}
\secondaryentry{engañarse}
\secondtranslation{lotyiñtyel}

\entry{engaño}
\partofspeech{m}
\spanishtranslation{lotyiya, lo'loya}

\entry{engendrar}
\partofspeech{vt}
\spanishtranslation{p'ol}

\entry{engordar}
\partofspeech{vt}
\spanishtranslation{jujp'el}

\entry{engrandecer}
\partofspeech{vt}
\spanishtranslation{ñuk'esañ}
\secondaryentry{engrandecerse}
\secondtranslation{ñuk'esäñtyel}

\entry{enjambrar}
\partofspeech{vt}
\secondaryentry{emjambrado}
\secondtranslation{wotswotsña}

\entry{enjuagar}
\partofspeech{vt}
\spanishtranslation{*um ja'}
\clarification{la boca}

\entry{enojar}
\onedefinition{1}
\partofspeech{vt}
\spanishtranslation{*mich'esañ}
\secondaryentry{enojarse}
\spanishtranslation{mich'añ}
\secondaryentry{enojarse con}
\spanishtranslation{mich'leñ}
\clarification{constantemente}
\secondaryentry{enojado}
\secondtranslation{mich', mich'ikña}

\entry{enredar}
\partofspeech{vt}
\onedefinition{1}
\spanishtranslation{\textsuperscript{1}sok}
\onedefinition{2}
\spanishtranslation{\textsuperscript{2}sowilañ}
\clarification{hilo}
\secondaryentry{enredarse}
\spanishtranslation{sojkel}
\secondaryentry{enredado}
\secondtranslation{sojkem, sojwik'tyik}

\entry{enrollar}
\partofspeech{vt}
\onedefinition{1}
\spanishtranslation{\textsuperscript{1}bäl,bälulañ}
\clarification{tela, papel, cobija}
\onedefinition{2}
\spanishtranslation{bech',bäch', bejch'iñ, jaxulañ}
\clarification{ixtle, hilo}
\onedefinition{3}
\spanishtranslation{xopiñ}
\clarification{ropa}
\onedefinition{4}
\spanishtranslation{xotyilañ}
\clarification{alambre, soga}
\secondaryentry{enrollarse}
\secondtranslation{bäch'tyäl}
\clarification{ixtle, hilo}
\spanishtranslation{\textsuperscript{2}sel}
\clarification{culebra}
\spanishtranslation{selulañtyel}
\clarification{alambre, soga}
\secondaryentry{enrollado}
\secondtranslation{bäch'äl}
\clarification{estado}
\secondtranslation{bäch'bil}
\clarification{por alguien}

\entry{enroscar}
\partofspeech{vt}
\spanishtranslation{\textsuperscript{2}wox}

\entry{ensartar}
\partofspeech{vt}
\onedefinition{1}
\spanishtranslation{beljul}
\clarification{gargantillas}
\onedefinition{2}
\spanishtranslation{säl}
\clarification{carne}
\secondaryentry{ensartado}
\secondtranslation{beljulbil}
\clarification{gargantillas}

\entry{enseñanza}
\partofspeech{f}
\spanishtranslation{käñtyesa, käñtyesaya, *käñtyesäbal, *päsbal, *tyoj'ijib}

\entry{enseñar}
\partofspeech{vt}
\spanishtranslation{käñtyesañ, päs}

\entry{ensuciar}
\partofspeech{vt}
\spanishtranslation{bib'esañ}
\secondaryentry{ensuciarse}
\secondtranslation{bib'añ}

\entry{entarimar}
\partofspeech{vt}
\secondaryentry{entarimado}
\spanishtranslation{ts'ältye'ebil}
\clarification{con palos}

\entry{enterrar}
\partofspeech{vt}
\spanishtranslation{muk}
\secondaryentry{enterrado}
\secondtranslation{ts'äpäl}
\clarification{poste}

\entry{entierro}
\partofspeech{m}
\secondaryentry{lugar de entierro}
\spanishtranslation{mukoñibäl, mujkibäl}

\entry{entonces}
\partofspeech{adv}
\spanishtranslation{che' jiñi}

\entry{entrada}
\partofspeech{f}
\spanishtranslation{*yochib}

\entry{entrar}
\partofspeech{vi}
\onedefinition{1}
\spanishtranslation{ochel}
\onedefinition{2}
\spanishtranslation{ts'ajmel}
\clarification{agua en casa}

\entry{entregar}
\partofspeech{vt}
\spanishtranslation{ke'ñañ}
\clarification{para cuidar taburete}

\entry{entronque}
\partofspeech{m}
\spanishtranslation{*xäk' bij}
\clarification{de camino}

\entry{entumirse}
\partofspeech{prnl}
\spanishtranslation{señ'añ}
\secondaryentry{entumido}
\secondtranslation{señ}

\entry{envejecer}
\partofspeech{vi}
\spanishtranslation{ñejp'añ, ñox'añ}

\entry{enviar}
\partofspeech{vt}
\spanishtranslation{wets'}
\secondaryentry{enviar acá}
\secondtranslation{chok tyilel}
\secondaryentry{enviar allá}
\secondtranslation{chok majlel}

\entry{envidia}
\partofspeech{f}
\onedefinition{1}
\spanishtranslation{*mich'lel}
\onedefinition{2}
\spanishtranslation{*k'uxlel ipusik'al}
\clarification{Sab.}

\entry{envidioso}
\partofspeech{adj}
\spanishtranslation{xuxjoño}
\clarification{Sab.}

\entry{envolver}
\partofspeech{vt}
\onedefinition{1}
\spanishtranslation{\textsuperscript{1}pix, tyep'}
\onedefinition{2}
\spanishtranslation{\textsuperscript{3}bäk'}
\clarification{criatura}
\secondaryentry{envuelto}
\secondtranslation{chuybil}
\clarification{envuelto con joloche}
\secondtranslation{pixil}
\clarification{estado}
\secondtranslation{tyep'bil}
\clarification{por alguien}
\secondtranslation{xipikña}
\clarification{ropa}
\secondaryentry{material para envolver}
\secondtranslation{*pixiñtyib}

\entry{epiglotis}
\partofspeech{f}
\spanishtranslation{ñup'ip}

\entry{epilepsia}
\partofspeech{f}
\spanishtranslation{chäñchäñ ik'}
\clarification{Sab.}

\entry{equipo}
\partofspeech{m}
\secondaryentry{equipo de trabajo}
\secondtranslation{e'tyijibäl}

\entry{eructo}
\partofspeech{m}
\spanishtranslation{keb}

\entry{ésa}
\partofspeech{pron}
\spanishtranslation{jiñi}

\entry{escaldar}
\partofspeech{vt}
\spanishtranslation{pits'}

\entry{escalera}
\partofspeech{f}
\spanishtranslation{tyek'oñib}

\entry{escalón}
\partofspeech{m}
\spanishtranslation{*kejp}

\entry{escama}
\partofspeech{f}
\spanishtranslation{*sujl}

\entry{escarabajo}
\partofspeech{m}
\spanishtranslation{jäx}
\clarification{ciervo volante}
\secondaryentry{escarabajo buey}
\secondtranslation{xkukluñtya'}

\entry{escarbar}
\partofspeech{vt}
\onedefinition{1}
\spanishtranslation{p'ik}
\clarification{con aguja o palillo}
\onedefinition{2}
\spanishtranslation{tyuksiñ}
\clarification{como el cerdo}

\entry{escarpa}
\partofspeech{f}
\spanishtranslation{pañch'eñ}

\entry{esclavitud}
\partofspeech{f}
\spanishtranslation{mosojil}
\clarification{Sab.}

\entry{escoba}
\partofspeech{f}
\spanishtranslation{*misujib}

\entry{escocer}
\partofspeech{vi}
\secondaryentry{escocido}
\secondtranslation{sak'}

\entry{escoger}
\partofspeech{vt}
\onedefinition{1}
\spanishtranslation{yajkañ}
\onedefinition{2}
\spanishtranslation{lajmañ}
\clarification{tortillas}
\secondaryentry{escogido}
\secondtranslation{yajkäbil}

\entry{esconder}
\partofspeech{vt}
\onedefinition{1}
\spanishtranslation{muk}
\onedefinition{2}
\spanishtranslation{puts'tyañ}
\clarification{de una persona}
\secondaryentry{esconder de}
\secondtranslation{mujkuñ}
\secondaryentry{escondido}
\secondtranslation{mukbil}

\entry{escopeta}
\partofspeech{f}
\spanishtranslation{juloñib}

\entry{escozor}
\partofspeech{m}
\spanishtranslation{*sak'el}

\entry{escriba}
\partofspeech{m}
\spanishtranslation{sts'ijb}

\entry{escribano}
\partofspeech{m}
\spanishtranslation{ajts'ijb}
\clarification{Sab.}

\entry{escribir}
\partofspeech{vt}
\onedefinition{1}
\spanishtranslation{ts'ijbañ, ts'ijbujel, ts'ijbuñ}
\onedefinition{2}
\spanishtranslation{otsañ}
\clarification{nombre}
\secondaryentry{el que escribe}
\secondtranslation{ts'ijbaya}

\entry{escritor}
\partofspeech{m}
\spanishtranslation{sts'ijbaya, sts'ijbujel}

\entry{escritura}
\partofspeech{f}
\spanishtranslation{ts'ijb, ts'ijbujel}
\secondaryentry{escritura falsa}
\secondtranslation{payxo tsi'ijba}

\entry{escuchar}
\partofspeech{vt}
\spanishtranslation{ñich'tyañ, ubiñ}

\entry{escupir}
\partofspeech{vt}
\spanishtranslation{tyujbañ}

\entry{ése}
\partofspeech{pron}
\spanishtranslation{jiñi, jiñjiñi}
\secondaryentry{ése es}
\secondtranslation{jiñäch}
\secondaryentry{todo ése}
\secondtranslation{pejtyel jiñi}

\entry{esférico}
\partofspeech{adj}
\spanishtranslation{wolol, woxol, wotyol, k'o'ol}

\entry{eso}
\partofspeech{pron}
\secondaryentry{eso es}
\secondtranslation{jiñäch, jiñkuyi}
\secondtranslation{jiñkwäyi}
\clarification{Sab.}

\entry{espacio}
\partofspeech{m}
\spanishtranslation{*jawtyälel}
\clarification{de una casa}

\entry{espalda}
\partofspeech{f}
\spanishtranslation{*paty}
\clarification{Sab.}

\entry{espantar}
\partofspeech{vt}
\spanishtranslation{bäk'tyesañ}

\entry{espatifilo}
\partofspeech{m}
\spanishtranslation{x'ik'ujts'}
\clarification{planta}

\entry{esperar}
\partofspeech{vt}
\onedefinition{1}
\spanishtranslation{pijtyañ}
\onedefinition{2}
\spanishtranslation{chijtyañ}
\clarification{presa}
\secondaryentry{esperado}
\secondtranslation{pijtyäbil}

\entry{esperma}
\partofspeech{f}
\spanishtranslation{*p'eñel}
\clarification{Sab.; del hombre}

\entry{espeso}
\partofspeech{adj}
\spanishtranslation{tyity}

\entry{espía}
\partofspeech{m, f}
\spanishtranslation{xchijty, xk'elojel, xyojch'oya}

\entry{espiar}
\partofspeech{vt}
\spanishtranslation{yojch'oñ}

\entry{espina}
\partofspeech{f}
\spanishtranslation{\textsuperscript{2}ch'ix}
\secondaryentry{lugar donde hay muchas espinas}
\secondtranslation{ch'ixol}

\entry{espíritu}
\partofspeech{m}
\spanishtranslation{*ch'ujlel, kuktyal}
\secondaryentry{espíritu bueno}
\secondtranslation{ik'ty'ojal}
\clarification{Tila}
\secondaryentry{espíritu malo}
\secondtranslation{ajal, bulu ok, ñek}
\clarification{Tila}
\secondtranslation{ajaw}
\clarification{de la tierra, del agua}
\secondtranslation{ajtso'}
\clarification{compañero del brujo}
\secondtranslation{wäläk paty ok, ch'ix wiñik}
\clarification{Tila}

\entry{esponja}
\partofspeech{f}
\spanishtranslation{tsutspuy}
\clarification{Sab.}

\entry{espontáneo}
\partofspeech{adj}
\secondaryentry{espontáneamente}
\secondtranslation{yilol jach}

\entry{esposa}
\partofspeech{f}
\onedefinition{1}
\spanishtranslation{*ijñam, pi'äl}
\onedefinition{2}
\spanishtranslation{si'im}
\clarification{del hermano de mi madre}

\entry{esposo}
\partofspeech{m}
\spanishtranslation{*ñoxi'al, pi'äl}

\entry{espuma}
\partofspeech{f}
\spanishtranslation{lojk, *säklojk}

\entry{espumajoso}
\partofspeech{adj}
\spanishtranslation{wotswotsña}

\entry{espumar}
\partofspeech{vi}
\secondaryentry{espumando}
\secondtranslation{potspotsña}

\entry{espumoso}
\partofspeech{adj}
\spanishtranslation{potsol, wotsol, potsokña}

\entry{esquina}
\partofspeech{vt}
\spanishtranslation{*xujk}

\entry{esta}
\partofspeech{adj}
\spanishtranslation{ili, iliyi}

\entry{ésta}
\partofspeech{pron}
\spanishtranslation{jiñi}

\entry{estaca}
\partofspeech{f}
\spanishtranslation{balisajlel}

\entry{estalagmita}
\partofspeech{f}
\spanishtranslation{chu'tyuñ}

\entry{estar}
\partofspeech{vi}
\secondaryentry{está bien}
\secondtranslation{yom}
\clarification{respuesta}

\entry{este}
\partofspeech{adj}
\spanishtranslation{ili, iliyi}

\entry{éste}
\partofspeech{pron}
\spanishtranslation{jiñi, jimbä}

\entry{estiércol}
\partofspeech{m}
\spanishtranslation{tya'}

\entry{estimar}
\partofspeech{vt}
\spanishtranslation{k'uxultyañ}
\secondaryentry{estimado}
\secondtranslation{pasaru}
\clarification{por su edad}

\entry{estirar}
\partofspeech{vt}
\spanishtranslation{\textsuperscript{1}säts'}
\secondaryentry{estirado}
\secondtranslation{säpäl}
\clarification{alambre, soga}

\entry{estómago}
\partofspeech{m}
\onedefinition{1}
\spanishtranslation{*chuyib, jo'ñal, *ñäk'}
\onedefinition{2}
\spanishtranslation{chuyo'}
\clarification{Sab.}

\entry{estornudo}
\partofspeech{m}
\spanishtranslation{ja'tsijm}

\entry{estrangular}
\partofspeech{vt}
\spanishtranslation{mil}

\entry{estrecho}
\partofspeech{adj}
\spanishtranslation{ñutsul}

\entry{estrella}
\partofspeech{f}
\spanishtranslation{\textsuperscript{1}ek'}

\entry{estreñimiento}
\partofspeech{m}
\spanishtranslation{mäktya'}

\entry{estruendoso}
\partofspeech{adv}
\secondaryentry{estruendosamente}
\secondtranslation{ty'äläkña}
\clarification{agua}
\secondtranslation{\textsuperscript{1}balakña}
\clarification{como una creciente}

\entry{estufa}
\partofspeech{f}
\spanishtranslation{tyik'äjib}

\entry{estupidez}
\partofspeech{f}
\spanishtranslation{*tyoñtyojlel}

\entry{examinar}
\partofspeech{vt}
\spanishtranslation{tsajiñ}

\entry{exceder}
\partofspeech{vi}
\spanishtranslation{ñumel}

\entry{excremento}
\partofspeech{m}
\spanishtranslation{tya'}

\entry{excusado}
\partofspeech{m}
\onedefinition{1}
\spanishtranslation{paxyal}
\onedefinition{2}
\spanishtranslation{tya'jibäl}
\clarification{lugar que usa como excusado}

\entry{explotar}
\partofspeech{vi}
\spanishtranslation{tyojmel}

\entry{exprimir}
\partofspeech{vt}
\spanishtranslation{pets', yäts'}
\secondaryentry{acción de exprimir}
\secondtranslation{yäts'oñel}

\entry{extender}
\partofspeech{vt}
\spanishtranslation{\textsuperscript{2}tsuy}
\secondaryentry{extendido}
\secondtranslation{tyich'il}

\entry{extraer}
\partofspeech{vt}
\secondaryentry{extraído}
\secondtranslation{jots'bil}

\entry{facción}
\partofspeech{f}
\spanishtranslation{partye}
\clarification{Sab.}

\entry{fácil}
\partofspeech{adj}
\spanishtranslation{mach wokolik}
\clarification{lit.: no es difícil}
\secondaryentry{muy fácil}
\spanishtranslation{ñuk ñumejach}

\entry{faisán real}
\partofspeech{m}
\spanishtranslation{chäkmuty}

\entry{faja}
\partofspeech{f}
\spanishtranslation{*kajchil, chujkiñäk'äl}

\entry{falda}
\partofspeech{f}
\spanishtranslation{majtsäl}

\entry{falso}
\partofspeech{adj}
\spanishtranslation{payxo}
\secondaryentry{escritura falsa}
\secondtranslation{payxo tsi'ijba}

\entry{faltar}
\partofspeech{vt}
\secondaryentry{falta todavía}
\secondtranslation{añtyo yom}

\entry{familia}
\partofspeech{f}
\spanishtranslation{majchil}
\clarification{extendida}

\entry{fantasma}
\partofspeech{m}
\onedefinition{1}
\spanishtranslation{sombreroñ, tyeñtsuñ}
\onedefinition{2}
\spanishtranslation{ch'ix wiñik, wicheñop}
\clarification{Tila}
\onedefinition{3}
\spanishtranslation{yesumil}
\clarification{Tila; se crea que se muere en la cruz}

\entry{fastidiarse}
\partofspeech{prnl}
\onedefinition{1}
\spanishtranslation{k'ojyel}
\onedefinition{2}
\spanishtranslation{bo'oyel}
\clarification{Sab.}

\entry{fatigoso}
\partofspeech{adj}
\secondaryentry{fatigadamente}
\secondtranslation{jesuña}

\entry{favor}
\partofspeech{m}
\secondaryentry{por favor}
\secondtranslation{awokolik, wokol ty'añ; \textsuperscript{1}poj}
\clarification{por tiempo limitado}

\entry{feliz}
\partofspeech{adj}
\secondaryentry{está feliz}
\secondtranslation{tyijikña ipusik'al}

\entry{fetidez}
\partofspeech{f}
\spanishtranslation{*tyuwel}

\entry{fibra}
\partofspeech{f}
\spanishtranslation{tyiñäm}
\clarification{de las capas de plátanos}

\entry{fiebre}
\partofspeech{f}
\secondaryentry{con fiebre}
\secondtranslation{yo'okñajax}

\entry{fiel}
\partofspeech{adj}
\onedefinition{1}
\spanishtranslation{xuk'ul}
\onedefinition{2}
\spanishtranslation{jump'ejl ipusik'al}
\clarification{lit.: un solo corazón}

\entry{fiesta}
\partofspeech{f}
\spanishtranslation{k'iñijel}
\secondaryentry{participante en una fiesta}
\secondtranslation{ajk'iñijel}
\clarification{Sab.}
\secondtranslation{xk'iñijel}
\secondaryentry{participante en la fiesta de carnaval}
\secondtranslation{\textsuperscript{2}pochob}

\entry{fijo}
\partofspeech{adj}
\spanishtranslation{tyulul}
\clarification{ojos}

\entry{fila}
\partofspeech{f}
\secondaryentry{en filas}
\secondtranslation{ji'ilob, tsolokña, tsolol}
\secondaryentry{por filas}
\secondtranslation{kelekña}
\secondaryentry{poner en fila}
\secondtranslation{tsol}

\entry{filín}
\partofspeech{m}
\spanishtranslation{lu'chäy}
\clarification{pez}

\entry{filo}
\partofspeech{m}
\spanishtranslation{*yej}
\secondaryentry{con mucho filo}
\secondtranslation{jaypochañ}

\entry{fin}
\partofspeech{m}
\spanishtranslation{*yujtyibal}

\entry{finca}
\partofspeech{f}
\spanishtranslation{piñka}

\entry{fingir}
\partofspeech{vt}
\onedefinition{1}
\spanishtranslation{kujyel}
\onedefinition{2}
\spanishtranslation{kuy}

\entry{fino}
\partofspeech{adj}
\spanishtranslation{ts'ubukña}

\entry{firme}
\partofspeech{adj}
\spanishtranslation{xuk'ul}

\entry{flaco}
\partofspeech{adj}
\spanishtranslation{yajyaj}

\entry{flamante}
\partofspeech{adj}
\spanishtranslation{lemla}

\entry{flauta}
\partofspeech{f}
\spanishtranslation{jaläl}
\clarification{hecha de carrizo}

\entry{flojera}
\partofspeech{f}
\spanishtranslation{*ts'u'lel}

\entry{flojo}
\partofspeech{adj}
\onedefinition{1}
\spanishtranslation{chojol, pojokña}
\clarification{objeto}
\onedefinition{2}
\spanishtranslation{ts'ub}
\clarification{persona}

\entry{flor}
\partofspeech{f}
\spanishtranslation{bob, ñichim}
\secondaryentry{flor de corazón}
\secondtranslation{xjol max}
\clarification{árbol}
\secondaryentry{flor de gusano}
\secondtranslation{ik' k'uts}
\clarification{hierba}
\secondaryentry{flor de maíz}
\secondtranslation{jañ}
\secondaryentry{flor de mayo}
\secondtranslation{xñichimtye'}
\clarification{árbol}
\secondaryentry{flor de muerto}
\secondtranslation{xtyiñjol}
\secondaryentry{flor de pasión}
\secondtranslation{ch'um'ak'}
\secondaryentry{flor de pato}
\secondtranslation{*pixol xiba}
\clarification{bejuco}

\entry{Flor del Agua}
\spanishtranslation{Ñichimbäja'}
\clarification{colonia}

\entry{floreciente}
\partofspeech{adj}
\spanishtranslation{ñichikña}

\entry{fofa}
\partofspeech{adj}
\spanishtranslation{k'uñluyañ}
\clarification{piel}

\entry{follaje}
\partofspeech{m}
\spanishtranslation{*yopotye'}

\entry{formación}
\partofspeech{f}
\secondaryentry{en formación}
\secondtranslation{\textsuperscript{2}ji'il}

\entry{fornicación}
\partofspeech{f}
\spanishtranslation{*ts'i'lel}

\entry{forzar}
\partofspeech{vt}
\spanishtranslation{tyujlañ}
\clarification{para soltarse}

\entry{foto}
\partofspeech{f}
\spanishtranslation{ejtyaläl, *yejtyal}

\entry{francolina}
\partofspeech{f}
\spanishtranslation{xkulukab}
\clarification{gallina de monte, gran tinamú; ave}

\entry{freír}
\partofspeech{vt}
\spanishtranslation{ch'il}

\entry{frente}
\partofspeech{f}
\spanishtranslation{pam}
\clarification{de la cara}

\entry{frijol}
\partofspeech{m}
\spanishtranslation{bu'ul}
\secondaryentry{frijol amarillo}
\secondtranslation{kañcheñek}
\clarification{Sab.}
\secondaryentry{frijol botil}
\secondtranslation{pojkäm}
\secondaryentry{frijol de castilla}
\secondtranslation{xwach' bu'ul}
\secondaryentry{frijol de rabia}
\secondtranslation{xpermax}
\secondaryentry{frijol de vara}
\secondtranslation{sts'äptye' bu'ul}
\secondaryentry{frijol del año}
\secondtranslation{xjabi bu'ul}
\secondaryentry{frijol pato}
\secondtranslation{xpech bu'ul}
\secondaryentry{frijol tierno}
\secondtranslation{ch'ok bu'ul}
\secondaryentry{tipo de frijol chico}
\secondtranslation{ts'ir bu'ul}
\secondaryentry{tipo de frijol grande}
\secondtranslation{xchu' bu'ul}
\secondaryentry{tipo de frijol de tierra}
\secondtranslation{xtya'lety}
\clarification{no se enreda y no tiene punta}
\secondtranslation{xlumil bu'ul}

\entry{frijolar}
\partofspeech{m}
\spanishtranslation{bu'lel}

\entry{frío}
\partofspeech{m}
\spanishtranslation{tsäñal, tsäñä}

\entry{frontera}
\partofspeech{f}
\spanishtranslation{kayajoñ}

\entry{frotar}
\partofspeech{vt}
\onedefinition{1}
\spanishtranslation{bilty'uñ}
\clarification{piel}
\onedefinition{2}
\spanishtranslation{biyulañ}
\clarification{cosa resbalosa}
\onedefinition{3}
\spanishtranslation{jajpiñ}
\clarification{con medicina}
\secondaryentry{frotarse}
\spanishtranslation{bijyel}

\entry{fruta}
\partofspeech{f}
\spanishtranslation{\textsuperscript{2}*wuty, yäk'bal}
\secondaryentry{fruta comestible de una planta que madura en junio}
\secondtranslation{x'ik'uts}
\secondaryentry{fruta comestible de un bejuco}
\spanishtranslation{tsots}
\secondaryentry{fruta del árbol}
\secondtranslation{*wuty tye'}

\entry{fuego}
\partofspeech{m}
\spanishtranslation{k'ajk}

\entry{fuerte}
\partofspeech{adj}
\onedefinition{1}
\spanishtranslation{ch'ejl}
\clarification{persona, animal}
\onedefinition{2}
\spanishtranslation{ch'e', chalakña}
\clarification{ruido}
\onedefinition{3}
\spanishtranslation{p'ätyäl}
\clarification{animal, persona, medicina}
\secondaryentry{fuertemente}
\secondtranslation{\textsuperscript{2}k'am}
\secondtranslation{ts'ilikña}
\clarification{Sab.; llorando}
\secondaryentry{hacerse fuerte}
\secondtranslation{p'äty'añ}
\secondaryentry{voz fuerte}
\secondtranslation{k'ambä ty'añ}

\entry{fuerza}
\partofspeech{f}
\onedefinition{1}
\spanishtranslation{\textsuperscript{1}wersa}
\onedefinition{2}
\spanishtranslation{*ts'a'ñal}
\clarification{de la cal}

\entry{fumar}
\onedefinition{1}
\partofspeech{vi}
\spanishtranslation{k'ujtsijel}
\onedefinition{2}
\partofspeech{vt}
\spanishtranslation{ñuk'}

\entry{gajo}
\partofspeech{m}
\spanishtranslation{*k'äbtye'}
\clarification{de un árbol}
\secondaryentry{gajos secos tirados en la milpa}
\secondtranslation{chax läktyik}
\secondaryentry{donde sale el gajo}
\secondtranslation{*xäk'}

\entry{gallina}
\partofspeech{f}
\onedefinition{1}
\spanishtranslation{muty, ña' muty}
\onedefinition{2}
\spanishtranslation{xpeloña}
\clarification{sin plumaje en el cuello}
\secondaryentry{gallina sin plumas}
\secondtranslation{xborox}
\secondaryentry{gallina crespa}
\secondtranslation{xwarach'}
\secondaryentry{gallina ciega}
\secondtranslation{k'ojlom}
\clarification{gusano}
\secondaryentry{gallina de monte}
\secondtranslation{xkulukab}

\entry{gallo}
\partofspeech{m}
\spanishtranslation{kayu, muty}

\entry{ganar}
\partofspeech{vt}
\spanishtranslation{chobejtyañ}

\entry{gancho}
\partofspeech{m}
\spanishtranslation{*jok'lib}
\clarification{para colgar cosas}

\entry{ganso de collar}
\spanishtranslation{xkañso pech}
\clarification{ave}

\entry{garabato}
\partofspeech{m}
\spanishtranslation{*jok'lib}
\secondaryentry{garabato de hierro o palo}
\secondtranslation{rok}

\entry{garganta}
\partofspeech{f}
\spanishtranslation{ch'och'}

\entry{gargantilla}
\partofspeech{vt}
\spanishtranslation{ujäl}
\clarification{sing.}
\spanishtranslation{ujañ}
\clarification{pl.}

\entry{garita}
\partofspeech{f}
\spanishtranslation{karitya}

\entry{garrafón}
\partofspeech{m}
\spanishtranslation{limetyoñ}

\entry{garrapata}
\partofspeech{f}
\spanishtranslation{sip}
\clarification{insecto}

\entry{garza}
\partofspeech{f}
\spanishtranslation{jojmay, ja'al pech}

\entry{gastar}
\partofspeech{vt}
\spanishtranslation{pujch'el}
\clarification{la punta, fila}
\secondaryentry{gastado}
\spanishtranslation{tsukul, *tsukulel}

\entry{gatear}
\partofspeech{vi}
\onedefinition{1}
\spanishtranslation{kojtyel}
\clarification{criatura}
\onedefinition{2}
\spanishtranslation{tsojtyel}
\clarification{niño}

\entry{gato}
\partofspeech{m}
\spanishtranslation{mis}

\entry{gavilán}
\partofspeech{m}
\onedefinition{1}
\spanishtranslation{xäye', xiye'}
\clarification{grande}
\onedefinition{2}
\spanishtranslation{xlilik, xuñxulu'}
\clarification{chico}
\secondaryentry{gavilán negro}
\secondtranslation{xtyow}

\entry{gemelos}
\partofspeech{m}
\onedefinition{1}
\spanishtranslation{\textsuperscript{2}loj}
\onedefinition{2}
\spanishtranslation{luty}
\clarification{Sab.}

\entry{gemido}
\partofspeech{m}
\spanishtranslation{ajkäñ}

\entry{gente}
\partofspeech{f}
\spanishtranslation{kixtyañu}
\clarification{Tila}
\secondaryentry{gente de Tenejapa}
\secondtranslation{tsutsob}
\secondaryentry{gente dentro de la tierra}
\secondtranslation{xma'ity}
\secondaryentry{gente que no es indígena}
\secondtranslation{kaxlañ}

\entry{glotón}
\partofspeech{m}
\spanishtranslation{wo'atyax}

\entry{glotonería}
\partofspeech{vt}
\spanishtranslation{*sits'lel}

\entry{golonchaco}
\partofspeech{m}
\spanishtranslation{chañwox}
\clarification{ave}

\entry{golondrina}
\partofspeech{f}
\spanishtranslation{wirischañ, xwirischañ, x'alum}
\clarification{ave}
\secondaryentry{golondrina de cueva}
\secondtranslation{xwilis}

\entry{goloso}
\partofspeech{adj}
\spanishtranslation{koloso}

\entry{golpear}
\partofspeech{vt}
\onedefinition{1}
\spanishtranslation{kuj}
\clarification{con martillo}
\onedefinition{2}
\spanishtranslation{ch'ojch'oj}
\clarification{ligeramente}
\onedefinition{3}
\spanishtranslation{jats'}
\clarification{con objeto}
\secondaryentry{golpear una herida}
\secondtranslation{tyajñañ}
\secondaryentry{golpeado}
\secondtranslation{ch'oj}
\clarification{la madera con piedra}
\secondaryentry{golpeando}
\secondtranslation{lajlaj}
\secondtranslation{chek'chek'ña}
\clarification{máquina de escribir}

\entry{gordo}
\onedefinition{1}
\partofspeech{adj}
\spanishtranslation{jujp'em}
\onedefinition{2}
\partofspeech{m}
\spanishtranslation{*boñtyilel}
\secondaryentry{muy gordo}
\secondtranslation{bayakña}

\entry{gorrioncillo}
\partofspeech{m}
\spanishtranslation{ak'bä ts'uñuñ}
\clarification{zacatonero; ave}

\entry{gota}
\partofspeech{f}
\secondaryentry{gota por gota}
\secondtranslation{ts'ujlaw}

\entry{gotear}
\partofspeech{vi}
\secondaryentry{goteando}
\secondtranslation{chäk', chäk'chäk'ña}

\entry{gotera}
\partofspeech{f}
\spanishtranslation{ochja'}

\entry{gracias}
\partofspeech{f}
\spanishtranslation{wokolix awälä, wokol awälä}

\entry{gradado}
\partofspeech{adj}
\spanishtranslation{kejpuktyik}

\entry{grajo verde}
\secondpartofspeech{m}
\secondtranslation{peazul, xkekex}
\clarification{queisque; ave}

\entry{gran tinamú}
\partofspeech{m}
\spanishtranslation{xkulukab}
\clarification{gallina de monte; ave}

\entry{granadilla}
\partofspeech{f}
\spanishtranslation{ch'um'ak'}
\clarification{fruta}

\entry{grande}
\partofspeech{adj}
\onedefinition{1}
\spanishtranslation{kolem, \textsuperscript{1}ñoj, \textsuperscript{1}ñuk}
\onedefinition{2}
\spanishtranslation{pitytyäl}
\clarification{animal, carga}
\onedefinition{3}
\spanishtranslation{ty'uñul}
\clarification{mazorca de maíz}
\secondaryentry{no es grande}
\secondtranslation{mach ñukik}
\secondaryentry{tamaño grande}
\secondtranslation{*ñojel}
\clarification{mazorca}

\entry{granizo}
\partofspeech{m}
\spanishtranslation{tyuñija'}

\entry{grano}
\partofspeech{m}
\onedefinition{1}
\spanishtranslation{k'osäl}
\clarification{en la cabeza}
\onedefinition{2}
\spanishtranslation{ty'äsläk}
\clarification{tiene la carne del cerdo}
\secondaryentry{granos podridos}
\secondtranslation{jom}
\clarification{mazorca}

\entry{grava}
\partofspeech{vt}
\spanishtranslation{bi'tyi xajlel}
\clarification{Sab.}

\entry{grieta}
\partofspeech{f}
\spanishtranslation{tyokolbä lum}

\entry{grillo}
\partofspeech{m}
\spanishtranslation{\textsuperscript{2}chil}
\secondaryentry{un tipo de grillo}
\secondtranslation{xña'ak'bal}

\entry{grisón}
\partofspeech{m}
\spanishtranslation{sakol}
\clarification{mamífero}

\entry{gritar}
\partofspeech{vt}
\spanishtranslation{otyañ}
\clarification{a una persona}
\spanishtranslation{mi icha'leñ oñel}

\entry{grito}
\partofspeech{m}
\spanishtranslation{oñel}
\secondaryentry{a gritos}
\secondtranslation{ts'elekña}

\entry{grosería}
\partofspeech{f}
\spanishtranslation{p'ajoñel, *tsukulel}

\entry{grosero}

\partofspeech{adj}
\spanishtranslation{leko}

\entry{grueso}
\onedefinition{1}
\partofspeech{adj}
\spanishtranslation{ty'uñul}
\onedefinition{2}
\partofspeech{adj}
\spanishtranslation{pim}
\clarification{tortilla, cobija, tabla}
\onedefinition{3}
\partofspeech{adj}
\spanishtranslation{p'ojp'ostyäl}
\clarification{frijol}
\onedefinition{4}
\partofspeech{m}
\spanishtranslation{\textsuperscript{1}*jaylel}
\clarification{de tabla, papel}

\entry{gruñir}
\partofspeech{vi}
\secondaryentry{gruñendo}
\secondtranslation{käräkña, jiñikña}

\entry{guabina}
\partofspeech{f}
\spanishtranslation{\textsuperscript{2}sok}
\clarification{pez}

\entry{guacamayo}
\partofspeech{m}
\spanishtranslation{jochitye'}
\clarification{árbol}

\entry{guachipilín}
\partofspeech{m}
\spanishtranslation{k'äñsijñ}
\clarification{árbol}

\entry{guaco}
\partofspeech{m}
\onedefinition{1}
\spanishtranslation{xchawa'ik'}
\clarification{planta}
\onedefinition{2}
\spanishtranslation{mo'och}
\clarification{ave}

\entry{guaje blanco}
\partofspeech{m}
\spanishtranslation{pipäl, amila}
\clarification{árbol}

\entry{guajolote}
\partofspeech{m}
\spanishtranslation{ak'ach}

\entry{guamúchil}
\partofspeech{m}
\spanishtranslation{tyuk'ul}
\clarification{patzagua; árbol}

\entry{guanábana}
\partofspeech{f}
\spanishtranslation{k'ätsats}
\clarification{árbol}

\entry{guano}
\partofspeech{m}
\spanishtranslation{xlecheñtyo'}
\clarification{palma}

\entry{guao}
\partofspeech{m}
\spanishtranslation{waw}
\clarification{tres lomas; tipo de tortuga}

\entry{guapaque}
\partofspeech{m}
\spanishtranslation{wäch'}
\clarification{árbol}

\entry{guaqueque negro}
\partofspeech{m}
\spanishtranslation{ujchib, xuchijp}
\clarification{agutí; mamífero}

\entry{guardabarranco}
\partofspeech{m}
\spanishtranslation{xwukpik}
\clarification{péndulo de corona; ave}

\entry{guardar}
\partofspeech{vt}
\spanishtranslation{\textsuperscript{2}loty}

\entry{guardia}
\partofspeech{f}
\spanishtranslation{xk'elojel}

\entry{guarumbo}
\partofspeech{m}
\spanishtranslation{k'olok'}
\clarification{chancarro; árbol}

\entry{guayabillo}
\partofspeech{m}
\spanishtranslation{pätyatye', xpapätyal}
\clarification{árbol}

\entry{guayabo}
\partofspeech{m}
\spanishtranslation{pätya}

\entry{guía}
\partofspeech{f}
\onedefinition{1}
\spanishtranslation{xpäsbij}
\onedefinition{2}
\spanishtranslation{*yä'k'il}
\clarification{de planta}

\entry{guiar}
\partofspeech{vt}
\spanishtranslation{tyoj'esañ}

\entry{guindar}
\partofspeech{vt}
\secondaryentry{guindado}
\spanishtranslation{litsil}
\clarification{cabeza de pavo, cuerpo de culebra}
\spanishtranslation{tsok'ol}
\clarification{racimo de plátanos}
\secondaryentry{guindado de la mano}
\spanishtranslation{jich'ye'el}

\entry{gusano}
\partofspeech{m}
\spanishtranslation{motso'}
\secondaryentry{tipo de gusano comestible}
\secondtranslation{\textsuperscript{2}säts'}
\secondaryentry{gusano de agua}
\secondtranslation{xtyajbachim}

\entry{gustar}
\partofspeech{vt}
\spanishtranslation{mulañ}
\secondaryentry{le está gustando}
\secondtranslation{k'ajakñayix yo}
\clarification{Sab.}

\entry{haber}
\partofspeech{vt}
\secondaryentry{hay}
\secondtranslation{añ}
\secondaryentry{lo hay}
\secondtranslation{añ ku}
\secondaryentry{no hay}
\secondtranslation{ma'añik, k'ebäch}
\clarification{respuesta al dar lo que se pide}
\secondaryentry{no hay}
\secondtranslation{ma'añ}
\clarification{Sab., Tila}
\secondaryentry{si hay}
\secondtranslation{añäch}
\secondaryentry{si hubiera}
\secondtranslation{añik}

\entry{habitación}
\partofspeech{f}
\spanishtranslation{chumlibäl}

\entry{habitante}
\partofspeech{m}
\onedefinition{1}
\spanishtranslation{ajchumtyäl}
\clarification{Sab.}
\onedefinition{2}
\spanishtranslation{xchumtyäl}
\clarification{pl.}

\entry{hablar}
\partofspeech{vi}
\spanishtranslation{\textsuperscript{1}pejkañ, mi icha'leñ ty'añ}
\secondaryentry{hablar mal}
\secondtranslation{pampañ}
\clarification{Sab.; de otro}
\secondaryentry{el que habla}
\secondtranslation{x'alty'añ; ajsubty'añ}
\clarification{Sab.}

\entry{hacer}
\partofspeech{vt}
\spanishtranslation{cha'leñ, mel, tyumbeñ}
\secondaryentry{hacer a un lado}
\secondtranslation{k'ej}
\clarification{Tila}
\secondaryentry{hacer calor}
\secondtranslation{k'ixiñ pañimil, tyikäw pañimil}
\secondaryentry{hacer casa}
\secondtranslation{\textsuperscript{2}päty}
\clarification{Tila}
\secondaryentry{hacer tacos}
\secondtranslation{p'ich}
\secondaryentry{hacer tortillas}
\secondtranslation{pechañ}
\secondtranslation{pechmañ}
\clarification{Sab.}
\secondaryentry{hace tiempo}
\secondtranslation{wajalix}
\secondaryentry{hace mucho tiempo}
\secondtranslation{oñiyi}

\entry{hacha}
\partofspeech{vt}
\spanishtranslation{jachaj}

\entry{hacinar}
\partofspeech{vt}
\spanishtranslation{ts'äl, läts, lätschokoñ}
\clarification{leña, maíz}
\secondaryentry{hacinado}
\secondtranslation{lätsäl, ts'äläl}
\clarification{leña, maíz}
\secondtranslation{ts'älbil}
\clarification{por alguien}

\entry{hallar}
\partofspeech{vt}
\secondaryentry{hallado}
\secondtranslation{k'ajyem}

\entry{hamaca}
\partofspeech{f}
\spanishtranslation{ab}
\secondaryentry{hamaca hecha de palitos}
\secondtranslation{ji'tye'ol}

\entry{hambre}
\partofspeech{f}
\spanishtranslation{wi'ñal}

\entry{haragán}
\partofspeech{m}
\spanishtranslation{sts'u'lel}

\entry{hasta}
\partofspeech{prep}
\spanishtranslation{k'äläl, jiñtyo}

\entry{hechicería}
\partofspeech{f}
\spanishtranslation{ch'äkoñel, wujty, xpejka xotyäl}

\entry{helado}
\partofspeech{adj}
\spanishtranslation{*ña'al tsäñal}

\entry{helecho}
\partofspeech{m}
\spanishtranslation{stsijb}

\entry{hembra}
\partofspeech{f}
\onedefinition{1}
\spanishtranslation{\textsuperscript{2}*ña'al}
\clarification{de animal}
\onedefinition{2}
\spanishtranslation{iña'al ña' muty}
\clarification{gallina}

\entry{hendidura}
\partofspeech{f}
\spanishtranslation{*jajp}
\clarification{hendedura}

\entry{henequén}
\partofspeech{m}
\spanishtranslation{k'ok'chi}

\entry{hepatitis}
\partofspeech{f}
\spanishtranslation{*k'äñajel}

\entry{herida}
\partofspeech{f}
\onedefinition{1}
\spanishtranslation{lojwel}
\onedefinition{2}
\spanishtranslation{*tsejpel}
\clarification{cortada}

\entry{hermana}
\partofspeech{f}
\onedefinition{1}
\spanishtranslation{*chich, chichäl, ñoj chich}
\clarification{mayor}
\onedefinition{2}
\spanishtranslation{*ijtyi'añ}
\clarification{de hombre}
\secondaryentry{hermana de la madre del padre}
\secondtranslation{ko'äl}

\entry{hermanastra}
\partofspeech{f}
\spanishtranslation{*majañ chich, *majañ ijtyi'añ}

\entry{hermanastro}
\partofspeech{m}
\onedefinition{1}
\spanishtranslation{*majañ ijtyi'añ}
\onedefinition{2}
\spanishtranslation{*majañ *äskuñ}
\clarification{mayor}

\entry{hermano}
\partofspeech{m}
\onedefinition{1}
\spanishtranslation{erañ, ermañu, \textsuperscript{2}*yaj}
\onedefinition{2}
\spanishtranslation{askuñ, ñoj *äskuñ}
\clarification{mayor}
\onedefinition{3}
\spanishtranslation{xuty}
\clarification{menor}
\secondaryentry{hermanito}
\secondtranslation{*ijts'iñ}
\secondaryentry{hermano de la abuela paterna}
\secondtranslation{mam}
\secondaryentry{hermano de mi padre}
\secondtranslation{*yumijel}

\entry{herrumbre}
\partofspeech{f}
\spanishtranslation{tya', *tsukulel}

\entry{hervido}
\partofspeech{adj}
\spanishtranslation{ch'äxbil}

\entry{hervir}
\partofspeech{vt}
\spanishtranslation{chäp, ch'äx}
\clarification{frijol, maíz}
\secondaryentry{hervirse}
\secondtranslation{echmäl}
\secondaryentry{hacer hervir}
\secondtranslation{lojkesañ}
\secondaryentry{acción de hervir}
\secondtranslation{lojk}

\entry{hielo}
\partofspeech{m}
\spanishtranslation{tyuñija', ña'al tsäñal}

\entry{hierba}
\partofspeech{f}
\spanishtranslation{pimel}
\secondaryentry{hierba laxante que se come}
\secondtranslation{\textsuperscript{1}tsuy}
\secondaryentry{hierba para condimentar carne}
\secondtranslation{xajb}
\secondaryentry{un tipe de hierba grande}
\secondtranslation{x'almis kajpe'}
\secondaryentry{hierba mora}
\secondtranslation{ch'ajuk'}
\secondaryentry{hierba santa}
\secondtranslation{momoy, serbatyañ}
\clarification{arbusto}
\secondaryentry{sin hierba}
\secondtranslation{chäk}

\entry{hierbabuena}
\partofspeech{f}
\spanishtranslation{araweño, x'araweño}

\entry{hierbatero}
\partofspeech{m}
\spanishtranslation{sts'äkaya}

\entry{hígado}
\partofspeech{m}
\spanishtranslation{*olmal, *soty'oty'}

\entry{higuerilla}
\partofspeech{f}
\spanishtranslation{ch'upujk}
\clarification{ricino; planta}

\entry{hija}
\partofspeech{f}
\onedefinition{1}
\spanishtranslation{*ajty'al, *ixik'al}
\clarification{de mujer}
\onedefinition{2}
\spanishtranslation{*ixikp'eñel}
\clarification{de hombre, grande}

\entry{hijastro}
\partofspeech{m}
\spanishtranslation{*majañ kalobil}

\entry{hijo}
\partofspeech{m}
\onedefinition{1}
\spanishtranslation{*ajty'al}
\clarification{de mujer}
\onedefinition{2}
\spanishtranslation{*p'eñel}
\clarification{Sab.; del padre}
\onedefinition{3}
\spanishtranslation{wiñikbä *alobil}
\clarification{adulto}
\secondaryentry{hijo de hermana de mi padre}
\secondtranslation{*ichak'}
\secondaryentry{hijos}
\secondtranslation{*p'olbal}

\entry{hilar}
\partofspeech{vt}
\spanishtranslation{ñok'ijel, ñok'iñ}

\entry{hilo}
\partofspeech{m}
\spanishtranslation{puy}

\entry{hinchar}
\partofspeech{vt}
\secondaryentry{hincharse}
\secondtranslation{sity'kuyel, sity'olajel}
\secondaryentry{hinchado}
\secondtranslation{sijty'em, sity'il}

\entry{hipo}
\partofspeech{m}
\spanishtranslation{jäch'bik'}

\entry{hipócrita}
\partofspeech{f}
\spanishtranslation{cha'chajp ipusik'al}
\clarification{lit.: tiene dos corazones}

\entry{hocofaisán}
\partofspeech{m}
\spanishtranslation{chäkmuty}
\clarification{faisán; ave}

\entry{hoja}
\partofspeech{f}
\onedefinition{1}
\spanishtranslation{yopol}
\onedefinition{2}
\spanishtranslation{yopmal}
\clarification{Sab.}
\secondaryentry{hoja de platana}
\secondtranslation{*pechäjib}
\clarification{done se echan las tortillas}
\secondaryentry{hoja de quequeste}
\secondtranslation{poko'}
\secondaryentry{hoja de tabaco}
\secondtranslation{pocholbä k'uts}
\secondaryentry{hoja grande para envolver pozol}
\secondtranslation{säktyo', tyuñtyo'}
\secondaryentry{hoja de pozol}
\spanishtranslation{säpäty yopom}
\secondaryentry{hoja de tamales}
\secondtranslation{xwaj'uñ}
\secondaryentry{una sola hoja}
\secondtranslation{yopom}
\secondaryentry{hojas podridas}
\secondtranslation{omos}
\secondaryentry{el cambio de hoja de los árboles}
\secondtranslation{ch'ajb}

\entry{hollar}
\partofspeech{vt}
\secondaryentry{hollado}
\secondtranslation{tyokbil}

\entry{hombre}
\partofspeech{m}
\spanishtranslation{wiñik}
\secondaryentry{hombre tzeltal}
\secondtranslation{amiku}
\secondaryentry{hombre de más edad}
\secondtranslation{askuñäl}
\secondaryentry{hombre con mucha barba}
\secondtranslation{tsutschoj}
\secondaryentry{hombre con lobanillo}
\secondtranslation{xbulubik'}
\secondaryentry{hombre que anda confuso}
\secondtranslation{xkajka}

\entry{hombro}
\partofspeech{m}
\onedefinition{1}
\spanishtranslation{kejlo'}
\onedefinition{2}
\spanishtranslation{*kejlab}
\clarification{Sab.}

\entry{homicidio}
\partofspeech{m}
\spanishtranslation{tsäñsa}

\entry{honda}
\partofspeech{f}
\spanishtranslation{päräñtyuñ}
\clarification{tirador}

\entry{hondo}
\partofspeech{adj}
\spanishtranslation{tyam}
\clarification{río, lodazal}
\spanishtranslation{*tyamlel}

\entry{hondonada}
\partofspeech{f}
\spanishtranslation{*lomtyilel}
\clarification{del terreno}

\entry{hondonado}
\partofspeech{adj}
\spanishtranslation{k'omokña, k'omol}

\entry{hongo}
\partofspeech{m}
\spanishtranslation{tyeñkech}
\secondaryentry{hongo que crece en los árboles}
\secondtranslation{ityeñkechlel tye'}
\secondaryentry{tipo de hongo}
\secondtranslation{k'o'loch}
\clarification{comestible}
\secondtranslation{x'uxkuruñwech patye'}
\clarification{gris, comestible}
\secondtranslation{xyok muty patye', x'yok muty patye'}
\clarification{blanco, comestible}
\secondtranslation{xkeräch' patye'}
\clarification{negro, comestible}
\secondtranslation{xlo' patye'}
\clarification{colorado, comestible}
\secondtranslation{tsuts chikiñ}
\clarification{de color café arriba y blanco abajo, comestible}
\secondtranslation{joch' lo' patye'}
\clarification{de árboles, blanco, comestible}

\entry{hora}
\partofspeech{f}
\spanishtranslation{*tyiempojlel}
\clarification{Sab.}
\secondaryentry{una hora antes}
\secondtranslation{juñwa'le}
\clarification{Sab.}

\entry{horcajada}
\secondaryentry{a horcajadas}
\secondtranslation{xak'}
\clarification{sobre un palo}

\entry{horcón}
\partofspeech{m}
\onedefinition{1}
\spanishtranslation{oy}
\onedefinition{2}
\spanishtranslation{iyoyel otyoty}
\clarification{de la casa}

\entry{hormiga}
\partofspeech{f}
\spanishtranslation{xiñich'}
\secondaryentry{tipo de homiga}
\secondtranslation{chakatyorex}
\clarification{colorada, pica fuerte}
\spanishtranslation{xty'ojty'ojbak}
\clarification{negra, grande}

\entry{hormiguillo}
\partofspeech{m}
\spanishtranslation{bu'ultye'}
\clarification{palo de marimba; árbol}

\entry{horqueta}
\partofspeech{f}
\spanishtranslation{xäk'tye'}

\entry{hortaliza}
\partofspeech{f}
\spanishtranslation{päk'äbäl}

\entry{hoy}
\partofspeech{adv}
\spanishtranslation{wäle}

\entry{hoyar}
\partofspeech{vi}
\secondaryentry{hoyado}
\secondtranslation{ch'ubul}
\clarification{chico}
\secondtranslation{boxol}
\clarification{adentro en forma redonda}

\entry{hoyuelo}
\partofspeech{m}
\spanishtranslation{*chojk'ib}

\entry{huarache}
\partofspeech{m}
\onedefinition{1}
\spanishtranslation{kaktye', warach, xäñäbäl}
\onedefinition{2}
\spanishtranslation{\textsuperscript{2}pats'}
\clarification{Sab.}

\entry{hueco}
\partofspeech{adj}
\spanishtranslation{ch'omol}
\clarification{terreno}
\secondaryentry{hueco de un árbol}
\secondtranslation{*ch'eñal}
\secondtranslation{*ch'ejñal tye'}
\clarification{Sab.}
\secondaryentry{hueco hondo}
\spanishtranslation{ch'eñ}

\entry{huérfano}
\partofspeech{m}
\spanishtranslation{meba', meba' aläl}
\secondaryentry{muchacho dejado huéfano}
\secondtranslation{meba' ch'ityoñ}
\secondaryentry{muchacha dejada huéfana}
\secondtranslation{meba' xch'ok}

\entry{hueso}
\partofspeech{m}
\spanishtranslation{bak}
\secondaryentry{persona con hueso quebrado}
\secondtranslation{xk'äskujel}

\entry{huésped}
\partofspeech{m}
\spanishtranslation{jula'}

\entry{huevo}
\partofspeech{m}
\onedefinition{1}
\spanishtranslation{tyumuty}
\onedefinition{2}
\spanishtranslation{tyuñ}
\clarification{Tila}
\secondaryentry{huevo que no brota}
\secondtranslation{puk'tya'}

\entry{huirse}
\partofspeech{prnl}
\spanishtranslation{puts'el}

\entry{humeante}
\partofspeech{adj}
\spanishtranslation{ts'obokña}
\clarification{al quemar milpa}

\entry{humedad}
\partofspeech{vt}
\spanishtranslation{*yäch'lel}

\entry{humillar}
\partofspeech{vt}
\secondaryentry{humillado}
\secondtranslation{pek'}

\entry{humo}
\partofspeech{m}
\spanishtranslation{buts'}

\entry{humus}
\partofspeech{m}
\spanishtranslation{omos}

\entry{hundir}
\partofspeech{vt}
\secondaryentry{hundido}
\secondtranslation{ch'omol}

\entry{huso}
\partofspeech{m}
\spanishtranslation{petyejty}
\clarification{palito para enrollar hilo}

\entry{idioma}
\partofspeech{m}
\spanishtranslation{*ty'añ}

\entry{ídolo}
\partofspeech{m}
\spanishtranslation{rus, sañtyo}

\entry{iglesia}
\partofspeech{f}
\spanishtranslation{lesia}

\entry{ignorante}
\partofspeech{adj}
\spanishtranslation{tyoñkots}

\entry{igual}
\partofspeech{adv}
\spanishtranslation{juñlajal, lajal}
\secondaryentry{no es igual}
\secondtranslation{mach lajalik}

\entry{igualar}
\partofspeech{vt}
\spanishtranslation{lajiñ, \textsuperscript{2}laj}

\entry{iguana}
\partofspeech{f}
\spanishtranslation{kolem p'ok, \textsuperscript{2}juj, jujl p'ok, mäch p'ok, tseljol}

\entry{ilama}
\partofspeech{f}
\spanishtranslation{säsäk k'ätsats}
\clarification{árbol}

\entry{imagen}
\partofspeech{f}
\secondaryentry{imagen de Cristo}
\secondtranslation{*ch'ujutyaty}

\entry{impensado}
\partofspeech{adj}
\secondaryentry{impensadamente}
\secondtranslation{ch'äñch'äña}
\clarification{hablar}

\entry{importancia}
\partofspeech{f}
\spanishtranslation{*k'äjñibal, *k'äjñibäyel, *ñuklel}

\entry{importante}
\partofspeech{adj}
\spanishtranslation{\textsuperscript{1}ñuk}

\entry{imposible}
\partofspeech{adj}
\spanishtranslation{mach mejlik}

\entry{impuesto}
\partofspeech{m}
\spanishtranslation{tyojoñel}

\entry{incendio}
\partofspeech{m}
\spanishtranslation{tyojklel}

\entry{incienso}
\partofspeech{m}
\spanishtranslation{pom}

\entry{inclinar}
\partofspeech{vt}
\onedefinition{1}
\spanishtranslation{lij}
\clarification{la cabeza}
\onedefinition{2}
\spanishtranslation{tyiñtyäl}
\clarification{Sab.}
\secondaryentry{inclinado}
\secondtranslation{ñäkäl}
\secondtranslation{boltyäl}
\clarification{Sab.}
\secondtranslation{lijil}
\clarification{la cabeza}
\secondtranslation{muyul}
\clarification{casa}
\secondtranslation{ñäkye'el}
\clarification{en la mano}
\secondtranslation{tyäkäl}
\clarification{postes, horcones}
\secondtranslation{wits}
\clarification{camino}
\secondtranslation{xewel}
\clarification{en el agua}

\entry{inconcluso}
\partofspeech{adj}
\spanishtranslation{kets'el}
\clarification{trabajo}

\entry{inconstante}
\partofspeech{adj}
\spanishtranslation{tyilel bixel lakpusik'al}

\entry{increíble}
\partofspeech{adj}
\spanishtranslation{jemachtyika}
\clarification{expresión de sorpresa}

\entry{infierno}
\partofspeech{m}
\spanishtranslation{tyojmulil}

\entry{infinito}
\partofspeech{adj}
\spanishtranslation{läbäkña}

\entry{inflamación}
\partofspeech{f}
\onedefinition{1}
\spanishtranslation{paslam}
\onedefinition{2}
\spanishtranslation{pots'lom}
\clarification{Sab.}

\entry{inflar}
\partofspeech{vt}
\secondaryentry{oprimir e inflar}
\secondtranslation{wosilañ}
\clarification{pelota}

\entry{inmediato}
\partofspeech{adj}
\onedefinition{1}
\spanishtranslation{ora jach, tyi ora}
\onedefinition{2}
\spanishtranslation{ñux}
\clarification{Tila}
\onedefinition{3}
\spanishtranslation{ora}
\clarification{Sab.}
\secondaryentry{inmediatamente}
\secondtranslation{bäläk'}
\clarification{Tila}

\entry{inmovilizado}
\partofspeech{adj}
\spanishtranslation{mech}

\entry{insecto}
\partofspeech{m}
\onedefinition{1}
\spanishtranslation{*chäñil pañimil}
\onedefinition{2}
\spanishtranslation{chäkäl aty}
\clarification{por ej.: zancudo}
\secondaryentry{insectos de agua}
\secondtranslation{*chäñil ja', xpampaja'}
\secondaryentry{insecto que come maíz}
\secondtranslation{chäm'al}
\secondaryentry{insecto que habita en la tierra}
\secondtranslation{xñätyechil}
\secondaryentry{insecto palo}
\secondtranslation{xtyuch' k'iñ}
\secondaryentry{insecto parecido a la avispa}
\secondtranslation{xbijlum}
\secondaryentry{insecto parecido a la lechuza}
\secondtranslation{xty'ojty'ojbak}
\secondaryentry{tipo de insecto}
\secondtranslation{\textsuperscript{2}juk}
\secondtranslation{xchäkäl aty}
\secondtranslation{xjuy'ul}
\secondtranslation{chikityíñ}
\clarification{abunda en tiempo de rozaduras}
\secondaryentry{tipo de insecto de agua}
\secondtranslation{mämäksijlel ja'}
\secondaryentry{tipo de insecto verde}
\secondtranslation{kukuch yopom}

\entry{insistentemente}
\partofspeech{adv}
\spanishtranslation{ch'ujch'uj}
\clarification{mirar}

\entry{instrucción}
\partofspeech{f}
\spanishtranslation{*käñtyesäñtyel, *tyoj'ijib}

\entry{inteligencia}
\partofspeech{f}
\spanishtranslation{*ña'tyäbal}

\entry{inteligente}
\partofspeech{adj}
\spanishtranslation{p'ip'}
\secondaryentry{eres inteligente}
\secondtranslation{añ ajol}
\clarification{Tum.}
\secondaryentry{ser inteligente}
\secondtranslation{p'ip'añ}

\entry{intención}
\partofspeech{f}
\spanishtranslation{\textsuperscript{1}ñak}
\clarification{del corazón}

\entry{interés}
\partofspeech{m}
\spanishtranslation{*jol tyak'iñ}
\clarification{de dinero}

\entry{interior}
\partofspeech{m}
\spanishtranslation{jo'ñal}
\clarification{de árbol, cuerpo}

\entry{inundar}
\partofspeech{vt}
\spanishtranslation{buty'ja'iyel}

\entry{inútil}
\partofspeech{adj}
\spanishtranslation{wojts}
\clarification{madera, tierra}

\entry{invertir}
\partofspeech{vt}
\spanishtranslation{ñukchokoñ}

\entry{inyectar}
\partofspeech{vt}
\spanishtranslation{\textsuperscript{1}joch'}

\entry{ir}
\partofspeech{vi}
\spanishtranslation{majlel, sami}
\secondaryentry{ir por primera vez para averiguar}
\secondtranslation{k'uñk'el}
\secondaryentry{fue}
\secondtranslation{\textsuperscript{1}tsajñi}
\secondaryentry{me voy}
\secondtranslation{koñix}
\secondaryentry{vamos}
\secondtranslation{koñla}
\secondaryentry{vete}
\secondtranslation{¡kuku!}
\secondaryentry{yendo}
\secondtranslation{welekña}

\entry{ixtle}
\partofspeech{m}
\spanishtranslation{chij, *chijil}

\entry{izote}
\partofspeech{m}
\spanishtranslation{koñjoyo}

\entry{jabalí}
\partofspeech{m}
\onedefinition{1}
\spanishtranslation{k'em}
\clarification{mamífero}
\onedefinition{2}
\spanishtranslation{matye' chityam}
\clarification{de collar; mamífero}

\entry{jabón}
\partofspeech{m}
\spanishtranslation{xapom}

\entry{jadeante}
\partofspeech{adj}
\onedefinition{1}
\spanishtranslation{woswosña}
\onedefinition{2}
\spanishtranslation{kojm}
\clarification{por falta de aire}

\entry{jadear}
\partofspeech{vi}
\spanishtranslation{jak' ik'}
\clarification{Sab.}
\secondaryentry{jadeando}
\secondtranslation{jäp'uña, lejlejña}

\entry{jaguar}
\partofspeech{m}
\spanishtranslation{bajlum, bo'lay, k'äñ bo'lay, xik' sajp}
\clarification{mamífero}

\entry{jaguarundi}
\partofspeech{m}
\spanishtranslation{stsukbajlum}
\clarification{mamífero}

\entry{jalar}
\partofspeech{vt}
\spanishtranslation{tyujk'añ, yu'}

\entry{jaspeado}
\partofspeech{adj}
\spanishtranslation{sälem}

\entry{jaula}
\partofspeech{f}
\secondaryentry{jaula grande}
\secondtranslation{kolotye'}
\clarification{para transportar aves}

\entry{jefe}
\partofspeech{m}
\secondaryentry{tener por jefe}
\secondtranslation{yumañ}

\entry{jengibre}
\partofspeech{m}
\spanishtranslation{kibre}
\clarification{tipo de zacate}

\entry{jicalpeste}
\partofspeech{m}
\spanishtranslation{pok'}

\entry{jícara}
\partofspeech{f}
\spanishtranslation{bux, tsima}
\secondaryentry{jícara para tortillas}
\secondtranslation{bux pok'}

\entry{jicotea}
\partofspeech{f}
\spanishtranslation{ajiñ}
\clarification{tipo de tortuga}

\entry{jilguero común}
\partofspeech{m}
\spanishtranslation{xwukpik}
\clarification{ave}

\entry{jilote}
\partofspeech{m}
\spanishtranslation{jajch}

\entry{jobo}
\partofspeech{m}
\spanishtranslation{luluy}
\clarification{ciruela; árbol}

\entry{jocotillo}
\partofspeech{m}
\onedefinition{1}
\spanishtranslation{yäxluluy}
\clarification{jobillo; árbol}
\onedefinition{2}
\spanishtranslation{ichitye'}
\clarification{Sab.; árbol}

\entry{joloche}
\partofspeech{m}
\spanishtranslation{jomojch'}
\clarification{cáscara del maíz}

\entry{jolocín}
\partofspeech{m}
\spanishtranslation{ojol}
\clarification{jonote; árbol}

\entry{jonote}
\partofspeech{m}
\onedefinition{1}
\spanishtranslation{poytye'}
\onedefinition{2}
\spanishtranslation{xwax}
\clarification{árbol}

\entry{joñón}
\spanishtranslation{joñochañ}
\clarification{insecto}

\entry{jornalero}
\partofspeech{m}
\spanishtranslation{kañar x'e'tyel}

\entry{jorobado}
\partofspeech{adj}
\spanishtranslation{k'uchu paty}

\entry{joven}
\partofspeech{m}
\spanishtranslation{alo'}
\clarification{Sab.}

\entry{judas}
\partofspeech{m}
\spanishtranslation{xiba yopom}

\entry{juego}
\partofspeech{m}
\onedefinition{1}
\spanishtranslation{alas}
\onedefinition{2}
\spanishtranslation{\textsuperscript{2}ñak}
\clarification{Sab.}

\entry{juez}
\partofspeech{m}
\secondaryentry{juez rural}
\secondtranslation{koñerol}

\entry{jugar}
\partofspeech{vt}
\spanishtranslation{mi icha'leñ alas}

\entry{jugo}
\partofspeech{m}
\spanishtranslation{\textsuperscript{1}*ya'lel}
\clarification{Sab., Tum.}

\entry{juguete}
\partofspeech{m}
\spanishtranslation{alasäl, *älas}

\entry{juguetear}
\partofspeech{vi}
\spanishtranslation{*asiñ}
\clarification{con un objeto}

\entry{julio}
\partofspeech{m}
\spanishtranslation{julio}
\clarification{Tila}

\entry{juntar}
\partofspeech{vt}
\onedefinition{1}
\spanishtranslation{tyempañ}
\onedefinition{2}
\spanishtranslation{\textsuperscript{2}*jop}
\clarification{una cosa seca}
\onedefinition{3}
\spanishtranslation{jop ochel, much'kiñ}
\clarification{frijol, café, maíz}
\onedefinition{4}
\spanishtranslation{moty, motyk'iñ}
\clarification{leña}
\onedefinition{5}
\spanishtranslation{ok'isañ}
\clarification{Sab.}
\onedefinition{6}
\spanishtranslation{xipulañ}
\clarification{ropa}
\secondaryentry{juntarse}
\secondtranslation{tyempäyel}
\spanishtranslation{chujkel}

\entry{juntos}
\partofspeech{adj}
\onedefinition{1}
\spanishtranslation{komol, tyemel}
\onedefinition{2}
\spanishtranslation{motyol}
\clarification{casas}

\entry{jurisdicción}
\partofspeech{f}
\spanishtranslation{*yumäñtyel}

\entry{justo}
\partofspeech{adj}
\spanishtranslation{tyoj}
\secondaryentry{justamente}
\secondtranslation{jaxakña}

\entry{juzgar}
\partofspeech{vt}
\spanishtranslation{mel}

\entry{la}
\partofspeech{artículo}
\spanishtranslation{jiñi}

\entry{labio}
\partofspeech{m}
\spanishtranslation{pächälel lakej}

\entry{ladear}
\partofspeech{vt}
\onedefinition{1}
\spanishtranslation{bech}
\clarification{avión}
\onedefinition{2}
\spanishtranslation{bets'}
\clarification{canoa}
\secondaryentry{ladeando}
\secondtranslation{xewuña}
\clarification{carga de mula}
\secondtranslation{\textsuperscript{1}yukiña}
\clarification{casa, árbol}

\entry{ladino}
\partofspeech{m}
\spanishtranslation{kaxlañ}

\entry{lado}
\partofspeech{m}
\spanishtranslation{*ty'ejl, *ts'ejtyäl}
\secondaryentry{lado derecho}
\spanishtranslation{\textsuperscript{2}*ñoj}
\secondaryentry{de lado}
\secondtranslation{bech, bets'el, ts'ej, ts'ejel}
\secondaryentry{de lado a lado}
\secondtranslation{ts'ejuña}
\secondaryentry{por un lado}
\secondtranslation{ñelel}

\entry{ladrar}
\partofspeech{vi}
\spanishtranslation{wojiñ}
\secondaryentry{ladrando}
\secondtranslation{wojwojña}

\entry{ladrido}
\partofspeech{m}
\spanishtranslation{woj}

\entry{ladrón}
\partofspeech{m}
\onedefinition{1}
\spanishtranslation{xujch'}
\onedefinition{2}
\spanishtranslation{ajxujch'}
\clarification{Sab.}

\entry{lagartija}
\partofspeech{f}
\secondaryentry{lagartija trepadora}
\secondtranslation{xmañchajk}
\clarification{anolis}
\secondaryentry{lagartija verde}
\secondtranslation{xyäx p'ok}

\entry{lagarto}
\partofspeech{m}
\spanishtranslation{p'ok}

\entry{lágrima}
\partofspeech{f}
\spanishtranslation{ya'lel lakwuty}

\entry{laguna}
\partofspeech{f}
\onedefinition{1}
\spanishtranslation{abañ, arayojil}
\onedefinition{2}
\spanishtranslation{petyem}
\clarification{Sab.}

\entry{lamentar}
\partofspeech{vt}
\spanishtranslation{uk'tyañ}

\entry{lamer}
\partofspeech{vt}
\spanishtranslation{lek', lem}
\clarification{perro}

\entry{lámina}
\partofspeech{f}
\spanishtranslation{*lamiñajlel}

\entry{lámpara}
\partofspeech{f}
\spanishtranslation{*yajñib k'ajk}

\entry{lana}
\partofspeech{f}
\spanishtranslation{\textsuperscript{2}tsuts, *tsutsel tyiñäme'}

\entry{langosta}
\partofspeech{f}
\spanishtranslation{meba' jijch}
\clarification{tipo que se come}
\secondaryentry{tipo de langosta}
\secondtranslation{xbäläl}
\clarification{de tierra caliente}
\secondtranslation{xk'o'k'obäläl}
\clarification{comestible, verde}

\entry{largo}
\onedefinition{1}
\partofspeech{adj}
\spanishtranslation{jältyäl, tyam}
\onedefinition{2}
\partofspeech{adj}
\spanishtranslation{päl}
\clarification{pantalón, falda}
\onedefinition{3}
\partofspeech{m}
\spanishtranslation{*kiñtyälel, *tyamlel}
\secondaryentry{palabra para describir un objeto largo}
\spanishtranslation{\textsuperscript{2}bäñ}
\secondaryentry{concuerda con todo lo largo}
\spanishtranslation{jälol}

\entry{larva}
\partofspeech{f}
\spanishtranslation{chup}
\secondaryentry{larva de mariposa}
\secondtranslation{xyäx chup}
\secondaryentry{larva de mosquito}
\spanishtranslation{xmäñäksi}

\entry{lástima}
\partofspeech{f}
\secondaryentry{tener lástima de}
\secondtranslation{p'uñtyañ}

\entry{lastimar}
\partofspeech{vt}
\spanishtranslation{low}
\secondaryentry{lastimarse}
\spanishtranslation{lojwel}
\secondaryentry{lastimado}
\secondtranslation{lojwem}

\entry{lastimoso}
\partofspeech{adj}
\spanishtranslation{p'ump'uñ}

\entry{lavar}
\partofspeech{vt}
\onedefinition{1}
\spanishtranslation{jo'}
\clarification{cabeza}
\onedefinition{2}
\spanishtranslation{pok}
\clarification{manos, cara, trastos}
\onedefinition{3}
\spanishtranslation{säk'}
\clarification{café, maíz}
\onedefinition{4}
\spanishtranslation{wuts'}
\clarification{ropa}
\onedefinition{5}
\spanishtranslation{wajyuñ}
\clarification{la cara}
\secondaryentry{acción de lavar}
\secondtranslation{säk'oñel}
\clarification{maíz}

\entry{lazar}
\partofspeech{vt}
\spanishtranslation{\textsuperscript{1}jich', kich'}

\entry{lechuza}
\partofspeech{f}
\onedefinition{1}
\spanishtranslation{xku}
\clarification{grande}
\onedefinition{2}
\spanishtranslation{xtyutyuy}
\clarification{chica}
\onedefinition{3}
\spanishtranslation{x'joch'}
\clarification{chica, de color amarillo}

\entry{leer}
\partofspeech{vt}
\spanishtranslation{\textsuperscript{1}pejkañ}
\clarification{en voz alta}

\entry{legua}
\partofspeech{vt}
\spanishtranslation{lewa}

\entry{lejos}
\partofspeech{adv}
\spanishtranslation{ñajty}

\entry{lengua}
\partofspeech{f}
\spanishtranslation{\textsuperscript{1}ak'}
\clarification{de la boca}

\entry{lento}
\partofspeech{adj}
\secondaryentry{lentamente}
\secondtranslation{ch'ujukña}

\entry{leña}
\partofspeech{f}
\spanishtranslation{si'}

\entry{leoncillo}
\partofspeech{m}
\spanishtranslation{stsukbajlum}
\clarification{jaguarundi}

\entry{leopardo}
\partofspeech{m}
\spanishtranslation{xik' sajp}
\clarification{tigre}

\entry{letra}
\partofspeech{f}
\spanishtranslation{ts'ijb}

\entry{levantar}
\partofspeech{vt}
\onedefinition{1}
\spanishtranslation{letsañ}
\onedefinition{2}
\spanishtranslation{ch'uy}
\clarification{cosa pesada}
\onedefinition{3}
\spanishtranslation{k'echulañ}
\clarification{pie}
\onedefinition{4}
\spanishtranslation{tyech}
\clarification{cama, piedra, palo}
\onedefinition{5}
\spanishtranslation{wa'chokoñ}
\clarification{casa}
\secondaryentry{levantarse}
\secondtranslation{ch'ojiyel, tyejchel}
\secondtranslation{ch'ujyel}
\clarification{cosa pesada}
\secondaryentry{levantado}
\secondtranslation{ch'uybil}
\clarification{cosa pesada}
\secondtranslation{k'ichil}
\clarification{pie}

\entry{ley}
\partofspeech{f}
\spanishtranslation{mañdar}

\entry{libélula}
\partofspeech{f}
\spanishtranslation{tyujlux}
\clarification{caballito del diablo; insecto}

\entry{libro}
\partofspeech{m}
\spanishtranslation{\textsuperscript{1}juñ}

\entry{ligero}
\partofspeech{adv}
\spanishtranslation{ora}
\clarification{Sab.}

\entry{lima}
\partofspeech{f}
\spanishtranslation{rima}
\clarification{para afilar}

\entry{limpiar}
\partofspeech{vt}
\onedefinition{1}
\spanishtranslation{sujkuñ}
\clarification{cara, zapatos}
\onedefinition{2}
\spanishtranslation{yäk'ñañ, ak'ñañ}
\clarification{milpa, cafetal}
\secondaryentry{limpiarse}
\secondtranslation{ak'ñäñtyel}

\entry{limpieza}
\partofspeech{f}
\spanishtranslation{ak'iñ}
\clarification{milpa, cafetal}

\entry{limpio}
\partofspeech{adj}
\onedefinition{1}
\spanishtranslation{säk}
\onedefinition{2}
\spanishtranslation{ak'äl}
\clarification{de vegetación}
\onedefinition{3}
\spanishtranslation{ak'ñibil}
\clarification{sembrado}
\onedefinition{4}
\spanishtranslation{chäk}
\clarification{un camino}
\onedefinition{5}
\spanishtranslation{chäkkolañ}
\clarification{de monte}
\onedefinition{6}
\spanishtranslation{\textsuperscript{2}säkpochañ}
\clarification{camisa}
\onedefinition{7}
\spanishtranslation{säkpojañ}
\clarification{río, cara}
\onedefinition{8}
\spanishtranslation{säkpoyañ}
\clarification{Tila; casa, cuarto}
\onedefinition{9}
\spanishtranslation{yäx}
\clarification{agua}
\secondaryentry{limpio todavía}
\secondtranslation{säktyo}

\entry{lindero}
\partofspeech{m}
\spanishtranslation{ñup'}

\entry{liquidámbar}
\partofspeech{m}
\spanishtranslation{suts'tye'}
\clarification{árbol}
\secondaryentry{arboleda de liquidámbar}
\secondtranslation{suts'tye'ol}

\entry{líquido}
\partofspeech{m}
\spanishtranslation{\textsuperscript{1}*ya'lel}

\entry{liso}
\partofspeech{adj}
\onedefinition{1}
\spanishtranslation{ts'ayakña, ulukña}
\onedefinition{2}
\spanishtranslation{piyikña}
\clarification{brilloso}
\secondaryentry{brilloso y liso}
\secondtranslation{elekña}

\entry{listo}
\partofspeech{adj}
\spanishtranslation{bibu}
\clarification{persona}

\entry{liviano}
\partofspeech{adj}
\spanishtranslation{sejb}
\secondaryentry{hacerse liviano}
\secondtranslation{sejb'añ}

\entry{llaga}
\partofspeech{f}
\spanishtranslation{tsoy}

\entry{llagar}
\partofspeech{vt}
\spanishtranslation{tsoy'ajel}

\entry{llama}
\partofspeech{f}
\spanishtranslation{*k'äk'al, *yaty k'ajk}
\clarification{de fuego}

\entry{llamada}
\partofspeech{f}
\secondaryentry{llamada que se hace al gato}
\secondtranslation{mix}

\entry{llamador}
\partofspeech{m}
\spanishtranslation{xpäyoñel}

\entry{llamar}
\partofspeech{vt}
\onedefinition{1}
\spanishtranslation{päy}
\onedefinition{2}
\spanishtranslation{lich'k'äbañ}
\clarification{señalando}
\onedefinition{3}
\spanishtranslation{\textsuperscript{2}mixuñ}
\clarification{gato}
\onedefinition{4}
\spanishtranslation{tyuxbañ}
\clarification{aves de corral}

\entry{llanto}
\partofspeech{m}
\onedefinition{1}
\spanishtranslation{uk'el}
\onedefinition{2}
\spanishtranslation{ch'e'lel}
\clarification{del niño}
\onedefinition{3}
\spanishtranslation{bajk'el}
\clarification{en extremo}

\entry{llave}
\partofspeech{f}
\spanishtranslation{*yabejlel}

\entry{llegar}
\partofspeech{vi}
\onedefinition{1}
\spanishtranslation{ajñel}
\clarification{a una parte}
\onedefinition{2}
\spanishtranslation{k'äjkel}
\clarification{a la cumbre}
\onedefinition{3}
\spanishtranslation{k'otyel}
\clarification{allá}
\onedefinition{4}
\spanishtranslation{julel}
\clarification{acá}
\onedefinition{5}
\spanishtranslation{yajñel}
\clarification{siempre}
\secondaryentry{llegar agarrado}
\secondtranslation{chuk majlel}
\secondaryentry{llegado}
\secondtranslation{k'otyem}
\clarification{allá}

\entry{llenar}
\partofspeech{vt}
\spanishtranslation{buty'}
\secondaryentry{llenarse}
\secondtranslation{ñaj'añ}
\clarification{de comida}

\entry{lleno}
\partofspeech{adj}
\onedefinition{1}
\spanishtranslation{buty'ul}
\onedefinition{2}
\spanishtranslation{buty'ukña}
\clarification{de gente}
\onedefinition{3}
\spanishtranslation{pamakña}
\clarification{bien}

\entry{llevar}
\partofspeech{vt}
\onedefinition{1}
\spanishtranslation{päy majlel}
\onedefinition{2}
\spanishtranslation{bejlañ}
\clarification{varios viajes}
\onedefinition{3}
\spanishtranslation{ch'äm}
\clarification{tomar}
\onedefinition{4}
\spanishtranslation{jich'ye'}
\clarification{colgado}
\onedefinition{5}
\spanishtranslation{ye', ch'äm majlel}
\clarification{en la mano}
\onedefinition{6}
\spanishtranslation{\textsuperscript{2}tsuy}
\clarification{fuego}
\secondaryentry{llevado}
\secondtranslation{kujchel}
\clarification{sobre la espalda}

\entry{llorar}
\partofspeech{vi}
\spanishtranslation{uk'el, mi icha'leñ uk'el}

\entry{lloriquear}
\partofspeech{vi}
\secondaryentry{lloriqueando}
\secondtranslation{jäk'jäk'ña}

\entry{llover}
\partofspeech{vi}
\spanishtranslation{mi icha'leñ ja'al}

\entry{lloviznar}
\partofspeech{vi}
\secondaryentry{lloviznando}
\secondtranslation{musmusña}

\entry{lluvia}
\partofspeech{f}
\spanishtranslation{ja'al}
\secondaryentry{lluvia recia}
\secondtranslation{k'am ja'al}

\entry{lluvioso}
\partofspeech{adj}
\spanishtranslation{weswesña}

\entry{lobanillo}
\partofspeech{m}
\onedefinition{1}
\spanishtranslation{\textsuperscript{1}bulul}
\clarification{con lobanillo}
\onedefinition{2}
\spanishtranslation{bulul ibik'}
\clarification{en el cuello}
\secondaryentry{hombre con lobanillo}
\secondtranslation{xbulubik'}

\entry{lóbulo}
\partofspeech{m}
\spanishtranslation{*k'uñel}

\entry{loco}
\partofspeech{adj}
\spanishtranslation{añ ajol}
\clarification{Sab.}

\entry{lodazal}
\partofspeech{m}
\spanishtranslation{ok'lel}

\entry{lodo}
\partofspeech{m}
\spanishtranslation{ok'ol, *yok'liyel}

\entry{loma}
\partofspeech{f}
\onedefinition{1}
\spanishtranslation{bujtyäl}
\onedefinition{2}
\spanishtranslation{ch'äktyäl}
\clarification{sirve como mirador}

\entry{lombriz}
\partofspeech{f}
\spanishtranslation{ch'ox}

\entry{lomo}
\partofspeech{m}
\spanishtranslation{*lomojlel ipaty}
\clarification{de animal}

\entry{loro}
\partofspeech{m}
\secondaryentry{loro verde}
\secondtranslation{ujrich'}
\clarification{loro de cabeza azul; ave}

\entry{luciérnaga}
\partofspeech{f}
\spanishtranslation{k'äjk'äs, xk'äjk'äs, xmäjmäs}
\clarification{insecto}

\entry{luego}
\partofspeech{adv}
\onedefinition{1}
\spanishtranslation{\textsuperscript{2}bäk', ñuñ, tyi ora}
\onedefinition{2}
\spanishtranslation{ñux}
\clarification{Tila}
\onedefinition{3}
\spanishtranslation{ora, saj ora, wa'}
\clarification{Sab.}
\secondaryentry{vete luego}
\secondtranslation{¡kukuñuñ!}

\entry{lugar}
\partofspeech{m}
\spanishtranslation{ajñibäl}
\secondaryentry{lugar donde hay muchas matas de cacaté}
\secondtranslation{*käkätye'ol}
\secondaryentry{lugar de descanso}
\secondtranslation{k'ajo'o'}
\clarification{para la noche}
\secondaryentry{lugar donde hay mucho bambú}
\secondtranslation{chejboy, ch'ijbol}
\secondaryentry{lugar donde hay muchas espinas}
\secondtranslation{ch'ixol}
\secondaryentry{lugar arenoso}
\secondtranslation{ji'lumil}
\secondaryentry{lugar de entierro}
\spanishtranslation{mukoñibäl, mujkibäl}
\secondaryentry{lugar encima de una cueva}
\secondtranslation{pañch'eñ}
\secondaryentry{lugar alto}
\secondtranslation{k'eloñib}
\secondaryentry{lugar para ver un paisaje}
\secondtranslation{k'elo'pañimil}
\secondaryentry{lugar de camarones}
\secondtranslation{xexkokil}

\entry{Lugar de Bambú a la Orilla del Agua}
\spanishtranslation{Chejopa'}
\clarification{Tila; ranchería}

\entry{Lugar de Bastante Pescado}
\spanishtranslation{Ma'chäyil}
\clarification{rancho}

\entry{Lugar de la Piedra Aquí}
\spanishtranslation{Tyumbalá}
\clarification{pueblo}

\entry{Lugar Donde se Cortan Flores}
\spanishtranslation{Tyuk'oñichim}
\clarification{Tila; colonia}

\entry{lumbre}
\partofspeech{f}
\spanishtranslation{tyik'äjib}

\entry{luna}
\partofspeech{f}
\onedefinition{1}
\spanishtranslation{uw}
\onedefinition{2}
\spanishtranslation{ch'ujuña'}
\clarification{lit.: madre santa}
\secondaryentry{luna llena}
\secondtranslation{pomol uw}
\secondaryentry{luna en cuarto menguante}
\secondtranslation{xiñ pañchañ uw}
\secondaryentry{rayos de luna}
\secondtranslation{ixojob uw}

\entry{lunar del ojo}
\spanishtranslation{päk'il iwuty}

\entry{luz}
\partofspeech{f}
\spanishtranslation{k'ajk, *säklel}
\secondaryentry{hay luz todavía}
\secondtranslation{säktyo}

\entry{macana}
\partofspeech{f}
\spanishtranslation{päk'ojib}

\entry{machete}
\partofspeech{m}
\spanishtranslation{machity}

\entry{macho}
\partofspeech{m}
\spanishtranslation{ityaty}
\clarification{de animales}

\entry{machucar}
\partofspeech{vt}
\spanishtranslation{k'uty, k'utyilañ}
\clarification{chile}

\entry{macizo}
\partofspeech{adj}
\spanishtranslation{ch'äjy}

\entry{madera}
\partofspeech{vt}
\spanishtranslation{tye'}

\entry{madrastra}
\partofspeech{f}
\spanishtranslation{*majañ ña'}

\entry{madre}
\partofspeech{f}
\spanishtranslation{ña'}
\secondaryentry{madre de cacao}
\secondtranslation{xchañtye'}
\clarification{árbol}

\entry{madrina}
\partofspeech{f}
\spanishtranslation{jalaña'}

\entry{madrugada}
\partofspeech{f}
\spanishtranslation{ik'tyo, ik'atyax}

\entry{madurar}
\partofspeech{vi}
\onedefinition{1}
\spanishtranslation{k'äñ'añ, ñejp'añ, ñejp'äyel}
\onedefinition{2}
\spanishtranslation{*k'äñajel}
\clarification{ponerse amarillo}
\onedefinition{3}
\spanishtranslation{chäk'añ}
\clarification{café}

\entry{maduro}
\onedefinition{1}
\partofspeech{adj}
\spanishtranslation{*k'äñel, k'äñ}
\onedefinition{2}
\spanishtranslation{chäkix}
\clarification{café}

\entry{maestro}
\partofspeech{m}
\spanishtranslation{maestyru, xkäñtyesa}

\entry{mafafa}
\partofspeech{f}
\spanishtranslation{\textsuperscript{2}juk', me'uñ}
\clarification{quequexte; planta}

\entry{maíz}
\partofspeech{m}
\spanishtranslation{ixim}
\secondaryentry{maíz amarillo}
\secondtranslation{k'añal}
\secondaryentry{maíz blanco}
\secondtranslation{säk waj}
\secondaryentry{maíz negro}
\secondtranslation{chäkchab, xchäk chab ixim}
\secondtranslation{yaxum}
\clarification{Sab.}
\secondaryentry{maíz picado}
\secondtranslation{\textsuperscript{3}joch'}
\secondaryentry{maíz podrido}
\secondtranslation{*jomil}

\entry{malacate}
\partofspeech{m}
\spanishtranslation{ñok'ijibäl}

\entry{malamujer}
\partofspeech{f}
\spanishtranslation{ixtye', x'ek'}
\clarification{ortiga; planta}

\entry{maldad}
\partofspeech{f}
\onedefinition{1}
\spanishtranslation{*kolosojlel, mulil}
\onedefinition{2}
\spanishtranslation{*simaroñiyel}
\clarification{Sab.}
\secondaryentry{su maldad}
\secondtranslation{*joñtyolel, *joñtyolil}

\entry{maldecir}
\onedefinition{1}
\partofspeech{vt}
\spanishtranslation{ch'äk, p'aj, ty'ojtyi'iñ}
\onedefinition{2}
\partofspeech{vi}
\spanishtranslation{ch'äkojel}

\entry{malo}
\partofspeech{adj}
\onedefinition{1}
\spanishtranslation{joñtyol}
\clarification{Sab.}
\onedefinition{2}
\spanishtranslation{simaroñ}
\clarification{Tila}
\secondaryentry{mal de ojo}
\secondtranslation{yats'}
\secondaryentry{mal olor}
\secondtranslation{leko iyujts'il}
\secondaryentry{mala vida}
\secondtranslation{*tyoñtyojlel}

\entry{mamá}
\partofspeech{f}
\spanishtranslation{ñaña}

\entry{mamar}
\partofspeech{vt}
\spanishtranslation{chu'uñ, cha'leñ chu'}
\secondaryentry{dar de mamar}
\secondtranslation{tsu'sañ}

\entry{mamífero}
\partofspeech{m}
\spanishtranslation{bätye'el}
\clarification{silvestre}

\entry{manaca}
\partofspeech{f}
\spanishtranslation{koroso}
\clarification{Tila; palma}

\entry{manchar}
\partofspeech{vt}
\onedefinition{1}
\spanishtranslation{\textsuperscript{2}päk'}
\onedefinition{2}
\spanishtranslation{xaxañ}
\clarification{extendiéndose}
\secondaryentry{mancharse}
\secondtranslation{\textsuperscript{2}päjk'el}
\secondaryentry{el que mancha}
\secondtranslation{xpäk'oñel}
\secondaryentry{manchado}
\secondtranslation{tya'tya'}
\secondtranslation{ik'selañ}
\clarification{con mancha redonda y negra; p. ej. de carbón, marca de nacimiento}

\entry{mancolón}
\partofspeech{m}
\spanishtranslation{xchäläl}
\clarification{gallina de monte; ave}

\entry{mango}
\partofspeech{m}
\spanishtranslation{mañko}
\clarification{fruta}

\entry{mano}
\partofspeech{f}
\spanishtranslation{k'äbäl}
\secondaryentry{mano de metate}
\secondtranslation{k'ä'tyuñ}
\secondaryentry{mano izquierda}
\secondtranslation{*ts'ej}
\secondaryentry{con la mano extendida}
\secondtranslation{tyich'ikña}

\entry{manojo}
\partofspeech{m}
\spanishtranslation{bombom jam}
\clarification{de zacate}

\entry{manosear}
\partofspeech{vt}
\spanishtranslation{pik'xuñ}

\entry{manteca}
\partofspeech{f}
\spanishtranslation{lew}

\entry{mantel}
\partofspeech{m}
\spanishtranslation{*tyasil}

\entry{mantener}
\partofspeech{vt}
\spanishtranslation{mäk'lañ}

\entry{mañana}
\partofspeech{f}
\spanishtranslation{ijk'äl}

\entry{mañoso}
\partofspeech{adj}
\spanishtranslation{maña}

\entry{mapache}
\partofspeech{m}
\spanishtranslation{ejmech}
\clarification{mamífero}

\entry{mar}
\partofspeech{m}
\spanishtranslation{kolem abal, ñajb, kolem ja'}

\entry{maraca}
\partofspeech{f}
\spanishtranslation{chikix}

\entry{marca}
\partofspeech{f}
\onedefinition{1}
\spanishtranslation{*yejtyal}
\onedefinition{2}
\spanishtranslation{ik'selañ}
\clarification{de nacimiento}

\entry{marcado}
\partofspeech{adj}
\spanishtranslation{japal}
\clarification{con cicatriz}

\entry{marear}
\partofspeech{vi}
\secondaryentry{mareado}
\secondtranslation{jämjämña}

\entry{mariposa}
\partofspeech{f}
\spanishtranslation{pejpem}
\secondaryentry{mariposa gavilana}
\secondtranslation{kolem pejpem}
\secondaryentry{tipo de mariposa}
\spanishtranslation{majmaj ul}
\clarification{chica; sale en mayo}
\secondaryentry{mariposa del comején}
\spanishtranslation{sulup}

\entry{martillo}
\partofspeech{m}
\spanishtranslation{ch'ijoñib, ch'ijo'lawux}

\entry{más}
\partofspeech{adv}
\spanishtranslation{bej}
\secondaryentry{así nada más}
\secondtranslation{chejachi, che' jach}

\entry{masa}
\partofspeech{f}
\spanishtranslation{sa'}

\entry{masticar}
\partofspeech{vt}
\spanishtranslation{jach'}

\entry{mástil}
\partofspeech{m}
\spanishtranslation{ts'omtye'}
\clarification{de una casa}

\entry{matapalo}
\partofspeech{m}
\spanishtranslation{tsu'um}
\clarification{amate, higuero; árbol}

\entry{matar}
\onedefinition{1}
\partofspeech{vt}
\spanishtranslation{jisañ, tsäñsañ}
\onedefinition{2}
\partofspeech{vi}
\spanishtranslation{tsäñsäñtyel}

\entry{matilisguate}
\partofspeech{m}
\spanishtranslation{makulis}
\clarification{palo de rosa; árbol}

\entry{matriz}
\partofspeech{vt}
\spanishtranslation{*chu'yib iyal}

\entry{mayordomo}
\partofspeech{m}
\secondaryentry{mayordomo en la iglesia}
\secondtranslation{ch'ujwañaj}
\clarification{Tila}

\entry{mazorca}
\partofspeech{f}
\spanishtranslation{bojlox}
\clarification{con pocos granos}

\entry{mecapal}
\partofspeech{m}
\onedefinition{1}
\spanishtranslation{tyajbal}
\onedefinition{2}
\spanishtranslation{tyajm}
\clarification{Sab.}
\onedefinition{3}
\spanishtranslation{pejk'}
\clarification{de cuero}

\entry{mecer}
\onedefinition{1}
\partofspeech{vt}
\spanishtranslation{jäjmañ, yäjmañ}
\clarification{en una hamaca}
\onedefinition{2}
\partofspeech{vt}
\spanishtranslation{jäjmesañ}
\clarification{criatura}
\onedefinition{3}
\partofspeech{vt}
\spanishtranslation{jämts'uñ, \textsuperscript{1}wejluñ}
\clarification{animal, objeto}
\onedefinition{4}
\partofspeech{vi}
\spanishtranslation{jäjmel}
\clarification{en una hamaca}
\secondaryentry{meciendo}
\secondtranslation{jämuña}

\entry{medicina}
\partofspeech{f}
\spanishtranslation{ts'ak, *ts'äkal}

\entry{medida}
\partofspeech{f}
\onedefinition{1}
\spanishtranslation{*p'isol}
\onedefinition{2}
\spanishtranslation{*xotytyilel}
\clarification{de un cafetal, potrero, casa}

\entry{medidor}
\partofspeech{m}
\spanishtranslation{*p'isoñib}

\entry{medio}
\partofspeech{adj}
\spanishtranslation{chäkkojañ}
\secondaryentry{en medio}
\secondtranslation{ximal, xiñ, xiñil}
\clarification{Sab.}
\secondaryentry{medio asar}
\secondtranslation{ch'ajtyañ}

\entry{mediodía}
\partofspeech{m}
\spanishtranslation{xiñk'iñil}

\entry{medir}
\partofspeech{vt}
\onedefinition{1}
\spanishtranslation{p'is}
\onedefinition{2}
\spanishtranslation{jajlañ}
\clarification{con brazadas}
\onedefinition{3}
\spanishtranslation{p'isbeñtyel}
\clarification{terreno}
\secondaryentry{medir con la cuarta de la mano}
\secondtranslation{ñajbañ}

\entry{mejilla}
\partofspeech{f}
\spanishtranslation{*choj}

\entry{mejor}
\partofspeech{adj}
\spanishtranslation{yom}
\clarification{de salud}
\secondaryentry{poco mejor}
\secondtranslation{k'uñche'ix}
\clarification{persona}

\entry{mejorar}
\partofspeech{vi}
\spanishtranslation{ty'ojläwel}

\entry{memela}
\partofspeech{f}
\spanishtranslation{bu'le waj}

\entry{menear}
\partofspeech{vt}
\spanishtranslation{k'ak xäñ}
\clarification{Sab.}
\secondaryentry{meneando}
\secondtranslation{luk'luk'ña}
\clarification{puente}
\secondtranslation{weluña}
\clarification{trapo}
\secondtranslation{\textsuperscript{2}yukiña}
\clarification{casa}

\entry{menor}
\partofspeech{m}
\spanishtranslation{tyak'}
\clarification{persona}

\entry{menos}
\partofspeech{adv}
\spanishtranslation{kojko}

\entry{mensajero}
\partofspeech{m}
\onedefinition{1}
\spanishtranslation{ak'juñ, suboñel}
\onedefinition{2}
\spanishtranslation{x'ak'juñ}
\clarification{por carta}

\entry{mentira}
\partofspeech{vt}
\spanishtranslation{\textsuperscript{1}loty}

\entry{mentiroso}
\partofspeech{m}
\spanishtranslation{xloty}

\entry{mentón}
\partofspeech{m}
\spanishtranslation{*xäk'tyi'}

\entry{menudo}
\partofspeech{adj}
\spanishtranslation{tyeñe}

\entry{meñique}
\partofspeech{m}
\spanishtranslation{*yal ik'äb}

\entry{mercancía}
\partofspeech{f}
\spanishtranslation{*p'olmäjel}

\entry{mes}
\partofspeech{m}
\spanishtranslation{tsik, uw}

\entry{mesa}
\partofspeech{f}
\spanishtranslation{*yeklib ibäl ñäk'äl}
\clarification{para comer}
\secondaryentry{mesita}
\secondtranslation{we'tye'}
\clarification{para comer}

\entry{metal}
\partofspeech{m}
\spanishtranslation{tsukul tyak'iñ, tsukutyak'iñ}

\entry{metate}
\partofspeech{m}
\spanishtranslation{juch'oñibäl, ña'atyuñ}

\entry{meteoro}
\partofspeech{m}
\spanishtranslation{tya'ek'}
\clarification{lit.: excremento de estrella}

\entry{meter}
\partofspeech{vt}
\onedefinition{1}
\spanishtranslation{otsañ}
\onedefinition{2}
\spanishtranslation{\textsuperscript{2}baj}
\clarification{Sab.}
\onedefinition{3}
\spanishtranslation{ch'ik}
\clarification{instrumento pequeño en un agujero}
\onedefinition{4}
\spanishtranslation{xuty' ochel}
\clarification{por pedazos}

\entry{metro}
\partofspeech{m}
\spanishtranslation{*p'isoñib}

\entry{mezclar}
\partofspeech{vt}
\onedefinition{1}
\spanishtranslation{xäb, walk'uñ}
\clarification{maíz con frijol}
\onedefinition{2}
\spanishtranslation{xäk'}
\clarification{arena, cal, cemento}
\onedefinition{3}
\spanishtranslation{walts'uñ}
\clarification{con condimento}

\entry{mico}
\partofspeech{m}
\spanishtranslation{max}
\secondaryentry{mico negro}
\secondtranslation{i'ik' max}
\secondaryentry{mico de noche}
\secondtranslation{k'äñk'äñ max, uyuj}
\clarification{mamífero}

\entry{miedo}
\partofspeech{m}
\spanishtranslation{bäk'eñ}

\entry{miel}
\partofspeech{f}
\spanishtranslation{chab}
\secondaryentry{miel colorada}
\secondtranslation{chäkchab}

\entry{miércoles}
\partofspeech{m}
\spanishtranslation{uxp'ejk'iñ}

\entry{migaja}
\partofspeech{vt}
\spanishtranslation{*sajl}

\entry{milpa}
\partofspeech{f}
\spanishtranslation{cholel}

\entry{mirada}
\partofspeech{vt}
\spanishtranslation{chañäl}

\entry{miradero}
\partofspeech{m}
\spanishtranslation{wa'lib}
\clarification{donde el cazador espera la caza}

\entry{mirar}
\partofspeech{vt}
\spanishtranslation{chäñtyañ}
\clarification{Sab.}
\secondaryentry{mira}
\secondtranslation{¡uñ tsa'!}
\secondaryentry{mira no más}
\secondtranslation{ujax}

\entry{mirón}
\partofspeech{m}
\spanishtranslation{xchañäl}
\secondaryentry{persona mirando al cielo}
\secondtranslation{pañchañ wuty}

\entry{misa}
\partofspeech{f}
\spanishtranslation{ch'uyijel}
\clarification{Tila}

\entry{mismo}
\partofspeech{adj}
\spanishtranslation{wäläk}

\entry{mitad}
\partofspeech{f}
\onedefinition{1}
\spanishtranslation{ojlil, xiñol}
\onedefinition{2}
\spanishtranslation{xujty'om}
\clarification{vela}
\onedefinition{3}
\spanishtranslation{lamityal}
\clarification{Sab.}

\entry{moco}
\partofspeech{m}
\spanishtranslation{tya'ñi'}

\entry{mococha}
\partofspeech{m, f}
\spanishtranslation{buk'tsu'}
\clarification{nauyaca saltadora}

\entry{moho}
\partofspeech{m}
\spanishtranslation{*kuxel}

\entry{mojar}
\partofspeech{vt}
\spanishtranslation{\textsuperscript{1}mul, wijts'añ}
\secondaryentry{mojarse}
\secondtranslation{ach'añ}
\secondaryentry{mojado}
\secondtranslation{ach', ach'esäbil}
\secondaryentry{muy mojado}
\secondtranslation{\textsuperscript{1}chijlaw}
\clarification{por el rocío}

\entry{mojarra}
\partofspeech{f}
\spanishtranslation{ik'chäy}
\clarification{pez negro}
\secondaryentry{mojarra jaspeada}
\secondtranslation{x'ik'chäy}
\clarification{pez}

\entry{moju}
\partofspeech{m}
\spanishtranslation{ax}
\clarification{pan de nuez; árbol}

\entry{molcate}
\partofspeech{m}
\spanishtranslation{bik'tyal}
\clarification{mazorca pequeña de maíz}

\entry{molendera}
\partofspeech{f}
\spanishtranslation{xmel waj}

\entry{moler}
\partofspeech{vt}
\onedefinition{1}
\spanishtranslation{kes}
\clarification{nixtamal}
\onedefinition{2}
\spanishtranslation{juch'}
\clarification{maíz, café}
\secondaryentry{molido}
\secondtranslation{ts'ubukña}
\clarification{fino}
\secondtranslation{ts'äbäkña}
\clarification{granos finos o remolidos de arena, azúcar}
\secondtranslation{juch'bil}
\clarification{maíz, café}

\entry{molestar}
\partofspeech{vt}
\onedefinition{1}
\spanishtranslation{tyik'lañ}
\onedefinition{2}
\spanishtranslation{wolts'iñ}
\clarification{con palabras}

\entry{molleja}
\partofspeech{f}
\spanishtranslation{*sos muty}

\entry{mollera}
\partofspeech{f}
\spanishtranslation{\textsuperscript{2}*jaylel}

\entry{momento}
\partofspeech{m}
\secondaryentry{de momento}
\secondtranslation{\textsuperscript{1}tsäy}

\entry{moneda}
\partofspeech{f}
\spanishtranslation{tyak'iñ}

\entry{mono}
\partofspeech{m}
\spanishtranslation{bats'}
\secondaryentry{mono araña}
\secondtranslation{ijk'al max}

\entry{montañoso}
\partofspeech{adj}
\secondaryentry{región montañosa}
\secondtranslation{*witsilel}

\entry{montar}
\partofspeech{vt}
\spanishtranslation{k'ächtyañ}
\secondaryentry{montar a}
\secondtranslation{k'ächchokoñ}

\entry{monte}
\partofspeech{m}
\secondaryentry{monte levantado}
\secondtranslation{kolol}

\entry{montón}
\partofspeech{m}
\spanishtranslation{busul}
\clarification{de tierra}

\entry{montura}
\partofspeech{f}
\spanishtranslation{*siyajlel}

\entry{mora}
\partofspeech{f}
\spanishtranslation{makom}

\entry{morado}
\partofspeech{adj}
\spanishtranslation{yäxmojañ}
\secondaryentry{ponerse morado}
\secondtranslation{yäx'añ}

\entry{morar}
\partofspeech{vi}
\spanishtranslation{yajñel}

\entry{morder}
\partofspeech{vt}
\onedefinition{1}
\spanishtranslation{\textsuperscript{1}k'ux}
\onedefinition{2}
\spanishtranslation{ch'oj}
\clarification{culebra}
\secondaryentry{mordido}
\spanishtranslation{k'uxul}

\entry{morir}
\partofspeech{vi}
\onedefinition{1}
\spanishtranslation{chämel}
\onedefinition{2}
\spanishtranslation{sajtyel}
\clarification{Tila}
\secondaryentry{estar a punto de morir}
\secondtranslation{chämeläyel}

\entry{morona}
\partofspeech{f}
\spanishtranslation{*ts'ubil}
\clarification{migajas de galletas}

\entry{morro}
\partofspeech{m}
\spanishtranslation{stsimajtye'}
\clarification{cuautecomate, árbol}

\entry{mortaja}
\partofspeech{f}
\spanishtranslation{*bäjk'il}

\entry{mosca}
\partofspeech{f}
\spanishtranslation{jaj, us}
\secondaryentry{mosca verde}
\secondtranslation{yäx jaj}

\entry{mosquera tijereta}
\spanishtranslation{x'ajlum}
\clarification{ave}

\entry{mostacilla}
\partofspeech{f}
\spanishtranslation{xch'asip}
\clarification{garrapata chica}

\entry{mostaza}
\partofspeech{f}
\spanishtranslation{xkulix}

\entry{mostrar}
\partofspeech{vt}
\spanishtranslation{päs}

\entry{mover}
\partofspeech{vt}
\onedefinition{1}
\spanishtranslation{ñijkañ, ñijkel}
\onedefinition{2}
\spanishtranslation{chijulañ}
\clarification{en un balde; maíz, frijol}
\onedefinition{3}
\spanishtranslation{juyts'iñ}
\clarification{atole, pinole, maíz cocido}
\onedefinition{4}
\spanishtranslation{lämulañ}
\clarification{líquido}
\onedefinition{6}
\spanishtranslation{päñts'uñ}
\clarification{mano, machete, palo}
\secondaryentry{mover con palo}
\secondtranslation{juy}
\clarification{atole, pinole}
\secondaryentry{manera de mover un bulto pesado}
\secondtranslation{wotsokña}
\secondaryentry{moviendo}
\secondtranslation{buchbuchña, buchiña}
\secondtranslation{\textsuperscript{2}balakña}
\clarification{como buena milpa}
\secondtranslation{xotyokña}
\clarification{en un círculo}

\entry{movimiento}
\partofspeech{m}
\onedefinition{1}
\spanishtranslation{le'}
\clarification{de mano para abrir costal}
\onedefinition{2}
\spanishtranslation{kajkaña}
\clarification{trastornado}

\entry{muchacho}
\partofspeech{m}
\onedefinition{1}
\spanishtranslation{ch'ityoñ}
\clarification{de siete hasta quince años}
\onedefinition{2}
\spanishtranslation{alo'}
\clarification{Sab.}

\entry{mucho}
\onedefinition{1}
\partofspeech{adj}
\spanishtranslation{bajk'äl, kabäl, ts'iwil}
\onedefinition{2}
\partofspeech{adj}
\spanishtranslation{joboñ}
\clarification{Tila}
\onedefinition{3}
\partofspeech{adv}
\spanishtranslation{bäjñel, putyuñ}
\onedefinition{4}
\partofspeech{adv}
\spanishtranslation{\textsuperscript{2}jal}
\clarification{tiempo}

\entry{mudar}
\partofspeech{vt}
\secondaryentry{acción de mudar}
\secondtranslation{k'exoñel}

\entry{mudo}
\partofspeech{m}
\spanishtranslation{uma', x'uma'}

\entry{muela}
\partofspeech{f}
\spanishtranslation{cha'am}

\entry{muerto}
\partofspeech{adj}
\onedefinition{1}
\spanishtranslation{chämeñ}
\onedefinition{2}
\spanishtranslation{sajtyem}
\clarification{Tila}

\entry{muesca}
\partofspeech{f}
\secondaryentry{hacer muescas}
\secondtranslation{kepuñ}

\entry{mugroso}
\partofspeech{adj}
\spanishtranslation{kuxeñtyik}

\entry{mujer}
\partofspeech{f}
\spanishtranslation{x'ixik}
\secondaryentry{mujer latina}
\secondtranslation{xiñolaj}
\clarification{que no es indígena}
\secondaryentry{mujer de edad}
\secondtranslation{xñejep'}
\secondaryentry{mujer de pelo largo}
\secondtranslation{xtyamijol}
\secondaryentry{mujer tzeltal}
\secondtranslation{x'amiku}

\entry{mula}
\partofspeech{f}
\spanishtranslation{k'ächlibäl}

\entry{mulato}
\partofspeech{m}
\spanishtranslation{chäkajl}
\clarification{copal; árbol}

\entry{mullir}
\partofspeech{vt}
\spanishtranslation{wotsilañ}
\clarification{cáscara de frijol}
\secondaryentry{mullido}
\secondtranslation{wotsol}

\entry{mundo}
\partofspeech{m}
\onedefinition{1}
\spanishtranslation{pañimil}
\onedefinition{2}
\spanishtranslation{mulawil}
\clarification{Sab., Tila}

\entry{murciélago}
\partofspeech{m}
\spanishtranslation{suts'}

\entry{murmullo}
\partofspeech{m}
\secondaryentry{con murmullo}
\secondtranslation{wulwulña}

\entry{murmuración}
\partofspeech{f}
\spanishtranslation{wälwäl ty'añ}

\entry{musgo}
\partofspeech{m}
\secondaryentry{tipo de musgo}
\secondtranslation{sujlum}
\clarification{brota de la tierra}
\secondtranslation{*tsuñtye'lel}
\clarification{verde oscuro; de árboles}
\secondaryentry{musgo verde}
\secondtranslation{ch'i'omal}
\clarification{de río y de árboles}

\entry{muslo}
\partofspeech{m}
\spanishtranslation{*tyomel, *ya'}

\entry{mútila mora}
\spanishtranslation{xk'ajk'äy aty}
\clarification{insecto}

\entry{nacer}
\partofspeech{vi}
\onedefinition{1}
\spanishtranslation{*ilañ pañimil, k'el pañimil}
\onedefinition{2}
\spanishtranslation{ch'ok'añ}
\clarification{Sab., Tila}

\entry{nada}
\partofspeech{pron}
\secondaryentry{así nada más}
\secondtranslation{chejachi, che' jach}
\secondaryentry{de nada}
\secondtranslation{k'ebtyo}
\clarification{respuesta}
\secondtranslation{k'etyo sajl}
\clarification{Sab.}

\entry{nadar}
\partofspeech{vi}
\spanishtranslation{ñuxijel}
\secondaryentry{nadando}
\secondtranslation{ñuxukña}

\entry{nalga}
\partofspeech{f}
\onedefinition{1}
\spanishtranslation{*choj'ity}
\onedefinition{2}
\spanishtranslation{kolo'ity, ñuchil}
\clarification{Tila}

\entry{nanche}
\partofspeech{m}
\spanishtranslation{chi'}
\clarification{árbol}

\entry{naranja}
\partofspeech{f}
\spanishtranslation{alaxax}

\entry{naranjal}
\partofspeech{m}
\spanishtranslation{alaxaxil}

\entry{nariz}
\partofspeech{f}
\spanishtranslation{*ñi'}

\entry{nauyaca}
\partofspeech{f}
\onedefinition{1}
\spanishtranslation{yäxk'äñcho}
\clarification{fer-de-lance; víbora}
\onedefinition{2}
\spanishtranslation{xbuk'utsu'}
\clarification{saltadora; reptil}
\secondaryentry{falsa nauyaca}
\secondtranslation{k'äñcho}
\clarification{colcuate; reptil}

\entry{navaja}
\partofspeech{f}
\spanishtranslation{ñawaxax}

\entry{neblina}
\partofspeech{f}
\spanishtranslation{*tya'tyokal}
\secondaryentry{con neblina baja}
\secondtranslation{pak'akña}

\entry{necesario}
\partofspeech{adj}
\spanishtranslation{\textsuperscript{2}wersa}

\entry{negativo}
\partofspeech{adj}
\onedefinition{1}
\spanishtranslation{kasä, käxtyi}
\onedefinition{2}
\spanishtranslation{kixtyä}
\clarification{Tila}

\entry{negro}
\partofspeech{adj}
\onedefinition{1}
\spanishtranslation{i'ik'}
\onedefinition{2}
\spanishtranslation{ik'jowañ}
\clarification{por humo}
\onedefinition{3}
\spanishtranslation{ik'motyañ}
\clarification{pelo, pluma}

\entry{nido}
\partofspeech{m}
\onedefinition{1}
\spanishtranslation{k'u', mety}
\clarification{aves, animales}
\onedefinition{2}
\spanishtranslation{*mujl}
\clarification{hormiga, tuza}
\secondaryentry{nido de hormiga}
\secondtranslation{*mujl xu'}
\secondaryentry{nido de la zacua}
\secondtranslation{chimk'ubul}

\entry{nieto}
\partofspeech{m}
\onedefinition{1}
\spanishtranslation{buts, bik'tyi ijts'iñ, *ij}
\onedefinition{2}
\spanishtranslation{mam}
\clarification{Tila}

\entry{niña}
\partofspeech{f}
\spanishtranslation{xch'ok}
\clarification{muchacha no casada}
\secondaryentry{niña del ojo}
\secondtranslation{ibäk' lakwuty}

\entry{niñez}
\partofspeech{f}
\spanishtranslation{*kolemal}

\entry{niño}
\partofspeech{m}
\onedefinition{1}
\spanishtranslation{alob, alp'eñel}
\onedefinition{2}
\spanishtranslation{bi'tyal}
\clarification{Sab.}

\entry{nivelar}
\partofspeech{vt}
\spanishtranslation{pamuñ}
\secondaryentry{nivelado}
\secondtranslation{pamal, tyäsäkña}
\clarification{terreno}

\entry{nixtamal}
\partofspeech{m}
\secondaryentry{nixtamal entero}
\secondtranslation{kestyo}

\entry{no}
\partofspeech{adv}
\onedefinition{1}
\spanishtranslation{ma', ma'añik}
\onedefinition{2}
\spanishtranslation{mach'añ}
\clarification{Sab.}
\onedefinition{3}
\spanishtranslation{ma'añ}
\clarification{Sab., Tila}
\secondaryentry{no es grande}
\secondtranslation{mach ñukik}
\secondaryentry{no es igual}
\secondtranslation{mach lajalik}
\secondaryentry{no está bueno}
\secondtranslation{mach weñik}
\secondaryentry{ya no}
\secondtranslation{jaymejl}

\entry{noche}
\partofspeech{f}
\onedefinition{1}
\spanishtranslation{ak'älel}
\onedefinition{2}
\spanishtranslation{abälel}
\clarification{Sab.}

\entry{nogal}
\partofspeech{m}
\spanishtranslation{tyoñtye'}
\clarification{árbol}

\entry{nómada}
\partofspeech{m}
\spanishtranslation{xchumchumñiyel}

\entry{nombrar}
\partofspeech{vt}
\spanishtranslation{wa'chokoñ}
\clarification{a un puesto}

\entry{nombre}
\partofspeech{m}
\spanishtranslation{*k'aba'}
\secondaryentry{nombre de una niña}
\secondtranslation{Xmikimañesa}

\entry{norte}
\partofspeech{m}
\spanishtranslation{chäk ik'lel}
\clarification{Sab.; mal tiempo}

\entry{nosotros}
\partofspeech{pron}
\spanishtranslation{lojoñ}
\clarification{excl.}

\entry{notata}
\partofspeech{f}
\spanishtranslation{juxlum}
\clarification{tipo de lagartija}

\entry{nube}
\partofspeech{f}
\spanishtranslation{tyokal}
\secondaryentry{nubes altas y delgadas a distintos niveles}
\secondtranslation{säkp'ilañ pañchañ}
\clarification{altas y delgadas a distintas niveles}
\secondaryentry{manera en que vienen nubes}
\secondtranslation{tyojokña}

\entry{nublar}
\partofspeech{vt}
\secondaryentry{nublarse}
\spanishtranslation{mäjkel}

\entry{nuca}
\partofspeech{f}
\spanishtranslation{*paty lakbik'}

\entry{nudo}
\partofspeech{m}
\spanishtranslation{*xu'il tye'}
\clarification{de madera}

\entry{nudoso}
\partofspeech{adj}
\spanishtranslation{tyuchul}
\clarification{cara, rama}

\entry{nuera}
\partofspeech{f}
\onedefinition{1}
\spanishtranslation{*ä'lib}
\onedefinition{2}
\spanishtranslation{a'libäl}
\clarification{Sab.}

\entry{nueve}
\partofspeech{adj}
\spanishtranslation{bolomp'ejl}

\entry{Nueve Espíritus}
\spanishtranslation{Boloñajaw}
\clarification{lugar}

\entry{nuevo}
\partofspeech{adj}
\spanishtranslation{tsijib}
\secondaryentry{hacer de nuevo}
\secondtranslation{tsijibtyesañ}

\entry{nutria}
\partofspeech{f}
\spanishtranslation{ja'al ts'i'}
\clarification{perro de agua; mamífero}

\entry{obedecer}
\partofspeech{vt}
\onedefinition{1}
\spanishtranslation{jak'}
\onedefinition{2}
\spanishtranslation{ch'ujbiñ}
\clarification{Tila}
\secondaryentry{acción de obedecer}
\secondtranslation{jak'ol}

\entry{obediente}
\partofspeech{adj}
\spanishtranslation{xjak'oñel}

\entry{obligar}
\partofspeech{vt}
\spanishtranslation{\textsuperscript{2}xik'}

\entry{obra}
\partofspeech{f}
\spanishtranslation{yäk'bal}

\entry{obrar}
\partofspeech{vi}
\secondaryentry{tiene ganas de obrar}
\secondtranslation{muk'uñ}

\entry{obsidiana}
\partofspeech{f}
\spanishtranslation{jacha lakmam, yäjyäx xajlel}

\entry{ocelote}
\partofspeech{m}
\onedefinition{1}
\spanishtranslation{ik'sajp, tsukbajlum, ik'bo'lay}
\clarification{mamífero}
\onedefinition{2}
\spanishtranslation{mek' ajtso'}
\clarification{Tila; mamífero}

\entry{ocho}
\partofspeech{adj}
\spanishtranslation{waxäkp'ejl}

\entry{ocotal}
\partofspeech{m}
\spanishtranslation{\textsuperscript{2}tyajol}

\entry{ocote}
\partofspeech{m}
\spanishtranslation{\textsuperscript{2}tyaj}
\secondaryentry{ocote agrio}
\secondtranslation{po'om}
\clarification{árbol}

\entry{ocre}
\partofspeech{m}
\spanishtranslation{almis}

\entry{octavo}
\partofspeech{adj}
\secondaryentry{octavo día}
\secondtranslation{waxäkñij}

\entry{odiar}
\partofspeech{vt}
\spanishtranslation{k'uxk'el}
\clarification{Sab.}

\entry{odioso}
\partofspeech{adj}
\secondaryentry{odiosamente}
\secondtranslation{ts'a'}

\entry{Oeste}
\partofspeech{m}
\spanishtranslation{*majlib k'iñ}
\clarification{Tila}

\entry{oír}
\partofspeech{vt}
\spanishtranslation{ubiñtyel}
\secondaryentry{oye}
\spanishtranslation{\textsuperscript{2}¡abi!}
\secondaryentry{oye tú}
\secondtranslation{¡jeyaj!}

\entry{ojo}
\partofspeech{m}
\spanishtranslation{\textsuperscript{1}*wuty}

\entry{oler}
\partofspeech{vt}
\spanishtranslation{sik'}

\entry{olfatear}
\partofspeech{vt}
\spanishtranslation{sik'}

\entry{olla}
\partofspeech{vt}
\spanishtranslation{ch'äxoñib, p'ejty, *p'ejtyal}

\entry{olor}
\partofspeech{m}
\spanishtranslation{ujts'il, *yujts'il}

\entry{oloroso}
\partofspeech{adj}
\spanishtranslation{chäbäkña}

\entry{olote}
\partofspeech{m}
\spanishtranslation{bäkäl}

\entry{olvidar}
\partofspeech{vt}
\spanishtranslation{ñajätyesañ}
\secondaryentry{olvidarse}
\secondtranslation{ñajäyel}
\secondaryentry{olvidado}
\secondtranslation{ñajäyem}

\entry{ombligo}
\partofspeech{m}
\spanishtranslation{*mujk}

\entry{omóplato}
\partofspeech{m}
\spanishtranslation{*pechkeñ}

\entry{once}
\partofspeech{adj}
\spanishtranslation{buluch, juñlujump'ejl}

\entry{ondeante}
\partofspeech{adj}
\spanishtranslation{ämlämña, sämäkña}
\clarification{agua}

\entry{ondear}
\partofspeech{vi}
\secondaryentry{ondeándose}
\secondtranslation{yulyulña}
\clarification{agua}

\entry{ondulante}
\partofspeech{adj}
\secondaryentry{manera ondulante}
\secondtranslation{lämuña}
\clarification{agua}

\entry{operar}
\partofspeech{vt}
\spanishtranslation{\textsuperscript{1}p'o'}
\clarification{persona}

\entry{oprimir}
\partofspeech{vt}
\secondaryentry{oprimir e inflar}
\secondtranslation{wosilañ}
\clarification{pelota}

\entry{orador}
\partofspeech{m}
\onedefinition{1}
\spanishtranslation{x'alty'añ}
\onedefinition{2}
\spanishtranslation{ajsubty'añ}
\clarification{Sab.}

\entry{orar}
\partofspeech{vi}
\spanishtranslation{ch'ujyijel}

\entry{orden}
\partofspeech{f}
\spanishtranslation{mañdal}

\entry{oreja}
\partofspeech{vt}
\spanishtranslation{chikiñ}
\secondaryentry{oreja de palo}
\secondtranslation{xk'o'loch}
\clarification{hongo}
\secondaryentry{oreja de palo}
\secondtranslation{xñojol}
\clarification{hongo colorado, no comestible}

\entry{orgulloso}
\partofspeech{adj}
\secondaryentry{orgullosamente}
\secondtranslation{wajawajal}
\clarification{Sab.}

\entry{orientar}
\partofspeech{vt}
\spanishtranslation{tyumbiñ}
\clarification{Sab.}

\entry{Oriente}
\partofspeech{m}
\onedefinition{1}
\spanishtranslation{*pasib k'iñ}
\onedefinition{2}
\spanishtranslation{paso' k'iñ}
\clarification{Sab.}

\entry{orilla}
\partofspeech{f}
\spanishtranslation{tyi'}
\secondaryentry{orilla de cerro}
\secondtranslation{ch'äktyäl}
\clarification{sirve como mirador}

\entry{orina}
\partofspeech{f}
\spanishtranslation{pich}

\entry{orinar}
\partofspeech{vt}
\spanishtranslation{cha'leñ pich}

\entry{Orión}
\partofspeech{m}
\spanishtranslation{rus ek'}

\entry{oruga}
\partofspeech{f}
\spanishtranslation{chup}
\secondaryentry{tipo de oruga}
\secondtranslation{tyo'ña'}
\secondaryentry{oruga agrimensora}
\spanishtranslation{ñajp'äk}
\clarification{larva}

\entry{oscuro}
\partofspeech{adj}
\onedefinition{1}
\spanishtranslation{ik'jowañ, mäkäl, sämlaw iyik'añ}
\onedefinition{2}
\spanishtranslation{ik'ch'ipañ, ik'yoch'añ}
\clarification{dentro de la casa o cueva}
\onedefinition{3}
\spanishtranslation{ik'ty'ojañ}
\clarification{bajo un nube}
\onedefinition{4}
\spanishtranslation{ik'ty'ojñal}
\clarification{Tila; de una cueva}
\secondaryentry{oscuramente}
\secondtranslation{ik'wa'añ}
\secondaryentry{muy oscuro}
\secondtranslation{i'ik'ax}

\entry{osito lanudo}
\spanishtranslation{chäkjocho chup}

\entry{oso hormiguero}
\spanishtranslation{ts'u' chab}
\clarification{mamífero}

\entry{otate verde}
\spanishtranslation{\textsuperscript{1}k'äñchejb}
\clarification{planta; con tallos sólidos}

\entry{otro}
\partofspeech{adj}
\spanishtranslation{yambä}
\secondaryentry{otra vez}
\secondtranslation{cha'}

\entry{oveja}
\partofspeech{f}
\spanishtranslation{tyäñäme'}

\entry{oxidar}
\partofspeech{vt}
\secondaryentry{oxidado}
\secondtranslation{tya'tya'}

\entry{óxido}
\partofspeech{m}
\spanishtranslation{tya'}

\entry{pacífico}
\partofspeech{adj}
\secondaryentry{hacerse pacífico}
\secondtranslation{k'uñ'añ}

\entry{padecer}
\partofspeech{vt}
\spanishtranslation{saliyel}
\clarification{sarna}

\entry{padrastro}
\partofspeech{m}
\spanishtranslation{*majañ tyaty}

\entry{padre}
\partofspeech{m}
\onedefinition{1}
\spanishtranslation{*tyaty, tyatyäl}

\onedefinition{2}
\spanishtranslation{pale}
\clarification{religioso}

\entry{padrino}
\partofspeech{m}
\spanishtranslation{jalatyaty}

\entry{padrón}
\partofspeech{m}
\spanishtranslation{chumtye'}
\clarification{de una casa}

\entry{paga}
\partofspeech{f}
\spanishtranslation{*tyojol}

\entry{pagar}
\partofspeech{vt}
\spanishtranslation{tyoj,tyojolañ}

\entry{país}
\partofspeech{m}
\onedefinition{1}
\spanishtranslation{pañimil}
\onedefinition{2}
\spanishtranslation{*lumal}
\clarification{de alguien}
\onedefinition{3}
\spanishtranslation{mulawil}
\clarification{Sab., Tila}

\entry{paisaje}
\partofspeech{m}
\secondaryentry{lugar para ver un paisaje}
\secondtranslation{k'elo'pañimil}

\entry{pajarera}
\partofspeech{f}
\spanishtranslation{uxix}
\clarification{Tila; reptil}

\entry{pájaro}
\partofspeech{m}
\spanishtranslation{muty, tye'lemuty}
\secondaryentry{pajarito}
\secondtranslation{alämuty}
\secondtranslation{bi'tyi muty}
\clarification{Sab., Tila}

\entry{pajón}
\partofspeech{m}
\spanishtranslation{tye'jam}
\clarification{tipo de zacate}

\entry{pajuil}
\partofspeech{vt}
\spanishtranslation{xch'ekejk}
\clarification{chachalaca negra; ave}

\entry{palabra}
\partofspeech{vt}
\spanishtranslation{*ty'añ}
\secondaryentry{con una sola palabra}
\secondtranslation{juñkujyel}

\entry{palidecer}
\partofspeech{vi}
\spanishtranslation{k'äñ'añ, yäx'añ}

\entry{pálido}
\partofspeech{adj}
\spanishtranslation{k'äñ, säklibañ}

\entry{palma}
\partofspeech{f}
\spanishtranslation{boñxañtye'}
\secondaryentry{palma real}
\secondtranslation{xañ}
\clarification{árbol}
\secondaryentry{palma de mano}
\secondtranslation{mal laj k'äb}
\secondaryentry{tipo de palma}
\spanishtranslation{xbäk ch'ip}
\clarification{sin espinas; chocón, chapalla}
\secondtranslation{ch'ib}
\clarification{da fruta comestible}
\secondtranslation{*ch'ibilel matye'el}
\clarification{del monte}

\entry{palmar}
\partofspeech{m}
\spanishtranslation{ch'ibol}

\entry{palmear}
\partofspeech{vt}
\secondaryentry{palmeando}
\secondtranslation{ty'ejty'ej}
\secondtranslation{lejles, lajlaj}
\clarification{acción repetida de palmear}

\entry{palo}
\partofspeech{m}
\onedefinition{1}
\spanishtranslation{tye'}
\onedefinition{2}
\spanishtranslation{pañtye'}
\clarification{atravesado en un arroyo}
\secondaryentry{palo de chapaya}
\secondtranslation{aktye'}
\secondaryentry{palo de humo}
\secondtranslation{uñtye'}
\clarification{laurel; árbol}
\secondaryentry{palo de maíz}
\secondtranslation{*sik'bal ixim}
\secondaryentry{palo que se usa para castigar a los niños}
\secondtranslation{xch'ich'itye'}
\secondaryentry{palo recortado para subir}
\secondtranslation{tyek'oñib}
\secondaryentry{palo verde}
\secondtranslation{yaxä'tye'}
\secondaryentry{palito para mover atole o pinole}
\secondtranslation{majas, juyib}

\entry{paloma}
\partofspeech{f}
\onedefinition{1}
\spanishtranslation{mukuy, pichoñ, ujkuts}
\onedefinition{2}
\spanishtranslation{xmukuy}
\clarification{con alas blancas}
\onedefinition{3}
\spanishtranslation{xpumuk}
\clarification{vive en el suelo}
\onedefinition{4}
\spanishtranslation{xtyuts}
\clarification{perdiz}
\onedefinition{5}
\spanishtranslation{x'ujkuts}
\clarification{ocotera}
\secondaryentry{paloma de collar}
\secondtranslation{x'ujkuts}

\entry{palpar}
\partofspeech{vt}
\spanishtranslation{lajlajye'}

\entry{palpitar}
\partofspeech{vi}
\secondaryentry{palpitando}
\secondtranslation{p'ump'uña, tyip'tyip'ña, tyumtyumña}

\entry{pan}
\partofspeech{m}
\spanishtranslation{kaxlañ waj}

\entry{panal}
\partofspeech{m}
\secondaryentry{panal de avispa}
\secondtranslation{otyoty xux, *yotyoty xux}

\entry{páncreas}
\partofspeech{m}
\spanishtranslation{lejlej}
\clarification{de ganado}

\entry{pandearse}
\partofspeech{prnl}
\spanishtranslation{lujk'el}

\entry{pantalón}
\partofspeech{m}
\spanishtranslation{-wex, wexäl}
\secondaryentry{sin pantalón}
\secondtranslation{wakal}

\entry{pantorrilla}
\partofspeech{f}
\spanishtranslation{tya'ok}

\entry{pañal}
\partofspeech{m}
\onedefinition{1}
\spanishtranslation{majts aläl, *tyasil}
\onedefinition{2}
\spanishtranslation{*bäjk'il}
\clarification{pl.}

\entry{pañuelo}
\partofspeech{m}
\spanishtranslation{pañujl}

\entry{papá}
\partofspeech{m}
\spanishtranslation{tyatya}

\entry{papausa}
\partofspeech{f}
\spanishtranslation{säsäk k'ätsats}
\clarification{ilama, tipo de anona; árbol}

\entry{papaya}
\partofspeech{vt}
\spanishtranslation{uch'uñtye'}

\entry{papel}
\partofspeech{m}
\spanishtranslation{\textsuperscript{1}juñ}

\entry{papera}
\partofspeech{f}
\spanishtranslation{ajkum bik', boroñch'och, yaj}

\entry{par}
\partofspeech{m}
\spanishtranslation{ñujp}
\clarification{animales, pájaros}
\secondaryentry{formar un par}
\secondtranslation{ñujpañ}

\entry{para}
\partofspeech{prep}
\spanishtranslation{cha'añ}

\entry{parar}
\partofspeech{vt}
\secondaryentry{pararse}
\secondtranslation{ty'uchtyäl}
\clarification{ave}
\secondtranslation{wa'tyäl}
\clarification{persona}
\secondtranslation{kotytyäl}
\clarification{animal}
\secondaryentry{parado}
\spanishtranslation{tyekel}
\clarification{árbol}
\spanishtranslation{wa'al}
\clarification{persona}
\spanishtranslation{kotyol}
\clarification{animal}
\spanishtranslation{ñäkwa'al}
\clarification{inclinado}
\spanishtranslation{xaty wa'al}
\clarification{con los pies separados}
\secondaryentry{poner parado}
\secondtranslation{kotychokoñ}
\clarification{objetos de cuatro patas}

\entry{pared}
\partofspeech{f}
\secondaryentry{pared de madera}
\secondtranslation{bojtye', tsal tye'}
\secondaryentry{pared de barro}
\secondtranslation{pajk'}
\secondaryentry{pared de cemento, pared de piedra}
\secondtranslation{ts'ajk}
\secondaryentry{sin paredes}
\secondtranslation{chaxal}
\clarification{casa}

\entry{parejo}
\partofspeech{adj}
\onedefinition{1}
\spanishtranslation{pamakña, k'iyikña}
\clarification{tierra}
\onedefinition{2}
\spanishtranslation{jaxal}
\clarification{Sab.}

\entry{pariente}
\partofspeech{m}
\onedefinition{1}
\spanishtranslation{majchil, pi'äl}
\onedefinition{2}
\spanishtranslation{ñi'tsil}
\clarification{políticos de los dos que se casan}

\entry{párpado}
\partofspeech{m}
\spanishtranslation{pächälel lakwuty}

\entry{parte}
\partofspeech{f}
\onedefinition{1}
\spanishtranslation{partye}
\clarification{Sab.}
\onedefinition{2}
\spanishtranslation{*chañelal}
\clarification{de arriba}

\entry{partera}
\partofspeech{f}
\onedefinition{1}
\spanishtranslation{ko'äl, mam}
\onedefinition{2}
\spanishtranslation{kolaj}
\clarification{Tila}

\entry{partido}
\partofspeech{m}
\spanishtranslation{partye}
\clarification{Sab.}

\entry{partir}
\onedefinition{1}
\partofspeech{vt}
\spanishtranslation{jaw}
\clarification{naranja, calabaza, corcho}
\onedefinition{2}
\partofspeech{vt}
\spanishtranslation{leb}
\clarification{piedra}
\onedefinition{3}
\partofspeech{vt}
\spanishtranslation{xoty'}
\clarification{leña}
\onedefinition{4}
\partofspeech{vt}
\spanishtranslation{xuty'}
\clarification{tortilla}
\secondaryentry{partirse}
\secondtranslation{jajwel}
\clarification{corcho, jonote}

\entry{parto}
\partofspeech{m}
\spanishtranslation{bajk'el}

\entry{pasado}
\partofspeech{adj}
\spanishtranslation{bejts'em}
\clarification{después del mediodía}
\secondaryentry{pasado mañana}
\secondtranslation{chabi}

\entry{pasar}
\onedefinition{1}
\partofspeech{vt}
\spanishtranslation{ñusañ}
\onedefinition{2}
\partofspeech{vi}
\spanishtranslation{bejts'el}
\clarification{mediodía}
\onedefinition{3}
\partofspeech{vi}
\spanishtranslation{k'axel}
\onedefinition{4}
\partofspeech{vi}
\spanishtranslation{tsuktsukñiyel}
\clarification{de casa en casa}
\onedefinition{5}
\partofspeech{vi}
\spanishtranslation{ñumel}
\clarification{persona en el camino}
\secondaryentry{pasar de fila en fila}
\spanishtranslation{tsoltsolñiyel}
\secondaryentry{pasar de largo}
\secondtranslation{päñ ñumel}
\clarification{sin saludar}
\secondaryentry{pasando}
\secondtranslation{k'äkäkña}
\clarification{flotando}
\secondaryentry{pasando encima}
\secondtranslation{k'äkäkña}
\clarification{lentamente}
\secondaryentry{pásate adelante}
\secondtranslation{¡ñumeñächix!}
\secondaryentry{pase adelante}
\secondtranslation{¡ocheñ!}
\clarification{dentro de la casa}
\secondaryentry{ya pasó}
\secondtranslation{xñumi}

\entry{pasear}
\partofspeech{vt}
\onedefinition{1}
\spanishtranslation{wa'akñiyel}
\onedefinition{2}
\spanishtranslation{chäñtyañ}
\clarification{Sab.}
\secondaryentry{paseando}
\secondtranslation{wa'wa'ña}
\clarification{tranquilamente}
\secondtranslation{xäñ pañimil}
\clarification{persona}

\entry{paseo}
\partofspeech{m}
\spanishtranslation{paxyal, xämbal}

\entry{pasivo}
\partofspeech{adj}
\spanishtranslation{k'uñ}

\entry{paso}
\partofspeech{m}
\onedefinition{1}
\spanishtranslation{k'axibäl}
\clarification{por un río}
\onedefinition{2}
\spanishtranslation{jajp wits}
\clarification{entre dos cerros}

\entry{patada}
\partofspeech{f}
\secondaryentry{dar patada}
\secondtranslation{tyoñtyek'}

\entry{pataste}
\partofspeech{m}
\spanishtranslation{tyuts'}
\clarification{patashte; árbol}

\entry{patear}
\partofspeech{vt}
\spanishtranslation{bo'tyek'}

\entry{patio}
\partofspeech{m}
\spanishtranslation{pam, paty otyoty}
\clarification{de la casa}

\entry{pato}
\partofspeech{m}
\spanishtranslation{pech}

\entry{patrón}
\partofspeech{m}
\secondaryentry{su patrón}
\secondtranslation{iyum}

\entry{patzagua}
\partofspeech{m}
\spanishtranslation{tyuk'ul}
\clarification{guamúchil; árbol}

\entry{pava}
\partofspeech{f}
\spanishtranslation{kox, ña' ak'ache}

\entry{pavo}
\partofspeech{m}
\spanishtranslation{ajtso', xpajlek'}
\clarification{ave doméstica}
\secondaryentry{pavo silvestre}
\secondtranslation{yäxak'ach}

\entry{paz}
\partofspeech{f}
\onedefinition{1}
\spanishtranslation{ñäch'chokoya, *ñäch'tyilel}
\onedefinition{2}
\spanishtranslation{*yajwälel}
\clarification{Sab.}

\entry{pecado}
\partofspeech{m}
\spanishtranslation{*malojlel}
\clarification{Sab.}

\entry{pecador}
\partofspeech{m}
\onedefinition{1}
\spanishtranslation{xmulil}
\onedefinition{2}
\spanishtranslation{ajmulil}
\clarification{Sab.}

\entry{pecho}
\partofspeech{m}
\onedefinition{1}
\spanishtranslation{chu', chu'äl}
\onedefinition{2}
\spanishtranslation{tyajñ}
\clarification{de aves}

\entry{pedazo}
\partofspeech{m}
\onedefinition{1}
\spanishtranslation{*xu'il}
\onedefinition{2}
\spanishtranslation{*lejbil}
\clarification{piedra, diente, madera}
\onedefinition{3}
\spanishtranslation{*xejty'il, \textsuperscript{2}*xujty'el}
\clarification{de papel, tortilla, tela, o madera}
\secondaryentry{pedazo de machete}
\secondtranslation{chojom}
\clarification{se usa para desgranar maíz}
\secondaryentry{pedazo de terreno}
\secondtranslation{*lumil}
\clarification{de una siembra}
\secondaryentry{pedazo de vidrio}
\secondtranslation{tyasa}
\secondaryentry{en pedacitos}
\secondtranslation{wijwis}
\secondaryentry{por pedazos}
\secondtranslation{kepkepña}

\entry{pedernal}
\partofspeech{m}
\spanishtranslation{ch'ijoñib}

\entry{pedir}
\partofspeech{vt}
\spanishtranslation{k'ajtyiñ, k'ajtyibeñ}
\secondaryentry{pedir fiado}
\secondtranslation{betyañ}
\secondaryentry{pedir prestado}
\secondtranslation{majñañ, majñäbeñ}
\secondaryentry{el que pide ayuda}
\secondtranslation{xk'ajtyiya}
\secondaryentry{el que pide esposa}
\secondtranslation{xk'ajtyiya}
\secondaryentry{pedido}
\spanishtranslation{k'ajtyibil}

\entry{pedo}
\partofspeech{m}
\spanishtranslation{tyis}

\entry{pedregal}
\partofspeech{m}
\spanishtranslation{xajlelol}

\entry{pegajoso}
\partofspeech{adj}
\spanishtranslation{ch'äyäkña, läp', tyäk', tsäyäkña}

\entry{pegamento}
\partofspeech{m}
\spanishtranslation{\textsuperscript{1}kolaj}

\entry{pegar}
\partofspeech{vt}
\onedefinition{1}
\spanishtranslation{bajbeñ}
\clarification{de golpes}
\onedefinition{2}
\spanishtranslation{jats'}
\clarification{persona o animal}
\onedefinition{3}
\spanishtranslation{les}
\clarification{con alas}
\onedefinition{4}
\spanishtranslation{läp', ñoty, \textsuperscript{2}tsuy}
\clarification{papel, tela}
\onedefinition{5}
\spanishtranslation{ñäp'}
\clarification{con pegamento}
\secondaryentry{pegarse}
\spanishtranslation{läp'tyäl}
\secondaryentry{pegado}
\secondtranslation{*tyoyol}
\secondtranslation{ñochol}
\clarification{una cosa con otra}
\secondtranslation{ñotyol}
\clarification{papel en la pared}
\secondtranslation{ñuty'ul}
\clarification{juntura}

\entry{peinar}
\partofspeech{vt}
\spanishtranslation{xibañ}

\entry{peine}
\partofspeech{m}
\spanishtranslation{xiyäb}

\entry{pelar}
\onedefinition{1}
\partofspeech{vt}
\spanishtranslation{poch, pajliñ}
\clarification{corteza, cáscara}
\onedefinition{2}
\partofspeech{vt}
\spanishtranslation{ts'ul}
\clarification{corteza de árbol, piel de animal}
\secondaryentry{pelarse}
\secondtranslation{ts'ujlel}
\secondaryentry{pelado}
\secondtranslation{chobil}
\clarification{frijol, plátano}
\secondtranslation{ts'ujlem}
\clarification{piel, cáscara}

\entry{pelear}
\partofspeech{vi}
\spanishtranslation{kajajtyañ}

\entry{peligro}
\partofspeech{m}
\spanishtranslation{bäbäk'eñ}

\entry{peligroso}
\partofspeech{adj}
\spanishtranslation{simaroñ}

\entry{pellizcar}
\partofspeech{vt}
\spanishtranslation{sep'}

\entry{pelo}
\partofspeech{m}
\spanishtranslation{*tsutsel}
\secondaryentry{pelo colorado}
\spanishtranslation{chäkjol}

\entry{peluquero}
\partofspeech{m}
\spanishtranslation{xlok' joläl}

\entry{pelusa}
\partofspeech{f}
\spanishtranslation{*jol ixim}

\entry{pendiente}
\partofspeech{adj}
\onedefinition{1}
\spanishtranslation{kep}
\clarification{limpia o rozadura de milpa}
\onedefinition{2}
\spanishtranslation{kepel}
\clarification{trabajo}

\entry{péndulo de corona}
\spanishtranslation{xwukip}
\clarification{momoto coroniazul; ave}

\entry{penetrar}
\partofspeech{vi}
\spanishtranslation{sojlel}
\clarification{frío, agua}

\entry{pensar}
\onedefinition{1}
\partofspeech{vt}
\spanishtranslation{ña'tyañ}
\onedefinition{2}
\partofspeech{vi}
\spanishtranslation{ña'tyäñtyel}
\secondaryentry{está pensando algo}
\secondtranslation{woli tyi ty'añ ipusik'al}
\secondaryentry{piensa mal}
\secondtranslation{leko ipusik'al}

\entry{pepita}
\partofspeech{f}
\spanishtranslation{bäkch'ujm}
\clarification{de calabaza}

\entry{pequeño}
\partofspeech{adj}
\spanishtranslation{alätyo}
\clarification{niño, animal}
\secondaryentry{hacerse más pequeño}
\secondtranslation{ch'och'okiyel}

\entry{percha}
\partofspeech{f}
\spanishtranslation{*ty'uchlib}
\clarification{lugar para pararse}

\entry{perder}
\partofspeech{vi}
\onedefinition{1}
\spanishtranslation{bujlel}
\clarification{la cáscara}
\onedefinition{2}
\spanishtranslation{sajtyel}
\clarification{Tila}
\secondaryentry{echar a perder}
\secondtranslation{säty}
\secondaryentry{perdido}
\secondtranslation{sajtyem}
\clarification{Tila}

\entry{perdiz}
\partofspeech{f}
\spanishtranslation{kulukab, xñakom}
\clarification{ave}
\secondaryentry{perdiz canela}
\secondtranslation{xñakow}
\secondaryentry{perdiz chica}
\secondtranslation{xchäläl}

\entry{perezoso}
\partofspeech{adj}
\spanishtranslation{ts'ub}

\entry{perico}
\partofspeech{m}
\spanishtranslation{uñix}
\clarification{ave}

\entry{periquito aliamarillo}
\partofspeech{m}
\spanishtranslation{tyuyub}
\clarification{ave}

\entry{perjudicial}
\partofspeech{adj}
\spanishtranslation{mech'}
\clarification{niño}

\entry{permanecer}
\partofspeech{vi}
\spanishtranslation{ajñel, ch'ujul}

\entry{permitir}
\partofspeech{vt}
\secondaryentry{no permite}
\secondtranslation{chäkbiñ}

\entry{perrito de agua}
\spanishtranslation{\textsuperscript{2}yol}
\clarification{Sab.; insecto}

\entry{perro}
\partofspeech{m}
\spanishtranslation{ts'i'}
\secondaryentry{perro de monte}
\secondtranslation{k'okil ts'i'}
\clarification{Sab.; agutí}

\entry{perseguir}
\partofspeech{vt}
\spanishtranslation{ajñesañ, yajñesañ}

\entry{persignar}
\partofspeech{vt}
\secondaryentry{persignarse}
\secondtranslation{rusiñ}

\entry{persona}
\partofspeech{f}
\spanishtranslation{kixtyañu}
\clarification{Tila}

\entry{pesado}
\partofspeech{adj}
\spanishtranslation{\textsuperscript{1}al}
\secondaryentry{hacerse pesado}
\secondtranslation{al'añ}
\clarification{físicamente}

\entry{pesca}
\partofspeech{f}
\spanishtranslation{chukchäy, luk'bäl}

\entry{pescado}
\partofspeech{m}
\spanishtranslation{\textsuperscript{2}chäy}

\entry{pescar}
\partofspeech{vt}
\spanishtranslation{lukijel}

\entry{pestaña}
\partofspeech{f}
\spanishtranslation{*tsutsel lakwuty}

\entry{petate}
\partofspeech{m}
\spanishtranslation{pojp}

\entry{petejul (reg.)}
\spanishtranslation{\textsuperscript{1}pats'}
\clarification{tamal hecho de masa y frijol tierno}

\entry{petróleo}
\partofspeech{m}
\spanishtranslation{kas}

\entry{pez}
\partofspeech{m}
\spanishtranslation{\textsuperscript{2}chäy, *chäñil ja'}

\entry{piar}
\partofspeech{vi}
\secondaryentry{piando}
\secondtranslation{ts'i'ts'i'ña}

\entry{picante}
\partofspeech{adj}
\spanishtranslation{ts'a'añ}
\clarification{Sab.}

\entry{picar}
\partofspeech{vt}
\onedefinition{1}
\spanishtranslation{ch'oj}
\clarification{gallina, culebra}
\onedefinition{2}
\spanishtranslation{jek'}
\clarification{con cuchillo o aguja}
\onedefinition{3}
\spanishtranslation{tyok}
\clarification{tierra}
\secondaryentry{picarse}
\secondtranslation{joch'añ}
\clarification{maíz}
\secondaryentry{picado}
\secondtranslation{ch'ä'ch'äñtyik}

\entry{pico}
\partofspeech{m}
\spanishtranslation{*ñi' muty}
\clarification{de ave}
\secondaryentry{pico blanco}
\secondtranslation{xkuway}
\clarification{arrendajo; ave}
\secondtranslation{xwakway}
\clarification{Tila; arrendajo; ave}
\secondaryentry{pico de canoa}
\secondtranslation{kolem päm}
\clarification{tucán; ave}
\secondaryentry{pico de garza}
\secondtranslation{piñtsik' päm}
\clarification{ave}
\secondaryentry{pico de hacha}
\secondtranslation{piñxik'päm}
\clarification{tucán; ave}
\secondaryentry{pico de oro}
\secondtranslation{stsijk}
\clarification{cerquero pico dorado; ave}

\entry{picotear}
\partofspeech{vt}
\spanishtranslation{ch'ojch'oñ}
\clarification{gallina, culebra}

\entry{pie}
\partofspeech{m}
\spanishtranslation{*ok}
\secondaryentry{concuerda con los pies encogidos}
\secondtranslation{pets}

\entry{piedra}
\partofspeech{f}
\onedefinition{1}
\spanishtranslation{xajlel}
\onedefinition{2}
\spanishtranslation{tyuñ}
\clarification{Tila}
\secondaryentry{piedra quebrada}
\secondtranslation{bi'tyi xajlel}
\secondaryentry{piedra para afilar}
\secondtranslation{jux}
\secondaryentry{piedra grande y redonda}
\secondtranslation{pityil}
\secondaryentry{piedra de chispa}
\secondtranslation{tyok', k'äñtyok'}
\secondaryentry{piedras chicas que sacan chispas}
\secondtranslation{tyok'bä tyuñ}
\secondaryentry{piedras del arroyo}
\secondtranslation{xajlelal pa'}
\secondaryentry{piedras chicas revueltas con barro}
\secondtranslation{xex}

\entry{piel}
\partofspeech{f}
\spanishtranslation{*pächälel}
\clarification{Sab.}

\entry{pierna}
\partofspeech{f}
\onedefinition{1}
\spanishtranslation{*ya'}
\onedefinition{2}
\spanishtranslation{tye'ok}
\clarification{de palo}

\entry{pigua}
\partofspeech{f}
\spanishtranslation{\textsuperscript{2}moch'}
\clarification{acamaya; crustáceo}

\entry{pimienta de Jamaica}
\partofspeech{f}
\spanishtranslation{ichtyo'}
\clarification{árbol}

\entry{pimienta de la tierra}
\spanishtranslation{ja'tye'}
\clarification{árbol}

\entry{pinole}
\partofspeech{m}
\spanishtranslation{ch'ilim}

\entry{pintar}
\partofspeech{vt}
\spanishtranslation{boñ}

\entry{pinto}
\partofspeech{adj}
\spanishtranslation{mistyuñtyik, p'ijlistyik}

\entry{pintura}
\partofspeech{f}
\spanishtranslation{*bojñil}

\entry{piña}
\partofspeech{f}
\spanishtranslation{pajch'}
\secondaryentry{piña chica}
\secondtranslation{tyuch pajch'}
\secondaryentry{piña de pino}
\secondtranslation{*wuty tyaj}
\clarification{piña de palo}
\secondaryentry{piña silvestre}
\secondtranslation{xch'ix pajch'}
\clarification{piñuela}

\entry{piñanona}
\partofspeech{f}
\onedefinition{1}
\spanishtranslation{juk'utyuñ}
\clarification{planta}
\onedefinition{2}
\spanishtranslation{poñch'ox}
\clarification{Sab., Tila; planta}

\entry{piñuela}
\partofspeech{f}
\onedefinition{1}
\spanishtranslation{xch'ix pajch'}
\clarification{piña silvestre}
\onedefinition{2}
\spanishtranslation{ty'utspajch'}
\clarification{Tila}

\entry{piojo}
\partofspeech{m}
\onedefinition{1}
\spanishtranslation{uch', sakaty}
\onedefinition{2}
\spanishtranslation{säsäk uch'}
\clarification{de cuerpo}
\onedefinition{3}
\spanishtranslation{yuch' muty}
\clarification{de gallina}

\entry{piquiamarillo}
\partofspeech{m}
\spanishtranslation{xkuway}
\clarification{ave}

\entry{pirinola}
\partofspeech{f}
\spanishtranslation{jetyejty}
\clarification{juguete}

\entry{pisar}
\partofspeech{vt}
\spanishtranslation{tyek', ty'uchtyañ}
\clarification{espina con los pies}

\entry{piscoy}
\partofspeech{m}
\spanishtranslation{xtyi'ja'}
\clarification{vaquero; ave}

\entry{piso}
\partofspeech{m}
\spanishtranslation{\textsuperscript{2}lum}
\secondaryentry{de tres pisos}
\secondtranslation{uxlajm}

\entry{pisón}
\partofspeech{m}
\spanishtranslation{kujoñib}

\entry{pito}
\partofspeech{m}
\spanishtranslation{amäy, \textsuperscript{1}pochob}

\entry{pitorreal}
\partofspeech{m}
\spanishtranslation{xch'aj päm}
\clarification{tucancillo collarejo; ave}

\entry{placenta}
\partofspeech{f}
\spanishtranslation{*chujyil}

\entry{planada}
\partofspeech{f}
\spanishtranslation{joktyäl}
\clarification{llanura}

\entry{plancha}
\partofspeech{f}
\spanishtranslation{juk'o' pisil}

\entry{planeta}
\partofspeech{m}
\spanishtranslation{ñoj ek'}

\entry{plano}
\partofspeech{adj}
\onedefinition{1}
\spanishtranslation{wechekña}
\onedefinition{2}
\spanishtranslation{pechel}
\clarification{piedra}
\onedefinition{3}
\spanishtranslation{welel}
\clarification{papel}
\onedefinition{4}
\spanishtranslation{wechel}
\clarification{libro, papel, tortilla, tabla}

\entry{planta}
\partofspeech{f}
\spanishtranslation{*tsuy me'}
\secondaryentry{tipo de planta}
\secondtranslation{boñxañ}
\clarification{alta; la hoja sirve para techos}
\secondtranslation{x'ichtyäk'}
\clarification{fruta verde, comida de palomas}
\spanishtranslation{mumo}
\clarification{su tallo tierno es comestible}
\secondtranslation{masamuñija'}
\clarification{medicinal}
\secondtranslation{tyiskok}
\clarification{medicinal}
\spanishtranslation{waj'um}
\clarification{Tila; que tiene el espíritu de abundancia}
\secondtranslation{wisik'iñ, xwayeñch'ix}
\clarification{tiene espinas}
\secondaryentry{planta del pie}
\secondtranslation{mal lakok}

\entry{plataforma}
\partofspeech{f}
\spanishtranslation{*kuñil}
\clarification{para maíz}

\entry{platanillo}
\partofspeech{m}
\spanishtranslation{p'o'tyo'}
\clarification{pico de gorrión; planta}

\entry{plátano}
\partofspeech{m}
\spanishtranslation{ja'as}
\secondaryentry{plátano enano}
\secondtranslation{ñokeb ja'as, xpek' ja'as}
\secondtranslation{yokä ja'as}
\clarification{Sab.}
\secondaryentry{plátano endosado}
\secondtranslation{wats}

\entry{plato}
\partofspeech{m}
\spanishtranslation{latyu}

\entry{pleito}
\partofspeech{m}
\onedefinition{1}
\spanishtranslation{letyo}
\onedefinition{2}
\spanishtranslation{periyal}
\clarification{Sab., Tila}

\entry{Pléyades}
\partofspeech{f}
\spanishtranslation{Xäñäb}
\clarification{constelación}

\entry{pluma}
\partofspeech{f}
\spanishtranslation{*tsutsel, *tsutsel muty}
\clarification{de ave}
\secondaryentry{sin pluma}
\secondtranslation{borox}

\entry{plumaje}
\partofspeech{m}
\spanishtranslation{*k'uk'mal, *k'uk'umlel}

\entry{pobre}
\partofspeech{adj}
\spanishtranslation{obol, p'ump'uñ, tyäwäl}

\entry{pochitoque}
\partofspeech{m}
\spanishtranslation{pochityok'}
\clarification{casquito; tipo de tortuga pequeña}

\entry{poco}
\onedefinition{1}
\partofspeech{adj}
\spanishtranslation{ts'itya'}
\onedefinition{2}
\partofspeech{adv}
\spanishtranslation{xep'}
\secondaryentry{poco líquido}
\secondtranslation{pomol}
\clarification{en su envase}
\secondaryentry{poquito}
\secondtranslation{wis}
\secondtranslation{ts'iñ}
\clarification{Sab.}
\secondtranslation{pomtyäl, ts'ujtyäl}
\clarification{líquido}
\secondaryentry{poquito en poquito}
\secondtranslation{bi'tyibi'tyäl}
\clarification{Sab.; comer}
\secondaryentry{por poco}
\secondtranslation{kolelix}
\secondaryentry{un poquito}
\secondtranslation{*ts'iñsaj}
\clarification{Sab.}

\entry{podar}
\partofspeech{vt}
\spanishtranslation{porajiñ}

\entry{poder}
\partofspeech{vi}
\onedefinition{1}
\spanishtranslation{mejlel}
\onedefinition{2}
\spanishtranslation{tyok'e}
\clarification{Tila}
\secondaryentry{no poder}
\secondtranslation{matsa'}
\secondaryentry{puede ser que podamos}
\secondtranslation{k'o'omejl}
\clarification{Tila}

\entry{podredumbre}
\partofspeech{f}
\spanishtranslation{*yok'beñal}
\clarification{putrefacción}

\entry{podrir}
\partofspeech{vi}
\onedefinition{1}
\spanishtranslation{ok'mäl}
\onedefinition{2}
\spanishtranslation{päk'mäyel}
\clarification{Sab.}
\secondaryentry{podrido}
\secondtranslation{ok'beñ, ok'beñal}
\secondaryentry{maíz podrido}
\secondtranslation{*jomil}

\entry{pollo}
\partofspeech{m}
\onedefinition{1}
\spanishtranslation{ch'ityoñ muty}
\onedefinition{2}
\spanishtranslation{bi'tyi muty}
\clarification{Sab.}
\onedefinition{3}
\spanishtranslation{chäkmuty}
\clarification{de pavón}
\secondaryentry{pollito}
\secondtranslation{alämuty}

\entry{polvo}
\partofspeech{m}
\spanishtranslation{ts'ubejñ}
\secondaryentry{su polvo}
\secondtranslation{*ts'ubeñal}

\entry{pólvora}
\partofspeech{f}
\spanishtranslation{sibik}
\clarification{de armas}

\entry{pomarrosa}
\partofspeech{f}
\spanishtranslation{xpomaros pätyal}
\clarification{manzana rosa; árbol}

\entry{poner}
\partofspeech{vt}
\onedefinition{1}
\spanishtranslation{k'äkchokoñ}
\clarification{sobre}
\onedefinition{2}
\spanishtranslation{jety}
\clarification{olla al fuego}
\onedefinition{3}
\spanishtranslation{läp}
\clarification{ropa, zapatos}
\onedefinition{4}
\spanishtranslation{tyasiñ}
\clarification{mantel en mesa; sudadero sobre espalda de mula}
\secondaryentry{ponerse el sol}
\secondtranslation{bäjlel, p'äjkel k'iñ}
\clarification{Sab.}
\secondaryentry{poner agachado}
\secondtranslation{wutschokoñ}
\secondaryentry{poner boca abajo}
\secondtranslation{ñukchokoñ, xitychokoñ}
\secondaryentry{poner en el fuego}
\secondtranslation{jejtyel}
\secondaryentry{poner en fila}
\secondtranslation{tsol}
\secondaryentry{poner parado}
\secondtranslation{kotychokoñ}
\clarification{objetos de cuatro patas}
\secondaryentry{poner piedras en lugar para hacer casa}
\secondtranslation{ts'ajkiñ}
\secondaryentry{poner un poco}
\secondtranslation{wischokoñ}
\clarification{de café, pozol}

\entry{Poniente}
\partofspeech{m}
\spanishtranslation{*bäjlib k'iñ}

\entry{por}
\partofspeech{prep}
\spanishtranslation{cha'añ}
\secondaryentry{de por sí}
\secondtranslation{ityilel, yilol jach}
\secondaryentry{por causa de}
\secondtranslation{kaj}
\secondaryentry{por favor}
\secondtranslation{awokolik; \textsuperscript{1}poj}
\clarification{por tiempo limitado}
\secondaryentry{por filas}
\secondtranslation{kelekña}
\secondaryentry{por pedazos}
\secondtranslation{kepkepña}
\secondaryentry{por poco}
\secondtranslation{kolelix}
\secondaryentry{por tramos}
\secondtranslation{kepekña}
\secondaryentry{por un lado}
\secondtranslation{ñelel}
\secondaryentry{por un rato}
\secondtranslation{tsäy}
\secondaryentry{por qué}
\secondtranslation{¿chukoch?}

\entry{porque}
\partofspeech{conj}
\spanishtranslation{kome}

\entry{posesión}
\partofspeech{f}
\spanishtranslation{*chubä'añ}

\entry{posible}
\partofspeech{adj}
\secondaryentry{lo más posible}
\secondtranslation{chilil}
\clarification{Sab.}
\secondaryentry{posiblemente}
\secondtranslation{tyik'äl}

\entry{postrar}
\partofspeech{vt}
\spanishtranslation{*päkleñ}

\entry{pozo}
\partofspeech{m}
\spanishtranslation{\textsuperscript{3}jok'}
\secondaryentry{pocito}
\secondtranslation{*chojk'ib}
\clarification{hoyuelo}

\entry{pozol, pozole}
\partofspeech{m}
\spanishtranslation{paj sa'}
\clarification{agrio}

\entry{precio}
\partofspeech{m}
\spanishtranslation{*tyojol, ityojol}

\entry{preciso}
\partofspeech{adj}
\spanishtranslation{\textsuperscript{2}wersa}

\entry{predicador}
\partofspeech{m}
\spanishtranslation{ajsubty'añ}
\clarification{Sab.}

\entry{predicar}
\partofspeech{vt}
\spanishtranslation{sub ty'añ}

\entry{preguntar}
\partofspeech{vt}
\spanishtranslation{k'ajtyiñ}
\secondaryentry{preguntado}
\secondtranslation{k'ajtyibil}
\secondaryentry{preguntarle}
\secondtranslation{k'ajtyibeñ}

\entry{prenda}
\partofspeech{vt}
\spanishtranslation{preñtya}

\entry{prensador de pelo}
\spanishtranslation{*ñejtyib joläl}

\entry{preocupación}
\partofspeech{vt}
\spanishtranslation{k'oj'ol}

\entry{preocupar}
\partofspeech{vt}
\onedefinition{1}
\spanishtranslation{k'oj'ojtyañ}
\onedefinition{2}
\spanishtranslation{peñsaliñ}
\clarification{Sab.}
\secondaryentry{preocupado}
\secondtranslation{mi imel ipusik'al}

\entry{preparar}
\partofspeech{vt}
\spanishtranslation{chajpañ}
\secondaryentry{preparado}
\secondtranslation{chajpäbil}

\entry{presa}
\partofspeech{f}
\spanishtranslation{*kolom}
\clarification{del perro}

\entry{presidente}
\partofspeech{m}
\spanishtranslation{pasaru}

\entry{presión}
\partofspeech{f}
\secondaryentry{ejercer presión}
\secondtranslation{yoty'}
\clarification{sobre el estómago}

\entry{preso}
\partofspeech{m}
\onedefinition{1}
\spanishtranslation{xkäjchel}
\onedefinition{2}
\spanishtranslation{ajkächol}
\clarification{Sab.}

\entry{prestado}
\partofspeech{adv}
\spanishtranslation{majañ}

\entry{prestar}
\partofspeech{vi}
\spanishtranslation{majñäñtyel}
\secondaryentry{no prestar por ser tacaño}
\secondtranslation{k'ebiñ}
\secondaryentry{recibir prestado}
\secondtranslation{ch'äm tyi majañ}

\entry{prevenido}
\partofspeech{adj}
\spanishtranslation{yäxäl}

\entry{primero}
\partofspeech{adv}
\onedefinition{1}
\spanishtranslation{ñaxañ, *yajtyi'al}
\onedefinition{2}
\spanishtranslation{ajapam, ñajañ}
\clarification{Sab.}
\onedefinition{3}
\spanishtranslation{pambij}
\clarification{en el camino}
\onedefinition{4}
\spanishtranslation{päñg}
\clarification{fruta}
\secondaryentry{primer hijo}
\secondtranslation{*yäx'al}
\secondaryentry{primer hijo o hija}
\secondtranslation{yäx'aläl}
\secondaryentry{primera vez}
\secondtranslation{juñyajlel}

\entry{primogénito}
\partofspeech{m}
\spanishtranslation{kojtyobä iyalobil}

\entry{principal}
\partofspeech{m}
\spanishtranslation{tyreñkipal}

\entry{principio}
\partofspeech{m}
\onedefinition{1}
\spanishtranslation{*tyejchibal}
\onedefinition{2}
\spanishtranslation{*tyejchilañ, *yajñelañ}
\clarification{Sab.}

\entry{probar}
\partofspeech{vt}
\onedefinition{1}
\spanishtranslation{ñop}
\onedefinition{2}
\spanishtranslation{\textsuperscript{2}*ilañ}
\clarification{el corazón}
\onedefinition{3}
\spanishtranslation{prowaliñ, yilbejtyañ}
\clarification{Sab.}
\secondaryentry{para probar}
\secondtranslation{lejeñsia}

\entry{proceso}
\partofspeech{m}
\spanishtranslation{meloñel, melojel}
\secondaryentry{proceso de moler}
\secondtranslation{juch'bal}
\clarification{maíz, café}

\entry{producir}
\partofspeech{vt}
\spanishtranslation{p'ol}
\clarification{animales}

\entry{prohibición}
\partofspeech{f}
\spanishtranslation{tyik'oñel}

\entry{prohibir}
\partofspeech{vt}
\spanishtranslation{tyik'}
\secondaryentry{prohibido}
\secondtranslation{jisil}

\entry{pronto}
\partofspeech{adv}
\spanishtranslation{mach jalik}
\clarification{lit.: no tarda}
\secondaryentry{por lo pronto}
\secondtranslation{leñ}

\entry{prostituta}
\partofspeech{f}
\spanishtranslation{xmoja}

\entry{proteger}
\partofspeech{vt}
\secondaryentry{protegido}
\secondtranslation{käñtyäbil}

\entry{próximo}
\partofspeech{adj}
\spanishtranslation{läk'äl}
\secondaryentry{próximo en edad}
\secondtranslation{tyak'äl}

\entry{pueblo}
\partofspeech{m}
\onedefinition{1}
\spanishtranslation{tyejklum}
\onedefinition{2}
\spanishtranslation{lum}
\clarification{Sab.}

\entry{puente}
\partofspeech{m}
\onedefinition{1}
\spanishtranslation{k'axibäl, pañtye'}
\onedefinition{2}
\spanishtranslation{k'ajtye', k'atye}
\clarification{Sab.}
\secondaryentry{puente de hamaca}
\secondtranslation{chiñcho'ak'}
\clarification{Sab.}

\entry{puerco}
\partofspeech{m}
\spanishtranslation{chityam}
\secondaryentry{puerco capado}
\secondtranslation{mañtyekaty}
\secondaryentry{puerco sin pelo}
\secondtranslation{\textsuperscript{2}xlu'}

\entry{puerco espín}
\partofspeech{s}
\spanishtranslation{ch'ix uch}

\entry{puerquito}
\partofspeech{m}
\spanishtranslation{ch'ämpäk'}
\clarification{ave}

\entry{puerro}
\partofspeech{m}
\spanishtranslation{welux}
\clarification{planta}

\entry{puerta}
\partofspeech{f}
\spanishtranslation{*ñujp'il, wertya}
\clarification{de casa}

\entry{puesto}
\partofspeech{adj}
\onedefinition{1}
\spanishtranslation{lakal}
\clarification{objeto largo}
\onedefinition{2}
\spanishtranslation{xetyel}
\clarification{objeto redondo}

\entry{pulga}
\partofspeech{f}
\spanishtranslation{ch'äk}

\entry{pulgón}
\partofspeech{m}
\spanishtranslation{k'äñk'äñ uch}
\clarification{enfermedad de maíz}

\entry{pulmón}
\partofspeech{m}
\spanishtranslation{*potsots}

\entry{pulso}
\partofspeech{m}
\spanishtranslation{*ch'ujlel}

\entry{puma}
\partofspeech{m}
\spanishtranslation{chäkbajlum}

\entry{punta}
\partofspeech{f}
\secondaryentry{hacer punta}
\secondtranslation{pajliñ}

\entry{puntiagudo}
\partofspeech{adj}
\spanishtranslation{ts'ukul}

\entry{puntilla}
\partofspeech{f}
\secondaryentry{de puntillas}
\secondtranslation{ty'ichty'ichña}

\entry{punto}
\partofspeech{m}
\spanishtranslation{*xuty wich'}
\clarification{de ala}
\secondaryentry{a punto de caer}
\secondtranslation{xuyiña}
\clarification{carga de animal}

\entry{puño}
\partofspeech{m}
\secondaryentry{con puño}
\secondtranslation{buj}

\entry{pupila}
\partofspeech{f}
\spanishtranslation{\textsuperscript{1}bäk', *ch'ujlel iwuty}

\entry{pupo barrigón}
\spanishtranslation{mäläl}
\clarification{pez}

\entry{puro}
\partofspeech{adj}
\spanishtranslation{k'ujts}
\clarification{cigarro}

\entry{pus}
\partofspeech{m}
\spanishtranslation{*pujil, *pujwil}
\secondaryentry{formar pus}
\secondtranslation{pujmäyel}

\entry{qué}
\partofspeech{pron}
\spanishtranslation{¿chuki?}
\secondaryentry{por qué}
\secondtranslation{¿chukoch?}
\secondaryentry{por qué será}
\secondtranslation{¿chukochka?; ¿chu'ochka?}
\secondaryentry{qué tal}
\secondtranslation{¿kojko?}

\entry{quebrar}
\onedefinition{1}
\partofspeech{vt}
\spanishtranslation{k'äs, xul}
\clarification{hueso, palo}
\onedefinition{2}
\partofspeech{vt}
\spanishtranslation{lom}
\clarification{olla, caja}
\onedefinition{3}
\partofspeech{vt}
\spanishtranslation{tyop'}
\clarification{piedra, tierra, vidrio}
\onedefinition{4}
\partofspeech{vt}
\spanishtranslation{ts'ij}
\clarification{piedra, leña, calabaza}
\secondaryentry{quebrarse}
\secondtranslation{k'äskujel, k'ästyäl, p'ijtyel, xujlel}
\clarification{palo, hueso}
\secondtranslation{lejbel}
\clarification{pedazo de piedra, madera, diente}
\secondtranslation{xejty'el}
\clarification{barro, tortilla}
\secondaryentry{quebrado}
\secondtranslation{xujlem}
\clarification{hueso, madera}
\secondaryentry{persona con hueso quebrado}
\secondtranslation{xk'äskujel}

\entry{quedar}
\partofspeech{vi}
\spanishtranslation{kälel, yajñel}
\secondaryentry{quedar pendiente}
\secondtranslation{keptyäl}
\secondaryentry{quedarse blanco}
\secondtranslation{säkwutyiyel}
\clarification{un lado de la tortilla}
\secondaryentry{quedarse firme}
\secondtranslation{xuk'tyäl}

\entry{queisque}
\partofspeech{m}
\onedefinition{1}
\spanishtranslation{xkekex}
\clarification{grajo verde; ave}
\onedefinition{2}
\spanishtranslation{peazul}
\clarification{Tila; ave}

\entry{quejarse}
\partofspeech{prnl}
\secondaryentry{quejándose}
\secondtranslation{ajakña, äjäkña}

\entry{quemadura}
\partofspeech{f}
\secondaryentry{quemadura aguda}
\secondtranslation{tyäjk'}

\entry{quemar}
\partofspeech{vt}
\spanishtranslation{pul}
\secondaryentry{quemarse}
\spanishtranslation{pulel}
\secondaryentry{quemado}
\secondtranslation{pulem}

\entry{quemazón}
\partofspeech{f}
\secondaryentry{quemazón de monte}
\secondtranslation{tyojklel}

\entry{quequeste}
\partofspeech{m}
\spanishtranslation{\textsuperscript{2}juk', me'uñ}
\clarification{mafafa; planta}

\entry{querer}
\partofspeech{vt}
\spanishtranslation{k'uxbiñ, *om}
\secondaryentry{querido}
\secondtranslation{k'uxbibil}
\secondaryentry{quisiera}
\secondtranslation{komkatsa'}
\secondaryentry{ya quiere}
\secondtranslation{yomox}

\entry{quetzal}
\partofspeech{m}
\spanishtranslation{xmañk'uk', xkeñzal}
\clarification{ave}

\entry{quiba}
\partofspeech{f}
\spanishtranslation{mojtyoy}
\clarification{palma}

\entry{quién}
\partofspeech{pron}
\onedefinition{1}
\spanishtranslation{¿majki?}
\onedefinition{2}
\spanishtranslation{¿majchki?}
\clarification{Tila}

\entry{quieto}
\partofspeech{adj}
\onedefinition{1}
\spanishtranslation{ch'ijiyem}
\clarification{en casa}
\onedefinition{2}
\spanishtranslation{k'uxtyayem}
\clarification{Sab.}
\onedefinition{3}
\spanishtranslation{lämäl, sämäl}
\clarification{persona, agua}
\onedefinition{4}
\spanishtranslation{ñäch'äl}
\clarification{persona}

\entry{quijada}
\partofspeech{f}
\spanishtranslation{xäk'tyi'}

\entry{quince}
\partofspeech{adj}
\spanishtranslation{jo'lujump'ejl}

\entry{quinto}
\partofspeech{adj}
\secondaryentry{quinto día del presente}
\secondtranslation{jo'ñij}

\entry{quitar}
\onedefinition{1}
\partofspeech{vt}
\spanishtranslation{chaw}
\clarification{pañales}
\onedefinition{2}
\partofspeech{vt}
\spanishtranslation{\textsuperscript{1}chil}
\clarification{algo o una persona}
\onedefinition{3}
\partofspeech{vt}
\spanishtranslation{joch}
\clarification{ropa}
\onedefinition{4}
\partofspeech{vt}
\spanishtranslation{\textsuperscript{2}poj}
\clarification{Sab.}
\secondaryentry{quitarse}
\secondtranslation{jojchel}
\clarification{ropa}

\entry{quizás}
\partofspeech{adv}
\spanishtranslation{machtyika, ma'tyika}

\entry{raíz}
\partofspeech{f}
\onedefinition{1}
\spanishtranslation{*tyomel, *wi'}
\onedefinition{2}
\spanishtranslation{chax läktyik, *chaxwi'}
\clarification{salen del suelo}
\secondaryentry{raíz de chayote}
\secondtranslation{yame}
\secondaryentry{raíz de plátano}
\secondtranslation{tyom}
\secondaryentry{raíz de yuca}
\secondtranslation{*wi'ts'ijñ}

\entry{rajada}
\partofspeech{f}
\spanishtranslation{*jajtyemal}
\clarification{de tabla}

\entry{rajadura}
\partofspeech{f}
\spanishtranslation{*jajp}

\entry{rajar}
\partofspeech{vt}
\onedefinition{1}
\spanishtranslation{jaty}
\clarification{leña}
\onedefinition{2}
\spanishtranslation{sil}
\clarification{tabla, mecapal}
\onedefinition{3}
\spanishtranslation{\textsuperscript{3}ts'ij}
\clarification{leña, piedra, calabaza}
\secondaryentry{rajarse}
\secondtranslation{jajtyel}
\secondaryentry{rajado}
\secondtranslation{xalal}

\entry{rana}
\partofspeech{f}
\spanishtranslation{*chäñil ja', tyujts', xtyujtye'}

\entry{ranera verde}
\spanishtranslation{yaxajachañ}
\clarification{reptil}

\entry{rápido}
\partofspeech{adv}
\spanishtranslation{seb}
\secondaryentry{rápidamente}
\secondtranslation{jäm}
\clarification{levantarse o tomar algún objeto}
\secondtranslation{xity'}
\clarification{levantarse, pararse}
\secondtranslation{\textsuperscript{1}yol}
\clarification{tomar algo líquido}

\entry{rascar}
\partofspeech{vt}
\onedefinition{1}
\spanishtranslation{lajchiñ, läch}
\clarification{piel}
\onedefinition{2}
\spanishtranslation{joty'}
\clarification{cabeza}

\entry{rasgadura}
\partofspeech{f}
\spanishtranslation{*tsijlemal}

\entry{raspar}
\partofspeech{vt}
\onedefinition{1}
\spanishtranslation{jujchiñ}
\clarification{barbas, pelo}
\onedefinition{2}
\spanishtranslation{susuñ}
\clarification{palos}
\secondaryentry{raspado}
\secondtranslation{susubil}
\clarification{cáscara de palo}

\entry{rastrojo}
\partofspeech{m}
\spanishtranslation{jaj sepoñel}
\clarification{Sab.}

\entry{rato}
\partofspeech{m}
\secondaryentry{por un rato}
\secondtranslation{tsäy}
\secondaryentry{ratito}
\secondtranslation{jumuk'}

\entry{ratón}
\partofspeech{m}
\spanishtranslation{tsuk}
\secondaryentry{las sobras del ratón}
\secondtranslation{*tsukil ixim}

\entry{raudal}
\partofspeech{m}
\secondaryentry{en raudal}
\secondtranslation{yolokña}

\entry{rayado}
\partofspeech{adj}
\spanishtranslation{barsiñ}
\clarification{perro}

\entry{rayo}
\partofspeech{m}
\spanishtranslation{chajk}
\clarification{sing.}
\spanishtranslation{*xojob}
\clarification{pl.}
\secondaryentry{rayos de sol}
\secondtranslation{ixojob k'iñ}
\secondaryentry{rayos de luna}
\secondtranslation{ixojob uw}

\entry{rebalsar}
\partofspeech{vi}
\secondaryentry{rebalsando}
\secondtranslation{buty'ukña}

\entry{rebanar}
\partofspeech{vt}
\spanishtranslation{sejluñ}

\entry{rebelde}
\partofspeech{adj}
\spanishtranslation{ch'aplom}

\entry{rebotar}
\partofspeech{vi}
\secondaryentry{rebotando}
\secondtranslation{tyip'tyip'ña}

\entry{rebozo}
\partofspeech{m}
\onedefinition{1}
\spanishtranslation{rebus}
\onedefinition{2}
\spanishtranslation{mujch'äl}
\clarification{Sab.}

\entry{recaudo}
\partofspeech{m}
\spanishtranslation{*ts'äkal}
\clarification{de comida}

\entry{rechinar}
\partofspeech{vi}
\spanishtranslation{äch'uñiyel}
\secondaryentry{rechinando}
\secondtranslation{äch'uña, kech'ekña}

\entry{recibir}
\partofspeech{vt}
\onedefinition{1}
\spanishtranslation{ñoxi'aliñ}
\clarification{como esposo}
\onedefinition{2}
\spanishtranslation{ochel}
\clarification{cargo de}

\entry{recio}
\partofspeech{adv}
\onedefinition{1}
\spanishtranslation{wo'okña}
\clarification{llorando}
\onedefinition{2}
\spanishtranslation{ts'ilikña}
\clarification{Sab.}

\entry{reclinar}
\partofspeech{vt}
\spanishtranslation{ñäktyäl}

\entry{recoger}
\partofspeech{vt}
\spanishtranslation{\textsuperscript{2}loty}
\secondaryentry{recogido}
\secondtranslation{tyempäbil}

\entry{recomendable}
\partofspeech{adj}
\onedefinition{1}
\spanishtranslation{*ty'ojol}
\onedefinition{2}
\spanishtranslation{k'otya}
\clarification{Tila, Sab.}

\entry{reconocer}
\partofspeech{vt}
\spanishtranslation{\textsuperscript{2}*ilañ}
\clarification{sus pensamientos}
\secondaryentry{reconocer como padre}
\secondtranslation{tyatyiñ}

\entry{recordar}
\partofspeech{vt}
\spanishtranslation{ña'tyañ}

\entry{recto}
\partofspeech{adj}
\onedefinition{1}
\spanishtranslation{tyoj}
\onedefinition{2}
\spanishtranslation{jump'ejl ipusik'al}
\clarification{lit.: un solo corazón}
\secondaryentry{es recto de corazón}
\secondtranslation{tyoj ipusik'al}

\entry{recuperar}
\partofspeech{vt}
\secondaryentry{recuperarse}
\secondtranslation{uts'atyiyel}
\clarification{Sab.}

\entry{red}
\partofspeech{f}
\onedefinition{1}
\spanishtranslation{chim}
\onedefinition{2}
\spanishtranslation{chimo' chäy}
\clarification{para pescar}
\secondaryentry{red chica}
\spanishtranslation{\textsuperscript{1}sewal}

\entry{redondo}
\partofspeech{adj}
\onedefinition{1}
\spanishtranslation{selel}
\clarification{piedra}
\onedefinition{2}
\spanishtranslation{selelob}
\clarification{gente en un círculo}
\onedefinition{3}
\spanishtranslation{xotyol}
\clarification{mesa, plato, sombrero}
\secondaryentry{en forma redonda}
\secondtranslation{xoty}
\secondaryentry{redondo y sentado}
\secondtranslation{petstyäl}

\entry{reducido}
\partofspeech{adj}
\spanishtranslation{tsätsä ñu'ty'ul}
\secondaryentry{muy reducido}
\secondtranslation{xep'el}

\entry{reflección}
\partofspeech{vt}
\spanishtranslation{*xojoblel}

\entry{refugio}
\partofspeech{m}
\spanishtranslation{puts'ibäl}

\entry{regalar}
\partofspeech{vt}
\spanishtranslation{p'ejw}
\clarification{en abundancia}

\entry{regalo}
\partofspeech{m}
\spanishtranslation{majtyañäl}

\entry{regañada}
\partofspeech{f}
\spanishtranslation{wulwul ty'añ}
\clarification{regaño}

\entry{regañar}
\partofspeech{vt}
\spanishtranslation{a'leñ}

\entry{regaño}
\partofspeech{m}
\onedefinition{1}
\spanishtranslation{a'leya}
\onedefinition{2}
\spanishtranslation{al'iya}
\clarification{Tila}
\secondaryentry{término de regaño}
\secondtranslation{xpajlek'}

\entry{regar}
\partofspeech{vt}
\onedefinition{1}
\spanishtranslation{\textsuperscript{1}mul}
\clarification{con agua}
\onedefinition{2}
\spanishtranslation{wejtyuñ, p'ujp'uñ, wejch'uñ}
\clarification{semilla}
\secondaryentry{regado}
\secondtranslation{ach'esäbil}
\secondtranslation{p'ujp'ubil}
\clarification{semilla, tierra}
\secondtranslation{pujkiktyik}
\clarification{piedra, semilla}

\entry{regidor}
\partofspeech{m}
\spanishtranslation{rejerol}

\entry{región}
\partofspeech{f}
\secondaryentry{región montañosa}
\secondtranslation{*witsilel}

\entry{regla}
\partofspeech{f}
\onedefinition{1}
\spanishtranslation{*p'isoñib}
\onedefinition{2}
\spanishtranslation{legra}
\clarification{Sab.}

\entry{regresar}
\partofspeech{vi}
\spanishtranslation{sujtyel}

\entry{rehusar}
\partofspeech{vt}
\spanishtranslation{tsäjyuñ}
\clarification{de dar algo}

\entry{reino}
\partofspeech{m}
\spanishtranslation{*reyil}

\entry{reír}
\partofspeech{vi}
\secondaryentry{reírse de}
\secondtranslation{tse'tyañ}

\entry{rejoya}
\partofspeech{f}
\spanishtranslation{k'omtyäl}
\clarification{de un terreno}

\entry{relación}
\partofspeech{f}
\secondaryentry{tener relación sexual}
\secondtranslation{pi'leñ}

\entry{relámpago}
\partofspeech{m}
\spanishtranslation{*xu'chajk, xojob chajk, ixu'il chajk}

\entry{reloj}
\partofspeech{m}
\spanishtranslation{k'elo'k'iñ}

\entry{remachar}
\partofspeech{vt}
\spanishtranslation{kuj}

\entry{remedio}
\partofspeech{m}
\spanishtranslation{ts'ak}

\entry{remendar}
\partofspeech{vt}
\spanishtranslation{läw}
\secondaryentry{remendarse}
\secondtranslation{läjwel}
\secondaryentry{remendado}
\secondtranslation{läwäl}
\secondaryentry{actividad de remendar ropa}
\spanishtranslation{läwoñel}

\entry{remiendo}
\partofspeech{m}
\spanishtranslation{*läjwil, *läwoñib}

\entry{remojar}
\partofspeech{vt}
\onedefinition{1}
\spanishtranslation{\textsuperscript{2}sul, ts'aj}
\onedefinition{2}
\spanishtranslation{ts'ajts'añ}
\clarification{cabeza}
\secondaryentry{remojado}
\secondtranslation{sulul}

\entry{remoler}
\partofspeech{vt}
\spanishtranslation{tyojsiñ, yajbiñ}
\clarification{nixtamal}

\entry{remolino}
\partofspeech{m}
\onedefinition{1}
\spanishtranslation{joyolbä ja'}
\clarification{agua}
\onedefinition{2}
\spanishtranslation{sutyuty ik'}
\clarification{de viento}

\entry{remover}
\partofspeech{vt}
\spanishtranslation{ñakulañ}
\clarification{piedra}

\entry{renacuajo}
\partofspeech{m}
\spanishtranslation{bujb}
\clarification{sing.}
\spanishtranslation{xbujb}
\clarification{pl.}

\entry{rendija}
\partofspeech{f}
\spanishtranslation{\textsuperscript{1}*yaj}
\clarification{de tabla}

\entry{rendir}
\partofspeech{vi}
\spanishtranslation{xijty'el}
\clarification{maíz, frijol}

\entry{renovar}
\partofspeech{vt}
\spanishtranslation{tsijibtyesañ}

\entry{repartir}
\partofspeech{vt}
\spanishtranslation{puk}

\entry{repellar}
\partofspeech{vt}
\spanishtranslation{boñ}

\entry{repentinamente}
\partofspeech{adv}
\onedefinition{1}
\spanishtranslation{tyul}
\onedefinition{2}
\spanishtranslation{pepets'}
\clarification{pisotear}

\entry{repetir}
\partofspeech{vt}
\secondaryentry{repetidas veces}
\secondtranslation{pik'os}
\clarification{amarrar}

\entry{repollo}
\partofspeech{m}
\spanishtranslation{xuxuk' pimel}

\entry{reproducir}
\partofspeech{vt}
\spanishtranslation{p'ojlel}

\entry{resbalar}
\partofspeech{vi}
\onedefinition{1}
\spanishtranslation{tyäjts'el}
\clarification{pies}
\onedefinition{2}
\spanishtranslation{jäjlel}
\clarification{en camino}
\onedefinition{3}
\spanishtranslation{juxk'iyel}
\clarification{sobre un palo o piedra}
\secondaryentry{resbalando}
\secondtranslation{juxukña}

\entry{resbaloso}
\partofspeech{adj}
\spanishtranslation{bojy}
\clarification{camino}

\entry{resembrar}
\partofspeech{vt}
\spanishtranslation{\textsuperscript{1}tsuts}
\clarification{maíz}

\entry{residente}
\partofspeech{adj}
\spanishtranslation{chumul}

\entry{resina}
\partofspeech{vt}
\spanishtranslation{*xu'ch'il}

\entry{resoplar}
\partofspeech{vi}
\secondaryentry{resoplando}
\secondtranslation{pujiña}
\clarification{nariz de caballo}

\entry{respirar}
\partofspeech{vt}
\spanishtranslation{jap ik'}
\secondaryentry{respirando}
\secondtranslation{wojsiña}

\entry{resplandor}
\partofspeech{m}
\spanishtranslation{*xojoblel}
\clarification{de la luz}

\entry{responsabilidad}
\partofspeech{vt}
\spanishtranslation{weñtya}

\entry{responsable}
\partofspeech{adj}
\spanishtranslation{bäjix, tsäts}

\entry{respuesta}
\partofspeech{vt}
\spanishtranslation{aja}

\entry{restregar}
\partofspeech{vt}
\spanishtranslation{juk'xiñ, juk'ilañ}

\entry{resultar}
\partofspeech{vi}
\onedefinition{1}
\spanishtranslation{lok'el}
\clarification{negocio, trabajo}
\onedefinition{2}
\spanishtranslation{majlel}
\clarification{un plan}

\entry{retirar}
\partofspeech{vt}
\spanishtranslation{tyäts'}

\entry{retoñar}
\partofspeech{vi}
\secondaryentry{retoñado}
\secondtranslation{ch'okiyem}
\clarification{nuevamente}

\entry{retoño}
\partofspeech{m}
\onedefinition{1}
\spanishtranslation{buts, k'ojty}
\onedefinition{2}
\spanishtranslation{chäläl}
\clarification{zacate nuevo después de cortar}

\entry{retorcer}
\partofspeech{vt}
\onedefinition{1}
\spanishtranslation{\textsuperscript{1}jax}
\clarification{ixtle, hilo}
\onedefinition{2}
\spanishtranslation{ts'otyiñ}
\clarification{mecate, lazo}

\entry{retrato}
\partofspeech{m}
\spanishtranslation{lok'oñbaj}
\clarification{Tila}

\entry{retumbante}
\partofspeech{adj}
\spanishtranslation{tsiñtsiña}
\clarification{sonido de campana}

\entry{reunidos}
\partofspeech{adj}
\spanishtranslation{selelob}
\clarification{personas}

\entry{reventar}
\onedefinition{1}
\partofspeech{vt}
\spanishtranslation{ts'ok}
\clarification{lazo, hilo}
\onedefinition{2}
\partofspeech{vi}
\spanishtranslation{tyojmel}
\secondaryentry{reventarse}
\secondtranslation{tyujk'el, ts'ojkel}
\clarification{alambre, cordón, hilo}

\entry{revolcar}
\partofspeech{vt}
\spanishtranslation{ñolch'iñ}
\clarification{persona con persona o animal con animal}
\secondaryentry{revolcando}
\secondtranslation{ñolok'}
\clarification{niño, caballo}
\secondaryentry{revolcarse}
\secondtranslation{bäläk'}
\clarification{Tila}
\secondtranslation{pajch'iñ}
\clarification{por dolor}
\secondtranslation{watyax}
\clarification{niño}
\secondaryentry{revuelto}
\secondtranslation{xäk'bil}

\entry{reyezuelo}
\partofspeech{m}
\spanishtranslation{tyikwak'iñ}
\clarification{pajarito}

\entry{rezar}
\partofspeech{vi}
\spanishtranslation{ch'uyijel}
\clarification{Tila}

\entry{rezo}
\partofspeech{m}
\spanishtranslation{resal}

\entry{rico}
\partofspeech{adj}
\spanishtranslation{chumul}

\entry{rifle}
\partofspeech{m}
\spanishtranslation{juloñib}

\entry{rincón}
\partofspeech{m}
\spanishtranslation{xo'tyäl}

\entry{riñón}
\partofspeech{m}
\spanishtranslation{*kuchi'tyuñ}

\entry{río}
\partofspeech{m}
\onedefinition{1}
\spanishtranslation{ja', ñoja'}
\onedefinition{2}
\spanishtranslation{ñojpa'}
\clarification{Sab.}
\secondaryentry{río grande}
\secondtranslation{kolem ja'}

\entry{risa}
\partofspeech{f}
\spanishtranslation{tse'ñal}

\entry{robar}
\partofspeech{vt}
\spanishtranslation{xujch'iñ}

\entry{roble}
\partofspeech{m}
\spanishtranslation{k'olol}
\clarification{árbol}

\entry{rociar}
\partofspeech{vt}
\spanishtranslation{pujbañ, tsijkañ}
\clarification{con agua}

\entry{rocío}
\partofspeech{m}
\spanishtranslation{tyijil}

\entry{rodar}
\partofspeech{vt}
\onedefinition{1}
\spanishtranslation{ñolk'iñ, waxk'uñ}
\clarification{piedra, bola}
\onedefinition{2}
\spanishtranslation{bälk'uñ}
\clarification{palo}
\onedefinition{3}
\spanishtranslation{selulañ}
\clarification{cosa redonda}
\secondaryentry{rodarse}
\secondtranslation{ñojlel}
\clarification{piedra}
\secondaryentry{rondando}
\secondtranslation{tsuktsukña}
\clarification{en busca de su presa}

\entry{rodear}
\partofspeech{vt}
\spanishtranslation{joy, wilijtyañ}
\secondaryentry{rodeando}
\secondtranslation{joyokña}

\entry{rodilla}
\partofspeech{f}
\spanishtranslation{\textsuperscript{2}pix}

\entry{rojo}
\partofspeech{adj}
\spanishtranslation{chäkwa'añ}
\clarification{llamas altas de fuego}

\entry{rollizo}
\partofspeech{adj}
\spanishtranslation{\textsuperscript{1}sel}
\clarification{objeto}

\entry{romper}
\partofspeech{vt}
\spanishtranslation{tsil}
\clarification{tela, papel}
\secondaryentry{roto}
\spanishtranslation{tyokol}
\spanishtranslation{lejbeñ}
\clarification{piedra, tabla o diente}
\spanishtranslation{p'e'}
\clarification{y abierto}
\spanishtranslation{tsijlem}
\clarification{ropa, costal}

\entry{roncar}
\partofspeech{vi}
\spanishtranslation{ñojk'ijel}

\entry{roncha}
\partofspeech{f}
\spanishtranslation{sal}

\entry{ronda}
\partofspeech{f}
\spanishtranslation{lajk'}
\clarification{hormiga grande}

\entry{ronquido}
\partofspeech{m}
\spanishtranslation{ñojk'}

\entry{ropa}
\partofspeech{vt}
\spanishtranslation{pisil}

\entry{rozadura}
\partofspeech{f}
\spanishtranslation{chobal, *chobälel}
\clarification{de milpa}

\entry{rozar}
\partofspeech{vt}
\spanishtranslation{choloñ}
\clarification{monte}
\secondaryentry{rozando}
\secondtranslation{\textsuperscript{2}jax}

\entry{rubio}
\partofspeech{adj}
\spanishtranslation{chäktyijañ}
\clarification{cabello}

\entry{ruidosamente}
\partofspeech{adv}
\onedefinition{1}
\spanishtranslation{burburña, ts'ij}
\onedefinition{2}
\spanishtranslation{\textsuperscript{2}silikña}
\clarification{sonido que da la madera al rajarse}
\onedefinition{3}
\spanishtranslation{sokokña}
\clarification{animal pasando por la hierba}
\onedefinition{4}
\spanishtranslation{ts'ilikña}
\clarification{Sab.}
\onedefinition{5}
\spanishtranslation{we'ekña}
\clarification{llanto}

\entry{sábado}
\partofspeech{m}
\spanishtranslation{*jilibal semaña}

\entry{saber}
\partofspeech{vt}
\onedefinition{1}
\spanishtranslation{*ujil, yujil}
\onedefinition{2}
\spanishtranslation{ubiñtyel}
\clarification{de algún rumor}
\secondaryentry{dar a saber}
\secondtranslation{tsiktyesañ}
\secondaryentry{no sé}
\secondtranslation{ba'ixtyi}

\entry{sabor}
\partofspeech{m}
\spanishtranslation{*sumuklel}
\secondaryentry{su sabor salado}
\secondtranslation{*yäts'mil}

\entry{saborear}
\partofspeech{vt}
\spanishtranslation{mits'tyi'añ}

\entry{sabroso}
\partofspeech{adj}
\spanishtranslation{sumuk}

\entry{sacar}
\partofspeech{vt}
\onedefinition{1}
\spanishtranslation{lok', lok'sañ}
\onedefinition{2}
\spanishtranslation{wets'}
\onedefinition{3}
\spanishtranslation{jeb}
\clarification{líquidos}
\onedefinition{4}
\spanishtranslation{\textsuperscript{1}jok'}
\clarification{líquido con la mano}
\onedefinition{5}
\spanishtranslation{jots'}
\clarification{diente}
\onedefinition{6}
\spanishtranslation{lech}
\clarification{alimento}
\onedefinition{7}
\spanishtranslation{luch}
\clarification{alimento o agua con taza}
\onedefinition{8}
\spanishtranslation{\textsuperscript{2}pejtyel}
\clarification{la olla del fuego}
\onedefinition{9}
\spanishtranslation{p'ik}
\clarification{con aguja o palillo}
\onedefinition{10}
\spanishtranslation{säy}
\clarification{del agua con la mano}
\onedefinition{11}
\spanishtranslation{wejk'añ}
\clarification{agua}
\secondaryentry{sacado}
\secondtranslation{jots'bil}
\secondaryentry{sacar líquido con las manos}
\secondtranslation{\textsuperscript{1}loch', loch'ilañ}
\secondaryentry{sacar tripa}
\secondtranslation{\textsuperscript{1}p'o'}

\entry{sacerdote}
\partofspeech{m}
\spanishtranslation{*ch'ujutyaty}

\entry{sacristán}
\partofspeech{m}
\spanishtranslation{xchuwaña}
\clarification{Tila}

\entry{sacudir}
\partofspeech{vt}
\onedefinition{1}
\spanishtranslation{k'ojlañ, tyijtyiñ}
\onedefinition{2}
\spanishtranslation{lijkañ}
\clarification{ropa}
\onedefinition{3}
\spanishtranslation{tyijkañ}
\clarification{cualquier cosa}
\onedefinition{4}
\spanishtranslation{tyojtyoñ}
\clarification{tierra de una canasta}
\onedefinition{5}
\spanishtranslation{yujkuñ}
\clarification{planta, árbol}

\entry{sal}
\partofspeech{f}
\spanishtranslation{ats'am}
\secondaryentry{sin sal}
\secondtranslation{sup}

\entry{salado}
\partofspeech{adj}
\secondaryentry{está salado}
\secondtranslation{iyäts'mil}

\entry{salamandra}
\partofspeech{f}
\onedefinition{1}
\spanishtranslation{ses p'ok}
\onedefinition{2}
\spanishtranslation{ajluk'}
\clarification{venenosa}

\entry{salir}
\partofspeech{vi}
\onedefinition{1}
\spanishtranslation{lok'el}
\onedefinition{2}
\spanishtranslation{pasel}
\clarification{el sol}

\entry{saliva}
\partofspeech{f}
\spanishtranslation{*sits', *tyujb, ya'lel lakej}

\entry{salpullido}
\partofspeech{m}
\spanishtranslation{tsaja bul'ich}

\entry{salud}
\partofspeech{f}
\spanishtranslation{k'ok'lel}
\clarification{del cuerpo}

\entry{saludar}
\partofspeech{vt}
\onedefinition{1}
\spanishtranslation{ña'iñ}
\clarification{a una vieja}
\onedefinition{2}
\spanishtranslation{ujts'iñ}
\clarification{besando la mano o tocando la frente}

\entry{saludo}
\partofspeech{m}
\spanishtranslation{kotyañety, kotyresiya}

\entry{salvación}
\partofspeech{vt}
\spanishtranslation{koltyäñtyel}

\entry{salvador}
\partofspeech{m}
\onedefinition{1}
\spanishtranslation{xkoltyaya}
\onedefinition{2}
\spanishtranslation{ajkoltyaya}
\clarification{Sab.}

\entry{salvar}
\partofspeech{vt}
\spanishtranslation{koltyañ}
\secondaryentry{salvado}
\secondtranslation{koltyäbil}

\entry{San Cristóbal de Las Casas}
\spanishtranslation{Jobel}

\entry{sanar}
\onedefinition{1}
\partofspeech{vt}
\spanishtranslation{k'ok'esañ, lajmesañ}
\onedefinition{2}
\partofspeech{vi}
\spanishtranslation{k'ok'añ}
\secondaryentry{sanarse}
\secondtranslation{lajmel}

\entry{sangre}
\partofspeech{f}
\spanishtranslation{ch'ich'}
\secondaryentry{sangre de dragón}
\secondtranslation{xch'ich'bäty}
\clarification{torote; árbol}

\entry{sano}
\partofspeech{adj}
\spanishtranslation{k'ok'}

\entry{santo}
\onedefinition{1}
\partofspeech{m}
\spanishtranslation{sañtyo}
\onedefinition{2}
\partofspeech{adj}
\spanishtranslation{ch'ujul}
\clarification{persona}

\entry{sapillo}
\partofspeech{m}
\spanishtranslation{ty'äsläk}
\clarification{granos de la carne del cerdo}

\entry{sapo}
\partofspeech{m}
\onedefinition{1}
\spanishtranslation{xpekejk}
\onedefinition{2}
\spanishtranslation{pokok, popok}
\clarification{Sab.}
\onedefinition{3}
\spanishtranslation{xpokok, x'oñkoñak}
\clarification{grande}

\entry{sarampión}
\partofspeech{m}
\spanishtranslation{pulibäl}

\entry{sardina}
\partofspeech{f}
\secondaryentry{sardina plateada}
\secondtranslation{stsats}
\clarification{pez}

\entry{sarna}
\partofspeech{f}
\spanishtranslation{sal}

\entry{sastre}
\partofspeech{m}
\spanishtranslation{sts'isoñel}

\entry{satisfecho}
\partofspeech{adj}
\spanishtranslation{tyijikña}

\entry{savia}
\partofspeech{f}
\spanishtranslation{*yetsel}

\entry{secar}
\partofspeech{vt}
\spanishtranslation{tyikesañ}
\secondaryentry{secarse}
\secondtranslation{ts'u'umiyel}

\entry{seco}
\partofspeech{adj}
\onedefinition{1}
\spanishtranslation{tyäkiñ}
\clarification{Sab.}
\onedefinition{2}
\spanishtranslation{tyikiñ}
\clarification{camino, madera, café, arroyo}
\onedefinition{3}
\spanishtranslation{tsäk'om, tsäk'ojm}
\clarification{frijol}
\onedefinition{4}
\spanishtranslation{chäkbulañ}
\clarification{tierra}

\entry{secretario}
\partofspeech{m}
\spanishtranslation{sts'ijbaya}

\entry{seguido}
\partofspeech{adv}
\spanishtranslation{belel}
\secondaryentry{seguidamente}
\secondtranslation{chäkächäkäjax}
\clarification{se enferma}

\entry{seguir}
\partofspeech{vt}
\onedefinition{1}
\spanishtranslation{esmañ}
\onedefinition{2}
\spanishtranslation{tyäp'leñ}
\clarification{a una persona}
\onedefinition{3}
\spanishtranslation{ty'um}
\clarification{en un camino}
\onedefinition{4}
\spanishtranslation{tsajkañ, tsäkleñ}

\entry{seguro}
\partofspeech{adj}
\secondaryentry{seguramente}
\secondtranslation{xuk'ul}

\entry{seis}
\partofspeech{adj}
\spanishtranslation{wäkp'ejl}
\secondaryentry{de hoy en seis días}
\secondtranslation{wäkñij}

\entry{semana}
\partofspeech{f}
\secondaryentry{media semana}
\secondtranslation{uxp'ejk'iñ}

\entry{sembrado}
\partofspeech{m}
\secondaryentry{sembrado de piña}
\secondtranslation{pajch'il}

\entry{sembrador}
\partofspeech{m}
\onedefinition{1}
\spanishtranslation{xpak', xpäk'oñel}
\onedefinition{2}
\spanishtranslation{ajp'ujpuya}
\clarification{Sab.}

\entry{sembrar}
\onedefinition{1}
\partofspeech{vt}
\spanishtranslation{\textsuperscript{1}päk', cha'leñ pak'}
\clarification{maíz, frijol}
\onedefinition{2}
\partofspeech{vt}
\spanishtranslation{ts'äp}
\clarification{poste}
\onedefinition{3}
\partofspeech{vt}
\spanishtranslation{wejtyuñ}
\clarification{regado}
\onedefinition{4}
\partofspeech{vi}
\spanishtranslation{\textsuperscript{1}päjk'el}
\clarification{maíz, frijol}

\entry{semejanza}
\partofspeech{f}
\spanishtranslation{ejtyaläl}

\entry{semifuerte}
\partofspeech{adv}
\spanishtranslation{mäsmäsña}

\entry{semilla}
\partofspeech{f}
\onedefinition{1}
\spanishtranslation{\textsuperscript{1}bäk', pak'}
\onedefinition{2}
\spanishtranslation{chikixchañ}
\clarification{de un tipo de planta}

\entry{senda}
\partofspeech{f}
\spanishtranslation{alä bij}

\entry{sentar}
\partofspeech{vt}
\spanishtranslation{buchchokoñ}
\clarification{a una persona}
\spanishtranslation{ñakchokoñ}
\clarification{Sab.; a una persona}
\secondaryentry{sentado}
\secondtranslation{buchul}
\secondtranslation{k'uch buchul}
\clarification{agachado}
\secondtranslation{ñakal}
\clarification{Sab.}
\secondtranslation{ñäkbuchul}
\clarification{inclinado hacia atrás}
\secondaryentry{sentarse}
\secondtranslation{buchtyäl}
\secondtranslation{ñaktyäl}
\clarification{Sab.}
\secondtranslation{tsotytyäl}
\clarification{agachado}
\secondaryentry{sentar sobre}
\secondtranslation{buchtyañ}

\entry{sentir}
\onedefinition{1}
\partofspeech{vt}
\spanishtranslation{ubiñ}
\onedefinition{2}
\partofspeech{vi}
\spanishtranslation{ubiñtyel}

\entry{señalar}
\partofspeech{vt}
\onedefinition{1}
\spanishtranslation{tyuch', tyuch'beñ}
\clarification{con el dedo}
\onedefinition{2}
\spanishtranslation{muts'wutyañ}
\clarification{con los ojos}

\entry{separado}
\partofspeech{adj}
\spanishtranslation{xatyal, xaty wa'al}
\clarification{los pies, las patas}

\entry{séptimo}
\partofspeech{adj}
\secondaryentry{séptimo día}
\secondtranslation{wukñij}

\entry{sepulcro}
\partofspeech{m}
\onedefinition{1}
\spanishtranslation{*ch'eñal, mukoñibäl, mujkibäl}
\onedefinition{2}
\spanishtranslation{ch'eñäl}
\clarification{Sab.}

\entry{ser}
\partofspeech{vi}
\secondaryentry{así es}
\secondtranslation{che'kuyi}
\clarification{respuesta}
\secondaryentry{es de}
\secondtranslation{ch'oyol}
\secondaryentry{es de nosotros juntos}
\secondtranslation{komol jach lakcha'añ}
\secondaryentry{no sea que}
\secondtranslation{ame}
\secondaryentry{será}
\secondtranslation{batyika}
\clarification{Sab.}

\entry{sereno}
\partofspeech{m}
\spanishtranslation{tyijil, ye'eb}

\entry{serranía}
\partofspeech{f}
\spanishtranslation{wits}

\entry{servir}
\partofspeech{vt}
\spanishtranslation{yumäñ}
\secondaryentry{no sirve}
\secondtranslation{*p'ajomal}
\clarification{frijol, maíz, animales}

\entry{seso}
\partofspeech{m}
\spanishtranslation{sikojk}

\entry{si}
\partofspeech{conj}
\spanishtranslation{mi, añ ku}
\secondaryentry{si no fuera por eso}
\secondtranslation{machiki}
\secondaryentry{si no hubiera}
\secondtranslation{machik}

\entry{sí}
\partofspeech{adv}
\onedefinition{1}
\spanishtranslation{yomäch, jiñkuyi}
\clarification{afirmativo}
\onedefinition{2}
\spanishtranslation{jiñkwäyi}
\clarification{Sab.; afirmativo}
\onedefinition{3}
\spanishtranslation{uts'aty}
\clarification{bueno}
\onedefinition{4}
\spanishtranslation{jaja'}
\clarification{pues}

\entry{sidra}
\partofspeech{f}
\spanishtranslation{silaj}

\entry{siempre}
\partofspeech{adv}
\onedefinition{1}
\spanishtranslation{bej, \textsuperscript{3}ñoj, pejtyel ora}
\onedefinition{2}
\spanishtranslation{ñuj}
\clarification{Sab.}

\entry{sien}
\partofspeech{f}
\spanishtranslation{*jayel}

\entry{siete}
\partofspeech{adj}
\spanishtranslation{wukp'ejl}

\entry{silbar}
\partofspeech{vt}
\spanishtranslation{tyowiñ}
\clarification{con los dedos en la boca}

\entry{silbido}
\partofspeech{m}
\spanishtranslation{ch'uyub}

\entry{silbo}
\partofspeech{m}
\spanishtranslation{tyow}
\clarification{producido con los dedos en la boca}

\entry{símbolo}
\partofspeech{m}
\spanishtranslation{weñtya}

\entry{simpático}
\partofspeech{adj}
\spanishtranslation{kuxu ity'ojol}

\entry{síndico}
\partofspeech{m}
\spanishtranslation{síñtyiko}

\entry{sinuoso}
\partofspeech{adj}
\spanishtranslation{k'ocholmetyel, xoyometyel}
\clarification{camino, vereda}

\entry{siquiera}
\partofspeech{conj}
\secondaryentry{ni siquiera}
\secondtranslation{\textsuperscript{2}yokä}
\clarification{Sab.}

\entry{sitit}
\spanishtranslation{x'obes}
\clarification{flor de cuaresma; árbol}

\entry{sobar}
\partofspeech{vt}
\spanishtranslation{jaxuñ}
\clarification{brazo}

\entry{sobra}
\partofspeech{f}
\spanishtranslation{*kolojbal}

\entry{sobrar}
\partofspeech{vi}
\spanishtranslation{sobrajiyel}
\clarification{comida}

\entry{sobrecocido}
\partofspeech{adj}
\spanishtranslation{echem}

\entry{sobrino}
\partofspeech{m}
\spanishtranslation{bik'tyi ijts'iñ}

\entry{Sol}
\partofspeech{m}
\spanishtranslation{*ch'ujutyaty}
\secondaryentry{ponerse el sol}
\secondtranslation{bäjlel, p'äjkel k'iñ}
\clarification{Sab.}
\secondaryentry{rayos de sol}
\secondtranslation{ixojob k'iñ}

\entry{soldado}
\partofspeech{m}
\spanishtranslation{soraru}
\secondaryentry{ser soldado}
\secondtranslation{sorarujiñtyel}

\entry{soledad}
\partofspeech{f}
\spanishtranslation{*ch'ijikñiyel}

\entry{solo}
\partofspeech{adj}
\spanishtranslation{bajñel,bajañ}

\entry{sólo}
\partofspeech{adv}
\spanishtranslation{kojach, jach, putyuñ}
\secondaryentry{sólo esa vez}
\secondtranslation{kojax}
\secondaryentry{sólo ése}
\secondtranslation{jiñi jach}

\entry{soltar}
\partofspeech{vt}
\spanishtranslation{kol}

\entry{soltera}
\partofspeech{f}
\spanishtranslation{xch'oktyobä x'ixik}

\entry{sombra}
\partofspeech{f}
\spanishtranslation{axñal, *äxñilel}

\entry{sombrero}
\partofspeech{m}
\spanishtranslation{pixoläl}

\entry{sonar}
\partofspeech{vt}
\onedefinition{1}
\spanishtranslation{ty'es}
\clarification{los dedos}
\onedefinition{2}
\spanishtranslation{sijmañ}
\clarification{nariz}
\secondaryentry{así suena}
\secondtranslation{p'äklaw}
\clarification{sonido de la lluvia}
\secondaryentry{sonando fuerte}
\secondtranslation{ty'olokña}
\clarification{lluvia}

\entry{sonido}
\partofspeech{m}
\secondaryentry{sonido que se produce al dar un manotazo}
\secondtranslation{\textsuperscript{2}ch'iñlaw}
\secondaryentry{manera en que hacen sonido las hojas y palitos secos}
\secondtranslation{tsoktsokña}
\secondaryentry{sonido al abrir un refresco}
\secondtranslation{ty'os}
\secondaryentry{sonido que da una cosa sobre lámina}
\secondtranslation{chilikña}

\entry{sonoro}
\partofspeech{adj}
\spanishtranslation{tsiñtsiña}
\clarification{guitarra}

\entry{sonrisa}
\partofspeech{f}
\spanishtranslation{tse'ñal}

\entry{sonrojarse}
\partofspeech{prnl}
\spanishtranslation{chäk'añ}
\secondaryentry{sonrojado}
\secondtranslation{chäkwaxañ}

\entry{sonzapote}
\partofspeech{m}
\spanishtranslation{pi}
\clarification{zapote amarillo; árbol}

\entry{soñar}
\partofspeech{vt}
\spanishtranslation{ñajleñ, cha'leñ ñajal}

\entry{soplar}
\partofspeech{vt}
\onedefinition{1}
\spanishtranslation{wujtyañ}
\onedefinition{2}
\spanishtranslation{wejlañ}
\clarification{con abanico}
\onedefinition{3}
\spanishtranslation{wus}
\clarification{hule, con soplo}
\secondaryentry{la manera de soplar}
\secondtranslation{jäpjäkña}
\clarification{susurrando}

\entry{sordo}
\onedefinition{1}
\partofspeech{adj}
\spanishtranslation{kojk}
\onedefinition{2}
\partofspeech{m}
\spanishtranslation{xkojk}

\entry{sosa}
\partofspeech{f}
\spanishtranslation{pajäl}
\clarification{planta}

\entry{suave}
\partofspeech{adj}
\spanishtranslation{k'uñluk'añ}
\clarification{madera}

\entry{subdesarrollado}
\partofspeech{adj}
\spanishtranslation{bujlux}
\clarification{granos, plumas}

\entry{súbdito}
\partofspeech{adj}
\secondaryentry{ser súbdito}
\secondtranslation{yumäñ}

\entry{subir}
\partofspeech{vi}
\onedefinition{1}
\spanishtranslation{letsel}
\onedefinition{2}
\spanishtranslation{k'äjkel}
\clarification{a la cumbre}
\secondaryentry{subido}
\secondtranslation{letsem}
\secondaryentry{subir de precio}
\secondtranslation{letsañ}
\secondaryentry{subir y bajar los extremos}
\secondtranslation{k'achulañ, k'achuña}
\clarification{subibaja}
\secondaryentry{sube pegándose}
\secondtranslation{ñotyñotyña}

\entry{suciedad}
\partofspeech{f}
\onedefinition{1}
\spanishtranslation{*bibi'lel}
\onedefinition{2}
\spanishtranslation{*kujklel}
\clarification{Tila}

\entry{sucio}
\partofspeech{adj}
\onedefinition{1}
\spanishtranslation{bibesäbil, bibi'}
\onedefinition{2}
\spanishtranslation{ik'sowañ}
\clarification{ropa, hilo}
\onedefinition{3}
\spanishtranslation{mistyuñtyik, tya'tya'}
\clarification{cara}
\onedefinition{4}
\spanishtranslation{tyity}
\clarification{líquido}

\entry{sudadero}
\partofspeech{m}
\spanishtranslation{*tyasil}

\entry{sudor}
\partofspeech{m}
\spanishtranslation{bu'lich}

\entry{suegra}
\partofspeech{f}
\spanishtranslation{*ñij'añ ña'}
\clarification{Sab.}

\entry{suegros}
\onedefinition{1}
\spanishtranslation{*ä'lib}
\clarification{de mujer}
\onedefinition{2}
\spanishtranslation{*ñij'al}
\clarification{del hombre}

\entry{sueldo}
\partofspeech{m}
\spanishtranslation{kañar}

\entry{suelo}
\partofspeech{m}
\spanishtranslation{\textsuperscript{2}lum}

\entry{suelto}
\partofspeech{adj}
\spanishtranslation{kolbil}

\entry{sueño}
\partofspeech{m}
\spanishtranslation{ñajal}

\entry{suficiente}
\partofspeech{adj}
\spanishtranslation{jasäl}

\entry{sufrir}
\partofspeech{vt}
\onedefinition{1}
\spanishtranslation{pajch'iñ}
\clarification{dolores de muerte}
\onedefinition{2}
\spanishtranslation{jijts'el}
\clarification{luxación}

\entry{sumergir}
\partofspeech{vt}
\spanishtranslation{sujp'el}
\secondaryentry{sumergido}
\secondtranslation{mujläyem}

\entry{sumirse}
\partofspeech{prnl}
\spanishtranslation{lojmel}

\entry{superficie}
\partofspeech{f}
\onedefinition{1}
\spanishtranslation{*pamlel}
\onedefinition{2}
\spanishtranslation{pañtyuñ}
\clarification{Sab.; de piedra}

\entry{suplente}
\partofspeech{m}
\spanishtranslation{rejerol}

\entry{suspirar}
\partofspeech{vi}
\spanishtranslation{jak' iyoj}

\entry{sustituir}
\partofspeech{vt}
\onedefinition{1}
\spanishtranslation{ñujpañ}
\onedefinition{2}
\spanishtranslation{k'exolañ}
\clarification{cargo, nombre}

\entry{tabaco}
\partofspeech{m}
\spanishtranslation{k'ujts}

\entry{tábano}
\partofspeech{m}
\spanishtranslation{\textsuperscript{1}k'oj}
\clarification{insecto}

\entry{tabaquillo}
\partofspeech{m}
\spanishtranslation{yäxñich}
\clarification{Sab.; tabaco cimarrón; árbol}

\entry{tablero}
\partofspeech{m}
\spanishtranslation{jobeñ}
\clarification{de moler}

\entry{taburete}
\partofspeech{m}
\spanishtranslation{pechäbtye'}

\entry{tacañería}
\partofspeech{f}
\spanishtranslation{*chäklel}

\entry{tacaño}
\partofspeech{adj}
\spanishtranslation{chäk, tsäytsäyña}

\entry{taco}
\partofspeech{m}
\spanishtranslation{k'omoch}

\entry{tal}
\partofspeech{adj}
\secondaryentry{tal vez}
\secondtranslation{tyik'äl}

\entry{talismecate}
\partofspeech{m}
\spanishtranslation{xpuy ch'ajañ}
\clarification{cuero de toro; árbol}

\entry{talla}
\partofspeech{f}
\spanishtranslation{*p'isol}
\clarification{medida}

\entry{tallo}
\partofspeech{m}
\spanishtranslation{*sik'bal ixim}
\clarification{de maíz}

\entry{talón}
\partofspeech{m}
\spanishtranslation{*tyuñ'ok, *tyuñil lakok}
\clarification{del pie}

\entry{tamal}
\partofspeech{m}
\secondaryentry{petejul (reg.)}
\spanishtranslation{\textsuperscript{1}pats'}
\clarification{tamal hecho de masa y frijol tierno}

\entry{tamaño}
\partofspeech{m}
\onedefinition{1}
\spanishtranslation{*ñuklel}
\onedefinition{2}
\spanishtranslation{*ty'ustyälel}
\clarification{cosa chica y redonda}
\secondaryentry{tamaño grande}
\secondtranslation{*ñojel}
\clarification{mazorca}

\entry{tambaleante}
\partofspeech{adj}
\spanishtranslation{tyätyäk}

\entry{también}
\partofspeech{adv}
\spanishtranslation{ja'el}
\secondaryentry{así también}
\secondtranslation{che' ja'el}

\entry{tambor}
\partofspeech{m}
\spanishtranslation{lajtye', tyamboril}

\entry{tamborcillo}
\partofspeech{m}
\spanishtranslation{kolem matye' chityam}
\clarification{mamífero}

\entry{tamular}
\partofspeech{vt}
\spanishtranslation{k'uty, k'utyilañ}
\clarification{machucar chile en el molcajete}

\entry{tapa}
\partofspeech{f}
\spanishtranslation{*majk}

\entry{tapacamino}
\partofspeech{m}
\spanishtranslation{xpujyu'}
\clarification{chotacabra; ave}

\entry{tapadera}
\partofspeech{vt}
\spanishtranslation{mosil}

\entry{tapar}
\partofspeech{vt}
\spanishtranslation{mäk, mos}
\secondaryentry{tapado}
\secondtranslation{mosbil, mosol}
\secondaryentry{taparse}
\secondtranslation{mojch'añ, moskiyel}

\entry{tapesco}
\partofspeech{m}
\onedefinition{1}
\spanishtranslation{*pañtye'lel, k'iyiñtyib}
\clarification{cama}
\onedefinition{2}
\spanishtranslation{ch'akutye'}
\clarification{para secar frijol, café}
\onedefinition{3}
\spanishtranslation{*ty'uchlib}
\clarification{percha para gallinas}

\entry{tapir}
\partofspeech{m}
\onedefinition{1}
\spanishtranslation{tsimiñ}
\clarification{anteburro; mamífero}
\onedefinition{2}
\spanishtranslation{jamoñ}
\clarification{Tila; danta, anteburro; mamífero}

\entry{tapisca}
\partofspeech{f}
\spanishtranslation{k'ajbal}
\clarification{de maíz}

\entry{tapiscar}
\partofspeech{vt}
\spanishtranslation{k'aj}
\clarification{maíz}

\entry{tarántula}
\partofspeech{f}
\onedefinition{1}
\spanishtranslation{xpetstyok'}
\onedefinition{2}
\spanishtranslation{am}
\clarification{venenosa}

\entry{tardar}
\partofspeech{vi}
\spanishtranslation{jalijel, jal'añ}

\entry{tarde}
\partofspeech{f}
\secondaryentry{ya (es) muy tarde}
\secondtranslation{ik'ix}
\secondaryentry{ya es tarde}
\secondtranslation{ochix k'iñ}
\clarification{en el día}

\entry{tartamudo}
\partofspeech{m}
\spanishtranslation{xles ak'}

\entry{tatarear}
\partofspeech{vt}
\secondaryentry{tatareando}
\secondtranslation{tyäklaw}

\entry{taza}
\partofspeech{f}
\onedefinition{1}
\spanishtranslation{uch'ibäl}
\onedefinition{2}
\spanishtranslation{lucho'ja'}
\clarification{para sacar agua}

\entry{techo}
\partofspeech{m}
\spanishtranslation{*yopol otyoty, *jol otyoty}

\entry{tecolote}
\partofspeech{m}
\spanishtranslation{pujyu'}
\clarification{ave}

\entry{tejer}
\partofspeech{vt}
\spanishtranslation{\textsuperscript{1}jal}
\clarification{red, hamaca, morral, costal}

\entry{tejido}
\partofspeech{m}
\spanishtranslation{jalbal}

\entry{tejón}
\partofspeech{m}
\spanishtranslation{kojtyom, ts'uts'ub}
\clarification{mamífero}

\entry{tela}
\partofspeech{f}
\spanishtranslation{pisil}
\secondaryentry{tela que se usa para cargar criaturas}
\spanishtranslation{*kujchil}
\secondaryentry{tela que se usa para envolver}
\spanishtranslation{pixoñib}

\entry{telaraña}
\partofspeech{f}
\spanishtranslation{jalba'am}

\entry{temanchile}
\partofspeech{m}
\spanishtranslation{xtyuñich}
\clarification{tipo de chile}

\entry{temazate}
\partofspeech{m}
\spanishtranslation{xwajch' me'}
\clarification{venado colorado}

\entry{temblar}
\partofspeech{vi}
\onedefinition{1}
\spanishtranslation{bäkbäkñiyel}
\onedefinition{2}
\spanishtranslation{tsiltsilñiyel}
\clarification{de frío o miedo}
\secondaryentry{temblando}
\secondtranslation{tsiltsilña}
\clarification{de miedo}
\secondtranslation{ty'äty'äña}
\clarification{Tila; de miedo}
\secondaryentry{temblando la tierra}
\secondtranslation{ty'älty'älña}
\clarification{por terremoto o rayo}

\entry{temblor}
\partofspeech{m}
\spanishtranslation{yujkel}

\entry{temer}
\partofspeech{vt}
\spanishtranslation{bäk'ñañ, mi icha'leñbäk'eñ}

\entry{temeroso}
\partofspeech{adj}
\spanishtranslation{bäk'eñbäk'eñ jax}

\entry{temperante}
\partofspeech{m}
\spanishtranslation{stsijtye'}
\clarification{árbol}

\entry{temporada}
\partofspeech{f}
\onedefinition{1}
\spanishtranslation{k'äñ jaläjp}
\clarification{de sequía; agosto o septiembre}
\onedefinition{2}
\spanishtranslation{k'iñi ja'lel}
\clarification{de lluvia}

\entry{temprano}
\partofspeech{adv}
\spanishtranslation{seb}

\entry{tenamaste}
\partofspeech{m}
\spanishtranslation{*yokejty}
\clarification{las piedras que componen el fogón}

\entry{tenazas}
\partofspeech{f}
\spanishtranslation{lecho'k'ajk}
\clarification{para sacar tizones}

\entry{tender}
\partofspeech{vt}
\onedefinition{1}
\spanishtranslation{lich', tyich'}
\clarification{ropa, cobija}
\onedefinition{2}
\spanishtranslation{tyäs}
\clarification{al suelo}
\secondaryentry{tenderse}
\spanishtranslation{k'iytyäl}
\clarification{café, chile, frijol, ropa}
\secondaryentry{tender en el sol}
\secondtranslation{k'iyiñ}
\secondaryentry{tendido}
\secondtranslation{\textsuperscript{2}ji'il}
\clarification{en suelo}
\secondtranslation{lich'bil}
\clarification{ropa}
\secondtranslation{k'iyil}
\clarification{en suelo o tapesco}
\secondtranslation{welel}
\clarification{plano}

\entry{tenguayaca}
\partofspeech{f}
\spanishtranslation{xch'ebak}
\clarification{pez}

\entry{tepeguaje}
\partofspeech{m}
\spanishtranslation{pipäl}
\clarification{árbol}

\entry{tepescuintle}
\partofspeech{m}
\onedefinition{1}
\spanishtranslation{tye'lal}
\clarification{mamífero}
\onedefinition{2}
\spanishtranslation{jalaw}
\clarification{Sab.; mamífero}

\entry{tercer}
\partofspeech{adj}
\secondaryentry{tercer tipo}
\secondtranslation{*uxchajplel}

\entry{terminación}
\partofspeech{vt}
\spanishtranslation{jilib}

\entry{terminar}
\partofspeech{vt}
\onedefinition{1}
\spanishtranslation{ujtyesañ}
\onedefinition{2}
\spanishtranslation{joloñtyesañ}
\clarification{un trabajo}
\secondaryentry{terminar culto o fiesta}
\spanishtranslation{xoty'}
\secondaryentry{terminarse}
\secondtranslation{jilel, joloñel, ujtyel}

\entry{término}
\partofspeech{m}
\secondaryentry{término de regaño}
\secondtranslation{xpajlek'}

\entry{terreno}
\partofspeech{m}
\spanishtranslation{\textsuperscript{1}lum}

\entry{testículos}
\partofspeech{m}
\spanishtranslation{bäk' iyaty, tyuñ'aty, *yaty}

\entry{testigo}
\partofspeech{m}
\onedefinition{1}
\spanishtranslation{tyestyiku}
\onedefinition{2}
\spanishtranslation{xlotyiya}
\clarification{falso}

\entry{teta}
\partofspeech{f}
\spanishtranslation{chu', *jol chu'}

\entry{tía}
\partofspeech{f}
\spanishtranslation{*ña'jel}
\clarification{hermana de la madre}

\entry{tibio}
\partofspeech{adj}
\spanishtranslation{k'uñlajam, k'uñk'ixiñ}

\entry{tiempo}
\partofspeech{m}
\onedefinition{1}
\spanishtranslation{*yorajlel}
\onedefinition{2}
\spanishtranslation{*tyiempojlel}
\clarification{Sab.}
\secondaryentry{hace tiempo}
\secondtranslation{wajalix}
\secondaryentry{hace mucho tiempo}
\secondtranslation{oñiyi}
\secondaryentry{su tiempo}
\secondtranslation{*yäklel}
\secondaryentry{ya tiene tiempo}
\secondtranslation{añix ora}
\secondaryentry{tiempo de seca}
\secondtranslation{ajmel, k'iñ tyuñil}

\entry{tienda}
\partofspeech{f}
\onedefinition{1}
\spanishtranslation{choñoñibäl}
\onedefinition{2}
\spanishtranslation{p'olmal}
\clarification{Tila}

\entry{tierno}
\partofspeech{adj}
\secondaryentry{hacerse tierno}
\secondtranslation{echmäl}

\entry{tierra}
\partofspeech{f}
\spanishtranslation{*lumal}
\clarification{de alguien}
\secondaryentry{tierra agria}
\secondtranslation{pajlumil}
\secondaryentry{tierra caliente}
\secondtranslation{*k'ixñilel, tyikwälel}
\secondaryentry{tierra fria}
\secondtranslation{tsäwañ pañimil}
\secondaryentry{tierra negra}
\secondtranslation{*yik'el lum}

\entry{tieso}
\partofspeech{adj}
\spanishtranslation{tsätsä bichil}

\entry{tigre}
\partofspeech{m}
\spanishtranslation{bajlum}
\clarification{jaguar}
\secondaryentry{tigre americano}
\secondtranslation{xik' sajp}
\secondaryentry{tigre frijolillo}
\secondtranslation{ik'bo'lay}
\clarification{ocelote}

\entry{tigrillo}
\partofspeech{m}
\spanishtranslation{tsukbajlum}

\entry{tijera}
\partofspeech{f}
\spanishtranslation{xtyexelex}
\clarification{tipo de insecto}
\secondaryentry{tijera gris}
\secondtranslation{x'ajlum}
\clarification{mosquera tijereta; ave}

\entry{tijeras}
\partofspeech{f}
\spanishtranslation{tyexelex}

\entry{tijerilla}
\partofspeech{f}
\spanishtranslation{xäl'ity, xtyexelex}
\clarification{insecto}

\entry{timbrillo}
\partofspeech{m}
\spanishtranslation{xchejkel}
\clarification{palo de pulque; arbusto}

\entry{tinamú canelo}
\spanishtranslation{xñakow}
\clarification{perdiz canela; ave}

\entry{tiña}
\partofspeech{f}
\spanishtranslation{pejpem wuty}
\clarification{roncha}

\entry{tío}
\partofspeech{m}
\spanishtranslation{*ichañ}
\clarification{hermano de mi madre}

\entry{tipo}
\partofspeech{m}
\secondaryentry{tercer tipo}
\secondtranslation{*uxchajplel}

\entry{tirador}
\partofspeech{m}
\spanishtranslation{päräñtyuñ, ulej}

\entry{tirar}
\partofspeech{vt}
\onedefinition{1}
\spanishtranslation{chok}
\onedefinition{2}
\spanishtranslation{jul}
\clarification{con escopeta, piedra, tirador, ensartar}
\secondaryentry{tirado}
\secondtranslation{chopol}
\clarification{rama de árbol}
\secondtranslation{ch'ebel}
\clarification{boca arriba}
\secondtranslation{jäläl}
\clarification{un objeto largo}
\secondtranslation{sowol}
\clarification{trapo}

\entry{tiro}
\partofspeech{m}
\onedefinition{1}
\spanishtranslation{ibäk' juloñib}
\clarification{cartucho de arma}
\onedefinition{2}
\spanishtranslation{*wa'tyilel}
\clarification{de pantalón}

\entry{tiuca}
\partofspeech{f}
\spanishtranslation{xtyukuk}
\clarification{huilota común, tiuta; ave}

\entry{tizón}
\partofspeech{m}
\onedefinition{1}
\spanishtranslation{k'äñk'äñ uch}
\clarification{enfermedad del maíz}
\onedefinition{2}
\spanishtranslation{xutyuñtye'}
\clarification{palo quemado}

\entry{tlacuache}
\partofspeech{m}
\spanishtranslation{uch}
\clarification{zorro pelón; mamífero}

\entry{tobillo}
\partofspeech{m}
\spanishtranslation{ibik' lakok, *wuty lakok}
\clarification{lit.: el ojo de nuestro pie}

\entry{tocar}
\partofspeech{vt}
\onedefinition{1}
\spanishtranslation{tyäl}
\onedefinition{2}
\spanishtranslation{wus}
\clarification{caracol, pito, trompeta}
\secondaryentry{tocar el corazón}
\secondtranslation{ñijkäbeñ ipusik'al}

\entry{tocayo}
\partofspeech{m}
\spanishtranslation{*k'exol}

\entry{todavía}
\partofspeech{adv}
\spanishtranslation{maxtyo}
\secondaryentry{falta todavía}
\secondtranslation{añtyo yom}

\entry{todo}
\partofspeech{adj}
\onedefinition{1}
\spanishtranslation{\textsuperscript{1}laj, \textsuperscript{1}pejtyel, pejtyelel, petyol}
\onedefinition{2}
\spanishtranslation{-jump'ejlel}
\clarification{cosa entera}
\secondaryentry{todo entero}
\secondtranslation{kotyol}
\secondaryentry{todo a su alrededor}
\secondtranslation{*sutyolel}
\secondaryentry{toda clase}
\secondtranslation{jujuñchajp}

\entry{todos}
\partofspeech{pron}
\spanishtranslation{lu'}

\entry{toloque}
\partofspeech{m}
\spanishtranslation{ty'orjol}
\clarification{Sab.; basilisco, pasarríos; una lagartija}

\entry{tomador}
\partofspeech{m}
\spanishtranslation{xlemoñel}
\clarification{de aguardiente}

\entry{tomar}
\onedefinition{1}
\partofspeech{vt}
\spanishtranslation{uch'eñ}
\clarification{líquido}
\onedefinition{2}
\partofspeech{vi}
\spanishtranslation{lem}
\clarification{bebidas alcohólicas}
\onedefinition{3}
\partofspeech{vt}
\spanishtranslation{ch'äm}
\clarification{llevar, agarrar}
\onedefinition{4}
\partofspeech{vt}
\spanishtranslation{jämch'äm}
\clarification{agarrar algo rápidamente}
\onedefinition{5}
\partofspeech{vt}
\spanishtranslation{ye'}
\clarification{en la mano}
\secondaryentry{acción de tomar}
\secondtranslation{lemoñel}
\clarification{bebidas alcohólicas}
\secondaryentry{acto de tomar, beber}
\secondtranslation{uch'el}
\secondaryentry{tomar en cuenta}
\secondtranslation{ch'ujbiñ}
\clarification{Tila}

\entry{tomate}
\partofspeech{m}
\spanishtranslation{koya'}

\entry{topota}
\partofspeech{m}
\spanishtranslation{xmäläl}
\clarification{tipo de pez chico}

\entry{tórax}
\partofspeech{m}
\spanishtranslation{kuktyal}

\entry{torcer}
\partofspeech{vt}
\onedefinition{1}
\spanishtranslation{k'ach}
\onedefinition{2}
\spanishtranslation{ts'oty}
\clarification{Sab.}
\onedefinition{3}
\spanishtranslation{bälch'uñ}
\clarification{con las manos}
\secondaryentry{torcer la boca de un lado al otro}
\secondtranslation{muyilañ}

\entry{torcido}
\partofspeech{adj}
\onedefinition{1}
\spanishtranslation{k'achal, \textsuperscript{2}sewel}
\clarification{tabla}
\onedefinition{2}
\spanishtranslation{k'ochol, lochol}
\clarification{alambre, palo, camino}
\onedefinition{3}
\spanishtranslation{pexel}
\clarification{la boca}
\onedefinition{4}
\spanishtranslation{ts'otyol}
\clarification{palo}
\onedefinition{5}
\spanishtranslation{ts'otyol metyel}
\clarification{camino}

\entry{tornamilpa}
\partofspeech{vt}
\spanishtranslation{mol, sijom, sijomal}

\entry{tornear}
\partofspeech{vt}
\spanishtranslation{bujchuñ}
\clarification{mover para sacar}

\entry{toro}
\partofspeech{m}
\spanishtranslation{wakax, xtyujk'atye' wakax}

\entry{tortilla}
\partofspeech{f}
\spanishtranslation{waj}
\secondaryentry{tortilla con frijol}
\secondtranslation{bu'le waj}
\secondaryentry{tortilla amarilla}
\secondtranslation{k'äñwaj}
\clarification{Sab.}
\secondaryentry{acción de hacer tortillas}
\secondtranslation{pechom}

\entry{tortillador}
\partofspeech{m}
\spanishtranslation{we'tye'}

\entry{tortolita}
\partofspeech{f}
\spanishtranslation{xpuruwok}
\clarification{ave}

\entry{tortuga}
\partofspeech{f}
\spanishtranslation{*chäñil ja'}
\secondaryentry{tortuga cocodrilo}
\secondtranslation{ch'ix ajk}
\secondaryentry{tortuga negra}
\secondtranslation{ajk}
\secondaryentry{tortuga chica}
\secondtranslation{ajkok}
\clarification{Sab.}

\entry{tos}
\partofspeech{f}
\spanishtranslation{ojbal}
\secondaryentry{tos ferina}
\secondtranslation{jijik' ojbal}

\entry{toser}
\partofspeech{vi}
\spanishtranslation{mi icha'leñ ojbal}

\entry{tostado}
\partofspeech{m}
\secondtranslation{woch'}
\clarification{tortilla, madera}

\entry{tostar}
\partofspeech{vt}
\onedefinition{1}
\spanishtranslation{woch'esañ}
\clarification{tortilla}
\onedefinition{2}
\spanishtranslation{ch'il}
\clarification{café, maíz}
\secondaryentry{tostado}
\secondtranslation{woch'esäbil}
\clarification{tortilla, por alguien}

\entry{trabado}
\onedefinition{1}
\partofspeech{adv}
\spanishtranslation{\textsuperscript{2}kats'}
\onedefinition{2}
\partofspeech{adj}
\spanishtranslation{kets'}

\entry{trabajador}
\partofspeech{m}
\spanishtranslation{x'e'tyel}

\entry{trabajar}
\partofspeech{vi}
\spanishtranslation{mi icha'leñ e'tyel}

\entry{trabajo}
\partofspeech{m}
\spanishtranslation{*e'tyel}
\secondaryentry{equipo de trabajo}
\secondtranslation{e'tyijibäl}
\secondaryentry{trabajo comunal}
\secondtranslation{komol e'tyel}
\secondaryentry{trabajo pequeño}
\secondtranslation{tyoñel}
\clarification{Tila}
\secondaryentry{trabajo grande}
\secondtranslation{tyroñel}
\clarification{Tila}

\entry{traer}
\partofspeech{vt}
\onedefinition{1}
\spanishtranslation{ch'äm tyilel, päy tyilel}
\onedefinition{2}
\spanishtranslation{wets'}
\clarification{gente, animales}
\secondaryentry{traer agarrado}
\secondtranslation{chuk tyilel}
\secondaryentry{traer carga}
\secondtranslation{bejlañ}
\clarification{en varios viajes}

\entry{tragar}
\partofspeech{vt}
\onedefinition{1}
\spanishtranslation{mäsañ}
\onedefinition{2}
\spanishtranslation{buk'}
\clarification{Sab.}

\entry{traicionero}
\partofspeech{m}
\spanishtranslation{judas}
\clarification{Sab.}

\entry{tramo}
\partofspeech{m}
\secondaryentry{por tramos}
\secondtranslation{kepekña}

\entry{trampa}
\partofspeech{f}
\spanishtranslation{yak}

\entry{tranquilo}
\partofspeech{adj}
\onedefinition{1}
\spanishtranslation{ch'iñch'iña, ñäjch'em}
\onedefinition{2}
\spanishtranslation{ts'ämäl}
\clarification{agua, río}
\onedefinition{3}
\spanishtranslation{sämäl}
\clarification{líquido}

\entry{transformar}
\partofspeech{vt}
\spanishtranslation{päñtyesañ}

\entry{trasto}
\partofspeech{m}
\spanishtranslation{uch'ijibäl}

\entry{trastornar}
\partofspeech{vt}
\secondaryentry{trastornarse}
\secondtranslation{sojkel}

\entry{tratable}
\partofspeech{adj}
\secondaryentry{es tratable}
\secondtranslation{k'uñ ipusik'al}

\entry{tratar}
\partofspeech{vt}
\spanishtranslation{\textsuperscript{2}jop}
\clarification{Sab.}
\secondaryentry{tratado}
\secondtranslation{ts'äkäbil}
\clarification{con medicina}

\entry{travesaño}
\partofspeech{m}
\spanishtranslation{*k'ätyloñtye'el}
\clarification{de casa}

\entry{trece}
\partofspeech{adj}
\spanishtranslation{uxlujump'ejl}

\entry{treinta}
\partofspeech{adj}
\spanishtranslation{lujump'ejl icha'k'al}

\entry{tremedal}
\partofspeech{m}
\spanishtranslation{ja' lumil}
\clarification{aguazal}

\entry{trementina}
\partofspeech{f}
\spanishtranslation{*xu'ch'il, *yetsel tye'}

\entry{trenzar}
\partofspeech{vt}
\secondaryentry{trenzado}
\secondtranslation{jalbil}

\entry{trepador}
\partofspeech{m}
\spanishtranslation{xchäkerech'}
\clarification{trepatroncos; ave}
\secondaryentry{trepadora anolis}
\secondtranslation{xmañchaje}

\entry{trepar}
\partofspeech{vt}
\secondaryentry{trepado}
\secondtranslation{ty'uchul}
\clarification{pollo en un palo}

\entry{trepatroncos gigante}
\spanishtranslation{ñoñochtye'}
\clarification{ave}

\entry{tres}
\partofspeech{adj}
\spanishtranslation{uxp'ejl}
\secondaryentry{tres noches de dormir}
\secondtranslation{uxwäyel}
\secondaryentry{tres veces}
\secondtranslation{uxyajl}

\entry{trescientos}
\partofspeech{adj}
\spanishtranslation{jo'lujuñk'al}

\entry{tribunal}
\partofspeech{m}
\spanishtranslation{melo'mulil}

\entry{Trinidad}
\partofspeech{f}
\spanishtranslation{*tyiñeral}
\clarification{un pueblo}

\entry{tripa}
\partofspeech{vt}
\spanishtranslation{soytya'}

\entry{triste}
\partofspeech{adj}
\onedefinition{1}
\spanishtranslation{ch'ijiyem}
\onedefinition{2}
\spanishtranslation{k'uxtyayem}
\clarification{Sab.}
\onedefinition{3}
\spanishtranslation{ch'ityikña}
\clarification{por estar ausente alguien}
\secondaryentry{triste y agachado}
\secondtranslation{lutsukña}
\clarification{gallo, gallina}
\secondaryentry{está triste}
\secondtranslation{ch'ijiyem ipusik'al, peñsaliñ}
\clarification{Sab.}
\secondaryentry{ponerse triste}
\secondtranslation{mel}

\entry{tristeza}
\partofspeech{f}
\spanishtranslation{*ch'ijikñiyel, peñsar, *xäp'äkñäyel}

\entry{troje}
\partofspeech{m}
\spanishtranslation{yotylel ixim}

\entry{trompo}
\partofspeech{m}
\spanishtranslation{setyejty}

\entry{tronar}
\onedefinition{1}
\partofspeech{vi}
\spanishtranslation{tyojmel}
\onedefinition{2}
\partofspeech{vt}
\spanishtranslation{tyojmesañ}
\clarification{cohete}

\entry{tronco}
\partofspeech{m}
\onedefinition{1}
\spanishtranslation{xujk'uñtye'}
\onedefinition{2}
\spanishtranslation{*xujk'uñtye'lel}
\clarification{árbol, maíz, plátano}

\entry{tropezar}
\partofspeech{vt}
\spanishtranslation{p'äk' oka, p'ojkiñ, p'osiñ}

\entry{trueque}
\partofspeech{m}
\spanishtranslation{*k'exol}

\entry{trunco}
\partofspeech{adj}
\spanishtranslation{borol, borolbä tye'}

\entry{tsílica}
\partofspeech{f}
\spanishtranslation{mäyäjl}
\clarification{tipo de calabaza}

\entry{tú}
\partofspeech{pron}
\spanishtranslation{jatyety}

\entry{tubería}
\partofspeech{f}
\spanishtranslation{*tyilib}

\entry{tucán}
\partofspeech{m}
\spanishtranslation{kolem päm}
\clarification{ave}
\secondaryentry{tucán cuello amarillo}
\secondtranslation{päm}
\clarification{ave}

\entry{tucancillo collarejo}
\onedefinition{1}
\spanishtranslation{xpiñtsik}
\onedefinition{2}
\spanishtranslation{xch'aj päm}
\clarification{pitorreal; ave}

\entry{tucaneta verde}
\spanishtranslation{bik'tyi päm}
\clarification{tucancillo; ave}

\entry{tuétano}
\partofspeech{m}
\spanishtranslation{yojlom}

\entry{tumbar}
\partofspeech{vt}
\spanishtranslation{sek'}
\secondaryentry{tumbado}
\secondtranslation{sek'el}

\entry{tumor}
\partofspeech{m}
\spanishtranslation{chäkajl, yaj}

\entry{tupido}
\partofspeech{adj}
\spanishtranslation{pim}

\entry{turbulento}
\partofspeech{adj}
\spanishtranslation{wamlaw, watylaw}
\clarification{agua}

\entry{turco real}
\partofspeech{m}
\spanishtranslation{xwukip}
\clarification{péndulo de corona; ave}

\entry{turnar}
\partofspeech{vi}
\spanishtranslation{jel}

\entry{tuza}
\partofspeech{f}
\onedefinition{1}
\spanishtranslation{\textsuperscript{1}baj}
\clarification{mamífero}
\onedefinition{2}
\spanishtranslation{ajbaj}
\clarification{Sab.; mamífero}

\entry{tzeltal}
\partofspeech{m}
\secondaryentry{hombre tzetal}
\spanishtranslation{amiku}
\secondaryentry{mujer tzetal}
\spanishtranslation{x'amiku}

\entry{tzumi}
\spanishtranslation{tyame ch'ijty}
\clarification{caimito, palo de canela; árbol}

\entry{ubre}
\partofspeech{f}
\spanishtranslation{chu'}

\entry{uco}
\partofspeech{m}
\spanishtranslation{ujchib, xuchijp}
\clarification{agutí; mamífero}

\entry{ujtui}
\spanishtranslation{luluy}
\clarification{jobo, ciruelo; árbol}

\entry{último}
\partofspeech{adj}
\spanishtranslation{kojix, wi'ilix}
\secondaryentry{última fruta}
\secondtranslation{*ñi' iwuty}
\secondaryentry{último hijo}
\secondtranslation{xutyäl}

\entry{único}
\partofspeech{adj}
\spanishtranslation{kojoñ}
\secondaryentry{únicamente}
\secondtranslation{putyuñ}

\entry{unir}
\partofspeech{vt}
\spanishtranslation{ñuts}

\entry{uno}
\partofspeech{adj}
\onedefinition{1}
\spanishtranslation{jump'ejl}
\onedefinition{2}
\spanishtranslation{juñsujm}
\clarification{género, clase}
\secondaryentry{cada uno}
\secondtranslation{jujump'ejl}
\secondaryentry{uno por uno}
\secondtranslation{jujump'ejl}

\entry{untar}
\partofspeech{vt}
\spanishtranslation{jaxuñ}
\clarification{grasa a una tortilla}

\entry{uña}
\partofspeech{f}
\spanishtranslation{ejk'ach}
\secondaryentry{uña de gato}
\secondtranslation{lujchuñ ej}
\clarification{bejuco}

\entry{urgente}
\partofspeech{adj}
\secondaryentry{ser urgente}
\secondtranslation{yomojax}

\entry{usado}
\partofspeech{adj}
\onedefinition{1}
\spanishtranslation{k'ämbil}
\onedefinition{2}
\spanishtranslation{wäytyäbil}
\clarification{para dormir}

\entry{usar}
\partofspeech{vt}
\onedefinition{1}
\spanishtranslation{si'ilañ}
\clarification{como leña}
\onedefinition{2}
\spanishtranslation{we'ibañ}
\clarification{para comer}
\secondaryentry{usar para dormir}
\secondtranslation{wäytyañ}
\secondaryentry{usarse}
\secondtranslation{k'äjñel, k'ämbil}

\entry{utilidad}
\partofspeech{vt}
\spanishtranslation{*k'äjñibal}

\entry{uva}
\partofspeech{f}
\onedefinition{1}
\spanishtranslation{ts'usub}
\onedefinition{2}
\spanishtranslation{xpajtyo' tsuk}
\clarification{silvestre}

\entry{úvula}
\partofspeech{f}
\spanishtranslation{ts'ujk, *ts'ujkil lakbik'}

\entry{vaca}
\partofspeech{f}
\spanishtranslation{wakax}

\entry{vaciar}
\partofspeech{vt}
\spanishtranslation{wuj}
\clarification{maíz, frijol, café, arroz}

\entry{vacío}
\partofspeech{adj}
\spanishtranslation{jochol}

\entry{vadoso}
\partofspeech{adj}
\spanishtranslation{säp'äl}

\entry{valiente}
\partofspeech{adj}
\spanishtranslation{ch'ejl, ch'ejl ipusik'al}

\entry{valle}
\partofspeech{m}
\spanishtranslation{joktyäl}

\entry{valor}
\partofspeech{m}
\spanishtranslation{*tyojol}

\entry{vano}
\partofspeech{m}
\spanishtranslation{pojts'emal}
\clarification{del café}
\secondaryentry{en vano}
\secondtranslation{tyo'ol jach}

\entry{vapor}
\partofspeech{m}
\spanishtranslation{yowix}

\entry{vara}
\partofspeech{f}
\secondaryentry{varita}
\secondtranslation{tye' bu'ul}
\clarification{para que suba el frijol}

\entry{varilla}
\partofspeech{f}
\spanishtranslation{*bik'itye'lel}
\clarification{del techo}
\secondaryentry{varillas del techo}
\secondtranslation{*säktye'lel}

\entry{varisco}
\partofspeech{m}
\spanishtranslation{xchilib}
\clarification{planta usada como escoba}

\entry{vasija}
\partofspeech{f}
\spanishtranslation{pokik'äbäl}

\entry{vecino}
\partofspeech{m}
\spanishtranslation{pi'äl, pi'äl tyi chumtyäl}

\entry{veinte}
\partofspeech{adj}
\spanishtranslation{juñk'al}
\secondaryentry{veintiuno}
\secondtranslation{jump'ejl icha'k'al}

\entry{vejiga}
\partofspeech{f}
\spanishtranslation{kuchib pich, *chujyil ipich, *wosmäjel}

\entry{vela}
\partofspeech{f}
\spanishtranslation{ñichim}

\entry{vello}
\partofspeech{m}
\spanishtranslation{*tsutsel}
\clarification{de hombre}
\spanishtranslation{*tsutsel laj k'äb}

\entry{vena}
\partofspeech{f}
\spanishtranslation{*chijil}

\entry{venado}
\partofspeech{m}
\spanishtranslation{me'}
\secondaryentry{venado colorado}
\secondtranslation{xwajch' me'}
\clarification{temazate mediano}
\secondaryentry{venado cabrito}
\secondtranslation{chäkme'}
\clarification{temazate chico}
\secondaryentry{venado azul}
\secondtranslation{yäxme'}
\clarification{grande}
\secondaryentry{venado común}
\secondtranslation{chijmay}
\clarification{con cola blanca}
\secondtranslation{säkme'}
\clarification{Tila}

\entry{vencejo listado}
\spanishtranslation{xwilis}
\clarification{vencejo collarejo; ave}

\entry{vencer}
\partofspeech{vt}
\spanishtranslation{mäl}

\entry{venda}
\partofspeech{f}
\spanishtranslation{*bejch'il}

\entry{vendedor}
\partofspeech{m}
\spanishtranslation{xchoñoñel}

\entry{vender}
\partofspeech{vt}
\spanishtranslation{choñ}
\secondaryentry{vendido}
\secondtranslation{chombil}

\entry{veneno}
\partofspeech{m}
\spanishtranslation{*chämib, chämibäl}

\entry{venir}
\partofspeech{vi}
\onedefinition{1}
\spanishtranslation{tyilel}
\onedefinition{2}
\spanishtranslation{tyälel}
\clarification{Sab.}
\secondaryentry{ven}
\secondtranslation{¡la'!}
\secondaryentry{vengo}
\secondtranslation{tyaloñ}
\secondaryentry{viene}
\secondtranslation{tyal}
\secondaryentry{viene de}
\secondtranslation{ch'oyol}

\entry{venoso}
\partofspeech{adj}
\spanishtranslation{chij ok, lo'chijtyik}
\clarification{en los pies}

\entry{ventana}
\partofspeech{f}
\spanishtranslation{k'eloñib}

\entry{ver}
\partofspeech{vt}
\spanishtranslation{\textsuperscript{1}*ilañ, k'el}
\secondaryentry{a ver}
\secondtranslation{la'tyika}
\clarification{Sab.}
\secondaryentry{ser visto}
\secondtranslation{k'ejlel}
\secondaryentry{ver con algún instrumento}
\secondtranslation{ñejñañ}
\secondaryentry{ya ves}
\secondtranslation{awilañ, meku}

\entry{verdad}
\partofspeech{f}
\spanishtranslation{*sujmlel}
\secondaryentry{verdad que así}
\secondtranslation{ñomach}
\secondaryentry{es verdad}
\secondtranslation{k'o'o'}
\clarification{Sab.}

\entry{verdadero}
\partofspeech{adj}
\spanishtranslation{isujm}
\secondaryentry{verdaderamente}
\secondtranslation{yokeñ}
\clarification{Sab.}

\entry{verde}
\partofspeech{adj}
\onedefinition{1}
\spanishtranslation{yäjyäx, *yäxel, yäxmulañ}
\clarification{color}
\onedefinition{2}
\spanishtranslation{yax}
\clarification{no maduro}
\onedefinition{3}
\spanishtranslation{yäxmojañ}
\clarification{reflejado}
\secondaryentry{verde todavía}
\secondtranslation{ch'oktyo}

\entry{verdoso}
\partofspeech{adj}
\spanishtranslation{yäxkich'añ}

\entry{verdura}
\partofspeech{f}
\secondaryentry{tipo de verdura}
\secondtranslation{axäñtye'}
\clarification{tiene hojas grandes y suaves; se comen}
\secondaryentry{verdura que comen los guajolotes}
\secondtranslation{*yuk'}

\entry{vereda}
\partofspeech{f}
\spanishtranslation{alä bij, k'ästye' bij, tyuk'o' bij}

\entry{vergonzoso}
\partofspeech{adj}
\secondaryentry{vergonzosamente}
\secondtranslation{kisiñtyik}

\entry{vergüenza}
\partofspeech{f}
\onedefinition{1}
\spanishtranslation{kisiñ}
\onedefinition{2}
\spanishtranslation{kisñil}
\clarification{Tila}
\onedefinition{3}
\spanishtranslation{yäjil}
\clarification{Sab.}

\entry{verruga}
\partofspeech{f}
\spanishtranslation{*4päk'}

\entry{vértice}
\partofspeech{m}
\spanishtranslation{pam}

\entry{vertiente}
\partofspeech{f}
\onedefinition{1}
\spanishtranslation{ja', *lok'ib ja'}
\onedefinition{2}
\spanishtranslation{tyokol ja'}
\clarification{de apertura}
\onedefinition{3}
\spanishtranslation{ch'eñbä ja'}
\clarification{nombre de una vertiente}

\entry{vestido}
\partofspeech{m}
\spanishtranslation{\textsuperscript{2}p'o'}
\clarification{Tila}

\entry{vestir}
\partofspeech{vt}
\spanishtranslation{xoj}

\entry{vez}
\partofspeech{f}
\secondaryentry{a veces}
\secondtranslation{añk'iñil, käñkäñ, \textsuperscript{1}tyajol}
\secondaryentry{cada vez}
\secondtranslation{jajayajl, jujuñyajl}
\secondaryentry{cada vez más}
\secondtranslation{utsi}
\secondaryentry{de una vez}
\secondtranslation{juñyajlel}
\secondaryentry{repetidas veces}
\secondtranslation{pik'os}
\clarification{amarrar}
\secondaryentry{tal vez}
\secondtranslation{tyik'äl}
\secondaryentry{una sola vez}
\secondtranslation{juññumel jach}
\secondaryentry{varias veces}
\secondtranslation{chä'chäk'}
\clarification{acción repetida}

\entry{Vía Láctea}
\partofspeech{f}
\spanishtranslation{Pa'as lum}

\entry{víbora}
\partofspeech{f}
\spanishtranslation{tsuñtye'chañ}
\clarification{verde, venenosa}

\entry{viborina}
\partofspeech{f}
\spanishtranslation{x'al mis}
\clarification{hierba medicinal}

\entry{vida}
\partofspeech{f}
\spanishtranslation{*kuxtyälel}

\entry{viejo}
\onedefinition{1}
\partofspeech{adj}
\spanishtranslation{tsukul}
\clarification{cualquier objeto}
\onedefinition{2}
\partofspeech{adj}
\spanishtranslation{ñox, xñox}
\clarification{persona}
\onedefinition{3}
\partofspeech{m}
\spanishtranslation{kolib}
\secondaryentry{ropa vieja}
\secondtranslation{tsukul pisil}
\secondaryentry{ya está viejo}
\secondtranslation{ñoxix}

\entry{viejo de monte}
\spanishtranslation{sakol}
\clarification{cabeza blanca; mamífero}

\entry{viento}
\partofspeech{m}
\spanishtranslation{ik'}
\secondaryentry{viento leve}
\secondtranslation{*jajts'}
\secondaryentry{viento fuerte}
\secondtranslation{ñuki ik'}
\secondtranslation{ña'ik'}
\clarification{madre de viento}

\entry{viga}
\partofspeech{f}
\spanishtranslation{kukujl}
\clarification{lateral de una casa}

\entry{vigilar}
\partofspeech{vt}
\spanishtranslation{yäx iyo}

\entry{viña}
\partofspeech{f}
\onedefinition{1}
\spanishtranslation{ts'usubil}
\onedefinition{2}
\spanishtranslation{ts'usu'ol}
\clarification{Sab.}

\entry{violar}
\partofspeech{vt}
\spanishtranslation{ñolch'iñ}
\clarification{a una mujer}

\entry{violento}
\partofspeech{adj}
\spanishtranslation{simaroñ}

\entry{visible}
\partofspeech{adj}
\spanishtranslation{k'ajpa'añ, tsikil}

\entry{visita}
\partofspeech{f}
\spanishtranslation{jula'al}
\clarification{Tila}

\entry{visitante}
\partofspeech{m}
\spanishtranslation{jula'}

\entry{visitar}
\partofspeech{vt}
\onedefinition{1}
\spanishtranslation{\textsuperscript{1}*ilañ}
\onedefinition{2}
\spanishtranslation{jula'añ, jula'tyañ}
\clarification{Sab.}

\entry{vista}
\partofspeech{f}
\secondaryentry{a la vista}
\secondtranslation{k'ajpam}

\entry{viuda}
\partofspeech{f}
\spanishtranslation{meba' x'ixik}

\entry{viudo}
\partofspeech{m}
\spanishtranslation{meba', meba' wiñik}

\entry{vivir}
\partofspeech{vi}
\spanishtranslation{kuxtyiyel, chumtyäl}
\secondaryentry{vivir y trabajar con un patrón por un sueldo bajo}
\secondtranslation{mosojäñtyel}

\entry{vivo}
\partofspeech{adj}
\spanishtranslation{kuxul}

\entry{vociferar}
\partofspeech{vi}
\secondaryentry{vociferando}
\secondtranslation{ju'ju'ña}

\entry{volar}
\partofspeech{vi}
\spanishtranslation{wejlel}
\secondaryentry{volando}
\secondtranslation{welekña, welwelña}

\entry{volcar}
\partofspeech{vt}
\secondaryentry{volcarse}
\secondtranslation{xejwel}

\entry{voltear}
\partofspeech{vt}
\onedefinition{1}
\spanishtranslation{sutyk'iñ}
\onedefinition{2}
\spanishtranslation{mäy}
\clarification{la cara}
\onedefinition{3}
\spanishtranslation{wäts'}
\clarification{objeto}
\secondaryentry{volteado}
\secondtranslation{wäts'äl}

\entry{vomitar}
\partofspeech{vt}
\spanishtranslation{xejtyañ}

\entry{vómito}
\partofspeech{m}
\spanishtranslation{xej}

\entry{voz}
\partofspeech{f}
\secondaryentry{voz fuerte}
\secondtranslation{k'ambä ty'añ}

\entry{vuelta}
\partofspeech{f}
\onedefinition{1}
\spanishtranslation{xoy bij}
\clarification{sing.}
\spanishtranslation{sutyujty}
\clarification{pl.}
\onedefinition{2}
\spanishtranslation{joyokñabä bij, joyol bij}
\clarification{desviación}
\secondaryentry{dando vueltas}
\secondtranslation{ñoliña}
\secondaryentry{dar vuelta}
\secondtranslation{\textsuperscript{2}xoy, wilts'uñ}
\secondaryentry{dar vueltas}
\secondtranslation{wijlel}
\secondaryentry{vuelta de camino}
\secondtranslation{joy bij}

\entry{vuelto}
\partofspeech{m}
\spanishtranslation{sujtyib}
\clarification{del dinero}

\entry{y}
\partofspeech{conj}
\spanishtranslation{yik'oty}

\entry{ya}
\partofspeech{adv}
\spanishtranslation{mux, tsa'ix}
\secondaryentry{ya no}
\secondtranslation{jaymejl}
\secondaryentry{ya pasó}
\secondtranslation{xñumi}
\secondaryentry{ya quiere}
\secondtranslation{yomox}
\secondaryentry{ya ves}
\secondtranslation{awilañ, meku}

\entry{yagual}
\partofspeech{m}
\spanishtranslation{ch'äch'ak'}
\clarification{canasta para guardar pozol}

\entry{yema}
\partofspeech{f}
\spanishtranslation{*k'äñel}
\clarification{de huevo}

\entry{yerno}
\partofspeech{m}
\spanishtranslation{*ñij'al}

\entry{yo}
\partofspeech{pron}
\spanishtranslation{joñoñ}

\entry{yolosóchil}
\partofspeech{m}
\spanishtranslation{xjol max}
\clarification{flor de corazón; árbol}

\entry{yuca}
\partofspeech{f}
\spanishtranslation{ts'ijñ}

\entry{yugo}
\partofspeech{m}
\spanishtranslation{*k'äklib ikuch}

\entry{yulo}
\partofspeech{m}
\spanishtranslation{jochojch}
\clarification{tipo de gusano}

\entry{zacatal}
\partofspeech{m}
\onedefinition{1}
\spanishtranslation{jamil}
\onedefinition{2}
\spanishtranslation{majamol}
\clarification{Sab.}

\entry{zacate}
\partofspeech{m}
\onedefinition{1}
\spanishtranslation{\textsuperscript{2}jam}
\onedefinition{2}
\spanishtranslation{chäk'ak}
\clarification{para techar la casa}
\secondaryentry{tipo de zacate que da flor}
\secondtranslation{chup jam}
\secondaryentry{tipo de zacate que sirve para hacer un té}
\secondtranslation{limóñ jam}
\secondaryentry{zacate duro como caña}
\secondtranslation{tye'jam}

\entry{zacatón}
\partofspeech{m}
\spanishtranslation{xtyuch' k'iñ}

\entry{zacua gigante}
\partofspeech{f}
\spanishtranslation{k'ubul}
\clarification{ave}

\entry{zacuilla}
\partofspeech{f}
\spanishtranslation{k'ubujl}
\clarification{zanate de oro; ave}

\entry{zafar}
\partofspeech{vt}
\spanishtranslation{chip}
\secondaryentry{zafarse}
\secondtranslation{chijpel}
\clarification{pantalón}

\entry{zambullir}
\partofspeech{vt}
\spanishtranslation{\textsuperscript{1}sul}

\entry{zanate de oro}
\spanishtranslation{k'ubujl}
\clarification{ave}

\entry{zancudo}
\partofspeech{m}
\spanishtranslation{uch'ja'}

\entry{zangolotear}
\partofspeech{vt}
\spanishtranslation{k'ojlañ}

\entry{zanja}
\partofspeech{f}
\spanishtranslation{*bijleyok ja', *yok ja'lel}

\entry{zanjón}
\partofspeech{m}
\spanishtranslation{japtyäl}

\entry{zapato}
\partofspeech{m}
\spanishtranslation{xäñäbäl, \textsuperscript{2}pats'}
\clarification{Sab.}

\entry{zapote}
\partofspeech{m}
\spanishtranslation{way ja'as}
\clarification{árbol}
\secondaryentry{zapote prieto}
\secondtranslation{i'ik'bä way ja'as}
\clarification{zapote negro; árbol}
\secondaryentry{zapote amarillo}
\secondtranslation{pi}
\secondaryentry{zapote de agua}
\secondtranslation{käktye'pa'}

\entry{zapotillo}
\partofspeech{m}
\spanishtranslation{chäkluñtye'}
\clarification{árbol}

\entry{zarza}
\partofspeech{f}
\spanishtranslation{xk'umajtye'}
\clarification{planta}

\entry{zontle}
\partofspeech{m}
\spanishtranslation{-bajk'}
\clarification{zonte}

\entry{zonzapote}
\partofspeech{m}
\spanishtranslation{pi}

\entry{zonzo}
\partofspeech{adj}
\spanishtranslation{soñso}

\entry{zopilote}
\partofspeech{m}
\onedefinition{1}
\spanishtranslation{tya'jol}
\onedefinition{2}
\spanishtranslation{xtya'jol}
\clarification{negro}

\entry{zorra}
\partofspeech{f}
\spanishtranslation{wax}
\clarification{gris}

\entry{zorrillo}
\partofspeech{m}
\spanishtranslation{pajäy}

\entry{zumbar}
\partofspeech{vi}
\secondaryentry{zumbando}
\secondtranslation{jäñäkña, säpla}
\secondtranslation{burukña}
\clarification{una creciente o una caída}
\end{multicols*}
\end{document}